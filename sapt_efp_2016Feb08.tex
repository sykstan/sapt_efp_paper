
\documentclass[journal=jctcce]{achemso}

\usepackage{amsmath}
\usepackage[T1]{fontenc}
\usepackage[utf8]{inputenc}
\usepackage{color}


% for esi.tex
\usepackage{pdflscape}
\usepackage{multirow}
% too many unprocessed floats error
\usepackage{morefloats}
\errorcontextlines 10000

%\usepackage{floatrow}
%\floatsetup[table]{font=footnotesize}
%\floatsetup[table]{capposition=top}

%% too many unprocessed floats error
\usepackage{morefloats}
\errorcontextlines 10000


\usepackage{graphicx}
\usepackage{subfigure}
\usepackage{appendix}

\usepackage{xstring}

% for symbols in footnotes
\usepackage[symbol*]{footmisc}

\usepackage{natbib}

\newcommand{\enUnit}{kJ$ \cdot \text{mol}^{-1}$}

% commands for all the anions
\newcommand{\bfl}{ \ensuremath{ \text{BF}_4^- } }
\newcommand{\br}{ \ensuremath{ \text{Br}^- } }
\newcommand{\cl}{ \ensuremath{ \text{Cl}^- } }
\newcommand{\dca}{ \ensuremath{ \text{Dca}^- } }
\newcommand{\mes}{ \ensuremath{ \text{Mes}^- } }
\newcommand{\ntf}{ \ensuremath{ \text{NTf}_{2}^{-} } }
\newcommand{\pf}{ \ensuremath{ \text{PF}_{6}^{-} } }
\newcommand{\tos}{ \ensuremath{ \text{Tos}^- } }

\newcommand{\bflb}{ $ [ \text{BF}_4 ] $ }
\newcommand{\brb}{ $ [ \text{Br} ] $ }
\newcommand{\clb}{ $ [ \text{Cl} ] $ }
\newcommand{\dcab}{ $ [ \text{Dca} ] $ }
\newcommand{\mesb}{ $ [ \text{Mes} ] $ }
\newcommand{\ntfb}{ $ [ \text{NTf}_{2} ] $ }
\newcommand{\pfb}{ $ [ \text{PF}_{6} ] $ }
\newcommand{\tosb}{ $ [ \text{Tos} ] $ }

% for cations
\newcommand{\cat}[2]{ $ \text{C}_{#2}\text{#1}^{+} $ }
\newcommand{\catb}[2]{ $ [ \text{C}_{#2}\text{#1} ] $ }

% for ion pairs
\newcommand{\ipair}[3]{
    \IfEqCase{#3} {
        {bfl} {\ensuremath{[ \text{C}_{\text{#2}}\text{#1}] [ \text{BF}_4 ]}}  
        {br} {\ensuremath{[ \text{C}_{\text{#2}}\text{#1}] [ \text{Br} ]}}
        {cl} {\ensuremath{[ \text{C}_{\text{#2}}\text{#1}] [ \text{Cl} ]}}
        {dca} {\ensuremath{[ \text{C}_{\text{#2}}\text{#1}] [ \text{Dca} ]}}
        {mes} {\ensuremath{[ \text{C}_{\text{#2}}\text{#1}] [ \text{Mes} ]}}
        {ntf} {\ensuremath{[ \text{C}_{\text{#2}}\text{#1}] [ \text{NTf}_{2} ]}} 
        {pf} {\ensuremath{[ \text{C}_{\text{#2}}\text{#1}] [ \text{PF}_6 ]}}
        {tos} {\ensuremath{[ \text{C}_{\text{#2}}\text{#1}] [ \text{Tos} ]}}   
    } 
    [ \PackageError{ipair}{Undefined option (anion) to ipair: #3}{} ]
}

% for energies
% 2nd arg optional; \energ[Ind]{EFP}
\newcommand{\energ}[2][]{ \ensuremath{ E^{\text{#1}}_{\text{#2}} }}

\title{Comparison of the Effective Fragment Potential Method with Symmetry-Adapted Perturbation Theory in the Calculation of Intermolecular Energies for Ionic Liquids}
%\date{}
\author{Samuel Y. S. Tan}
\email{samuel.tan@monash.edu} 
\author{Ekaterina I. Izgorodina}
\email{katya.pas@monash.edu}
\affiliation[Monash University]
    {School of Chemistry, Monash University, 17 Rainforest Walk, Clayton, Victoria 3800, AUSTRALIA}

\begin{document}

\maketitle

\begin{abstract}
    The effective fragment potential (EFP) method that decomposes the interaction energy as a sum of the five fundamental forces---electrostatic, exchange-repulsion, polarisation, dispersion and charge-transfer---was applied to a large test set of ionic liquid ion pairs and compared against the state-of-the-art method, Symmetry-Adapted Perturbation Theory (SAPT).
    The ion pairs include imidazolium and pyrrolidinium cations combined with anions that are routinely used in the field of ionic liquids.
    The aug-cc-pVDZ, aug-cc-pVTZ, and 6-311++G(d,p) basis sets were used for EFP, while SAPT2+3/aug-cc-pVDZ provided the benchmark energies.
    Differences between the two methods were found to be large, and strongly dependent on the anion type. 
    For the aug-cc-pVTZ basis set, which produced the least errors, average relative errors were between 2.3--18.4\% for pyrrolidinium ion pairs and 2.1--27.7\% for imidazolium ion pairs for each individual energetic component (excluding charge-transfer) as well as the total interaction energy.
    Charge transfer gave the largest relative errors, 56\% and 63\% on average for pyrrolidinium- and imidazolium-based ion pairs, respectively.
    Scaling of the EFP components against SAPT2+3 showed improvement for polarisation (induction) and dispersion terms, thus indicating potential for the development of cost-effective alternatives for intermolecular induction and dispersion potentials for ionic liquids. 
    
    
\end{abstract}

\section{Introduction}
\label{sec:intro}

% Wed 18 Mar 2015
% section

Intermolecular interactions have an important effect on the physical and chemical properties of condensed chemical systems, especially where noncovalent interactions dominate. 
In ionic liquids (ILs), calculating the interacting energy not only requires accounting for the inherent covalent interactions and the ionic character that dictates much of the intermolecular dynamics, but also accurately including interactions such as hydrogen-bonding, $\pi$-$\pi$ stacking, van der Waals forces, etc.
\mautocite{Wendler2012, Bedrov2010, Izgorodina2011}
The noncovalent interactions in ionic liquids are often dominated by electrostatics (Coulomb), dispersion and induction (also known as polarization), as well as exchange-repulsion to a smaller extent. 
The complex interplay of all these interactions means that characterising the intermolecular dynamics of ionic liquids is a challenging task. 
\mautocite{Izgorodina2011}


Symmetry-adapted perturbation theory (SAPT) is the state-of-the-art method for calculating intermolecular interactions, and the separation of its components provides important insight into how the interactions affect the sturcture and properties of the chemical system in consideration.
\mautocite{Stone1996, Turney2012}
However, while accurate, it is very expensive computationally. This theory often partitions the intermolecular interaction energy into electrostatic, exchange, induction and dispersion components. The charge-transfer is considered a part of the induction energy. 


The general effective fragment potential (EFP) method was developed by Gordon et al. 
\mautocite{Jensen1998, Gordon2001, Gordon2009, Mullin2009, Gordon2012} 
as a computationally inexpensive method to model intermolecular interactions. 
This method was originally created to model solvents,
\mautocite{Day1996, Chen1996, Adamovic2006} 
but has then been generalised.
\mautocite{Gordon2007, Ghosh2010}
It belongs to class of fragmentation methods, and is an \emph{ab-initio} based method, without any empirical parameters, with each term developed independently of the rest. Each term in the EFP method represents an individual fundamental component of interaction energy such as electrostatics, exchange, polarization, dispersion and charge transfer. Thus it calculates the interaction energy as a sum of terms directly comparable with SAPT.


This work extends on the work done by Flick et al.,
\mautocite{Flick2012}
who undertook a systematic study on the performance of EFP compared against a raft of semi-empirical and correlated methods.
They used the S22 and S66 test sets of Hobza et al.
\mautocite{Jurecka2006, Rezac2011}    % get these on Mon
However, to date no systematic study has been done on the suitability of the EFP method for charged species like ionic liquids. 
While this method was not originally designed for charged species, its computational efficiency shows promise. 
This study attempts to identify how well EFP performs for ionic liquids in representing the intermolecular interactions.
The test set is a suite of ionic liquids at various configurations, and the EFP data will be compared against the SAPT results.
Three basis sets will be used for EFP to determine the accuracy gained when larger basis sets are used.



\section{Theoretical Background}
\label{sec:bkgrd}

% Wed 18 Mar 2015   

% section: theoretical background

\subsection{SAPT}
SAPT was first used by London \emph{et al.}
\cite{Eisenschitz1930a}
to describe the intermolecular interaction operator as a multipole expansion. 
The theory has been further improved and refined and is the current benchmark for calculating the intermolecular interaction energy between two molecules.
Jeziorski, Moszynski and Szalewicz have a comprehensive description of the theory elsewhere. 
\cite{Jeziorski1994a}
In the context of this work, ``dimer'' refers to an ion pair, while ``monomer'' refers to an individual ion.
It has to be pointed out that within the SAPT formulation the intermolecular interaction energy, defined as the difference between the total energies of the dimer and constituent monomers, is calculated free of basis set superposition error. 

In order to obtain physically sound concepts of intermolecular forces such as the electrostatic, dispersion and induction, a non-symmetric decomposition of the Hamiltonian is used.
This means that electrons are no longer indistinguishable, and the corresponding zeroth-order wavefunction no longer obeys the Pauli exclusion principle. 
As a result, anti-symmetrisation is required making the anti-symmetrised wavefunction no longer an eigenfunction of the unperturbed sum of the Hamiltonian's of constituent monomers, $H_A + H_B$, where $A$ and $B$ are monomers.
To circumvent the issue, the \emph{symmetry-adapted} perturbation procedure is applied to keep the $H_A + H_B$ sum as the unperturbed operator whilst still utilising the anti-symmetrised wavefunction.


The SAPT method has the Hamiltonian partitioned as
\begin{equation}
    H = F_A + F_B + W_A + W_B + V
\end{equation}
where $ F_A, F_B $ are the Fock operators for monomers $A$ and $B$ respectively. 
Similarly, $W_A, W_B$ are the differences between the exact Coulomb operator and the Fock operator for each monomer, whereas $V$ contains all the intermolecular terms.
SAPT perturbs all of $W_A, W_B, V$ through various orders in order to calculate the individual energy terms.
The different energy components are grouped to produce five fundamental forces as follows:

\def\doubleunderline#1{\underline{\underline{#1}}}
% note that blank lines make align unhappy, no blank lines anywhere!
\begin{flalign}
    %\begin{split}
     E_{\text{electrostatic}} = & E^{(10)}_{\text{Elst,Repl}} +
             \underline{\textcolor{blue}{E^{(12)}_{\text{Elst,Repl}}}}  +
             \doubleunderline{\textcolor{red}{E^{(13)}_{\text{Elst,Repl}}}} \\ 
    E_{\text{exchange}} = & E^{(10)}_{\text{Exch}} +
            \underline{\textcolor{blue}{E^{(11)}_{\text{Exch}}}} +
            \underline{\textcolor{blue}{E^{(12)}_{\text{Exch}}}} \\ 
    E_{\text{induction}}    = & E^{(20)}_{\text{Ind,Repl}} +
                \doubleunderline{\textcolor{red}{E^{(30)}_{\text{Ind,Repl}}}} +
                \underline{\textcolor{blue}{E^{(22)}_{\text{Ind}}}} +
                E^{(20)}_{\text{Exch-Ind,Repl}} + \\  \nonumber
                    & \doubleunderline{\textcolor{red}{E^{(30)}_{\text{Exch-Ind,Repl}}}} +
                        \underline{\textcolor{blue}{E^{(22)}_{\text{Exch-Ind}}}} +
                        \underline{\textcolor{blue}{\delta E^{(2)}_{\text{HF}}}} +
                        \doubleunderline{\textcolor{red}{\delta E^{(3)}_{\text{HF}}}} \\ 
    E_{\text{dispersion}} = & E^{(20)}_{\text{Disp}} +
                                \doubleunderline{\textcolor{red}{E^{(30)}_{\text{Disp}}}} +
                                \underline{\textcolor{blue}{E^{(21)}_{\text{Disp}}}} +
                                \underline{\textcolor{blue}{E^{(22)}_{\text{Disp}}}} + 
                                E^{(20)}_{\text{Exch-Disp}} + \\    \nonumber
                                    &   \doubleunderline{\textcolor{red}{E^{(30)}_{\text{Exch-Disp}}}} +
                                        \doubleunderline{\textcolor{red}{E^{(30)}_{\text{Ind-Disp}}}} +
                                        \doubleunderline{\textcolor{red}{E^{(30)}_{\text{Exch-Ind-Disp}}}} \\ 
    E_{\text{charge-transfer}} = & E_{\text{Ind}}(\text{\small dimer basis}) - 
                                    E_{\text{Ind}}(\text{\small monomer basis})
\end{flalign}

The superscripts in parenthesis denote the perturbation order of $V$ and $W = W_A + W_B$ respectively. 
In this work the 2+3 truncation was used in the SAPT expansion
\cite{Turney2012a}.
In the equations above, blue singly underlined terms refer to \underline{\textcolor{blue}{second order}}, whereas red doubly underlined terms represent \doubleunderline{\textcolor{red}{third order}}.
The first order has only electrostatics and exchange terms, while induction and dispersion occur in the second order.
Also present in the second order is quenching via exchange-repulsion in the intramolecular contributions to electrostatics and exchange.
SAPT2+ further includes intramolecular electron correlation terms pertaining to dispersion.
The third order, SAPT2+3, consists of additional terms for dispersion, as well as quenching of induction and dispersion by third order exchange.
In perturbation theory, the induction energy can be separated into two categories: those involving excitations from the occupied orbitals of a molecule to virtual orbitals of the same molecule, and excitations from the occupied orbitals of a molecule to the virtual orbitals of another molecule.
\cite{Stone2009a}
The latter is known as the charge-transfer energy (CT).
To calculate this interaction, the SAPT induction energy from the monomer basis set, where no charge-transfer is permitted, is subtracted from the SAPT induction energy from the dimer basis set.
Furthermore, note that induction and dispersion include exchange components due to the quenching of forces as a result of proximity of the interaction species and non-negligible orbital overlap.
\cite{Jeziorski1994a, Hohenstein2010a, Hohenstein2010b, Hohenstein2010c, Hohenstein2011a, Hohenstein2012a}



\subsection{EFP}
The effective fragment potential method is an \emph{ab-initio}-based potential method that models the intermolecular interactions of non-covalently bound systems using a cost-effective formulation.
\cite{Gordon2001a, Gordon2007a, Gordon2009a, Mullin2009a, Ghosh2010a}.
In the EFP method, the system is broken into fragments.
Typically each of the interacting molecules is a fragment. 
In this study, since only ion pairs are considered, individual ions are treated as single fragments. 
Each fragment is treated separately at the Hartree--Fock level of theory in order to generate potentials.
Then the cation and anion potentials are allowed to interact and the interaction energy is decomposed into individual components.


The EFP method partitions the interaction energy in the following way:
\begin{equation}
    \label{eq:efp-decomp}
    E_{\text{total interaction}} = E_{\text{Elst}} + E_{\text{Pol}} + E_{\text{Disp}} + 
                                    E_{\text{Repl}} + E_{\text{CT}}
\end{equation}
In order of appearance on the right hand side of Equation \ref{eq:efp-decomp}, these are the electrostatic, induction (polarization), dispersion, exchange-repulsion and charge-transfer components.
Coulomb, induction and dispersion are considered long-range interactions.
They decay as $R^{-n}$, with $n = 1$ for Coulomb, $n = 2$ to 4 for induction, and $n = 6$ for dispersion.
The short-range interactions which decay exponentially are exchange-repulsion and charge-transfer.
In the EFP formulation, the Coulomb interaction uses Stone's distributed multipolar analysis 
\cite{Stone1996a} truncated at the octopole term.
\corrected{
To correct for charge penetration effects arising from orbital overlap, a damping function is employed.
This overlap-based screening has exponential dependence on the separation.
\cite{Slipchenko2007a}
}
Induction, also known as polarisation in the EFP method, is the effect of inducing a dipole moment in a molecule by the electric field of another.
This term is treated with dipole polarisability tensors located at the centroids of localised bond and lone pair orbitals of the molecules.
\cite{Li2006a}
\corrected{
To prevent `polarisation collapse' at short intermolecular distances, EFP employs Gaussian-type damping, due to its mathematical simplicity and independence of the choice of Coulomb damping.
% even though exponential damping is more physically meaningful and has slightly better accuracy wrt SAPT
\cite{Slipchenko2009b}
}


Polarisation in EFP is analogous to induction in SAPT with one exception. 
The SAPT induction term also contains the charge-transfer energy, whereas it is calculated separately in the EFP method and is defined as the interaction between the occupied orbitals on one EFP fragment with the virtual orbitals of another fragment. 
For charge-transfer, the EFP method uses a second order perturbation at the HF level of theory
\cite{Li2006a}.
As these calculations involve the virtual orbitals, it becomes slower with increasing number of basis functions. 
For example, the water molecule has five occupied orbitals and 60 virtual orbitals in the 6-31++G(3df,2p) basis set. 
Thus the calculation for the charge-transfer term is usually 20-30 times slower than that for the other terms.
\cite{Li2006a}
In order to trim the expense of CT calculations, quasiatomic minimal-basis-set orbitals 
\cite{Lu2004a} 
are used that include the valence virtual orbitals. The latter ensures recovery of the most important CT interactions in the virtual space.


Dispersion is treated by the sum of two terms,
\begin{equation}
    \energ{Disp} = \frac{C_6}{R^6} + \frac{C_8}{R^8}.
\end{equation}
The first term is the induced dipole--induced dipole interaction.
In the EFP method, the coefficients for this $C_6$ term are calculated through the interactions between pairs of localised molecular orbitals of each ion using the time-dependent Hartree-Fock method, with the $C_8$ coefficients being approximated as $1/3$ of those of $C_6$.
\cite{Adamovic2005a}.
\corrected{
This expression is corrected for short-range charge penetration effects through a distance-dependent damping function.
}
It should be noted that this dispersion term was formulated by comparing with SAPT dispersion as the benchmark.


The exchange-repulsion is also calculated using a static localised molecular orbital basis by expanding the intermolecular overlap integral, with truncation at the quadratic term for exchange-repulsion.
\begin{equation}
   \begin{split}
    E^{\text{exch}}_{ij} = & -4 \sqrt{\frac{-2}{\pi} \ln \lvert S_{ij} \rvert } \frac{S^2_{ij}}{R_{ij}} 
                             -2 S_{ij} \left( \sum_{k \in A} F^A_{ik} S_{kj} + \sum_{l \in B} F^B_{jl}S_{il} - 2 T_{ij} \right) \\
                             &  -2 S^2_{ij} \left( \sum_{I \in A} \frac{Z_I}{R_{Ij}}  + 2 \sum_{k \in A} \frac{1}{R_{kj}} + 
                                 \sum_{J \in B} \frac{Z_J}{R_{iJ}} + 2 \sum_{l \in B} \frac{1}{R_{il}} - \frac{1}{R_{ij}} \right)
   \end{split}
\end{equation}
where $A,B$ are the effective fragments, $i, j, k$ and $l$ are the LMOs, and $I, J$ are the nuclei. 
$S$ refers to the intermolecular overlap integral, and $T$ to the kinetic energy integral.
The Fock matrix element is represented by $F$
\cite{Ghosh2010a}.
It is expected that higher order correlation effects are not well accounted for in second order exchange-repulsion.

While the computational costs for each component varies depending on system size and complexity, in general the most expensive interactions to calculate by means of EFP are the exchange-repulsion and charge-transfer interactions.
These two components might be more than five times as computationally demanding than the other three components, which are of roughly the same cost relative to each other.


Originally the exchange-repulsion and charge-transfer were designed with optimisations for neutral molecules and therefore it is suggested that these terms might not perform as well for charged species such as ionic liquids.
These interactions will be stronger due to greater orbital overlap among ions.
While stronger interaction energies in ionic liquid ion pairs might not result in higher relative errors, absolute errors would be expected to be larger.


In comparing EFP and SAPT, Table \ref{tab:sapt-efp-energy-comp} describes which terms from each method will be compared against each other.
\corrected{
It has to be pointed out that due to the difference in the definitions of charge transfer in the EFP and SAPT approaches the EFP polarization term was directly compared to the SAPT induction without inclusion of the CT energy.
}

\begin{table}
    \centering
    \caption{Energetic components from SAPT and EFP compared with one another, and abbreviation of these components used in the text.}
    \label{tab:sapt-efp-energy-comp}
    \begin{tabular}{c|c|c}
        \hline
        SAPT name               & EFP name      & Abbreviation   \\ \hline
        \energ{electrostatics}  & \energ{Elst}  & \energ{Elst}          \\
        \energ{exchange}        & \energ{Repl}  & \energ{Exch}          \\
        \energ{induction}       & \energ{Pol}   & \energ{Ind}           \\
        \energ{dispersion}      & \energ{Disp}  & \energ{Disp}          \\
        \energ{charge-transfer} & \energ{CT}    & \energ{CT}            \\ \hline
    \end{tabular}
\end{table}




\section{Methodology}
\label{sec:method}
%%Note that these numbers do not include \ipair{mpyr}{3}{dca} (p5) which is an outlier (0.331 \enUnit).

% section: method

\subsubsection{Chemical systems studied}
The chemical systems studied were single ion pairs of ionic liquids. 
Routinely used anions  such as tetrafluoroborate (\bfl), bromide (\br), chloride (\cl), dicyanamide (\dca), mesylate (\mes), tosylate (\tos), hexafluorophosphate (\pf) and bis\{(tri\-fluoro\-meth\-yl)\-sulf\-onyl\}\-amide (\ntf) were used in this study. 
These anions were combined with N-alkyl-N'-pyrrolidinium (denoted here as $ \text{C}_n\text{mpyr}$) and 1-methyl-3-alkyl-imidazolium (denoted here as $ \text{C}_n\text{mim}$) cations with varying alkyl chain from methyl, ethyl, propyl to butyl.
The names and abbreviations of the different cations and anions are tabulated in 
Table \ref{tab:cation-anion-list}. Further in the text, halides are abbreviated as "Hal", whereas the rest of the anions are referred to as typical ionic liquid anions (TILA).

\begin{table}[ht]
    \begin{centering}
    \footnotesize
    \begin{tabular}{c|c|c|c}
        \hline 
        Cations  & abbreviation & Anions & abbreviation\tabularnewline
        \hline 
        1-methyl-3-methyl-imidazolium & $\text{C}_{1}\text{mim}^{+}$ & tetrafluoroborate &  $\text{BF}_{\text{4}}^{-}$\tabularnewline
        1-methyl-3-ethyl-imidazolium & $\text{C}_{2}\text{mim}^{+}$ & bromide &  $\text{Br}^{-}$\tabularnewline
        1-methyl-3-propyl-imidazolium & $\text{C}_{3}\text{mim}^{+}$ & chloride  & $\text{Cl}^{-}$\tabularnewline
        1-methyl-3-butyl-imidazolium & $\text{C}_{4}\text{mim}^{+}$ & dicyanamide  & $\text{Dca}^{-}$\tabularnewline
        N,N’-dimethyl-pyrrolidinium & $\text{C}_{1}\text{mpyr}^{+}$ & mesylate &  $\text{Mes}^{-}$\tabularnewline
        N-ethyl-N'-methyl-pyrrolidinium & $\text{C}_{2}\text{mpyr}^{+}$ & bis\{(trifluoromethyl)sulfonyl\}amide  & $\text{NTf}_{\text{2}}^{-}$\tabularnewline
        N-propyl-N'-methyl-pyrrolidinium & $\text{C}_{3}\text{mpyr}^{+}$ & hexafluorophosphate &  $\text{PF}_{\text{6}}^{-}$\tabularnewline
        N-butyl-N'-methyl-pyrrolidinium & $\text{C}_{4}\text{mpyr}^{+}$ & tosylate  & $\text{Tos}^{-}$\tabularnewline
        \hline 
    \end{tabular}
    \caption{List of cations and anions }
    \label{tab:cation-anion-list}
    \par\end{centering}
\end{table}
%\begin{multicols}{2}


For each cation-anion combination, different configurations of these ion pairs were incorporated. 
These configurations differ by how the anion interacts with the cation. 
In the imidazolium-based cation  the anion can interact with the cation above and below the imidazolium ring and these configurations are referred to as p1 and p4, respectively.
When the anion interacts with the cation in the $\text{C}_2\text{--H}$ bond plane this configuration is referred to as in-plane interactions and denoted further in the text as p2 and p3.
The different configurations for \ipair{mim}{3}{br} are presented in Figure
\protect\ref{fig:conf-c3mim-br}
as an example. 

%\end{multicols}

% original graphics located at ~/Dropbox/QuantumChem/il_structure_images
\begin{figure}
    \centering
    \mbox{
    \subfigure[\protect\ipair{mim}{3}{br} (p1) ]{\includegraphics[scale=0.3]{./images/c3mim-br-p1.pdf}}
    \subfigure[\protect\ipair{mim}{3}{br} (p2) ]{\includegraphics[scale=0.3]{./images/c3mim-br-p2.pdf}}
    }
    \mbox{
    \subfigure[\protect\ipair{mim}{3}{br} (p3) ]{\includegraphics[scale=0.3]{./images/c3mim-br-p3.pdf}}
    \subfigure[\protect\ipair{mim}{3}{br} (p4) ]{\includegraphics[scale=0.3]{./images/c3mim-br-p4.pdf}}
    }                                 
    % need the \protect to make hyperref and the macro happy together
    \caption{Different configurations of \protect\ipair{mim}{3}{br} \label{fig:conf-c3mim-br}}
\end{figure}

For the case of pyrrolidinium-based ion pairs, the configurations studied are different as the anion tends to interact with the nitrogen centre of the cation from three energetically domineering positions denoted here as p1, p2 and p3.
\cite{Izgorodina2014a}
The \ntf anion has multiple interaction sites such as the central nitrogen and the oxygens on the sulfonyl groups, as shown in Figure \ref{fig:conf-c2mpyr-ntf2}.
Therefore, there are more ion pair configurations corresponding to the anion interacting through different centres compared to halides.
In the case of the \ntf anion there are six possible configurations (for more detail see Figure \ref{fig:conf-c2mpyr-ntf2}). 


% original graphics located at ~/Dropbox/QuantumChem/il_structure_images
\begin{figure}
    \centering
    \mbox{
    \subfigure[\protect\ipair{mpyr}{2}{ntf} (p1) ]{\includegraphics[scale=0.27]{./images/c2mpyr-ntf2-p1.pdf}}
    \subfigure[\protect\ipair{mpyr}{2}{ntf} (p2)]{\includegraphics[scale=0.27]{./images/c2mpyr-ntf2-p2.pdf}}
    }
    \mbox{
    \subfigure[\protect\ipair{mpyr}{2}{ntf} (p3) ]{\includegraphics[scale=0.27]{./images/c2mpyr-ntf2-p3.pdf}}
    \subfigure[\protect\ipair{mpyr}{2}{ntf} (p4) ]{\includegraphics[scale=0.27]{./images/c2mpyr-ntf2-p4.pdf}}
    }                                 
    \mbox{                            
    \subfigure[\protect\ipair{mpyr}{2}{ntf} (p5) ]{\includegraphics[scale=0.27]{./images/c2mpyr-ntf2-p5.pdf}}
    \subfigure[\protect\ipair{mpyr}{2}{ntf} (p6) ]{\includegraphics[scale=0.27]{./images/c2mpyr-ntf2-p6.pdf}}
    }
    % need the \protect to make hyperref and the macro happy together
    \caption{Different configurations of \protect\ipair{mpyr}{2}{ntf} \label{fig:conf-c2mpyr-ntf2}}
\end{figure}


The \catb{mim}{n}X series of ion pairs, where X represents chloride or bromide, were optimised at the MP2/aug-cc-pVDZ level, whilst for the other TILAs, MP2/6-31+G(d,p) was used.
For the \catb{mpyr}{n} series of ion pairs, geometry optimisation was performed at the B3LYP/6-31+G(d) level.
The geometries of these configurations have previously been published by our group.
\cite{Izgorodina2014a, Rigby2014a}


\subsubsection{Software}

% SAPT
The \textsc{Psi4} quantum chemistry package was used for the SAPT2+3 calculations. 
\cite{Turney2012a}
All SAPT calculations were performed using the aug-cc-pVDZ basis set, unless stated otherwise.
\cite{Izgorodina2014a}

% EFP
The GAMESS-US software package was used to perform the EFP calculations
\cite{Schmidt1993a, Gordon2005a}.
Three basis sets were used for EFP calculations: aug-cc-pVDZ, aug-cc-pVTZ and the Pople basis set 6-311++G**.\footnote{For the bromide anion, since it is not included in the 6-311++G** basis set, 6-311G** basis functions were used instead.}
Three basis sets were employed to see how consistently the method performs for this series of basis sets, and the effect of the basis set on the EFP performance for ionic liquids.

\subsubsection{Basis sets}
While three basis sets were used to give an indication of basis set dependency for the EFP method, only the aug-cc-pDVZ basis set was used for SAPT2+3. 
A number of test calculations were performed using aug-cc-pVTZ, indicated that aug-cc-pVDZ gave satisfactory accuracy.
The calculations using aug-cc-pVTZ were run on select representative systems, largely with halide anions, namely \catb{mim}{n}X, where X = Cl and Br.
The rest of the ion pairs for which SAPT2+3/(aug-cc-pVTZ) calculations were run are \catb{mpyr}{n=1,3}X, where X = \bfl and \cl.
Halides---in particular, chlorides---were selected due to smaller system sizes (monoatomic anions). 
These are strongly bound to the cation, representing challenging systems from the theoretical point of view.
\cite{Lehmann2010a}


The differences between these two basis sets are reported in the ESI.
The consistency between the two basis sets is immediately apparent: with the triple-$\zeta$ basis set, electrostatics is always underestimated by 2.7 \enUnit~ on average (3.0 \enUnit~ for imidazolium systems and 1.9 \enUnit~for pyrrolidinium systems). 
Exchange follows the same trend (an overestimation by 4.9 \enUnit~on average).
It is not surprising that the larger aug-cc-pVTZ basis set leads to greater recovery of the dispersion energy by 5.6 \enUnit~on average.
There is no such pattern observed in the induction energy, with both basis sets giving excellent agreement to each other below 1 \enUnit.
Note that the difference between the two basis sets for charge-transfer is a 8.6 \enUnit~difference on average with a standard deviation of 1.7 \enUnit.
Charge-transfer is expected to decrease with increasing basis set size in SAPT, as observed. 
Due to this, agreement between basis sets cannot be based on the comparison of charge-transfer energies.
In the total interaction energy, the largest difference of 9 \enUnit~ comes from \ipair{mim}{1}{cl}, specifically the in-plane configuration p2.
The average difference for the total interaction energy is 7.3 \enUnit~ with a standard deviation of 1.2 \enUnit.
Charge-transfer has the largest contribution to this large difference, and as the small standard deviation attests, this difference is consistent.

Excluding charge-transfer, the differences between the two basis sets for each energetic component (that is, electrostatics, exchange, induction and dispersion) range between -7.0 to 4.8 \enUnit.
Considering the low standard deviations  (0.5 to 1.5 \enUnit), this indicates that these two basis sets differ \emph{consistently} for each energetic component. 
It has to be noted that the aug-cc-pVTZ basis set requires tremendous amounts of CPU time---in most cases more than double the amount of that required for the aug-cc-pVDZ basis set.
For example, for \ipair{mim}{4}{cl}, aug-cc-pVTZ required 329 CPU hours and 22 GB of memory, compared to 26  hours and 4 GB for aug-cc-pVDZ. Taking the computational expense into account, the aug-cc-pVDZ is the largest basis set possible for a number of the bulky ionic liquid ion pairs studied, such as \ipair{pyr}{4}{ntf}.

SAPT2+3 calculations on the intermolecular complexes in the S22 and S66 data sets were also performed with charge-transfer using aug-cc-pVDZ.
\cite{Jurecka2006a, Takatani2010a, Rezac2011a}


\section{Results and Discussion}
\label{sec:results}

% due to MAE SD formulas etc

% subsection: CCSD(T) and SAPT
% and aug-cc-pVTZ as well

To ensure the results given by SAPT agree well with other benchmark methods, the CCSD(T)/CBS energies were also calculated for the ion pairs. 
In this comparison, only the total energy, the HF energy, and the correlation correction can be compared, since CCSD(T)/CBS only has these values. 
The statistics for the differences are tabulated below, with the halide systems separated from the rest.
Note that these numbers do not include \ipair{mpyr}{3}{dca} (p5) which is an outlier (0.331 \enUnit).

% numbers exclude c3mpyr-dca-p5!!!
%\end{multicols}

\begin{table}[h]
\centering
\small
    \begin{tabular}{ccc|cc}
        \multicolumn{5}{c}{$\Delta$(HF Energy) (\enUnit)}                                                \\ \hline
                & \multicolumn{2}{c}{Non-halides}        & \multicolumn{2}{c}{Halides}          \\ \hline
                & Statistics & System                    & Statistics & System                  \\ \hline
        Mean    & -1.49E-06      & -                         &  1.73E-06      & -                       \\
        Median  & -8.61E-07      & -                         &  1.03E-06      & -                       \\
        Std dev & 2.40E-05       & -                         &  1.27-05       & -                       \\
        Min     & -7.49E-05      & \ipair{mim}{4}{tos} (p1)  & -1.95E-05      & \ipair{mpyr}{2}{br} (p2) \\ 
        Max     & 6.25E-05       & \ipair{mpyr}{1}{dca} (p2) & 4.09E-05       & \ipair{mpyr}{4}{cl} (p1) \\ \hline
    \end{tabular}
    \caption{CCSD(T)/aug-cc-pVDZ and SAPT2+3 differences for HF energy}
    \label{tab:ccsd-sapt-hf}
\end{table}

\begin{table}[h]
\centering
\small
    \begin{tabular}{ccc|cc}
        \multicolumn{5}{c}{$\Delta$(Correlation correction) (\enUnit)}                                    \\ \hline
                & \multicolumn{2}{c}{Non-halides}        & \multicolumn{2}{c}{Halides}          \\ \hline
                & Statistics & System                    & Statistics & System                  \\ \hline
        Mean    & -1.89      & -                         & -3.27      & -                       \\
        Median  & -1.40      & -                         & -1.76      & -                       \\
        Std dev & 2.95       & -                         & 3.36       & -                       \\
        Min     & -9.07      & \ipair{mim}{4}{tos} (p1)  & -10.68     & \ipair{mim}{4}{cl} (p4) \\ 
        Max     & 2.37       & \ipair{mpyr}{1}{bfl} (p2) & 0.59       & \ipair{mpyr}{1}{br} (p2) \\ \hline
    \end{tabular}
    \caption{CCSD(T)/aug-cc-pVDZ and SAPT2+3 differences for correlation correction}
    \label{tab:ccsd-sapt-corr}
\end{table}

\begin{table}[h]
\centering
\small
    \begin{tabular}{ccc|cc}
        \multicolumn{5}{c}{$\Delta$(Total Energy) (\enUnit)}                                    \\ \hline
                & \multicolumn{2}{c}{Non-halides}        & \multicolumn{2}{c}{Halides}          \\ \hline
                & Statistics & System                    & Statistics & System                  \\ \hline
        Mean    & -1.77      & -                         & -1.90      & -                       \\
        Median  & -1.05      & -                         & -0.76      & -                       \\
        Std dev & 2.81       & -                         & 2.84       & -                       \\
        Min     & -8.67      & \ipair{mim}{4}{tos} (p1)  & -8.19      & \ipair{mim}{4}{cl} (p4) \\ 
        Max     & 2.13       & \ipair{mpyr}{1}{dca} (p2) & 1.50       & \ipair{mpyr}{1}{br} (p2) \\ \hline
    \end{tabular}
    \caption{CCSD(T)/CBS and SAPT2+3 differences for Total Energy}
    \label{tab:ccsd-sapt-Etot}
\end{table}

%\begin{multicols}{2}


In table \ref{tab:ccsd-sapt-hf}, the HF energies from both methods agree to the fourth decimal place in all cases.
This degree of agreement is expected, since the HF method is the same in principle for both methods.

CCSD(T)/CBS improves the HF energy by adding on the correlation correction; i.e. there are only two components, as opposed to SAPT.
Hence the difference between the SAPT2+3 Total Energy and the SAPT Hartree--Fock energy was compared against the correlation correction; this is done in the second table above.
The difference between the two correlation correction values showed excellent agreement. 
In fact, this difference can be considered as the difference between the SAPT and CCSD(T), since the difference in the HF energy is so small.
Thus that is why the statistics for the correlation correction are almost identical to the statistics for the difference between CCSD(T)/aug-cc-pVDZ and SAPT2+3 total energies.
Therefore the differences in total energy between CCSD(T)/aug-cc-pVDZ and SAPT2+3 are not shown. 


In table \ref{tab:ccsd-sapt-Etot} the difference between total energy for CCSD(T)/aug-cc-pVQZ and SAPT are shown.
The reason for the smaller values than the correlation correction are due to the fact that CCSD(T)/CBS results are used, while in the correlation comparison the aug-cc-pVDZ basis set was used for consistency. 
This indicates that the SAPT total energy results are actually converging with CCSD(T)/CBS, and points very strongly to SAPT2+3 being a robust method that approaches to the complete basis set limit.

%\end{multicols}

\begin{table}[h]
\centering
\small
    \begin{tabular}{ccccccc}
\hline
Statistic & $\Delta$(HF Energy) & System                 & $\Delta$(Corr) & System                   & $\Delta$(Total Energy) & System                  \\ \hline  
Mean      & 1.68E-06            & -                      & -10.52         & -                        & -10.39                 & -                       \\   
Median    & -8.99E-07           & -                      & -10.40         & -                        & -10.29                 & -                       \\   
Std dev   & 5.67E-06            & -                      & 2.80           & -                        & 2.76                   & -                       \\    
Min       & -5.54E-06           & \ipair{mim}{4}{cl} (p4)& -15.04         & \ipair{mim}{4}{cl} (p4)  & -14.87                 & \ipair{mim}{4}{cl} (p4) \\    
Max       & 1.17E-05            & \ipair{mim}{2}{cl} (p2)& -6.04          & \ipair{mpyr}{3}{cl} (p2) & -5.95                  & \ipair{mpyr}{3}{cl} (p2)\\ 
\hline
    \end{tabular}
    \caption{CCSD(T) and SAPT2+3/aug-cc-pVTZ differences in \enUnit}
    \label{tab:ccsd-sapt-atz}
\end{table}

%\begin{multicols}{2}

Comparing the SAPT2+3/aug-cc-pVTZ with the CCSD(T) results, the same conclusion can be derived. 
The statistics for the correlation correction difference are given in table \ref{tab:ccsd-sapt-atz}.

These errors are relatively large because of the difficulty in modelling halides. 
Furthermore, there are a lot less pyrrolidinium systems compared to imidazolium, which tend to have larger variances in interaction energies.
However, HF energy differences are on the same scale as the previous aug-cc-pVDZ:

Indeed, comparing the aug-cc-pVTZ total energies with CCSD(T)/CBS yields statistics very similar to that of the differences between the correlation correction.
This indicates most of the differences are coming from the correlation correction. 
Strictly speaking, since SAPT2+3 treats the intermolecular interaction in a very different way mathematically, it is does not correct for electronic correlation.
The correlation correction obtained as the difference of the Total Energy and the HF energy is purely for comparison purposes with CCSD(T).

The agreement between SAPT and CCSD(T)/CBS results validates the reliability of SAPT as a method  and aug-cc-pVDZ as a basis set to accurately determine the decomposition of the total interaction energy.



\subsection{Differences between SAPT2+3 and EFP}
\label{subsec:comparison}
% absolute and relative (ratio diffferences)
% subsection: differences between SAPT and EFP


%\begin{footnotesize}

%(probably in supplementary info)

\begin{table}[h]
\centering
\scriptsize
\caption{Statistics on differences between SAPT2+3 and EFP by basis set, cation and energy component, classified into halides and TILAs.}
\label{tab:sapt-efp-diff-stats}
\begin{tabular}{lll|rrrr|rrrr}
\hline
 Cation                               & Basis                          & Component & \multicolumn{4}{c}{Halides}                   & \multicolumn{4}{c}{TILAs}                        \\
                                      &                                &           & MAE    & SD    & Max    &                     & MAE     & SD    & Max    &                      \\ \hline
 \multirow{18}{*}{\catb{mim}{n}}      & \multirow{6}{*}{AVDZ}          & Elst      & 10.5   & 9.0   & 23.5   & \ipair{mim}{3}{cl}  & 13.0    & 12.5  & 69.7   & \ipair{mim}{4}{tos}  \\
                                      &                                & Exch      & 6.8    & 7.9   & -18.0  & \ipair{mim}{3}{br}  & 13.9    & 6.2   & 23.8   & \ipair{mim}{2}{mes}  \\
                                      &                                & Ind       & 11.6   & 7.4   & -30.6  & \ipair{mim}{4}{cl}  & 8.4     & 3.8   & -25.6  & \ipair{mim}{1}{ntf}  \\
                                      &                                & Disp      & 8.9    & 1.8   & -13.4  & \ipair{mim}{4}{br}  & 16.6    & 6.2   & 27.9   & \ipair{mim}{3}{ntf}  \\
                                      &                                & CT        & 40.7   & 16.4  & -64.8  & \ipair{mim}{3}{br}  & 5.1     & 1.8   & -9.4   & \ipair{mim}{4}{mes}  \\
                                      &                                & Total     & 18.0   & 20.1  & -41.2  & \ipair{mim}{3}{br}  & 36.8    & 15.1  & 92.6   & \ipair{mim}{4}{tos}  \\ \cline{2-11}
                                      & \multirow{6}{*}{AVTZ}          & Elst      & 9.3    & 11.6  & -37.3  & \ipair{mim}{4}{br}  & 9.2     & 6.9   & 25.5   & \ipair{mim}{2}{tos}  \\
                                      &                                & Exch      & 8.6    & 7.7   & -16.8  & \ipair{mim}{4}{br}  & 15.5    & 11.5  & 28.0   & \ipair{mim}{4}{mes}  \\
                                      &                                & Ind       & 12.6   & 14.0  & 37.7   & \ipair{mim}{3}{br}  & 7.4     & 3.4   & -20.7  & \ipair{mim}{1}{ntf}  \\
                                      &                                & Disp      & 4.5    & 2.5   & -9.8   & \ipair{mim}{4}{br}  & 17.1    & 6.0   & 25.1   & \ipair{mim}{3}{dca}  \\
                                      &                                & CT        & 31.2   & 20.9  & -62.5  & \ipair{mim}{1}{br}  & 3.1     & 2.0   & -6.6   & \ipair{mim}{3}{mes}  \\
                                      &                                & Total     & 22.9   & 26.3  & 50.9   & \ipair{mim}{3}{br}  & 36.3    & 12.5  & 69.7   & \ipair{mim}{2}{tos}  \\ \cline{2-11}
                                      & \multirow{6}{*}{6-311++G(d,p)} & Elst      & 13.5   & 13.1  & -33.9  & \ipair{mim}{4}{br}  & 4.8     & 7.6   & 32.8   & \ipair{mim}{1}{ntf}  \\
                                      &                                & Exch      & 21.7   & 21.8  & 48.8   & \ipair{mim}{2}{br}  & 31.2    & 13.7  & 50.2   & \ipair{mim}{4}{mes}  \\
                                      &                                & Ind       & 32.7   & 7.9   & -47.8  & \ipair{mim}{4}{cl}  & 11.0    & 3.4   & -27.5  & \ipair{mim}{1}{ntf}  \\
                                      &                                & Disp      & 27.2   & 3.9   & -34.1  & \ipair{mim}{4}{br}  & 6.7     & 6.3   & 16.5   & \ipair{mim}{3}{ntf}  \\
                                      &                                & CT        & 37.4   & 16.7  & -62.5  & \ipair{mim}{3}{br}  & 4.1     & 1.5   & -8.1   & \ipair{mim}{4}{mes}  \\
                                      &                                & Total     & 46.2   & 21.3  & -104.3 & \ipair{mim}{4}{cl}  & 29.4    & 15.6  & 72.6   & \ipair{mim}{3}{ntf}  \\ \hline \hline
 \multirow{18}{*}{\catb{mpyr}{n}}     & \multirow{6}{*}{AVDZ}          & Elst      & 18.3   & 7.4   & 29.4   & \ipair{mpyr}{4}{br} & 9.0     & 9.5   & 44.7   & \ipair{mpyr}{4}{tos} \\
                                      &                                & Exch      & 5.0    & 5.3   & 14.3   & \ipair{mpyr}{4}{cl} & 13.7    & 6.6   & 27.8   & \ipair{mpyr}{1}{mes} \\
                                      &                                & Ind       & 5.1    & 3.5   & -12.1  & \ipair{mpyr}{2}{cl} & 7.8     & 1.5   & -12.2  & \ipair{mpyr}{1}{tos} \\
                                      &                                & Disp      & 9.8    & 1.2   & -11.5  & \ipair{mpyr}{3}{br} & 6.2     & 2.7   & 11.9   & \ipair{mpyr}{1}{dca} \\
                                      &                                & CT        & 26.1   & 3.4   & -31.1  & \ipair{mpyr}{4}{br} & 5.4     & 1.9   & -10.6  & \ipair{mpyr}{1}{mes} \\
                                      &                                & Total     & 12.2   & 13.6  & 29.0   & \ipair{mpyr}{4}{cl} & 20.5    & 10.7  & 61.4   & \ipair{mpyr}{4}{tos} \\ \cline{2-11}
                                      & \multirow{6}{*}{AVTZ}          & Elst      & 9.2    & 4.7   & -19.7  & \ipair{mpyr}{3}{br} & 4.4     & 5.1   & 15.1   & \ipair{mpyr}{2}{ntf} \\
                                      &                                & Exch      & 5.0    & 4.9   & -10.2  & \ipair{mpyr}{3}{br} & 16.0    & 7.7   & 25.2   & \ipair{mpyr}{1}{mes} \\
                                      &                                & Ind       & 3.0    & 3.8   & 8.5    & \ipair{mpyr}{2}{br} & 7.6     & 1.5   & -9.9   & \ipair{mpyr}{1}{ntf} \\
                                      &                                & Disp      & 5.8    & 1.0   & -7.9   & \ipair{mpyr}{2}{br} & 6.7     & 2.4   & 12.1   & \ipair{mpyr}{1}{dca} \\
                                      &                                & CT        & 14.4   & 2.9   & -19.8  & \ipair{mpyr}{1}{br} & 2.0     & 1.5   & -5.2   & \ipair{mpyr}{1}{mes} \\
                                      &                                & Total     & 8.6    & 9.1   & -21.8  & \ipair{mpyr}{3}{br} & 22.4    & 8.1   & 45.5   & \ipair{mpyr}{2}{tos} \\ \cline{2-11}
                                      & \multirow{6}{*}{6-311++G(d,p)} & Elst      & 9.3    & 3.7   & -15.8  & \ipair{mpyr}{3}{br} & 6.8     & 9.8   & 43.7   & \ipair{mpyr}{1}{tos} \\
                                      &                                & Exch      & 25.2   & 19.3  & 48.1   & \ipair{mpyr}{1}{br} & 29.7    & 9.4   & 42.4   & \ipair{mpyr}{1}{tos} \\
                                      &                                & Ind       & 21.1   & 4.9   & -27.3  & \ipair{mpyr}{1}{br} & 11.8    & 2.4   & -18.0  & \ipair{mpyr}{1}{tos} \\
                                      &                                & Disp      & 24.5   & 2.6   & -28.7  & \ipair{mpyr}{4}{br} & 2.9     & 3.6   & 6.8    & \ipair{mpyr}{3}{dca} \\
                                      &                                & CT        & 19.8   & 3.3   & -25.0  & \ipair{mpyr}{4}{br} & 3.8     & 1.5   & -6.8   & \ipair{mpyr}{1}{mes} \\ 
                                      &                                & Total     & 23.8   & 13.3  & -45.6  & \ipair{mpyr}{3}{cl} & 26.6    & 16.1  & 77.3   & \ipair{mpyr}{1}{tos} \\ \hline
\end{tabular}
\end{table}


\begin{table}[h]
\centering
\scriptsize
\caption{Percentage of differences between SAPT2+3 and EFP for each energetic component}
\label{tab:sapt-efp-perc-stats}
\begin{tabular}{lll|rrrr|rrrr}
\hline
Cation                            & Basis                           & Component & \multicolumn{4}{c}{Halides}                & \multicolumn{4}{c}{TILAs}                   \\
                                  &                                 &        & MAE   & SD   & Max   &                     & MAE & SD  & Max  &                         \\ \hline       
\multirow{18}{*}{\catb{mim}{n}}   & \multirow{6}{*}{AVDZ}           & Elst   & 2.4   & 1.6  & 5.5   & \ipair{mim}{3}{cl}  & 3.4  & 2.9  & 16.9 & \ipair{mim}{4}{tos}   \\              
                                  &                                 & Exch   & 3.8   & 2.2  & 9.1   & \ipair{mim}{3}{cl}  & 12.2 & 4.6  & 18.5 & \ipair{mim}{2}{bfl}   \\              
                                  &                                 & Ind    & 12.9  & 7.3  & 28.0  & \ipair{mim}{4}{cl}  & 17.3 & 5.4  & 38.2 & \ipair{mim}{1}{ntf}   \\              
                                  &                                 & Disp   & 18.0  & 5.8  & 30.0  & \ipair{mim}{4}{br}  & 26.8 & 7.8  & 42.0 & \ipair{mim}{3}{dca}   \\              
                                  &                                 & CT     & 88.3  & 9.3  & 99.0  & \ipair{mim}{4}{cl}  & 67.8 & 10.2 & 79.9 & \ipair{mim}{2}{dca}   \\              
                                  &                                 & Total  & 4.5   & 2.9  & 10.1  & \ipair{mim}{3}{br}  & 9.7  & 3.4  & 22.4 & \ipair{mim}{4}{tos}   \\ \cline{2-11}
                                  & \multirow{6}{*}{AVTZ}           & Elst   & 2.1   & 2.1  & 8.1   & \ipair{mim}{4}{br}  & 2.4  & 1.5  & 6.1  & \ipair{mim}{2}{tos}   \\              
                                  &                                 & Exch   & 4.7   & 2.7  & 9.4   & \ipair{mim}{4}{br}  & 13.7 & 7.7  & 22.8 & \ipair{mim}{4}{bfl}   \\              
                                  &                                 & Ind    & 13.9  & 7.9  & 40.1  & \ipair{mim}{3}{br}  & 15.7 & 6.0  & 30.9 & \ipair{mim}{1}{ntf}   \\              
                                  &                                 & Disp   & 9.7   & 6.4  & 22.1  & \ipair{mim}{4}{br}  & 27.7 & 7.7  & 41.2 & \ipair{mim}{3}{dca}   \\              
                                  &                                 & CT     & 62.7  & 26.5 & 93.4  & \ipair{mim}{2}{cl}  & 41.6 & 18.8 & 66.4 & \ipair{mim}{3}{pf}    \\              
                                  &                                 & Total  & 5.7   & 3.9  & 13.2  & \ipair{mim}{3}{br}  & 9.6  & 2.8  & 16.8 & \ipair{mim}{2}{tos}   \\ \cline{2-11}
                                  & \multirow{6}{*}{6-311++G(d,p)}  & Elst   & 3.0   & 2.5  & 8.4   & \ipair{mim}{4}{br}  & 1.3  & 1.7  & 9.5  & \ipair{mim}{1}{ntf}   \\              
                                  &                                 & Exch   & 12.2  & 10.1 & 28.9  & \ipair{mim}{3}{br}  & 28.4 & 12.4 & 42.6 & \ipair{mim}{3}{ntf}   \\              
                                  &                                 & Ind    & 36.5  & 7.4  & 50.4  & \ipair{mim}{3}{br}  & 24.0 & 5.1  & 41.1 & \ipair{mim}{1}{ntf}   \\              
                                  &                                 & Disp   & 53.0  & 4.4  & 61.2  & \ipair{mim}{3}{br}  & 10.4 & 6.8  & 23.0 & \ipair{mim}{4}{dca}   \\              
                                  &                                 & CT     & 79.9  & 11.6 & 93.6  & \ipair{mim}{3}{br}  & 57.7 & 17.0 & 85.5 & \ipair{mim}{3}{pf}    \\              
                                  &                                 & Total  & 11.4  & 5.2  & 25.6  & \ipair{mim}{4}{cl}  & 7.9  & 4.1  & 20.2 & \ipair{mim}{3}{ntf}   \\ \hline \hline
\multirow{18}{*}{\catb{mpyr}{n}}  & \multirow{6}{*}{AVDZ}           & Elst   & 4.5   & 1.9  & 7.6   & \ipair{mpyr}{4}{br}  & 2.6  & 2.1  & 12.2 & \ipair{mpyr}{4}{tos}  \\              
                                  &                                 & Exch   & 3.6   & 2.9  & 10.5  & \ipair{mpyr}{4}{cl}  & 15.5 & 5.6  & 23.0 & \ipair{mpyr}{1}{bfl}  \\              
                                  &                                 & Ind    & 6.9   & 4.2  & 16.3  & \ipair{mpyr}{2}{cl}  & 16.5 & 3.7  & 24.9 & \ipair{mpyr}{2}{ntf}  \\              
                                  &                                 & Disp   & 21.6  & 1.8  & 24.9  & \ipair{mpyr}{3}{br}  & 13.6 & 5.2  & 28.0 & \ipair{mpyr}{1}{dca}  \\              
                                  &                                 & CT     & 102.0 & 4.6  & 113.1 & \ipair{mpyr}{4}{br}  & 77.9 & 6.7  & 92.2 & \ipair{mpyr}{3}{dca}  \\              
                                  &                                 & Total  & 3.2   & 2.4  & 7.7   & \ipair{mpyr}{4}{cl}  & 5.7  & 2.7  & 16.2 & \ipair{mpyr}{4}{tos}  \\ \cline{2-11}
                                  & \multirow{6}{*}{AVTZ}           & Elst   & 2.3   & 1.2  & 4.8   & \ipair{mpyr}{3}{br}  & 1.3  & 1.0  & 4.9  & \ipair{mpyr}{2}{ntf}  \\              
                                  &                                 & Exch   & 3.4   & 2.1  & 6.9   & \ipair{mpyr}{3}{br}  & 18.4 & 7.6  & 26.6 & \ipair{mpyr}{1}{bfl}  \\              
                                  &                                 & Ind    & 4.1   & 3.3  & 11.8  & \ipair{mpyr}{2}{br}  & 16.1 & 4.3  & 21.8 & \ipair{mpyr}{2}{ntf}  \\              
                                  &                                 & Disp   & 12.8  & 1.9  & 17.3  & \ipair{mpyr}{2}{br}  & 14.8 & 4.2  & 28.5 & \ipair{mpyr}{1}{dca}  \\              
                                  &                                 & CT     & 55.8  & 4.4  & 64.8  & \ipair{mpyr}{1}{br}  & 27.3 & 13.1 & 44.9 & \ipair{mpyr}{2}{pf}   \\              
                                  &                                 & Total  & 2.3   & 1.8  & 5.7   & \ipair{mpyr}{3}{br}  & 6.3  & 2.0  & 11.4 & \ipair{mpyr}{2}{tos}  \\ \cline{2-11}
                                  & \multirow{6}{*}{6-311++G(d,p)}  & Elst   & 2.3   & 0.9  & 3.9   & \ipair{mpyr}{3}{br}  & 1.9  & 2.5  & 11.7 & \ipair{mpyr}{3}{tos}  \\              
                                  &                                 & Exch   & 17.2  & 12.9 & 32.6  & \ipair{mpyr}{4}{br}  & 35.0 & 10.1 & 45.8 & \ipair{mpyr}{3}{ntf}  \\              
                                  &                                 & Ind    & 28.7  & 7.1  & 38.7  & \ipair{mpyr}{1}{br}  & 24.7 & 4.5  & 30.5 & \ipair{mpyr}{4}{dca}  \\              
                                  &                                 & Disp   & 54.2  & 3.0  & 58.1  & \ipair{mpyr}{2}{br}  & 7.1  & 5.6  & 17.3 & \ipair{mpyr}{2}{pf}   \\              
                                  &                                 & CT     & 76.7  & 3.0  & 81.5  & \ipair{mpyr}{4}{br}  & 54.3 & 11.9 & 72.0 & \ipair{mpyr}{4}{bfl}  \\              
                                  &                                 & Total  & 6.2   & 3.4  & 11.5  & \ipair{mpyr}{3}{cl}  & 7.4  & 4.2  & 20.0 & \ipair{mpyr}{1}{tos}  \\ \hline
\end{tabular}
\end{table}


% theory
%The names for the components differ between the methods. 
%For SAPT, it is electrostatics, exchange-repulsion, induction, and dispersion that make up the total interaction energy. 
%The SAPT charge-transfer energy is calculated in Psi4 as the difference in total induction between the dimer and mononmer basis sets. 
%This is because in SAPT, the charge-transfer energy is included in the total induction energy, i.e.
%
%\begin{equation*}
%    \energ[tot Ind]{SAPT} = \energ[Ind]{SAPT} + \energ[CT]{SAPT}
%\end{equation*}
%
%
%The components that make up the total EFP interaction energy, in order corresponding to their SAPT equivalents, are electrostatics, repulsion, polarizaton, dispersion, and charge-transfer. 
%In the EFP method, the charge-transfer energy is considered separate from polarization as a part of the total energy. 
%
%In comparing the two methods, the full induction energy from SAPT will be compared with the polarization energy from EFP; i.e. not the sum of polarization energy with charge-transfer.

%Therefore, to compare the induction/polarization component between the two methods, the polarization energy will be added to the charge-transfer energy in the EFP method. 
%That is, compare 
%\energ[Ind]{SAPT} with
%$ \energ[Pol]{EFP} + \energ[CT]{EFP} $.

% data analysis
The statistics on the differences between SAPT and EFP energies for each energetic component, basis set and cation type are tabulated in Table \ref{tab:sapt-efp-diff-stats}.
The columns on the right indicate the ion pair with the largest errors.
For the total interaction energy, EFP produced errors that ranged up to 20.5 \enUnit~ for pyrrolidinium based systems and to 36.8 \enUnit~ for imidazolium based systems. 
Both of these maxima were found with the aug-cc-pVDZ basis set. 
All three basis sets generated large standard deviations; for typical TILAs, it ranged from 8.1 to 16.1 \enUnit, whereas for halide systems, the standard deviation was found to be wider, from 9.1 to 26.3 \enUnit.
Out of the three basis sets,  aug-cc-pVTZ produced the smallest errors. 
EFP underperforms SAPT with a maximum error of 92.6 \enUnit~ for \ipair{mim}{4}{tos} among TILA-based systems in the aug-cc-pVDZ basis set, and -104.3 \enUnit~ for \ipair{mim}{4}{cl} among halide-based systems in the 6-311++G(d,p) basis set.
Generally, pyrrolidinium-based ion pairs had smaller errors (largest of 77.3 \enUnit in 6-311++G(d,p)) compared to imidazolium; however, these are still unacceptable.
On the absolute scale, all four components produced large errors for imidazolium ion pairs, whereas for pyrrolidinium the largest errors came from electrostatics and exchange components.


% CT comment bit?


Relative errors calculated using the SAPT energy for each energetic component, are tabulated in Table \ref{tab:sapt-efp-perc-stats}.
On average, the relative errors fall within 20\% for the TILAs. 
Exceptions to this trend include the exchange and induction terms in the 6-311++G(d,p) basis set, and the dispersion term for Dunning's basis sets.
%The analysis of these errors indicates that across all three basis sets studied, the average deviation of EFP from SAPT falls outside the 20\% range for all energetic components, except for electrostatics for TILAs, and exchange-repulsion for halides with Dunning's basis sets.
For systems with halides, it is notable that Dunning's basis sets had smaller errors for exchange-repulsion compared to the TILAs, whereas the opposite trend is observed in the 6-311++G(d,p) basis set.
With Dunning's basis sets, the relative errors are usually smaller for the halides for all components except charge-transfer, compared to the other ionic liquid anions.
Across all basis sets, it is notable that electrostatics produced the least relative errors, below 4.5\% on average.
%However, considering the same subset of EFP/aug-cc-pVTZ calculations for pyrrolidinium-based ion pairs, the halides have a lower MAE compared to the TILAs for all other components; exchange (3.4\% \emph{vs.} 18.4\%), induction (4.1\% \emph{vs.} 16.1\%), dispersion (12.8\% \emph{vs.} 14.8\%) and total interaction energy (2.3\% \emph{vs.} 6.3\%).
The aug-cc-pVTZ basis set had the least errors for the charge-transfer energy across all systems studied.


Surprisingly, the 6-311++G(d,p) basis set gave the largest relative errors of 60\% for dispersion for halides, even though it usually performed better than Dunning's basis sets for the TILAs by reducing the error to about 8\%.
%In the case of induction, the 6-311++G(d,p) basis set also gives slightly higher errors for halide systems. 
%This is rather surprising as the EFP method was designed for use with Pople basis sets.
% need to write about halide calculations (ghost atoms and manual addition)
EFP charge-transfer energies disagree with SAPT numbers by at least twofold regardless of basis set or system. 
The largest differences in charge-transfer are usually observed in systems that have anions with sulfonyl bonds such as mesylates and \ntf or $\pi$-conjugation, such as dicyanamide and tosylate.
Similar relative errors in EFP for electrostatics, exchange, dispersion and induction were observed for other intermolecular complexes from the S22 and S66 databases.
\cite{Flick2012a}
These findings indicate that errors observed in the application of EFP to ionic liquids are not specific to charged intermolecular complexes as previously suggested.
Due to the increased strength of these four components in ionic liquids, absolute errors with the EFP method fall in a much wider range.
Overall, based on these statistics presented, EFP could not be recommended for accurate calculations of interaction energies of ionic liquids where chemical accuracy is required.




%table 1 (in supplementary information) shows the mean and standard deviation of differences between sapt and efp, as well as the system with the largest difference. 
%the rows are grouped by basis set, cation base, and energy component. 
%halides and non-halides are separated side-by-side. 
%
%The averages and standard deviations are all quite large; the agreement between SAPT and EFP 
%
%The energies from each method are also plotted against each other in correlation plots, and a linear regression applied.
%
%
%%The format for the graphs plotting the difference in energy is very similar to the graphs for the energy plots. 
%%Instead of plotting the different configurations, the difference is averaged across the configurations, and the different basis sets used by EFP are compared. 
%%There are six plots, one each for electrostatics, exchange-repulsion, induction (polarization), dispersion, charge-transfer and the total interaction energy.
%%These graphs are meant to illustrate how the energy differences across the different ion pairs and basis sets.
%
%
%%The Boltzmann distribution is used to determine the average energy ($\overline{E}_{\text{comp}}$, the 'comp' refers to a generic component of the interaction energy) of an ion pair system,
%%\begin{equation*}
%%\overline{E}_{\text{comp}} = \frac{\sum_{i=1}^{N} [E_{\text{comp}}^i \times e^{ \frac{- E_{\text{comp}}^i}{RT} }]}{\sum_{i=1}^N e^{\frac{- E_{\text{comp}}^i}{RT}}}
%%\end{equation*}
%%where $N$ is the number of configurations for a particular ion pair, the $E_{\text{comp}}^i$ are the energies for the $i$-th configuration, and $RT$ is room temperature.
%
%% is RT really room temperature???
%
%
%
%
%\paragraph{Electrostatics}
%Looking at the correlation plots for electrostatics, the agreement between the two methods is clear.
%
%Considering the plots of the differences, most of the EFP values fall within 25 \enUnit of the SAPT results; only for some instances of the tosylates does EFP overestimate the energy beyond 25 \enUnit. 
%For the imidazoliums, the 6-311++G(d,p)le basis set tends to underestimate the electrostatic energy, whilst the Dunning basis sets overestimate if we exclude the halides.
%However, for pyrrolidinium systems, in general the triple zeta basis set has the weakest electrostatic interactions (except for \ntf), and is often closer to the SAPT values.
%cor both cationsf often the aug-cc-pVDZ has the largest overestimations; exceptions include the \ntf anion. 
%The EFP results indicate that a basis set of at least up to aug-cc-pVTZ quality is required to treat the tosylate systems well, especially when the system gets larger. 
%
%
%The relative difference in energy across the three basis sets showed the error to be within 5\%, except for the tosylates and some halides.
%In terms of relative error, the electrostatic energy is the best treated out of all the components of the interaction energy.
%It is crucial that electrostatics is treated well, since this is typically the largest component in the interaction energy. 
%
%
%Looking at the correlation scatterplot, the trend linse follow the centre diagonal fairly closely.
%The anions with lower electrostatic energies, i.e. towards the top right, show better agreement; it is the halides and the tosylates that deviate more at the higher energies.
%
%\paragraph{Exchange-Repulsion}
%The correlation plots show clear separation between different anions and cations.
%It is further broken down into halides and non-halides.
%
%This component has the lowest $R^2$ values, aside from charge-transfer from which no meaningful correlation could be observed. 
%This is due to the different trends that the mesylates and tosylates follow. 
%However sub-dividing with this further distinction would lead to too much fragmentation. 
%The important thing to note is that linear correlation between the two methods is still prominent.
%
%Here the separation of the different basis sets is clearly seen from the difference plots.
%As expected, the Dunning basis sets perform much better than the 6-311++G(d,p)le basis set. 
%However, excluding the halides and dicyanamide, surprisingly the triple zeta is worse than the double zeta basis set.
%In general, the exchange-repulsion interaction is underestimated by the EFP method. 
%This can clearly be seen in the correlation scatterplot, figure
%\ref{fig:sapt-efp-corr}.
%The scatterplot also shows the very clear separation between anions.
%Chloride is handled relatively well, considering that both halides have higher exchange energies compared to the rest of the anions; this is likely due to its smaller size.
%EFP tends to slightly overestimate the repulsion for imidazolium chloride systems in the 6-311++G(d,p)le and triple zeta basis sets, and overestimate it in the double zeta basis set.
%In pyrrolidinium chloride systems, EFP slightly underestimates for all three basis sets. 
%Bromide systems were well treated in the aug-cc-pVDZ basis set, but when the 6-311++G(d,p)le basis set is used the error is comparable with that of the other anions, in fact it is the highest amongst the pyrrolidinium systems.
%When using aug-cc-pVTZ basis set though, the repulsion energy for the bromides is overestimated in both imidazolium and pyrrolidinium.
%After chloride, dicyanamide is the anion with the lowest errors.
%Here, the triple zeta basis set performs the best, slightly overestimating the energy for \ipair{mim}{n}{dca} systems, and underestimating it in all other cases.
%For the rest of the anions (\bfl, \mes, \ntf, \pf, and \tos), the repulsion is underestimated. 
%The double zeta basis set gives the closest results, followed by triple zeta and then lastly the 6-311++G(d,p)le basis set.
%Excluding the 6-311++G(d,p)le basis set, most errors were under 25 \enUnit, or around 20\% relative error.
%The 6-311++G(d,p)le basis set gave errors up to nearly 50 \enUnit, for example in the case of the imidazolium mesylates.
%% what kind of systems are the mesylates/tosylates and how are they related to the halides?
%There is a very slight suggestion that the repulsion energy error increases for imidazolium systems as the length of the alkyl chain increases. 
%There is an equally slight but opposite indication for the pyrrolidinium systems.
%Referring back to the SAPT results, the trend for exchange to increase for longer alkyl chains in imidazolium is observed, whilst this decreases in pyrrolidinium. 
%Comparing with the EFP results, this pattern is also seen, but to a lesser degree.
%Hence this is also seen in the difference between the two methods.
%
%%When comparing the absolute and relative differences between methods, the electrostatic energy is in good agreement, and while the EFP method overestimates the repulsion energy when compared to the exchange energy of SAPT, the trends across the test set are largely in agreement. 
%
%
%\paragraph{Induction (Polarization)}
%Note that to compare the SAPT induction and EFP polarization energies, the EFP charge-transfer energy is summed with the EFP polarization energy. 
%This is because the SAPT method calculates the total induction, which includes the charge-transfer energy.
%% should we instead do SAPT_Ind - SAPT_CT - EFP_Pol ???
%Excepting a few cases of the halides (oddly enough, from the triple zeta basis set), the induction energy is consistently underestimated by the EFP method; most of the points in the correlation scatterplot fall above the line $ y= x $.
%The scatterplot once again highlights the separation between anions.
%Due to the variation in the halide values, no correlation was done for them; the one equation refers to the linear correlation for all the non-halides only.
%
%Looking at the difference plots, errors are usually within 10 \enUnit; the worst errors come from the halides in the 6-311++G(d,p)le basis set, which go over 30 \enUnit for the imidazolium bromides.  
%In terms of relative error this translates to within 20\%, excluding results from the 6-311++G(d,p)le basis.
%Here again the 6-311++G(d,p)le basis set has the largest deviations from the SAPT numbers. 
%The Dunning basis sets are comparable for the non-halides, except in tosylates where the triple zeta does better. 
%No data is available for the \ipair{mim}{n}{tos} systems in the 6-311++G** basis set, though it can be surmised from the corresponding pyrrolidinium systems that they would have larger errors than the Dunning basis sets.
%Furthermore none of the \ipair{mim}{n}{ntf} nor \ipair{mpyr}{n}{ntf} results are available since the SAPT results required excessive computational time.
%
%From the SAPT data, the induction energy increases with the length of the alkyl chain on the cation. 
%This trend is reflected in the EFP data as well; for the tetrafluoroborates, the mesylates, hexafluorophosphates and to a lesser extent the dicyanamides and \ntf, though the latter two show less constancy in the energy difference between the two methods.
%This is more clearly seen in the plots of the relative error, with the error being larger for smaller for the bulkier cations.
%
%The scatterplot also shows how the basis sets agree much better for the anions that have lower energies.
%The three basis sets diverge when it comes to the more strongly binding anions such as the halides.
%Surprisingly however, it is the trend line from the 6-311++G(d,p)le basis set that follows the centre diagonal the closest with these more problematic anions; the other two basis sets underestimate considerably.
%Nevertheless, while the mean of the results may agree better, the deviations within are just as large for the 6-311++G(d,p)le basis as for the Dunning basis.
%
%
%\paragraph{Dispersion}
%If the halides are excluded, the dispersion energy is usually overestimated, except for \bfl and \pf when using the 6-311++G(d,p)le basis set. 
%The halides on the other hand, usually have their dispersion underestimated.
%Looking at the correlation scatterplot, once again the clustering of the anions is observed.
%The two trend lines are differentiated once again by halides and non-halides. 
%The 6-311++G(d,p)le basis set in many cases gives results closer to the SAPT values than the other two basis sets.
%This is surprising, as one would expect the Dunning basis sets to have allow a better treatement of dispersion.
%Again excluding the halides, the absolute difference between SAPT and EFP for dispersion is usually within 10 \enUnit for pyrrolidinium, and within 25 \enUnit for imidazolium.
%In terms of relative energy, this means within 20\% for pyrrolidinium and 40\% for imidazolium.
%The halides have much larger errors, due to the 6-311++G(d,p)le basis set. 
%If the 6-311++G(d,p)le basis set is not considered, than the halides have error ranges comparable with the other anions.
%The SAPT results indicate that the dispersion interaction in general increases for bulkier cations, and this trend is well reflected in the EFP results, as there is only slight variation across the different alkyl chain lengths in the errors. 
%
%
%\paragraph{Charge-transfer}
%The absolute error for charge-transfer is small compared to the other interactions (within 10 \enUnit excluding halides), but the relative error is the highest out of all the components, with most of the pyrrolidinium results above 25\% and imidazolium results above 50\%.
%On an absolute scale this difference is not significant, but the relative error makes it obvious.
%Considering the correlation plot, it is clear that this is the component with the worst agreement between the two methods.
%Thus no attempt to linearly correlate the two methods for this component has been made.
%While the halides have much larger absolute errors, they also experience stronger charge-transfer interactions, so their relative errors are comparable with the rest of the anions.
%However, this means the halides dominate the scatterplot; if they are excluded, then the other anions show better agreement in pattern.
%The EFP method generally underestimates charge-transfer when compared with SAPT.
%Charge-transfer was best treated with the aug-cc-pVTZ basis set in every case.
%In fact, with the triple zeta basis set, the error decreases with increasing alkyl chain length, while it increases for the other two basis sets.
%
%
%\paragraph{Total interaction energy}
%Interestingly, the halides have the lowest error in the total interaction energy if the 6-311++G(d,p)le basis set is disregarded.
%This is likely due to the errors from exchange being lower for the halides.
%The electrostatic energy is usually the most dominant interaction, and the exchange-repulsion cancels this energy out.
%For imidazolium halides, the tendency is to underestimate the total interaction energy, especially when using the 6-311++G(d,p)le basis set.
%In the double zeta basis set, the pyrrolidinium bromides have a couple of systems overestimated, while all the pyrrolidinium chloride systems in this basis are overestimated.
%The rest of the pyrrolidinium halides are underestimated, with the triple zeta basis giving the closest results overall.
%
%
%The next two anions with the lowest errors are \dca and \bfl. 
%If the 6-311++G(d,p)le set is excluded, then \pf would belong to this group as well.
%Here, surprisingly, the 6-311++G(d,p)le basis set seems to have the lowest errors across the different basis sets.
%For these three systems, in general the 6-311++G(d,p)le basis set slightly underestimates the electrostatic energy, and underestimates the induction energy.
%However, it treats dispersion better than the Dunning basis sets.
%This seems to indicate that the 6-311++G(d,p)le basis set is sufficient for smaller ion pairs, but do not treat ion pairs with halides, or larger anions like mesylate and \ntf as well.
%
%
%For \ipair{mim}{n}{ntf}, there is very little difference between the two Dunning basis sets, but in \ipair{mpyr}{n}{ntf} the double zeta basis gives better results. 
%
%
%Lastly the tosylates and the mesylates have the highest errors, especially the tosylates. 
%This is probably because of the larger errors from the electrostatic and dispersion components, which were overestimated, coupled with the fact that the repulsion was underestimated. 
%This is more severe for the imidazolium tosylates.
%The mesylates show little difference between basis sets, across cations with different alkyl chain lengths.
%
%
%An interesting trend is seen when looking at the scatterplot---ion pairs with less intermolecular attraction are usually better treated than those with high binding energies.
%While the agreement between SAPT and EFP is not perfect, the EFP energy is usually higher; this consistency is not seen in the other anions.
%The energies tend to be more scattered towards the left, as the interaction energy increases; another observation is that certain anions tend to be underestimated, whilst others tend to be overestimated.
%
%%\end{par}
%
%To give a sense of how the different components contribute to the total interaction energy, figures 
%\ref{fig:corr-all_En} and
%\ref{fig:adiff_en-barplot_all}
%show all the energies on the same plot.
%In the scatterplot (\ref{fig:corr-all_En}), the scales of the different components can be seen.
%In this plot the colours now refer to different basis sets, and the shapes of the points correspond to the different energies.
%The grid of bar plots in figure \ref{fig:adiff_en-barplot_all} is of the absolute errors.
%It is meant to convey the magnitudes of the error from each component, and how they sum to give the final difference.
%The colours correspond to the different energies, and the depth of the colour indicates the basis set.
%Once again, the Boltzmann distribution was used to average the energies across different configurations.
%% make sure the average across chain lengths is done right!!!!!
%
%%\subsection{SAPT results}
%%
%%The raw SAPT energies are presented here to give an indication of how the different energies behave across the systems studied.
%%
%%\paragraph{Electrostatics} 
%%The results from this graph tend to fall into neat bands. 
%%For example, in the imidazolium cation systems, only the first and fourth configurations of bromide                                 and chloride fall below the $-450$ \enUnit ~ mark. 
%%The second and third energetically favourable configurations are between -400 and -450 \enUnit, along with systems that have the mesylates and tosylates as anions. 
%%Next up are the tetrafluoraborates and dicyanamides, followed by \pf and \ntf systems. 
%%The results are less spread out in pyrrolidinium, but the trends are the same. 
%%The mesylate, tosylate and halide systems are again similar in energy, with much less visible separation this time between the different halide configurations.  
%%Next up are \bfl, \dca, \pf and \ntf, again in that order. 
%%The consistent trend across all systems is that the electrostatic interaction weakens as the length of the alkyl chain on the cation increases.
%%This is consistent with our understanding of the chemistry, as the bulkier cations mean a greater inter-ion distance. 
%%Moreover, this matches up very well with previously proposed proton affinity scale of Izgorodina et al.
%%\cite{Izgorodina2007}
%%
%%\paragraph{Exchange}
%%Once again a strong separation into bands is observed in all the results. 
%%For the imidazolium halide systems, where previously the above plane and below plane configurations showed stronger electrostatic interactions, here they exhibit stronger exchange forces. 
%%This is because both electrostatic and exchange are strongly distance dependent. 
%%Just as the closer separations mean the electron-nuclei attraction is stronger, in the same way the electron-electron and nuclear-nuclear repulsions are stronger.
%%The exchange results tend to mirror the electrostatic numbers, but in reverse. 
%%After the halides, the mesylates and tosylates have the strongest exchange interactions, followed by \ntf and \dca. \bfl and \pf6 have the weakest exchanges. 
%%On the other hand, in the pyrrolidinium systems, \pf and \ntf have the weakest interactions, then \dca and \bfl.
%%The next few anions in order of increasing exchange force are \tos, \mes, \cl and \br. 
%%It is even more evident here that the pyrrolidinium results appear uniform than the imidazolium results.
%%This difference between the two cations is observed for all the energy components in SAPT.
%%As the length of the alkyl chain increases, in pyrrolidinium a slight reduction in exchange is noted in chloride, mesylate and tosylate systems; perhaps tetrafluoroborate, hexafluorophosphate and even dicyanamide systems too.
%%For example, in \ipair{mpyr}{n}{tos} (p2), as the alkyl chain goes from methyl to butyl, the exchange decreases as 107.9, 105.2, 104.1 and 103.6 \enUnit.
%%However, the overall trend in the imidazolium systems seems to be increasing exchange as the chain length increases. 
%%This is most clearly seen in \tos, \ntf, the first configurations of \mes and \pf, and \bfl as well as the second and third configurations of \br and \cl. 
%%Using \ipair{mim}{n}{tos} (p1) as an example, the exchange increases as 127.6, 135.7, 140.2 and 143.0 going from dimethyl imidazolium to butyl-methyl-imidazolium.
%%The second configurations of \mes and \pf, the first and fourth configurations of the halides, and all the \dca configurations show little or no variation. 
%%
%%\paragraph{Induction}
%%Separation between different chemical systems is again evident in the plot for the induction component.
%%The halides have the strongest interactions; all of them fall below -70 \enUnit~ in both cations, with no other anions having results lower.
%%The mesylates and tosylates follow the halides, as previously seen in electrostatics and exchange, though this is less obvious in the imidazolium row since the second and third configurations of \ntf fall into the same range of energies, between -70 and -50 \enUnit.
%%For pyrrolidinium, the mesylates and tosylates occupy the band between -70 and -60 \enUnit.
%%Except for in systems with dicyanamide, longer alkyl chains mean stronger induction in imidazolium systems. 
%%For instance, for the first configuration of \ipair{mim}{1}{tos} the induction energy is -59.7 \enUnit~ while for \ipair{mim}{4}{tos} it is -67.9 \enUnit.
%%This is more noticeably manifest in imidazolium, but only weakly observed in pyrrolidinium, e.g. for the mesylates and tosylates there is no visually discernable pattern.
%%
%%\paragraph{Dispersion}
%%In similar fashion as induction, dispersion increases as the chain increases in length.
%%In pyrrolidinium systems, the are roughly two bands, one containing \bfl and \pf2 (~ -40 to -30 \enUnit), and the other containing the rest (-60 to -40 \enUnit).
%%With imidazolium such a distinction is even more blurred, with the \ntf, \mes, \tos and perhaps \dca and the first and fourth configurations of the halides having stronger dispersion than the rest, at energies below -40 \enUnit.
%%This is due to the fact that imidazolium, with a delocalised ring, allows for more dispersion compared to pyrrolidnimium.
%%The electrons an anion has, the greater the dispersion.
%%The anions listed previously have greater electron density compared with the other anions.
%%The upper band is occupied by the second and third configurations of the halides, \bfl, and \pf. 
%%\ipair{1}{mim}{dca} and
%%\ipair{1}{mim}{ntf}
%%lie in the upper band as well. 
%%In all the systems, the above-plane configuration (and below-plane, if it exists) always has a stronger dispersion interaction compared to the in-plane configurations; this is due to direct interaction with the delocalised ring system.
%%
%%\paragraph{Charge-transfer}
%%There is very little variation across varying chain lengths for this energy. 
%%Once again, the halides have the strongest interactions, all of them occupying the band below -20 \enUnit. 
%%The rest of the anions all have absolute energies below 15 \enUnit.
%%There does not appear to be much separation between different configurations, except for the stark instance in the halides, between the in-plane and above/below-plane configurations.
%%The mesylates and tosylates have slightly higher energies in pyrrolidinium; this is less obvious in imidazolium.
%%
%%\paragraph{Total interaction energy}
%%This energy is the sum of all the components discussed previously, except for charge-transfer, which is included in the induction energy.
%%The interesting thing to note here is that the mesylates and tosylates have the highest energies here, followed by the halides, then the tetrafluoroborates, and then the rest.
%%While the halides had higher energies for all the interactions, the large repulsion they possessed meant that they had a weaker interaction overall. 
%%In general, and more evidently for the pyrrolidiniums, the longer the alkyl chain the lesser the energy. 
%%This energy difference may be up to 12.6 \enUnit~ (\ipair{mpyr}{n}{pf}), but is usually below 10 \enUnit.
%%The same is hard to say for the imidazoliums, with some decreasing (first configurations of \mes and \pf), while others increasing (\tos, \ntf and to some extent the first configurations of \mes and \pf).
%%For the other cations, the interaction energy does not much with increasing alkyl chain due to the different interplay between terms for each combination.
%%
%%\subsection{EFP results}
%%
%%By way of comparison with the SAPT results, a similar analysis of the raw EFP energies is given here. 
%%The results from the aug-cc-pVTZ basis set is used as a representative case, since it would be tedious to do the same for all three basis sets.
%%
%%\paragraph{Electrostatic}
%%The most obvious result from a cursory glance at the graph is that the electrostatic interaction weakens as the length of the alkyl chain increases.
%%This is not surprising as the cation gets bulkier it sterically hinders the anion. 
%%
%%The next evident fact is that the second and third (if it exists) configurations are usually lower in energy (less negative) than the first and fourth (if it exists) configurations. 
%%The first and fourth configurations correspond to above and below plane geometries; it seems that the in-plane interactions of the second and third configurations gives a lower interaction energy. 
%%
%%The pyrrolidinium ion pairs tend to be less distributed, and somewhat weaker, whereas the imidazolium systems are more spread out, and overall have stronger interactions. 
%%This is reflected as well in the SAPT results, though the EFP plots are less ordered.
%%This trend appears later on in many of the other components of the interaction energy. 
%%Another trend that is also observed in the other energies is that the halides, followed by the mesylates and tosylates, usually have stronger interactions.
%%This is more easily seen in the electrostatic interaction for the pyrrolidinium row. 
%%Once again, this was seen in the SAPT raw energies, so the EFP method does somewhat capture the differences in chemical systems, albeit in a less precise fashion.
%%
%%
%%\paragraph{Repulsion}
%%For both cations, the separation is quite clear between halides and the rest of the anions.
%%For those with an imidazolium ion, only the halides have a repulsion energy > 140 \enUnit ; those with a pyrrilidinium cation have repulsion energies above 120 \enUnit.
%%There is a possible outlier in 
%%\ipair{mim}{2}{br}.
%%Less obvious is the clustering of the non-halides. 
%%For the imidazolium cation, the lowest cluster is made up of all the ion pairs with \bfl and \pf as anions. 
%%No other anions have repulsion energies below 80 \enUnit .
%%
%%With the pyrrolidinium cation the middle cluster is composed of ion pairs with either mesylate or tosylate as the anion; the minimum repulsion for these two anions is around 81 \enUnit , and the maximum at about 101 \enUnit.
%%No other anions fall within this band. 
%%The anions that form the group with the lowest repulsion energies are hexafluorophosphate, tetrafluoroborate (as before), and \ntf, from around 65 to 53 \enUnit.
%%
%%Trends across increasing alkyl chain length not evident. 
%%In imidazolium systems the longer alkyl chains have slightly less repulsion, probably because the larger molecules are further apart. 
%%However, the pyrrolidinium systems either show no difference or a very slight increase in repulsion as che chain length increases. 
%%
%%\paragraph{Polarization}
%%Three bands can clearly be seen for the pyrrolidinium cation: the lowest from around -68 to -81 \enUnit comprising of the halides, the middle from about -50 down to -64 \enUnit representing only the mesylates and the tosylates, and the rest fall into the highest from -30 to -42 \enUnit .
%%The imidazolium cation exhibits the same behaviour, though the bands are less clear.
%%The halides are from -66 down to -135, the mesylates and tosylates from -50 to -65 \enUnit .
%%Everything else as above -50 \enUnit .
%%In both imidazolium and pyrrolidinium, the tetrafluoroborates and hexafluorophosphates are very similar to dicyanamide and \ntf .
%%Overall, the longer the alkyl chain length results in slightly stronger polarization.
%%
%%\paragraph{Dispersion}
%%For the pyrrolidinium ion two clear groups are observed, with \bfl , \br , \cl , and \pf falling in between -34 and -45 \enUnit , whereas \mes , \dca , and \ntf fall within -49 to -55 \enUnit .
%%The data for the imidazolium cation is more disperse, with \bfl , \pf and the halides having the lowest (less negative) dispersion energies. 
%%However, they do not form their own band because the \cat{mim}{1} of \dca and \ntf have lower dispersion energies too; this is likely due to the shorter alkyl chain. 
%%As the alkyl chain length increases, the dispersion force increases. 
%%This is more obvious in the imidazolium species, but can also be seen for pyrrolidinium. 
%%
%%\paragraph{Charge-transfer}
%%The second and third configurations of the halides immediately stand out for the charge-transfer energy for imidazolium.
%%These configurations have the anion interacting with the ring \textbf{in} the plane. 
%%While these configurations result in lower electrostatic interactions, they have stronger charge-transfer energies. 
%%This does not happen for pyrrolidinium systems, as no discernable difference is seen between configurations.
%%
%%Notice that the second configurations for \dca and \ntf are also lower relative to the other configurations, but not the third configurations.
%%This is because while the second configuration is still interacting side-on with the ring, because of the size of \dca and \ntf the third configuration is positioned differently.
%%In \dca , if the first configuration is thought of as above the plane, than the third configuration is below the plane, hence the similar energies.
%%In \ntf , the first configuration has the anion above the ring, the second has it perpendicular to the ring, to the side. 
%%In both configurations, it is the amide that is interacting with imidazolium. 
%%For the third configuration however, it is the carbonyl groups on either side of the amide that interacts with the ring. 
%%This occurs with the anion obliquely positioned relative to the ring, neither side-on nor fully above the plane.
%%
%%
%%Like the halides, electrostatics and charge-transfer seem to be inversely correlated for ion pairs with \dca , as can be seen from the second configurations of 
%%\ipair{mim}{1}{dca} and 
%%\ipair{mim}{3}{dca}. 
%%However, this is only for the imidazolium cation, and furthermore this is cannot be clearly seen for \ntf .
%%\ipair{mpyr}{1}{dca} and 
%%\ipair{mpyr}{3}{dca}
%%also weaker electrostatic energies, but their charge-transfer energies do not show any deviation from the other configurations.
%%
%%On the other hand, the pyrrolidinium results are recognisably less spread out. 
%%All the halides have charge-transfer energies less than -9 \enUnit , while the mesylates and tosylates occupy the narrow band from -6.6 to -8.5 \enUnit .
%%This is followed closely by the dicyanamides, then \bfl , \ntf and \pf .
%%
%%
%%
%
%
%% followed by
%% subsubsection: correlations between in SAPT and EFP for individual components
%%\end{footnotesize}


\subsection{Correlation between SAPT and EFP}
\label{subsec:corr-elst}
%% correlations and scaling results, both indiv and ratio-wise
% subsubsection 
% subsection: Scaled EFP


% correlation
In order to understand the trends between the energy components of the two methods, they were plotted against each other. 
These plots are shown in Figure \ref{fig:sapt-efp-corr-all}.
Since aug-cc-pVTZ showed the best agreement with SAPT2+3,  it will be further discussed in the text as a representative of EFP performance, since including the other basis sets would only serve to bog the reader down with excessive minutiae.
More detail on the other two basis sets can be found in ESI.
The linear correlation is visually obvious for electrostatics, exchange, induction and dispersion components.
Since both methods decompose the interaction energy into the same physically meaningful components, this comes as no surprise.
Furthermore, electrostatics and dispersion are expected to have strong linear correlations.
This is because the electrostatic component, calculated with Stone's distributed multipole analysis, was formulated against benchmark CCSD(T) and SAPT energies.
\cite{Slipchenko2007a}
As it is the largest contributor to the total interaction energy, accurate treatment of electrostatics is essential.
% TODO
Dispersion was also formulated against SAPT dispersion, and so is expected to agree well.
\cite{Adamovic2005a}
As such, attempting a linear fit between the two methods is one way of analysing the differences between the methods, thus providing more insight into why EFP might deviate from SAPT2+3 for ionic liquids.


\begin{figure}
    \caption{Comparison of the electrostatic component between SAPT2+3 and EFP. (\enUnit)}
    \label{fig:sapt-efp-corr-all}
    \centering
    % original
    %\includegraphics[scale=0.75]{\string~/GoogleDrive/SAPT-EFP/images/sapt_efp_corr/all_corr.pdf}
    \includegraphics[scale=0.75]{./images/all_corr.pdf}
\end{figure}


\begin{table}[h]
\centering
\scriptsize
\caption{Statistics from linear regression analysis of individual energy component calculated with EFP/aug-cc-pVTZ and fitted to SAPT2+3/aug-cc-pVDZ. All energies are in \enUnit~.}
\label{tab:scaled-efp-atz-coef}
\begin{tabular}{ll|rrrrr|rrrrr}
    \hline
       &          & \multicolumn{5}{c}{Halides}                                                 & \multicolumn{5}{c}{TILAs}                                            \\
           Cation & Component   & $a$    & MAE      & SD         & Max      &                                 & $a$      & MAE        & SD       & Max       &                      \\ \hline
\catb{mim}{n}     & Elst   & 1.014    & 9.3        & 11.8     & -31.2     & \ipair{mim}{4}{br}  & 0.978    & 5.3        & 6.7      & -18.6     & \ipair{mim}{4}{bfl}  \\
                  & Exch   & 0.964    & 5.8        & 7.2      & 17.1      & \ipair{mim}{3}{cl}  & 1.124    & 11.4       & 12.6     & -27.7     & \ipair{mim}{2}{dca}  \\
                  & Ind    & 1.031    & 11.4       & 14.5     & 41.8      & \ipair{mim}{3}{br}  & 1.164    & 3.2        & 4.1      & -13.1     & \ipair{mim}{1}{ntf}  \\
                  & Disp   & 1.083    & 2.8        & 3.3      & -7.0      & \ipair{mim}{4}{br}  & 0.777    & 2.5        & 3.5      & -9.8      & \ipair{mim}{2}{ntf}  \\
                  & Total  & 1.020    & 20.8       & 26.7     & 59.8      & \ipair{mim}{3}{br}  & 0.910    & 7.0        & 9.8      & 26.1      & \ipair{mim}{2}{tos}  \\ \hline
\catb{mpyr}{n}    & Elst   & 1.023    & 4.1        & 4.9      & -10.8     & \ipair{mpyr}{3}{br} & 0.994    & 4.1        & 5.1      & 13.1      & \ipair{mpyr}{2}{ntf} \\
                  & Exch   & 0.975    & 4.1        & 4.6      & 7.3       & \ipair{mpyr}{2}{cl} & 1.222    & 6.4        & 7.8      & -22.1     & \ipair{mpyr}{4}{dca} \\
                  & Ind    & 0.985    & 3.0        & 3.8      & -7.9      & \ipair{mpyr}{3}{cl} & 1.169    & 2.4        & 2.7      & 6.2       & \ipair{mpyr}{2}{tos} \\
                  & Disp   & 1.148    & 0.7        & 1.0      & -2.4      & \ipair{mpyr}{2}{br} & 0.866    & 1.2        & 1.5      & 4.8       & \ipair{mpyr}{1}{dca} \\
                  & Total  & 1.016    & 7.8        & 9.3      & -16.1     & \ipair{mpyr}{3}{br} & 0.940    & 5.2        & 6.8      & 18.7      & \ipair{mpyr}{2}{tos} \\ \hline
\end{tabular}
\end{table}


Linear regression was performed assuming a linear relationship of the form $ y = a \cdot x $.
Scaled EFP energies are obtained by multiplying the original EFP energy with the scaling coefficient.
The coefficient $a$, as well as the associated regression statistics are given in Table \ref{tab:scaled-efp-atz-coef}.
Since all of the $R^2$ values are high, falling between 0.9737 and 0.9999, they are not shown in the table (for more detail see the ESI).
Clear differences in the actual magnitude of the regression coefficient are observed between the halides and the typical ionic liquid anions.
To achieve better fitting, linear regression analysis was performed separately for the imidazolium and pyrrolidinium cations, further highlighting the importance of the cation-anion combination in determining the component-specific correlations in ionic liquids.


Analysis of Table \ref{tab:scaled-efp-atz-coef} reveals that there is a drastic improvement for the scaled dispersion and induction components, with the MAE falling below 3.2 \enUnit. 
Imidazolium halides are an exception, due to relatively large errors in the induction component (MAE = 11.4 \enUnit).
Except for these systems, maximum errors fall in a much narrower range of -13.1 to 6.2 \enUnit.
Among the components studied, the gradient falls quite close to 1 for electrostatics and exchange-repulsion within 3.5\%, whereas for halides much larger differences, between 12 and 22\%, are observed in the exchange-repulsion component for TILAs.
In general, for halides, EFP underestimates electrostatics and overestimates exchange-repulsion compared to SAPT2+3, and vice versa for the other typical IL anions.
As for the other components, EFP dose not show clear trends.
EFP tends to  underestimate induction up to 17\% for the TILAs, whereas for halides, the scaling coefficient ranges between 0.985 and 1.031.
In the case of halides, EFP underestimates dispersion between 8 and 15\%, whereas for the TILAs this component is overestimated between 13 and 23\%. 


For electrostatics and exchange, the maximum errors still fall in a wide range between -31.2 and 17.1 \enUnit.
The standard deviation is above 5 \enUnit~on average, which is still unacceptable where chemical accuracy is required.


\begin{table}[h]
\centering
\scriptsize
\caption{Relative differences (given in \%) for EFP scaled energetic components and total interaction energy compared to SAPT2+3.}
\label{tab:scaled-perc-stats-atz}
\begin{tabular}{ll|rrrr|rrrr}
\hline
 Cation                & Component & \multicolumn{4}{c}{Halides}                  & \multicolumn{4}{c}{TILAs}              \\
                       &        & MAE  & SD  & Max  &                          & MAE  & SD  & Max  &                   \\ \hline
\catb{mim}{n}          & Elst   & 2.1  & 1.6 & 6.8  & \ipair{mim}{4}{br}   & 1.4  & 1.0 & 4.9  & \ipair{mim}{4}{bfl}   \\
                       & Exch   & 3.3  & 2.5 & 9.9  & \ipair{mim}{3}{cl}   & 10.2 & 4.8 & 25.7 & \ipair{mim}{2}{dca}   \\
                       & Ind    & 12.5 & 9.3 & 44.4 & \ipair{mim}{3}{br}   & 6.2  & 4.1 & 19.6 & \ipair{mim}{1}{ntf}   \\
                       & Disp   & 5.9  & 4.2 & 15.6 & \ipair{mim}{4}{br}   & 4.2  & 4.3 & 18.8 & \ipair{mim}{1}{ntf}   \\
                       & Total  & 5.2  & 4.1 & 15.5 & \ipair{mim}{3}{br}   & 1.8  & 1.7 & 6.3  & \ipair{mim}{2}{tos}   \\ \hline
\catb{mpyr}{n}         & Elst   & 1.0  & 0.6 & 2.6  & \ipair{mpyr}{3}{br}  & 1.2  & 0.9 & 4.3  & \ipair{mpyr}{2}{ntf}  \\
                       & Exch   & 2.9  & 1.4 & 5.3  & \ipair{mpyr}{3}{cl}  & 7.8  & 6.0 & 31.6 & \ipair{mpyr}{4}{dca}  \\
                       & Ind    & 4.0  & 3.0 & 10.4 & \ipair{mpyr}{3}{cl}  & 4.8  & 2.4 & 9.1  & \ipair{mpyr}{2}{tos}  \\
                       & Disp   & 1.7  & 1.3 & 5.1  & \ipair{mpyr}{2}{br}  & 2.9  & 2.3 & 11.3 & \ipair{mpyr}{1}{dca}  \\
                       & Total  & 2.1  & 1.2 & 4.2  & \ipair{mpyr}{3}{br}  & 1.4  & 1.2 & 4.7  & \ipair{mpyr}{2}{tos}  \\ \hline
\end{tabular}
\end{table}


% message: errors *are* reducing, but not enough
Table \ref{tab:scaled-perc-stats-atz} shows the relative errors in percentage of the scaled EFP energetic components and the total interaction energy with respect to SAPT.
One can clearly see that the errors for the induction and dispersion components reduce to below 20\%.
For the dispersion term, this can be expected, as the dispersion component in EFP was formulated against SAPT dispersion.
\cite{Adamovic2005a}
It is important to note that \catb{mpyr}{n}[Hal] systems performed exceptionally well, with a maximum error of 10.4\% for induction, and only 5.1\% for dispersion.
The exchange component still gives errors between 5.3\% for \catb{mpyr}{n} halides and 31.6\% for the pyrrolidinium ion pairs coupled with the TILAs.
Although the relative error for electrostatics drops below 10\% for all systems under study (with all three basis sets), due to the magnitude of this component, maximum errors can still exceed -30 \enUnit~(\emph{e.g.} in the case of \ipair{mim}{4}{br}).
The same applies to the exchange-repulsion component that gives relative errors below 3.3\% on average, whereas the maximum errors can still be up to -27.7 \enUnit~ as observed for \ipair{mim}{2}{dca}.


\begin{figure}[h]
    \caption{Distribution of the raw and scaled differences (\enUnit) between SAPT and EFP, by energetic component and anion type, coloured by cation.}
    \label{fig:hist-raw-scaled-atz}
    \centering
    % original
    %\includegraphics[scale=0.5]{\string~/GoogleDrive/SAPT-EFP/images/histograms/hist_by_En_atz_raw_scaled.pdf}
    \includegraphics[scale=0.5]{./images/hist_by_En_atz_raw_scaled.pdf}
\end{figure}


In Figure \ref{fig:hist-raw-scaled-atz} the distribution of EFP errors, raw and scaled, for individual energetic component is shown.
The solid histograms are the raw differences, whereas the line density plots are the errors obtained after scaling.
The line density distributions are kernel distributions plotted using the normal kernel.
They represent the smoothed count and are analogous to the histograms.
The plots are separated in a grid based on the component of the interaction energy, as well as the type of anion, halides and TILAs.
As expected, the scaling shifts the mean of the distribution for each component towards zero.
In particular, the scaling achieves narrower distributions for some components, such as dispersion and induction.
It appears that linear regression did not improve the spread of errors for electrostatics and exchange-repulsion, indicating that there are shortcomings in the EFP formulation of these components when applied to ionic liquids.


\begin{table}
\centering
\footnotesize
\caption{Statistics of the differences between first order electrostatics and exchange-repulsion calculated in SAPT2+3 and their full components in EFP. All energies are given in \enUnit.}
\label{tab:trunc-stats-atz}
\begin{tabular}{lllrrrr}
\hline
 Component   &  Cation                          & Anion-type & MAE    & SD     & Max     &                           \\ \hline
 Elst     &  \multirow{2}{*}{\catb{mim}{n}}  & Hal        &  9.7   & 12.1   & -31.0   & \ipair{mim}{4}{br}    \\ 
          &                                  & TILA        &  10.9  & 6.3    & 28.5    & \ipair{mim}{2}{tos}   \\
          &  \multirow{2}{*}{\catb{mpyr}{n}} & Hal        &  5.0   & 4.8    & -14.8   & \ipair{mpyr}{3}{br}   \\
          &                                  & TILA        &  4.9   & 4.9    & 15.0    & \ipair{mpyr}{2}{ntf}  \\ \hline
 Exch     &  \multirow{2}{*}{\catb{mim}{n}}  & Hal        & 27.5   & 10.1   & -41.3   & \ipair{mim}{4}{br}    \\
          &                                  & TILA        & 4.2    & 5.3    & -19.2   & \ipair{mim}{2}{dca}   \\
          &  \multirow{2}{*}{\catb{mpyr}{n}} & Hal        & 21.4   & 5.0    & -28.0   & \ipair{mpyr}{3}{br}   \\
          &                                  & TILA        & 4.0    & 3.3    & -11.0   & \ipair{mpyr}{4}{dca}  \\ \hline
\end{tabular}
\end{table}


It is hypothesized that the errors from electrostatics and exchange-repulsion come from the fact that higher-order contributions are not included in the EFP formulation.
In order to test this hypothesis, first order electrostatic and first order exchange components calculated with SAPT2+3 were compared with their corresponding EFP components. 
the statistics on these differences are given in Table \ref{tab:trunc-stats-atz}.
Only a small difference in errors was observed for electrostatics, thus indicating that perhaps higher-order terms do not contribute significantly to the electrostatic interaction.
As for the exchange-repulsion component, the difference between first-order SAPT and full EFP is reduced to below 4.2 \enUnit~on average for all ion pairs with TILAs.
For comparison, the errors for EFP exchange-repulsion were almost triple that for halides when compared with the first-order exchange-repulsion energy in SAPT2+3.
For example, for the imidazolium based halides the MAE increases from 8.6 to 27.5 \enUnit.


These findings clearly indicate the importance of higher-order terms in exchange-repulsion for typical ionic liquids anions but not halides.
Compared to the correlations between SAPT2+3 and EFP in Table \ref{tab:scaled-efp-atz-coef} for the halide systems, exchange is slightly overestimated by EFP, by about 3\%, whereas in the TILAs, EFP underestimates exchange between 12 and 22\% on average. 
While halides have a higher exchange energy in general, roughly 60~\enUnit~ more than that of the other TILAs, they also give smaller errors. 
Due to the increased orbital overlap between the halide anion and the cation, the optimal geometry of the ion pair has shorter distances between the cation and anion, thus resulting in higher exchange-repulsion energies.
It is suggested that having a single atom as the anion reduces the complexity of the interaction, and is thus better modelled by EFP.
On the other hand, the TILAs are more complex as they have multiple interaction sites.
In this case, the optimal geometry of the ion pair positions the anion further away from the cation, leading to lower exchange-repulsion energies.
At the same time, due to the presence of multiple interaction sites, the overall error increases with the number of possible interaction sites on the anion.


\begin{table}[ht]
\centering
\footnotesize
\caption{Contributions (given in \enUnit) from higher-order terms in electrostatics and exchange-repulsion for ionic liquid ion pairs and the intermolecular complexes in the S22 and S66 databases.}
\label{tab:s22-s66-trunc}
\begin{tabular}{llrrrr}
  \hline
Component & Set & Mean  & SD & Max &                            \\ \hline
Elst   & \catb{mim}{n}  & 2.92 & 2.74 & -6.52 & \ipair{mim}{2}{br}    \\ 
       & \catb{mpyr}{n} & 2.13 & 1.90 & -5.15 & \ipair{mpyr}{1}{br}   \\ 
       & S22 & 1.36 & 1.67 & 6.13  & Uracil dimer        \\ 
       & S66 & 1.19 & 1.46 & 5.32  & Complex of acetic acid and Uracil \\ \hline
Exch   & \catb{mim}{n}  & 18.07 & 6.86 & 28.40 & \ipair{mim}{4}{tos}      \\ 
       & \catb{mpyr}{n} & 14.05 & 4.89 & 20.93 & \ipair{mpyr}{2}{mes}     \\ 
       & S22 & 8.08  & 8.30 & 21.73 & Complex of 2-Pyridoxine and 2-Aminopyridine  \\
       & S66 & 5.30  & 5.89 & 29.85 & Acetic acid dimer    \\ \hline
\end{tabular}
\end{table}


\begin{table}[ht]
\centering
\footnotesize
\caption{Relative contributions (given in \%) from higher order terms in electrostatics and exchange for ionic liquid ion pairs and the intermolecular complexes in the S22 and S66 databases.}
\label{tab:s22-s66-perc}
\begin{tabular}{llrrrr}
  \hline
 Component & Set & Mean  & SD     &  Max   &                           \\ \hline
 Elst   & \catb{mim}{n}  & 0.72 &  0.66  &  1.41  & \ipair{mim}{2}{br}    \\ 
        & \catb{mpyr}{n} & 0.58 &  0.49  &  1.26  & \ipair{mpyr}{1}{br}   \\ 
        & S22 & 4.05 &  5.03  & -19.09 & Complex of Benzene and hydrogen cyanide  \\
        & S66 & 6.32 &  7.87  & -19.43 & Ethyne dimer          \\ \hline
 Exch   & \catb{mim}{n}  & 13.47 & 4.95 & 20.16 & \ipair{mim}{4}{bfl}       \\ 
        & \catb{mpyr}{n} & 14.31 & 3.86 & 19.14 & \ipair{mpyr}{4}{bfl}     \\ 
        & S22 & 10.86 & 4.53 & 16.50 & Phenol dimer                 \\
        & S66 & 11.97 & 4.62 & 18.61 & Complex of methanol and model peptide \\ \hline
\end{tabular}
\end{table}


Tables \ref{tab:s22-s66-trunc} and \ref{tab:s22-s66-perc} show the actual contributions of higher-order terms to the electrostatic and exchange components in SAPT2+3.
The tables contrast contributions for the ionic liquid systems studied to the neutral intermolecular complexes taken from the S22 and S66 databases.
Surprisingly, for the higher-order contributions of the electrostatic component in ionic liquids, only slight differences were observed between ionic liquids and neutral complexes.
The maximum contribution did not exceed 6.5 \enUnit~for all systems, be it ionic or neutral.
On the relative scale, these contributions constitute a much smaller portion of the electrostatic component, usually below 1\%. 
On the other hand, the neutral complexes from S22 and S66 show a much higher relative contribution, up to 19\% in some cases, for example in the ethyne dimer.

For exchange-repulsion, the higher-order contributions are larger for ionic liquids on the absolute scale when compared to the neutral complexes.
Nevertheless, the maximum contributions could be not be split between ionic and neutral systems, with both types of complexes having higher-order contributions of up to 30 \enUnit.
Hydrogen-bonded complexes from the S22 and S66 databases tend to have a larger exchange-repulsion component due to increased orbital overlap.
Furthermore, on the relative scale, the higher-order contributions fall in a narrow range of 11\% for neutral complexes and 14\% for ionic liquids.
As established above, for ionic liquids EFP matches SAPT2+3 exchange-repulsion energies up to the first order only as seen in Table \ref{tab:s22-s66-perc}.
Higher-order exchange-repulsion terms are equally important in ionic and neutral complexes, thus suggesting that EFP shortcomings in exchange are not restricted to semi-Coulombic systems such as ionic liquids.
The magnitude of the error for exchange-repulsion in EFP is dependent on the complexity of the intermolecular interaction and proximity of interacting charged species. 
It has to be pointed out that the absolute error also increases as the magnitude of the energy increases, despite the relative error remaining practically constant.

To summarise, good linear correlations have been achieved for the induction and dispersion components between the EFP and SAPT2+3 methods for ionic liquids. This clearly introduces huge potential for improvement of these terms in the EFP formulation. Electrostatic and exchange-repulsion components calculated with EFP show the largest deviations. For the former this finding was attributed to deficiency in the EFP method for recovering electrostatics, whereas for the latter higher-order terms (currently missing in the EFP formulation) were found to be important for ionic liquids, especially those containing anions with multiple interaction sites.   



%\subsection{Charge-transfer}

\paragraph{Charge-transfer and the geodesic scheme}
The charge-transfer energies from SAPT and EFP differ by a significant amount. 
Across all systems, SAPT had an average charge-transfer of 15.1 \enUnit, with a similar standard deviation of 15.0 \enUnit, coupled with a minimum error of -68.0 \enUnit~and a maximum error of -3.9 \enUnit.
For EFP/aug-cc-pVTZ, the average charge-transfer energy was 7.0 \enUnit~and a standard deviation of 4.5 \enUnit, coupled with a minimum error -22.9 and a maximum error of -1.8 \enUnit.
Although the charge transfer energies have very high maximum relative errors of 66.4\% for TILAs and 93.4\% for halides, the actual contribution of the charge-transfer energy to the total interaction energy is relatively small, less than 4\% on average.


It has been hypothesised that the charge transfer and induction energies may correlate with the actual net charge transfer occurring between the cation and anion.
To further investigate the extent of the charge transfer effect, the geodesic charge calculation scheme was used to predict the net charge transfer (NCT) between the cation and anion. 
Our group has previously shown that this scheme for fitting atomic charges to reproduce electrostatic potentials was particularly reliable for ionic liquids ions. 
\cite{Rigby2013a, Spackman1996a}


\begin{table}[ht]
\centering
\footnotesize
\caption{Net charge transfer statistics from the geodesic scheme.}
\label{tab:geod-stats}
\begin{tabular}{llrrrl}
\hline
  Basis & Cation & Mean & SD & Max & IL       \\ \hline
  AVDZ & im & 0.1781 & 0.0372 & 0.2777 & \ipair{mim}{3}{ntf} (p2) \\ 
       & pyr & 0.1403 & 0.0416 & 0.2340 & \ipair{mpyr}{2}{ntf} (p1) \\ \cline{2-6}
  AVTZ & im & 0.1799 & 0.0377 & 0.2784 & \ipair{mim}{3}{ntf} (p2) \\ 
       & pyr & 0.1420 & 0.0412 & 0.2362 & \ipair{mpyr}{2}{ntf} (p1) \\ \hline
\end{tabular}
\end{table}


The NCT values were calculated as the difference between the unity charge, \emph{i.e.} 1$e$ and the total charge on the cation/anion.   
The statistics for the net charge-transfer from the geodesic scheme are given in Table \ref{tab:geod-stats}.
The same two Dunning basis sets were used in the geodesic scheme, and both basis sets showed very close agreement, with the largest differences being less than 0.01$e$, where $e$ is the elementary charge. This finding indicates that the geodesic scheme is independent of the basis set used.
Both aug-cc-pVDZ and aug-cc-pVTZ produced the largest NCT values for \ipair{mim}{3}{ntf} and \ipair{mpyr}{2}{ntf}.
 
Analysis of the NCT data reveals that the amount of net charge transfer observed is dependent on the cation type.
The average NCT is 0.18$e$ for \catb{mim}{n}[Anion] and 0.14$e$ for \catb{mpyr}{n}[Anion], meaning that the average charges on the ions are $\pm 0.82$ and $\pm 0.86$ for imidazolium and pyrrolidinium ion pairs, respectively.
The standard deviation for charge-transfer for both classes of cations is rather small, averaging 0.044$e$.


\begin{table}
    \centering
    \footnotesize
\caption{Correlation coefficients between geodesic NCT and charge-transfer/induction from SAPT and EFP.}
\label{tab:geod-corr}
\begin{tabular}{lll|rr|rr}
        \hline
               &            &            & \multicolumn{2}{c}{SAPT} & \multicolumn{2}{c}{EFP} \\
  Basis        & Halide     & Energy     & Pearson    & Spearman    & Pearson    & Spearman   \\ \hline
  aug-cc-pVDZ  & Halides    & Induction  & -0.617     & -0.691      & -0.351     & -0.307     \\
  aug-cc-pVDZ  & Halides    & CT         & -0.675     & -0.837      & -0.776     & -0.869     \\
  aug-cc-pVTZ  & Halides    & Induction  & -0.612     & -0.692      & -0.383     & -0.303     \\
  aug-cc-pVTZ  & Halides    & CT         & -0.697     & -0.842      & -0.150     & -0.058     \\
  aug-cc-pVDZ  & TILA       & Induction  & -0.002     & 0.010       & 0.093      & 0.104      \\
  aug-cc-pVDZ  & TILA       & CT         & 0.066      & 0.119       & -0.216     & -0.101     \\
  aug-cc-pVTZ  & TILA       & Induction  & -0.012     & -0.015      & 0.058      & 0.037      \\
  aug-cc-pVTZ  & TILA       & CT         & 0.064      & 0.110       & -0.024     & 0.032      \\ \hline
\end{tabular}
\end{table}
  

Correlation coefficients, both Pearson's product-moment correlation coefficient and Spearman's rank correlation coefficient, are presented in Table \ref{tab:geod-corr}.
\cite{Mukaka2012a, Pearson1895a, Edwards1976a}
They represent the correlation between the NCT and an energetic component calculated using either SAPT or EFP. 
Both charge-transfer and induction energies were correlated with NCT values.


Pearson's correlation coefficient ($r$) is commonly used to measure the linear correlation between two variables, $x$ and $y$, and is calculated by dividing the covariance ($\sigma_{xy}$) between the two variables with the product of their standard deviations:

\begin{equation}
    r = \frac{\sigma_{xy}}{\sigma_x \sigma_y}.
\end{equation}

This correlation coefficient has a range from -1 to 1.
A value of 0 indicates no correlation; 1 means perfect positive correlation, while -1 corresponds to perfect negative correlation.
%% A very rough rule-of-the-thumb regarding correlation coefficients is that if the coefficient is below -0.5 or above 0.5, then the correlation is significant.
Spearman's correlation coefficient is a rank correlation coefficient and has similar properties to Pearson's coefficient, also having a range of -1 to 1. 
While Pearson's coefficient measures how well a linear function describes the relationship between two variables, Spearman's coefficient measures how well a monotonic function captures that relationship.
It is the Pearson correlation coefficient between the \emph{ranked} variables, and is less susceptible to non-linear behaviour and outliers.


It is surprising that most of the correlation coefficients are negative, and positive coefficients tend to be small.
This means that the smaller the NCT is, the larger the stabilising effect is.
This negative correlation is more prominent for ion pairs with halides as the anion, whereas for the TILAs the small coefficients point to very little correlation.
The charge-transfer energy always had a larger coefficient when compared with the corresponding induction coefficient for SAPT.
While the same trend is not always observed for EFP, both SAPT and EFP have the largest correlation coefficients for charge-transfer energy calculated using aug-cc-pVDZ for systems with halide anions.
This agreement is not observed for aug-cc-pVTZ, due to the larger charge-transfer energies and larger standard deviations EFP gave for the larger basis set, which was not expected as charge-transfer should decrease with increasing basis set size. 
NCT from the geodesic scheme changed very little between basis sets.
The low values for the correlation coefficients for TILAs reinforce the complex nature of the charge-transfer phenomenon.
To this end, the charge-transfer stabilising energy has a very weak relationship with the actual NCT.

%@inbook{Edwards1976a,
%    author    = "Edwards, A. L."
%    %editor  = "",
%    title   = "An Introduction to Linear Regression and Correlation.",
%    chapter = "Ch. 4 The Correlation Coefficient",
%    pages   = "33-46",
%    publisher= "W. H. Freeman",
%    %volume = "",
%    %number = "",
%    %series = "",
%    %type   = "",
%    %address= "",
%    %edition= "",
%    year    = "1976",
%    %month  = "",
%    %note   = "",
%}
%
%@inproceedings{Pearson1895a,
%    author      = "Karl Pearson",
%    title       = "Notes on regression and inheritance in the case of two parents",
%    booktitle   = "Proceedings of the Royal Society of London",
%    %editor     = "",
%    volume     = "58",
%    %number     = "",
%    %series     = "",
%    pages      = "240-242",
%    %address    = "",
%    %organization   = "",
%    %publisher  = "",
%    year        = "1895",
%    %month      = "",
%    %note       = "",
%}


%For Kendall correlation, the coefficents are both 0.30; using Spearman's correlation, both 0.41.
%discuss the geodesic results ... correlate with electrostatics = 0.79 (pearson), 0.69 (kendall), 0.87 (spearman)
% halides only: kendall = 0.655, spearman = 0.837
% non-halides: kendall = -0.064, spearman = -0.097


\section{Conclusion}
\label{sec:conclusion}
% section

The interaction energies and their individual fundamental components calculated by means of the EFP method were compared with those from SAPT2+3 for an extensive series of single ion pairs of ionic liquids.
Overall, the deviations between the two methods for total interaction energy and the four fundamental components such as electrostatic, exchange-repulsion, induction and dispersion were much larger on the absolute scale than expected, falling in the range of -63 to 70 \enUnit. 
Out of all three basis sets studied for EFP, aug-cc-pVTZ gave the lowest errors.
The largest absolute errors came from the charge-transfer energy for the halide-based ion pairs and the exchange-repulsion contribution for the ILs combined with other routinely used anions. 
On the relative scale, the EFP method did not deviate from SAPT by more than 20\% \emph{on average} for all systems and energetic components (with the exception of charge transfer energy for halides). 
Electrostatics in particular showed very small relative errors, below 3\% for all basis sets on average.
As expected, out of the five fundamental components, charge-transfer from EFP produced the largest relative errors, with EFP overestimating SAPT2+3 by 40\% on average for aug-cc-pVTZ.
Apart from charge-transfer, EFP relative errors for ILs are comparable to those reported for the S22 and S66 databases.
\cite{Flick2012a}
This finding indicates that the large EFP errors are not specific to charged intermolecular complexes, as previously suggested.
Due to the sheer magnitude of each individual interaction in ionic liquids reaching up to -479 \enUnit~for electrostatics (-378 \enUnit~on average) it is not surprising that on the absolute scale the average errors were well beyond chemical accuracy, thus rendering EFP inapplicable for ILs at present.


The importance of higher-order terms in the EFP formulation has been established for the exchange-repulsion contribution in the case of typical ionic liquid anions, whereas EFP exchange-repulsion was treated rather well for halide systems.  
It has to be noted that similar relative errors from higher-order terms for electrostatics and exchange-repulsion were observed in neutral intermolecular complexes from the S22 and S66 databases. 
Electrostatics did not show appreciable contribution from higher-order terms for all ILs studied.


Although linear regression analysis of EFP against SAPT2+3 per each energetic component had very high values for the coefficient of determination, $R^2$, significant reduction in error was achieved only for the induction and dispersion terms. 
Moreover, better statistics for linear regression were found when the ion pairs were separated into cation-specific groups. 
Small errors for scaled induction and dispersion in the range of 1.2--3.2 and 0.7--11.4~\enUnit~on average were observed for both for TILAs and halides, respectively. 
This is particularly important highlighting the applicability of the EFP formulation for these two terms for ionic liquid systems. 
This finding will assist in the future development of intermolecular potentials for these complex systems.  



%\section*{Acknowledgements}
\begin{acknowledgement}
The authors gratefully acknowledge a generous allocation of computer resources through the Monash \emph{e}Research
Centre and the National Computational Infrastructure. This
work is generously supported by the Australian Research Council through a Discovery Project Grant and a Future Fellowship for EII.
\end{acknowledgement}

\begin{suppinfo}
    The supplementary information includes data not displayed in the main text.
    The statistics on the differences between the two SAPT2+3 basis sets, aug-cc-pVDZ and aug-cc-pVTZ, linear regression statistics for the other two EFP basis sets, aug-cc-pVDZ and 6-311++G(d,p), as well as the raw energies from both SAPT2+3 and EFP for each ion pair are all presented here.
\end{suppinfo}

%
% tables of data

\documentclass[a4paper]{article}

\usepackage{etex}
\reserveinserts{28}


% generic preamble

%%%%%%%%%%%%%%%%%%%%%%%%%%%%%%%%%
% leave macro definitons and stuff in the actual file preamble
% also packages only used by that .tex file
% can \input{} this file
%%%%%%%%%%%%%%%%%%%%%%%%%%%%%%%%%


%\usepackage{gfsartemisia}
%\usepackage[math]{iwona}    %no ink traps
%\usepackage[math]{kurier}    %gaps between certain letters
%usepackage[sc]{mathpazo}
\linespread{1.1}
\usepackage[T1]{fontenc}
\usepackage[utf8]{inputenc}
\usepackage[a4paper,margin=2.0cm]{geometry}
%\usepackage{changepage}       %change page margins for figures
%\usepackage{marginnote}
%\usepackage{multicol}
%\usepackage{amsmath}
%\usepackage{breqn}
% allows onehalf & double & arbitrary spacing
\usepackage{setspace}
\usepackage[demo]{graphicx}
%\usepackage{pdfpages}
%\usepackage{subfigure}
%\usepackage{lscape}
%\usepackage{rotating}

% for pretty tables
\usepackage{booktabs}

\usepackage[toc, header]{appendix}

% for \IfEqCase conditionals
\usepackage{xstring}

%\usepackage{fancyhdr}
%%\setlength{\headhight}{14pt}
%\pagestyle{fancy}
%\fancyhf{}
%\renewcommand{\headrulewidth}{0.2pt}
%\lhead{}
%\chead{\rightmark}
%\rhead{}

\usepackage[backend=biber,sorting=none,style=numeric-comp,mcite,subentry,
                firstinits=true, autocite=superscript]{biblatex}
\bibliography{\string~/Dropbox/Papers/zot_il_lib}

% autocite capable of multicite (high-level citation markup)
% only relevant to numeric styles, since using superscript
% I mainly want it to swap between before/after punctuation superscript
\newrobustcmd*{\mautocite}{\mcitelike\autocite}

% hyperref must be at the end of all used packages, and before other settings
\usepackage[colorlinks=true,linkcolor=magenta,citecolor=blue]{hyperref}


\usepackage{multirow}

\usepackage{amsmath}


\newcommand{\ipair}[3]{
    \IfEqCase{#3} {
        {bfl} {\ensuremath{[ \text{C}_{\text{#2}}\text{#1}] [ \text{BF}_4 ]}}  
        {br} {\ensuremath{[ \text{C}_{\text{#2}}\text{#1}] [ \text{Br} ]}}
        {cl} {\ensuremath{[ \text{C}_{\text{#2}}\text{#1}] [ \text{Cl} ]}}
        {dca} {\ensuremath{[ \text{C}_{\text{#2}}\text{#1}] [ \text{Dca} ]}}
        {mes} {\ensuremath{[ \text{C}_{\text{#2}}\text{#1}] [ \text{Mes} ]}}
        {ntf} {\ensuremath{[ \text{C}_{\text{#2}}\text{#1}] [ \text{NTf}_{2} ]}} 
        {pf} {\ensuremath{[ \text{C}_{\text{#2}}\text{#1}] [ \text{PF}_6 ]}}
        {tos} {\ensuremath{[ \text{C}_{\text{#2}}\text{#1}] [ \text{Tos} ]}}   
    } 
    [ \PackageError{ipair}{Undefined option (anion) to ipair: #3}{} ]
}

% for cations
\newcommand{\cat}[2]{ $ \text{C}_{#2}\text{#1}^{+} $ }
\newcommand{\catb}[2]{ $ [ \text{C}_{#2}\text{#1} ] $ }


\usepackage{pdflscape}
\usepackage{longtable}
% make font for tables smaller
\usepackage{floatrow}
\floatsetup[table]{font=footnotesize}
\floatsetup[table]{capposition=top}

% too many unprocessed floats error
\usepackage{morefloats}

\errorcontextlines 10000

\begin{document}

All energies given in kiloJoules per mole.

% latex table generated in R 3.2.0 by xtable 1.7-4 package
% Thu Oct 15 07:24:59 2015
\begin{table}[ht]
\centering
\footnotesize
\caption{Difference between aug-cc-pVTZ and aug-cc-pVDZ for SAPT2+3 by energetic components of interaction energy}
\begin{tabular}{llrrr}
  \hline
Cation & Energy & MAD & SD & max \\ 
  \hline
  \multirow{6}{*}{\catb{mim}{n}}   & Elst & 3.0 & 1.5 & 4.8 \\ 
                                   & Exch & 5.1 & 0.9 & -6.2 \\ 
                                   & Ind & 0.6 & 0.5 & 1.1 \\ 
                                   & Disp & 6.0 & 0.6 & -7.0 \\ 
                                   & CT & 8.9 & 1.6 & 12.2 \\ 
                                   & Total & 7.7 & 0.7 & -9.0 \\ \hline
  \multirow{6}{*}{\catb{mpyr}{n}}  & Elst & 1.9 & 0.2 & 2.0 \\ 
                                   & Exch & 4.0 & 1.0 & -4.6 \\ 
                                   & Ind & 0.8 & 0.1 & 0.9 \\ 
                                   & Disp & 4.1 & 0.5 & -4.5 \\ 
                                   & CT & 6.2 & 0.4 & 6.5 \\ 
                                   & Total & 5.5 & 1.6 & -6.4 \\  \hline
\end{tabular}
\end{table}


%This table is of differences between CCSD(T) and SAPT2+3 exceeding 3 kJ/mol.

%% latex table generated in R 3.1.3 by xtable 1.7-4 package
%% Wed Aug 19 17:09:28 2015
%\begin{table}[ht]
%\centering
%\footnotesize
%\begin{tabular}{lr}
%  \hline
%name & CCSD\_SAPT\_Etot\_ACCQ.Diff \\ 
%  \hline
%4-im-tos-p1 & -8.67 \\ 
%  3-im-tos-p2 & -8.28 \\ 
%  4-im-tos-p2 & -8.24 \\ 
%  3-im-tos-p1 & -8.23 \\ 
%  4-im-cl-p4 & -8.19 \\ 
%  3-im-cl-p4 & -7.96 \\ 
%  2-im-tos-p2 & -7.95 \\ 
%  2-im-tos-p1 & -7.62 \\ 
%  4-im-br-p4 & -7.51 \\ 
%  3-im-br-p4 & -7.35 \\ 
%  1-im-tos-p1 & -7.22 \\ 
%  4-im-mes-p2 & -6.84 \\ 
%  3-im-mes-p2 & -6.64 \\ 
%  2-im-cl-p4 & -6.59 \\ 
%  1-im-cl-p1 & -6.38 \\ 
%  4-im-mes-p1 & -6.32 \\ 
%  2-im-mes-p2 & -6.26 \\ 
%  2-im-ntf2-p2 & -6.13 \\ 
%  3-im-mes-p1 & -5.98 \\ 
%  2-im-ntf2-p1 & -5.87 \\ 
%  2-im-br-p4 & -5.85 \\ 
%  1-im-br-p1 & -5.73 \\ 
%  2-im-ntf2-p4 & -5.56 \\ 
%  4-im-cl-p1 & -5.55 \\ 
%  2-im-mes-p1 & -5.54 \\ 
%  1-im-mes-p1 & -5.44 \\ 
%  3-im-cl-p1 & -5.42 \\ 
%  2-im-ntf2-p3 & -5.39 \\ 
%  1-im-ntf2-p2 & -5.37 \\ 
%  2-im-cl-p1 & -5.08 \\ 
%  1-im-ntf2-p1 & -5.07 \\ 
%  1-im-ntf2-p3 & -5.02 \\ 
%  3-pyr-tos-p1 & -4.92 \\ 
%  4-im-br-p1 & -4.82 \\ 
%  3-im-br-p1 & -4.68 \\ 
%  2-pyr-tos-p1 & -4.48 \\ 
%  2-im-br-p1 & -4.18 \\ 
%  3-pyr-ntf2-p1 & -3.95 \\ 
%  3-pyr-tos-p2 & -3.83 \\ 
%  2-pyr-ntf2-p1 & -3.83 \\ 
%  1-pyr-tos-p1 & -3.80 \\ 
%  2-pyr-ntf2-p2 & -3.76 \\ 
%  1-pyr-ntf2-p2 & -3.76 \\ 
%  1-pyr-ntf2-p1 & -3.67 \\ 
%  1-pyr-ntf2-p5 & -3.62 \\ 
%  3-pyr-ntf2-p6 & -3.61 \\ 
%  3-pyr-ntf2-p4 & -3.59 \\ 
%  4-pyr-mes-p1 & -3.55 \\ 
%  3-pyr-ntf2-p2 & -3.54 \\ 
%  2-pyr-ntf2-p4 & -3.48 \\ 
%  2-pyr-ntf2-p6 & -3.47 \\ 
%  1-pyr-tos-p2 & -3.42 \\ 
%  3-pyr-mes-p1 & -3.36 \\ 
%  2-pyr-tos-p2 & -3.34 \\ 
%  1-pyr-ntf2-p3 & -3.30 \\ 
%  3-pyr-ntf2-p5 & -3.27 \\ 
%  3-pyr-ntf2-p3 & -3.27 \\ 
%  2-pyr-ntf2-p3 & -3.10 \\ 
%   \hline
%\end{tabular}
%\end{table}



\begin{landscape}


\begin{table}[h]
\centering
\footnotesize
\caption{Linear regression of each energy component, EFP(aug-cc-pVDZ) against SAPT}
\label{tab:scaled-EFP-adz-coef}
\begin{tabular}{ll|rrrrr|rrrrr}
    \hline
        &         & \multicolumn{5}{c}{Halides}                                                   & \multicolumn{5}{c}{TILAs}                                                  \\
Cation  & Energy  & $a$      & MAE         &         SD  &        Max  &                          &   $a$      &  MAE        &         SD  &        Max  &                           \\ \hline
    im  &   Elst  & 0.980    &  7.7        &  9.1        & -14.7       &  \ipair{mim}{2}{br}(p1)  &  0.967     &    7.6      &  11.9       &  53.9       &  \ipair{mim}{4}{tos}(p2)  \\
    im  &   Exch  & 0.993    &  6.6        &  7.8        &  16.7       &  \ipair{mim}{3}{cl}(p2)  &  1.135     &    5.6      &   6.5       & -19.5       &  \ipair{mim}{2}{dca}(p1)  \\
    im  &    Ind  & 1.140    &  6.3        &  8.1        & -19.6       &  \ipair{mim}{4}{cl}(p2)  &  1.206     &    2.5      &   3.7       & -17.1       &  \ipair{mim}{1}{ntf}(p2)  \\
    im  &   Disp  & 1.192    &  2.9        &  3.3        &  -7.4       &  \ipair{mim}{4}{br}(p2)  &  0.780     &    3.1      &   3.6       &  -8.7       &  \ipair{mim}{1}{ntf}(p2)  \\
    im  &  Total  & 1.017    & 16.6        & 20.4        &  45.8       &  \ipair{mim}{3}{cl}(p2)  &  0.909     &    8.3      &  12.4       &  46.6       &  \ipair{mim}{4}{tos}(p2)  \\ \hline
   pyr  &   Elst  & 0.957    &  5.9        &  7.3        & -15.8       & \ipair{mpyr}{2}{br}(p1)  &  0.980     &    6.6      &   9.2       &  36.3       & \ipair{mpyr}{4}{tos}(p2)  \\
   pyr  &   Exch  & 1.023    &  4.2        &  5.6        & -11.7       & \ipair{mpyr}{2}{br}(p1)  &  1.190     &    4.9      &   5.8       & -17.0       & \ipair{mpyr}{4}{dca}(p1)  \\
   pyr  &    Ind  & 1.067    &  3.1        &  3.8        &  -8.0       & \ipair{mpyr}{2}{cl}(p2)  &  1.181     &    1.8      &   2.2       &  -4.7       & \ipair{mpyr}{1}{ntf}(p3)  \\
   pyr  &   Disp  & 1.277    &  0.9        &  1.0        &  -1.9       & \ipair{mpyr}{3}{br}(p2)  &  0.874     &    1.6      &   1.9       &   5.0       & \ipair{mpyr}{1}{dca}(p3)  \\
   pyr  &  Total  & 0.981    & 10.9        & 13.4        & -30.8       & \ipair{mpyr}{2}{br}(p1)  &  0.944     &    6.6      &   9.4       &  36.8       & \ipair{mpyr}{4}{tos}(p2)  \\ \hline
\end{tabular}
\end{table}

\begin{table}[ht]
\centering
\footnotesize
\caption{Linear regression of each energy component, EFP(6-311++G(d,p)) against SAPT}
\label{tab:scaled-EFP-pop-coef}
\begin{tabular}{ll|rrrrr|rrrrr}
  \hline
          &        & \multicolumn{5}{c}{Halides}                                             & \multicolumn{5}{c}{TILAs}                         \\
   Cation & Energy & $a$      & MAE        & SD       & Max       &                          & $a$      & MAE        & SD       & Max       &          \\ \hline
   im     & Elst   & 1.026    & 10.7       & 13.4     & 31.4      & \ipair{mim}{4}{cl}(p4)   & 0.998 & 5.0  & 7.6  & 32.2  & \ipair{mim}{1}{ntf}(p2)   \\ 
   im     & Exch   & 1.088    & 22.4       & 23.9     & 36.5      & \ipair{mim}{3}{br}(p3)   & 1.351 & 17.6 & 18.1 & -31.0 & \ipair{mim}{2}{dca}(p1)   \\ 
   im     & Ind    & 1.557    & 8.6        & 10.7     & -21.4     & \ipair{mim}{3}{br}(p2)   & 1.299 & 2.5  & 3.9  & -15.7 & \ipair{mim}{1}{ntf}(p2)   \\ 
   im     & Disp   & 2.081    & 3.8        & 4.5      & -8.3      & \ipair{mim}{3}{br}(p2)   & 0.901 & 4.3  & 4.8  & -8.7  & \ipair{mim}{4}{pf}(p1)   \\ 
   im     & Total  & 1.127    & 18.4       & 23.3     & -65.9     & \ipair{mim}{4}{cl}(p2)   & 0.924 & 11.7 & 14.0 & 39.9  & \ipair{mim}{3}{ntf}(p1)   \\ \hline
   pyr    & Elst   & 1.024    & 3.0        & 3.7      & 6.7       & \ipair{mpyr}{3}{br}(p2)  & 0.982 & 6.1  & 9.5  & 36.1  & \ipair{mpyr}{1}{tos}(p2)  \\ 
   pyr    & Exch   & 1.192    & 21.2       & 22.1     & 28.5      & \ipair{mpyr}{1}{br}(p1)  & 1.505 & 10.6 & 11.8 & -24.6 & \ipair{mpyr}{4}{dca}(p1)  \\ 
   pyr    & Ind    & 1.384    & 6.0        & 7.3      & 18.4      & \ipair{mpyr}{4}{cl}(p1)  & 1.309 & 2.6  & 2.8  & 6.2   & \ipair{mpyr}{1}{dca}(p3)  \\ 
   pyr    & Disp   & 2.186    & 2.8        & 3.0      & 4.4       & \ipair{mpyr}{4}{cl}(p1)  & 0.992 & 2.8  & 3.5  & 6.4   & \ipair{mpyr}{3}{dca}(p3)  \\ 
   pyr    & Total  & 1.066    & 11.8       & 13.6     & 22.7      & \ipair{mpyr}{3}{br}(p2)  & 0.928 & 11.5 & 14.2 & 43.7  & \ipair{mpyr}{1}{tos}(p2)  \\ \hline
\end{tabular}
\end{table}


\end{landscape}


\begin{table}[h]
\centering
\footnotesize
\caption{Statistics for errors from scaling EFP against SAPT}
\label{tab:scaled-diff-stats}
\begin{tabular}{lll|rrrr|rrrr}
\hline
Basis                 & Cation                & Energy & \multicolumn{4}{c}{Halides}                     & \multicolumn{4}{c}{TILAs}                 \\
                      &                       &        & MAE  & SD   & Max   &                           & MAE  & SD   & Max   &                           \\ \hline
\multirow{5}{*}{AVDZ} & \multirow{15}{*}{im}  & Elst   & 7.7  & 9.1  & -14.7 & \ipair{mim}{2}{br} (p1)  & 7.6  & 11.9 & 53.9  & \ipair{mim}{4}{tos} (p2)  \\
                      &                       & Exch   & 6.6  & 7.8  & 16.7  & \ipair{mim}{3}{cl} (p2)  & 5.6  & 6.5  & -19.5 & \ipair{mim}{2}{dca} (p1)  \\
                      &                       & Ind    & 6.3  & 8.1  & -19.6 & \ipair{mim}{4}{cl} (p2)  & 2.5  & 3.7  & -17.1 & \ipair{mim}{1}{ntf} (p2)  \\
                      &                       & Disp   & 2.9  & 3.3  & -7.4  & \ipair{mim}{4}{br} (p2)  & 3.1  & 3.6  & -8.7  & \ipair{mim}{1}{ntf} (p2)  \\
                      &                       & Total  & 16.6 & 20.4 & 45.8  & \ipair{mim}{3}{cl} (p2)  & 8.3  & 12.4 & 46.6  & \ipair{mim}{4}{tos} (p2)  \\ \hline
\multirow{5}{*}{AVTZ} &                       & Elst   & 9.3  & 11.8 & -31.2 & \ipair{mim}{4}{br} (p4)  & 5.3  & 6.7  & -18.6 & \ipair{mim}{4}{bfl} (p2)  \\
                      &                       & Exch   & 5.8  & 7.2  & 17.1  & \ipair{mim}{3}{cl} (p2)  & 11.4 & 12.6 & -27.7 & \ipair{mim}{2}{dca} (p1)  \\
                      &                       & Ind    & 11.4 & 14.5 & 41.8  & \ipair{mim}{3}{br} (p2)  & 3.2  & 4.1  & -13.1 & \ipair{mim}{1}{ntf} (p2)  \\
                      &                       & Disp   & 2.8  & 3.3  & -7.0  & \ipair{mim}{4}{br} (p2)  & 2.5  & 3.5  & -9.8  & \ipair{mim}{2}{ntf} (p2)  \\
                      &                       & Total  & 20.8 & 26.7 & 59.8  & \ipair{mim}{3}{br} (p2)  & 7.0  & 9.8  & 26.1  & \ipair{mim}{2}{tos} (p1)  \\ \hline
\multirow{5}{*}{pop}  &                       & Elst   & 10.7 & 13.4 & 31.4  & \ipair{mim}{4}{cl} (p4)  & 5.0  & 7.6  & 32.2  & \ipair{mim}{1}{ntf} (p2)  \\
                      &                       & Exch   & 22.4 & 23.9 & 36.5  & \ipair{mim}{3}{br} (p3)  & 17.6 & 18.1 & -31.0 & \ipair{mim}{2}{dca} (p1)  \\
                      &                       & Ind    & 8.6  & 10.7 & -21.4 & \ipair{mim}{3}{br} (p2)  & 2.5  & 3.9  & -15.7 & \ipair{mim}{1}{ntf} (p2)  \\
                      &                       & Disp   & 3.8  & 4.5  & -8.3  & \ipair{mim}{3}{br} (p2)  & 4.3  & 4.8  & -8.7  & \ipair{mim}{4}{pf} (p1)   \\
                      &                       & Total  & 18.4 & 23.3 & -65.9 & \ipair{mim}{4}{cl} (p2)  & 11.7 & 14.0 & 39.9  & \ipair{mim}{3}{ntf} (p1)  \\ \hline \hline
\multirow{5}{*}{AVDZ} & \multirow{14}{*}{pyr} & Elst   & 5.9  & 7.3  & -15.8 & \ipair{mpyr}{2}{br} (p1) & 6.6  & 9.2  & 36.3  & \ipair{mpyr}{4}{tos} (p2) \\
                      &                       & Exch   & 4.2  & 5.6  & -11.7 & \ipair{mpyr}{2}{br} (p1) & 4.9  & 5.8  & -17.0 & \ipair{mpyr}{4}{dca} (p1) \\
                      &                       & Ind    & 3.1  & 3.8  & -8.0  & \ipair{mpyr}{2}{cl} (p2) & 1.8  & 2.2  & -4.7  & \ipair{mpyr}{1}{ntf} (p3) \\
                      &                       & Disp   & 0.9  & 1.0  & -1.9  & \ipair{mpyr}{3}{br} (p2) & 1.6  & 1.9  & 5.0   & \ipair{mpyr}{1}{dca} (p3) \\
                      &                       & Total  & 10.9 & 13.4 & -30.8 & \ipair{mpyr}{2}{br} (p1) & 6.6  & 9.4  & 36.8  & \ipair{mpyr}{4}{tos} (p2) \\ \hline
\multirow{5}{*}{AVTZ} &                       & Elst   & 4.1  & 4.9  & -10.8 & \ipair{mpyr}{3}{br} (p1) & 4.1  & 5.1  & 13.1  & \ipair{mpyr}{2}{ntf} (p2) \\
                      &                       & Exch   & 4.1  & 4.6  & 7.3   & \ipair{mpyr}{2}{cl} (p1) & 6.4  & 7.8  & -22.1 & \ipair{mpyr}{4}{dca} (p1) \\
                      &                       & Ind    & 3.0  & 3.8  & -7.9  & \ipair{mpyr}{3}{cl} (p1) & 2.4  & 2.7  & 6.2   & \ipair{mpyr}{2}{tos} (p1) \\
                      &                       & Disp   & 0.7  & 1.0  & -2.4  & \ipair{mpyr}{2}{br} (p2) & 1.2  & 1.5  & 4.8   & \ipair{mpyr}{1}{dca} (p3) \\
                      &                       & Total  & 7.8  & 9.3  & -16.1 & \ipair{mpyr}{3}{br} (p1) & 5.2  & 6.8  & 18.7  & \ipair{mpyr}{2}{tos} (p1) \\ \hline
\multirow{5}{*}{pop}  &                       & Elst   & 3.0  & 3.7  & 6.7   & \ipair{mpyr}{3}{br} (p2) & 6.1  & 9.5  & 36.1  & \ipair{mpyr}{1}{tos} (p2) \\
                      &                       & Exch   & 21.2 & 22.1 & 28.5  & \ipair{mpyr}{1}{br} (p1) & 10.6 & 11.8 & -24.6 & \ipair{mpyr}{4}{dca} (p1) \\
                      &                       & Ind    & 6.0  & 7.3  & 18.4  & \ipair{mpyr}{4}{cl} (p1) & 2.6  & 2.8  & 6.2   & \ipair{mpyr}{1}{dca} (p3) \\
                      &                       & Disp   & 2.8  & 3.0  & 4.4   & \ipair{mpyr}{4}{cl} (p1) & 2.8  & 3.5  & 6.4   & \ipair{mpyr}{3}{dca} (p3) \\
                      & pyr                   & Total  & 11.8 & 13.6 & 22.7  & \ipair{mpyr}{3}{br} (p2) & 11.5 & 14.2 & 43.7  & \ipair{mpyr}{1}{tos} (p2) \\ \hline
\end{tabular}
\end{table}


\begin{table}[h]
\centering
\footnotesize
\caption{Percentage differences between scaled EFP and SAPT}
\label{tab:scaled-perc-stats}
\begin{tabular}{lll|rrrr|rrrr}
\hline
Basis                 & Cation                & Energy & \multicolumn{4}{c}{Halides}                  & \multicolumn{4}{c}{TILAs}               \\
                      &                       &        & MAE  & SD  & Max  &                          & MAE  & SD  & Max  &                           \\ \hline
\multirow{5}{*}{AVDZ} & \multirow{15}{*}{im}  & Elst   & 1.7  & 1.0 & 3.4  & \ipair{mim}{3}{cl} (p2)  & 2.0  & 2.3 & 13.1 & \ipair{mim}{4}{tos} (p2)  \\
                      &                       & Exch   & 3.8  & 2.4 & 9.7  & \ipair{mim}{3}{cl} (p2)  & 5.0  & 2.7 & 18.0 & \ipair{mim}{2}{dca} (p1)  \\
                      &                       & Ind    & 6.8  & 5.0 & 18.0 & \ipair{mim}{4}{cl} (p2)  & 4.9  & 4.2 & 25.4 & \ipair{mim}{1}{ntf} (p2)  \\
                      &                       & Disp   & 6.0  & 4.0 & 16.6 & \ipair{mim}{4}{br} (p2)  & 5.2  & 3.3 & 17.6 & \ipair{mim}{1}{ntf} (p2)  \\
                      &                       & Total  & 4.1  & 2.9 & 11.2 & \ipair{mim}{3}{cl} (p2)  & 2.2  & 2.4 & 11.3 & \ipair{mim}{4}{tos} (p2)  \\ \hline
\multirow{5}{*}{AVTZ} &                       & Elst   & 2.1  & 1.6 & 6.8  & \ipair{mim}{4}{br} (p4)  & 1.4  & 1.0 & 4.9  & \ipair{mim}{4}{bfl} (p2)  \\
                      &                       & Exch   & 3.3  & 2.5 & 9.9  & \ipair{mim}{3}{cl} (p2)  & 10.2 & 4.8 & 25.7 & \ipair{mim}{2}{dca} (p1)  \\
                      &                       & Ind    & 12.5 & 9.3 & 44.4 & \ipair{mim}{3}{br} (p2)  & 6.2  & 4.1 & 19.6 & \ipair{mim}{1}{ntf} (p2)  \\
                      &                       & Disp   & 5.9  & 4.2 & 15.6 & \ipair{mim}{4}{br} (p2)  & 4.2  & 4.3 & 18.8 & \ipair{mim}{1}{ntf} (p2)  \\
                      &                       & Total  & 5.2  & 4.1 & 15.5 & \ipair{mim}{3}{br} (p2)  & 1.8  & 1.7 & 6.3  & \ipair{mim}{2}{tos} (p1)  \\ \hline
\multirow{5}{*}{pop}  &                       & Elst   & 2.4  & 1.7 & 6.7  & \ipair{mim}{4}{cl} (p4)  & 1.4  & 1.6 & 9.3  & \ipair{mim}{1}{ntf} (p2)  \\
                      &                       & Exch   & 12.4 & 4.6 & 22.6 & \ipair{mim}{3}{br} (p3)  & 16.2 & 4.6 & 28.7 & \ipair{mim}{2}{dca} (p1)  \\
                      &                       & Ind    & 9.5  & 6.4 & 22.7 & \ipair{mim}{3}{br} (p2)  & 4.9  & 4.7 & 23.4 & \ipair{mim}{1}{ntf} (p2)  \\
                      &                       & Disp   & 7.7  & 5.2 & 19.3 & \ipair{mim}{3}{br} (p2)  & 7.8  & 4.7 & 16.2 & \ipair{mim}{4}{pf} (p1)   \\
                      &                       & Total  & 4.6  & 3.5 & 16.1 & \ipair{mim}{4}{cl} (p2)  & 3.2  & 2.1 & 11.1 & \ipair{mim}{3}{ntf} (p1)  \\ \hline \hline
\multirow{5}{*}{AVDZ} & \multirow{15}{*}{pyr} & Elst   & 1.5  & 1.0 & 3.8  & \ipair{mpyr}{2}{br} (p1) & 1.9  & 1.8 & 9.9  & \ipair{mpyr}{4}{tos} (p2) \\
                      &                       & Exch   & 3.0  & 2.5 & 8.4  & \ipair{mpyr}{4}{cl} (p2) & 5.8  & 4.0 & 24.4 & \ipair{mpyr}{4}{dca} (p1) \\
                      &                       & Ind    & 4.2  & 2.8 & 10.7 & \ipair{mpyr}{2}{cl} (p2) & 3.7  & 2.5 & 11.3 & \ipair{mpyr}{2}{ntf} (p3) \\
                      &                       & Disp   & 1.9  & 1.1 & 4.2  & \ipair{mpyr}{3}{br} (p2) & 3.7  & 2.6 & 11.9 & \ipair{mpyr}{1}{dca} (p3) \\
                      &                       & Total  & 2.8  & 2.0 & 8.1  & \ipair{mpyr}{2}{br} (p1) & 1.9  & 1.8 & 9.7  & \ipair{mpyr}{4}{tos} (p2) \\ \hline
\multirow{5}{*}{AVTZ} &                       & Elst   & 1.0  & 0.6 & 2.6  & \ipair{mpyr}{3}{br} (p1) & 1.2  & 0.9 & 4.3  & \ipair{mpyr}{2}{ntf} (p2) \\
                      &                       & Exch   & 2.9  & 1.4 & 5.3  & \ipair{mpyr}{3}{cl} (p2) & 7.8  & 6.0 & 31.6 & \ipair{mpyr}{4}{dca} (p1) \\
                      &                       & Ind    & 4.0  & 3.0 & 10.4 & \ipair{mpyr}{3}{cl} (p1) & 4.8  & 2.4 & 9.1  & \ipair{mpyr}{2}{tos} (p1) \\
                      &                       & Disp   & 1.7  & 1.3 & 5.1  & \ipair{mpyr}{2}{br} (p2) & 2.9  & 2.3 & 11.3 & \ipair{mpyr}{1}{dca} (p3) \\
                      &                       & Total  & 2.1  & 1.2 & 4.2  & \ipair{mpyr}{3}{br} (p1) & 1.4  & 1.2 & 4.7  & \ipair{mpyr}{2}{tos} (p1) \\ \hline
\multirow{5}{*}{pop}  &                       & Elst   & 0.7  & 0.5 & 1.7  & \ipair{mpyr}{3}{br} (p2) & 1.7  & 1.9 & 9.7  & \ipair{mpyr}{3}{tos} (p2) \\
                      &                       & Exch   & 14.8 & 3.1 & 19.7 & \ipair{mpyr}{4}{br} (p2) & 13.4 & 7.2 & 35.3 & \ipair{mpyr}{4}{dca} (p1) \\
                      &                       & Ind    & 8.1  & 5.6 & 24.0 & \ipair{mpyr}{4}{cl} (p1) & 5.4  & 2.8 & 14.3 & \ipair{mpyr}{1}{dca} (p3) \\
                      &                       & Disp   & 6.2  & 1.7 & 9.7  & \ipair{mpyr}{4}{cl} (p1) & 6.9  & 6.0 & 17.9 & \ipair{mpyr}{2}{pf} (p2)  \\
                      &                       & Total  & 3.1  & 1.7 & 6.3  & \ipair{mpyr}{3}{br} (p2) & 3.2  & 2.2 & 11.3 & \ipair{mpyr}{1}{tos} (p2) \\ \hline
\end{tabular}
\end{table}



%%%% ================== SAPT0 results =================== %%%
%% latex table generated in R 3.1.3 by xtable 1.7-4 package
%% Wed Aug 19 10:32:26 2015
%\begin{table}[ht]
%%\begin{center}
%\centering
%%{\scriptsize 
%\begin{tabular}{cccccrrrrr}
%  \hline 
%Basis & Chain & Cation & Anion & Conf & Electrostatics & Exchange & Induction & Dispersion & Total.SAPT0 \\ 
%  \hline 
%VDZ &  1 & im & bf4 & p1 & -399.7 & 91.8 & -35.4 & -23.0 & -368.0 \\ 
%  VQZ &  1 & im & bf4 & p1 & -388.1 & 87.4 & -39.0 & -41.6 & -386.6 \\ 
%  VTZ &  1 & im & bf4 & p1 & -391.7 & 88.5 & -38.2 & -36.0 & -381.4 \\ 
%  aVDZ &  1 & im & bf4 & p1 & -389.5 & 91.9 & -40.7 & -39.3 & -384.2 \\ 
%  aVTZ &  1 & im & bf4 & p1 & -387.1 & 88.8 & -39.9 & -43.5 & -388.5 \\ 
%  VDZ &  1 & im & dca & p1 & -396.5 & 114.0 & -31.8 & -44.6 & -357.6 \\ 
%  VQZ &  1 & im & dca & p1 & -384.3 & 108.0 & -35.0 & -64.8 & -377.3 \\ 
%  VTZ &  1 & im & dca & p1 & -388.2 & 109.7 & -34.3 & -59.3 & -372.6 \\ 
%  aVDZ &  1 & im & dca & p1 & -384.4 & 113.8 & -36.5 & -61.2 & -371.9 \\ 
%  aVTZ &  1 & im & dca & p1 & -382.5 & 109.6 & -35.9 & -66.2 & -377.6 \\ 
%  VDZ &  1 & im & ntf2 & p2 & -363.8 & 108.1 & -59.6 & -32.3 & -347.6 \\ 
%  VQZ &  1 & im & ntf2 & p2 & -349.2 & 108.0 & -62.1 & -46.3 & -349.6 \\ 
%  VTZ &  1 & im & ntf2 & p2 & -352.6 & 108.5 & -61.5 & -42.6 & -348.2 \\ 
%  aVDZ &  1 & im & ntf2 & p2 & -348.5 & 108.6 & -62.3 & -44.2 & -346.3 \\ 
%  aVTZ &  1 & im & ntf2 & p2 & -348.1 & 108.2 & -62.6 & -47.3 & -349.8 \\ 
%  VDZ &  1 & im & tos & p1 & -427.2 & 107.1 & -45.9 & -37.9 & -403.8 \\ 
%  VQZ &  1 & im & tos & p1 & -416.0 & 103.3 & -51.4 & -60.2 & -424.3 \\ 
%  VTZ &  1 & im & tos & p1 & -418.9 & 104.0 & -50.2 & -53.7 & -418.8 \\ 
%  aVDZ &  1 & im & tos & p1 & -415.6 & 104.4 & -51.6 & -57.7 & -420.6 \\ 
%  aVTZ &  1 & im & tos & p1 & -415.2 & 103.4 & -51.8 & -62.1 & -425.7 \\ 
%  VDZ &  1 & pyr & bf4 & p1 & -385.4 & 85.3 & -44.3 & -19.5 & -363.9 \\ 
%  VQZ &  1 & pyr & bf4 & p1 & -372.8 & 82.2 & -45.8 & -34.9 & -373.7 \\ 
%  VTZ &  1 & pyr & bf4 & p1 & -376.5 & 83.1 & -45.5 & -30.6 & -370.8 \\ 
%  aVDZ &  1 & pyr & bf4 & p1 & -373.2 & 86.0 & -47.1 & -32.8 & -372.2 \\ 
%  aVTZ &  1 & pyr & bf4 & p1 & -371.4 & 83.4 & -46.4 & -36.3 & -374.6 \\ 
%  VDZ &  1 & pyr & dca & p1 & -361.6 & 85.6 & -41.0 & -30.0 & -345.8 \\ 
%  VQZ &  1 & pyr & dca & p1 & -351.1 & 81.6 & -42.6 & -43.3 & -355.6 \\ 
%  VTZ &  1 & pyr & dca & p1 & -354.5 & 82.8 & -42.3 & -39.9 & -353.6 \\ 
%  aVDZ &  1 & pyr & dca & p1 & -349.6 & 85.8 & -43.5 & -40.8 & -351.0 \\ 
%  aVTZ &  1 & pyr & dca & p1 & -349.0 & 82.7 & -43.0 & -44.2 & -355.1 \\ 
%  VDZ &  2 & im & br & p3 & -438.8 & 153.3 & -78.1 & -26.6 & -390.3 \\ 
%  VQZ &  2 & im & br & p3 & -415.2 & 146.0 & -81.9 & -43.0 & -394.1 \\ 
%  VTZ &  2 & im & br & p3 & -421.4 & 147.8 & -80.8 & -38.6 & -392.9 \\ 
%  aVDZ &  2 & im & br & p3 & -412.0 & 144.2 & -81.0 & -37.8 & -386.5 \\ 
%  aVQZ &  2 & im & br & p3 & -412.1 & 145.1 & -82.6 & -44.9 & -394.4 \\ 
%  aVTZ &  2 & im & br & p3 & -412.1 & 145.1 & -82.5 & -43.5 & -392.9 \\ 
%  VDZ &  2 & im & mes & p2 & -436.4 & 109.5 & -43.9 & -37.7 & -408.5 \\ 
%  VQZ &  2 & im & mes & p2 & -424.1 & 105.2 & -48.7 & -60.9 & -428.5 \\ 
%  VTZ &  2 & im & mes & p2 & -427.5 & 106.1 & -47.8 & -54.1 & -423.2 \\ 
%  aVDZ &  2 & im & mes & p2 & -423.5 & 106.3 & -49.0 & -58.1 & -424.3 \\ 
%  aVQZ &  2 & im & mes & p2 & -422.6 & 104.9 & -49.2 & -64.2 & -431.1 \\ 
%  aVTZ &  2 & im & mes & p2 & -422.8 & 105.2 & -49.2 & -62.9 & -429.7 \\ 
%  VDZ &  2 & im & ntf2 & p1 & -365.1 & 108.7 & -33.9 & -54.2 & -344.4 \\ 
%  VQZ &  2 & im & ntf2 & p1 & -351.9 & 106.9 & -36.8 & -82.4 & -364.3 \\ 
%  VTZ &  2 & im & ntf2 & p1 & -355.7 & 107.4 & -36.1 & -74.3 & -358.7 \\ 
%  aVDZ &  2 & im & ntf2 & p1 & -351.9 & 108.0 & -37.2 & -80.1 & -361.2 \\ 
%  aVQZ &  2 & im & ntf2 & p1 & -351.0 & 106.8 & -37.3 & -86.5 & -368.1 \\ 
%  aVTZ &  2 & im & ntf2 & p1 & -351.2 & 107.1 & -37.3 & -85.2 & -366.6 \\ 
%  VDZ &  3 & im & br & p4 & -482.9 & 195.6 & -70.8 & -40.3 & -398.4 \\ 
%  VQZ &  3 & im & br & p4 & -461.6 & 184.4 & -75.0 & -65.3 & -417.5 \\ 
%  VTZ &  3 & im & br & p4 & -467.4 & 187.2 & -73.7 & -58.3 & -412.2 \\ 
%  aVDZ &  3 & im & br & p4 & -457.5 & 182.5 & -74.2 & -58.6 & -407.9 \\ 
%  aVQZ &  3 & im & br & p4 & -458.5 & 183.2 & -75.9 & -68.3 & -419.4 \\ 
%  aVTZ &  3 & im & br & p4 & -458.6 & 183.3 & -75.7 & -66.1 & -417.1 \\ 
%  VDZ &  3 & im & cl & p2 & -459.9 & 168.0 & -98.1 & -25.5 & -415.5 \\ 
%  VQZ &  3 & im & cl & p2 & -438.9 & 161.7 & -102.8 & -42.9 & -422.9 \\ 
%  VTZ &  3 & im & cl & p2 & -445.7 & 164.0 & -102.2 & -38.4 & -422.3 \\ 
%  aVDZ &  3 & im & cl & p2 & -433.4 & 159.9 & -101.5 & -38.1 & -413.0 \\ 
%  aVQZ &  3 & im & cl & p2 & -433.2 & 160.0 & -103.3 & -44.9 & -421.4 \\ 
%  aVTZ &  3 & im & cl & p2 & -433.2 & 160.0 & -103.1 & -43.6 & -420.0 \\ 
%   \hline
%
%\caption{SAPT0 results for problematic systems across various Dunning basis sets} \\ % need this for longtable
%\end{tabular}
%%}
%%\end{center}
%\end{table}


%The following tables are statistics for the differences between the ratios of each component relative to the total energy. 
%
%\begin{equation*} 
%    E_{\text{diff}} = 100 \cdot \left( 
%            \frac{E^{\text{SAPT}}_{\text{component}}}{E^{\text{SAPT}}_{\text{Total E}}} - 
%                    \frac{E^{\text{EFP}}_{\text{component}}}{E^{\text{EFP}}_{\text{Total E}}} \right)
%\end{equation*}
%
%Naturally this is not meaningful for the total energy. 

% %%%%%%%%%%%%%%%% ratio stats %%%%%%%%%%%%%%%%%% 
% 
% \begin{table}[ht]
% \centering
% \begin{tabular}{lllrrrrrll}
%   \hline
% Halide & Cation & Energy & mean & med & sd & min & max & min.name & max.name \\ 
%   \hline
% Hal & im & Elec & 109.74 & 107.40 & 3.67 & 104.83 & 114.82 & \ipair{mim}{4}{br} (p2) & \ipair{mim}{1}{br} (p1) \\ 
%   Hal & im & Exch.Repl & 44.71 & -42.65 & 3.33 & -49.88 & -39.50 & \ipair{mim}{4}{br} (p4) & \ipair{mim}{1}{br} (p2) \\ 
%   Hal & im & Ind.Pol & 22.20 & 22.47 & 2.44 & 18.82 & 26.77 & \ipair{mim}{1}{br} (p1) & \ipair{mim}{4}{cl} (p2) \\ 
%   Hal & im & Disp & 12.77 & 12.83 & 1.97 & 9.57 & 15.93 & \ipair{mim}{1}{cl} (p2) & \ipair{mim}{4}{br} (p4) \\ 
%   Hal & pyr & Elec & 106.37 & 106.51 & 1.37 & 104.29 & 108.33 & \ipair{mpyr}{4}{cl} (p1) & \ipair{mpyr}{1}{br} (p2) \\ 
%   Hal & pyr & Exch.Repl & 37.55 & -37.44 & 2.11 & -40.13 & -35.08 & \ipair{mpyr}{2}{br} (p2) & \ipair{mpyr}{1}{cl} (p1) \\ 
%   Hal & pyr & Ind.Pol & 19.35 & 19.43 & 0.48 & 18.49 & 20.37 & \ipair{mpyr}{1}{cl} (p1) & \ipair{mpyr}{4}{cl} (p2) \\ 
%   Hal & pyr & Disp & 11.84 & 11.59 & 0.90 & 10.36 & 13.35 & \ipair{mpyr}{1}{cl} (p2) & \ipair{mpyr}{4}{br} (p1) \\ 
%   TILA & im & Elec & 101.28 & 101.20 & 1.52 & 97.96 & 104.53 & \ipair{mim}{3}{ntf} (p3) & \ipair{mim}{1}{dca} (p1) \\ 
%   TILA & im & Exch.Repl & 30.40 & -30.91 & 3.28 & -36.45 & -24.18 & \ipair{mim}{3}{ntf} (p1) & \ipair{mim}{1}{bfl} (p1) \\ 
%   TILA & im & Ind.Pol & 12.82 & 11.95 & 2.42 & 9.84 & 19.50 & \ipair{mim}{2}{dca} (p6) & \ipair{mim}{1}{ntf} (p2) \\ 
%   TILA & im & Disp & 16.30 & 16.76 & 2.83 & 11.54 & 24.73 & \ipair{mim}{1}{bfl} (p1) & \ipair{mim}{3}{ntf} (p1) \\ 
%   TILA & pyr & Elec & 97.68 & 98.38 & 2.43 & 91.72 & 102.65 & \ipair{mpyr}{3}{ntf} (p1) & \ipair{mpyr}{1}{mes} (p2) \\ 
%   TILA & pyr & Exch.Repl & 24.03 & -22.98 & 2.88 & -31.98 & -19.81 & \ipair{mpyr}{1}{mes} (p2) & \ipair{mpyr}{2}{ntf} (p1) \\ 
%   TILA & pyr & Ind.Pol & 13.71 & 13.15 & 1.93 & 10.88 & 17.97 & \ipair{mpyr}{1}{dca} (p4) & \ipair{mpyr}{4}{mes} (p3) \\ 
%   TILA & pyr & Disp & 12.63 & 12.81 & 1.63 & 9.47 & 16.17 & \ipair{mpyr}{1}{bfl} (p2) & \ipair{mpyr}{3}{ntf} (p1) \\ 
%    \hline
% \end{tabular}
% \caption{Statistics on raw ratios as percentages, SAPT(aug-cc-pVDZ)}
% \end{table}
% 
% \begin{table}[ht]
% \centering
% \begin{tabular}{llllrrrrrll}
%   \hline
% Basis & Halide & Cation & Energy & mean & med & sd & min & max & min.name & max.name \\ 
%   \hline
% atz & Hal & im & Elec & 110.98 & 111.56 & 8.82 & 91.05 & 122.44 & 3-im-br-p2 & 2-im-br-p1 \\ 
%   atz & Hal & im & Exch.Repl & 47.65 & -47.98 & 7.32 & -58.95 & -34.47 & 3-im-br-p4 & 3-im-br-p2 \\ 
%   atz & Hal & im & Ind.Pol & 24.69 & 25.23 & 5.10 & 17.57 & 35.18 & 1-im-cl-p1 & 3-im-br-p2 \\ 
%   atz & Hal & im & Disp & 11.98 & 11.96 & 3.03 & 7.85 & 17.18 & 1-im-br-p2 & 4-im-br-p4 \\ 
%   atz & Hal & pyr & Elec & 105.75 & 105.84 & 2.61 & 100.87 & 110.03 & 3-pyr-cl-p2 & 2-pyr-br-p2 \\ 
%   atz & Hal & pyr & Exch.Repl & 39.19 & -39.32 & 3.99 & -45.40 & -33.72 & 2-pyr-br-p2 & 2-pyr-cl-p1 \\ 
%   atz & Hal & pyr & Ind.Pol & 22.95 & 22.93 & 1.28 & 20.67 & 24.92 & 1-pyr-cl-p2 & 4-pyr-br-p3 \\ 
%   atz & Hal & pyr & Disp & 10.50 & 10.48 & 0.87 & 9.10 & 12.21 & 1-pyr-cl-p2 & 4-pyr-br-p1 \\ 
%   atz & TILA & im & Elec & 94.50 & 94.06 & 2.93 & 89.06 & 100.53 & 4-im-tos-p2 & 1-im-dca-p1 \\ 
%   atz & TILA & im & Exch.Repl & 24.52 & -24.47 & 4.34 & -30.95 & -17.36 & 2-im-dca-p3 & 1-im-bf4-p1 \\ 
%   atz & TILA & im & Ind.Pol & 10.98 & 10.28 & 2.26 & 8.02 & 16.01 & 1-im-pf6-p1 & 1-im-ntf2-p2 \\ 
%   atz & TILA & im & Disp & 19.04 & 18.90 & 3.63 & 13.28 & 28.98 & 2-im-bf4-p2 & 3-im-ntf2-p1 \\ 
%   atz & TILA & pyr & Elec & 92.16 & 92.67 & 2.77 & 86.76 & 97.14 & 4-pyr-ntf2-p1 & 1-pyr-dca-p4 \\ 
%   atz & TILA & pyr & Exch.Repl & 18.33 & -17.38 & 2.48 & -23.77 & -14.97 & 1-pyr-mes-p2 & 2-pyr-ntf2-p1 \\ 
%   atz & TILA & pyr & Ind.Pol & 12.22 & 11.40 & 2.05 & 9.84 & 17.10 & 1-pyr-pf6-p1 & 4-pyr-mes-p3 \\ 
%   atz & TILA & pyr & Disp & 13.95 & 14.38 & 2.12 & 9.80 & 18.20 & 1-pyr-bf4-p2 & 4-pyr-ntf2-p1 \\ 
%    \hline
% \end{tabular}
% \caption{Statistics on raw ratios as percentages, EFP(aug-cc-pVTZ)}
% \end{table}
% 
% \begin{table}[ht]
% \centering
% \begin{tabular}{llllrrrrrll}
%   \hline
% Basis & Halide & Cation & Energy & mean & med & sd & min & max & min.name & max.name \\ 
%   \hline
% adz & Hal & im & Elec & 114.45 & 116.65 & 6.57 & 101.78 & 127.27 & 3-im-cl-p2 & 3-im-br-p4 \\ 
%   adz & Hal & im & Exch.Repl & 46.02 & -47.26 & 6.95 & -60.39 & -34.94 & 3-im-br-p4 & 1-im-br-p2 \\ 
%   adz & Hal & im & Ind.Pol & 20.68 & 20.37 & 2.37 & 17.70 & 26.08 & 2-im-cl-p1 & 3-im-cl-p2 \\ 
%   adz & Hal & im & Disp & 10.89 & 11.29 & 2.65 & 6.96 & 14.98 & 1-im-br-p2 & 3-im-br-p4 \\ 
%   adz & Hal & pyr & Elec & 109.29 & 109.33 & 3.27 & 103.50 & 115.56 & 4-pyr-cl-p2 & 2-pyr-br-p1 \\ 
%   adz & Hal & pyr & Exch.Repl & 36.07 & -35.04 & 4.14 & -44.40 & -30.02 & 2-pyr-br-p1 & 4-pyr-cl-p2 \\ 
%   adz & Hal & pyr & Ind.Pol & 17.66 & 17.70 & 0.87 & 15.87 & 19.29 & 2-pyr-cl-p2 & 3-pyr-br-p3 \\ 
%   adz & Hal & pyr & Disp & 9.12 & 9.02 & 0.77 & 7.98 & 10.63 & 1-pyr-cl-p2 & 2-pyr-br-p1 \\ 
%   adz & TILA & im & Elec & 95.32 & 95.46 & 1.97 & 90.20 & 101.50 & 3-im-ntf2-p4 & 1-im-ntf2-p2 \\ 
%   adz & TILA & im & Exch.Repl & 24.36 & -24.92 & 3.16 & -31.21 & -18.59 & 1-im-ntf2-p2 & 2-im-bf4-p2 \\ 
%   adz & TILA & im & Ind.Pol & 10.18 & 9.58 & 1.77 & 7.35 & 14.27 & 3-im-dca-p5 & 1-im-ntf2-p2 \\ 
%   adz & TILA & im & Disp & 18.86 & 18.61 & 3.69 & 13.39 & 28.68 & 1-im-bf4-p1 & 3-im-ntf2-p1 \\ 
%   adz & TILA & pyr & Elec & 93.89 & 94.54 & 2.62 & 87.72 & 98.06 & 3-pyr-ntf2-p1 & 1-pyr-dca-p4 \\ 
%   adz & TILA & pyr & Exch.Repl & 19.12 & -18.92 & 1.79 & -23.16 & -16.18 & 2-pyr-tos-p1 & 1-pyr-pf6-p2 \\ 
%   adz & TILA & pyr & Ind.Pol & 11.28 & 10.79 & 1.79 & 8.54 & 15.54 & 1-pyr-dca-p4 & 2-pyr-tos-p1 \\ 
%   adz & TILA & pyr & Disp & 13.96 & 13.96 & 2.47 & 9.76 & 19.56 & 1-pyr-bf4-p2 & 3-pyr-ntf2-p1 \\ 
%    \hline
% \end{tabular}
% \caption{Statistics on raw ratios as percentages, EFP(aug-cc-pVDZ)}
% \end{table}
% 
% \begin{table}[ht]
% \centering
% \begin{tabular}{llllrrrrrll}
%   \hline
% Basis & Halide & Cation & Energy & mean & med & sd & min & max & min.name & max.name \\ 
%   \hline
% pop & Hal & im & Elec & 120.89 & 119.73 & 6.69 & 108.52 & 132.37 & 3-im-br-p3 & 3-im-cl-p4 \\ 
%   pop & Hal & im & Exch.Repl & 45.75 & -46.01 & 8.62 & -59.30 & -30.50 & 4-im-cl-p2 & 3-im-br-p3 \\ 
%   pop & Hal & im & Ind.Pol & 18.01 & 17.86 & 2.83 & 14.01 & 25.14 & 1-im-br-p1 & 3-im-cl-p2 \\ 
%   pop & Hal & im & Disp & 6.85 & 7.25 & 1.46 & 4.15 & 8.68 & 1-im-br-p2 & 3-im-cl-p4 \\ 
%   pop & Hal & pyr & Elec & 110.85 & 110.43 & 3.05 & 106.39 & 115.40 & 3-pyr-br-p1 & 2-pyr-cl-p3 \\ 
%   pop & Hal & pyr & Exch.Repl & 33.00 & -32.97 & 4.50 & -38.97 & -26.90 & 3-pyr-cl-p1 & 3-pyr-br-p2 \\ 
%   pop & Hal & pyr & Ind.Pol & 16.39 & 16.34 & 1.76 & 13.79 & 19.95 & 1-pyr-br-p2 & 4-pyr-cl-p1 \\ 
%   pop & Hal & pyr & Disp & 5.76 & 5.77 & 0.30 & 5.11 & 6.33 & 1-pyr-br-p2 & 3-pyr-cl-p1 \\ 
%   pop & TILA & im & Elec & 94.24 & 94.10 & 3.94 & 84.79 & 100.55 & 3-im-ntf2-p4 & 1-im-dca-p2 \\ 
%   pop & TILA & im & Exch.Repl & 20.15 & -19.16 & 4.78 & -26.86 & -13.74 & 2-im-dca-p1 & 2-im-bf4-p2 \\ 
%   pop & TILA & im & Ind.Pol & 9.55 & 9.10 & 1.82 & 7.35 & 13.95 & 4-im-dca-p3 & 2-im-ntf2-p2 \\ 
%   pop & TILA & im & Disp & 16.36 & 16.26 & 3.87 & 10.61 & 24.02 & 1-im-bf4-p1 & 3-im-ntf2-p1 \\ 
%   pop & TILA & pyr & Elec & 92.03 & 93.13 & 3.76 & 84.49 & 97.65 & 4-pyr-ntf2-p1 & 1-pyr-dca-p4 \\ 
%   pop & TILA & pyr & Exch.Repl & 14.27 & -13.29 & 2.98 & -19.87 & -9.95 & 1-pyr-mes-p2 & 3-pyr-ntf2-p1 \\ 
%   pop & TILA & pyr & Ind.Pol & 10.34 & 9.88 & 1.55 & 8.55 & 14.23 & 1-pyr-dca-p4 & 1-pyr-mes-p2 \\ 
%   pop & TILA & pyr & Disp & 11.90 & 12.59 & 2.36 & 7.77 & 15.91 & 1-pyr-pf6-p2 & 4-pyr-ntf2-p1 \\ 
%    \hline
% \end{tabular}
% \caption{Statistics on raw ratios as percentages, EFP(6-311++G(d,p))}
% \end{table}
% 
% % Tue Aug 25 22:32:30 2015
% \begin{table}[ht]
% \centering
% \begin{tabular}{llllrrrrr}
%   \hline \\
% Basis & Halide & Cation & Energy & mean & med & sd & min & max \\ 
%   \hline \\
% atz & TILA & im & Elec & 6.77 & 6.68 & 2.13 & 1.32 & 10.80 \\ 
%   atz & TILA & im & Exch.Repl & 5.95 & -6.93 & 3.04 & -10.29 & 1.33 \\ 
%   atz & TILA & im & Ind.Pol & 2.96 & 2.83 & 0.92 & 1.60 & 6.55 \\ 
%   atz & TILA & im & Disp & 2.73 & -2.36 & 1.30 & -5.23 & -0.07 \\ 
%   atz & TILA & im & CT & 0.90 & 0.81 & 0.47 & -0.14 & 1.68 \\ 
%   atz & Hal & im & Elec & 5.13 & -3.19 & 6.04 & -7.72 & 14.30 \\ 
%   atz & Hal & im & Exch.Repl & 4.84 & 4.03 & 4.76 & -6.85 & 9.09 \\ 
%   atz & Hal & im & Ind.Pol & 2.14 & 1.64 & 2.31 & -5.79 & 3.56 \\ 
%   atz & Hal & im & Disp & 1.08 & 0.72 & 1.16 & -1.26 & 2.95 \\ 
%   atz & Hal & im & CT & 7.63 & 3.79 & 5.02 & 1.89 & 15.27 \\ 
%   atz & TILA & pyr & Elec & 5.11 & 4.89 & 1.67 & 2.03 & 7.95 \\ 
%   atz & TILA & pyr & Exch.Repl & 5.59 & -6.08 & 2.20 & -8.30 & 0.70 \\ 
%   atz & TILA & pyr & Ind.Pol & 2.86 & 2.94 & 0.40 & 1.48 & 3.59 \\ 
%   atz & TILA & pyr & Disp & 1.04 & -0.93 & 0.57 & -2.53 & -0.13 \\ 
%   atz & TILA & pyr & CT & 0.62 & 0.72 & 0.42 & -0.39 & 1.48 \\ 
%   atz & Hal & pyr & Elec & 1.46 & 0.48 & 1.79 & -2.18 & 4.12 \\ 
%   atz & Hal & pyr & Exch.Repl & 2.15 & 1.85 & 2.10 & -1.61 & 5.28 \\ 
%   atz & Hal & pyr & Ind.Pol & 0.81 & -0.57 & 0.83 & -2.20 & 0.93 \\ 
%   atz & Hal & pyr & Disp & 1.34 & 1.34 & 0.26 & 0.83 & 1.79 \\ 
%   atz & Hal & pyr & CT & 3.73 & 3.61 & 0.78 & 2.47 & 5.11 \\ 
%    \hline
% \end{tabular}
% \caption{Statistics on differences between ratios: EFP(aug-cc-pVTZ)}
% \end{table}
% 
% 
% 
% \begin{table}[ht]
% \centering
% \begin{tabular}{llllrrrrr}
%   \hline
% Basis & Halide & Cation & Energy & mean & med & sd & min & max \\ 
%   \hline
% adz & TILA & im & Elec & 6.00 & 5.98 & 1.80 & -0.71 & 9.25 \\ 
%   adz & TILA & im & Exch.Repl & 5.99 & -5.68 & 1.70 & -10.23 & -1.42 \\ 
%   adz & TILA & im & Ind.Pol & 3.19 & 2.76 & 1.11 & 1.88 & 7.21 \\ 
%   adz & TILA & im & Disp & 2.60 & -2.30 & 1.31 & -5.39 & 0.43 \\ 
%   adz & TILA & im & CT & 1.40 & 1.27 & 0.39 & 0.59 & 2.27 \\ 
%   adz & Hal & im & Elec & 5.33 & -4.75 & 4.26 & -13.07 & 3.73 \\ 
%   adz & Hal & im & Exch.Repl & 3.73 & 2.76 & 4.18 & -7.13 & 10.54 \\ 
%   adz & Hal & im & Ind.Pol & 2.53 & 2.36 & 1.44 & 0.14 & 6.11 \\ 
%   adz & Hal & im & Disp & 1.91 & 1.97 & 0.76 & 0.77 & 3.16 \\ 
%   adz & Hal & im & CT & 10.06 & 9.24 & 3.91 & 5.72 & 15.88 \\ 
%   adz & TILA & pyr & Elec & 3.42 & 3.35 & 1.38 & 0.75 & 6.59 \\ 
%   adz & TILA & pyr & Exch.Repl & 4.86 & -4.86 & 1.92 & -9.73 & -0.16 \\ 
%   adz & TILA & pyr & Ind.Pol & 2.85 & 2.74 & 0.46 & 1.80 & 4.17 \\ 
%   adz & TILA & pyr & Disp & 1.06 & -1.00 & 0.87 & -3.49 & 0.99 \\ 
%   adz & TILA & pyr & CT & 1.53 & 1.40 & 0.45 & 0.73 & 2.78 \\ 
%   adz & Hal & pyr & Elec & 3.02 & -3.06 & 2.30 & -7.84 & 0.85 \\ 
%   adz & Hal & pyr & Exch.Repl & 2.39 & -1.16 & 2.62 & -6.08 & 4.91 \\ 
%   adz & Hal & pyr & Ind.Pol & 1.57 & 1.60 & 0.75 & 0.11 & 3.31 \\ 
%   adz & Hal & pyr & Disp & 2.72 & 2.68 & 0.38 & 2.06 & 3.51 \\ 
%   adz & Hal & pyr & CT & 6.86 & 6.77 & 1.04 & 5.64 & 8.21 \\ 
%    \hline
% \end{tabular}
% \caption{Statistics on differences between ratios: EFP(aug-cc-pVDZ)}
% \end{table}
% 
% \begin{table}[ht]
% \centering
% \begin{tabular}{llllrrrrr}
%   \hline
% Basis & Halide & Cation & Energy & mean & med & sd & min & max \\ 
%   \hline
% pop & TILA & im & Elec & 7.16 & 6.74 & 3.06 & 0.24 & 14.66 \\ 
%   pop & TILA & im & Exch.Repl & 9.95 & -10.89 & 4.02 & -16.59 & -2.65 \\ 
%   pop & TILA & im & Ind.Pol & 3.72 & 3.32 & 1.13 & 2.69 & 9.03 \\ 
%   pop & TILA & im & Disp & 1.30 & 0.30 & 1.49 & -2.86 & 1.98 \\ 
%   pop & TILA & im & CT & 1.15 & 1.11 & 0.34 & 0.53 & 2.02 \\ 
%   pop & Hal & im & Elec & 11.10 & -11.78 & 6.10 & -25.09 & -1.51 \\ 
%   pop & Hal & im & Exch.Repl & 7.22 & 1.41 & 8.15 & -11.36 & 16.86 \\ 
%   pop & Hal & im & Ind.Pol & 6.27 & 6.34 & 2.35 & 2.03 & 10.17 \\ 
%   pop & Hal & im & Disp & 5.95 & 5.89 & 1.10 & 4.29 & 8.03 \\ 
%   pop & Hal & im & CT & 9.00 & 8.22 & 4.10 & 4.65 & 15.30 \\ 
%   pop & TILA & pyr & Elec & 5.15 & 5.46 & 1.91 & 1.28 & 8.11 \\ 
%   pop & TILA & pyr & Exch.Repl & 9.39 & -10.34 & 2.77 & -13.20 & -2.76 \\ 
%   pop & TILA & pyr & Ind.Pol & 4.06 & 4.06 & 0.75 & 2.39 & 6.36 \\ 
%   pop & TILA & pyr & Disp & 1.06 & 1.02 & 0.84 & -1.08 & 2.32 \\ 
%   pop & TILA & pyr & CT & 1.11 & 1.23 & 0.38 & 0.46 & 1.88 \\ 
%   pop & Hal & pyr & Elec & 4.71 & -4.03 & 4.12 & -10.42 & 1.38 \\ 
%   pop & Hal & pyr & Exch.Repl & 6.33 & -4.51 & 6.54 & -12.76 & 3.76 \\ 
%   pop & Hal & pyr & Ind.Pol & 4.61 & 4.64 & 1.83 & 0.81 & 6.76 \\ 
%   pop & Hal & pyr & Disp & 6.08 & 5.99 & 0.99 & 4.89 & 7.27 \\ 
%   pop & Hal & pyr & CT & 5.09 & 5.12 & 0.98 & 3.88 & 6.50 \\ 
%    \hline
% \end{tabular}
% \caption{Statistics on differences between ratios: EFP(6-311++G(d,p))}
% \end{table}
% 
% 
% Using the same methods for ratios, except comparing sum of polarization and charge-transfer with induction from SAPT.
% 
% \begin{table}[ht]
% \centering
% \begin{tabular}{llllrrrrrll}
% 
%   \hline
% Basis & Halide & Cation & Energy & mean & med & sd & min & max & min.name & max.name \\ 
%  \hline
%   atz & Hal & im & Elec & 8.10 & 2.21 & 10.32 & -24.93 & 14.79 & 3-im-br-p2 & 1-im-cl-p1 \\ 
%   atz & Hal & im & Ind.Pol & 8.69 & -4.16 & 10.69 & -17.66 & 25.80 & 1-im-cl-p1 & 3-im-br-p2 \\ 
%   atz & Hal & im & Disp & 4.34 & -1.30 & 5.04 & -10.21 & 7.68 & 3-im-br-p2 & 4-im-br-p4 \\ 
%   atz & Hal & pyr & Elec & 3.25 & -0.14 & 3.84 & -5.97 & 6.39 & 3-pyr-br-p1 & 1-pyr-cl-p2 \\ 
%   atz & Hal & pyr & Ind.Pol & 2.91 & 0.37 & 3.43 & -5.74 & 5.52 & 3-pyr-cl-p1 & 2-pyr-br-p1 \\ 
%   atz & Hal & pyr & Disp & 0.87 & -0.43 & 1.06 & -1.75 & 2.03 & 2-pyr-br-p3 & 3-pyr-cl-p1 \\ 
%   atz & TILA & im & Elec & 5.69 & 1.15 & 6.89 & -12.13 & 14.62 & 2-im-dca-p1 & 1-im-ntf2-p2 \\ 
%   atz & TILA & im & Ind.Pol & 2.73 & 0.18 & 3.30 & -5.29 & 7.20 & 2-im-ntf2-p1 & 3-im-tos-p2 \\ 
%   atz & TILA & im & Disp & 2.64 & -0.50 & 3.55 & -8.29 & 6.97 & 3-im-ntf2-p2 & 3-im-dca-p6 \\ 
%   atz & TILA & pyr & Elec & 2.97 & 0.62 & 3.62 & -10.90 & 4.75 & 4-pyr-dca-p1 & 2-pyr-ntf2-p2 \\ 
%   atz & TILA & pyr & Ind.Pol & 2.42 & -1.36 & 2.65 & -3.92 & 5.61 & 1-pyr-ntf2-p1 & 4-pyr-dca-p1 \\ 
%   atz & TILA & pyr & Disp & 1.38 & -0.59 & 1.67 & -2.36 & 4.49 & 2-pyr-bf4-p3 & 1-pyr-dca-p3 \\ 
%    \hline
% \end{tabular}
% \caption{Statistics on differences between ratio-scaled energies (Elec + Exch.Repl): EFP(aug-cc-pVTZ)}
% \end{table}
% 
% \begin{table}[ht]
% \centering
% \begin{tabular}{llllrrrrrll}
%   \hline
% Basis & Halide & Cation & Energy & mean & med & sd & min & max & min.name & max.name \\ 
%   \hline
% adz & Hal & im & Elec & 4.15 & -0.60 & 5.49 & -9.13 & 14.01 & 2-im-br-p4 & 4-im-cl-p2 \\ 
%   adz & im & Ind.Pol & 4.47 & -1.22 & 5.24 & -12.43 & 7.57 & 4-im-cl-p2 & 2-im-br-p3 \\ 
%   adz & Hal & im & Disp & 4.16 & -0.69 & 4.53 & -7.96 & 6.14 & 3-im-cl-p2 & 3-im-br-p4 \\ 
%   adz & Hal & pyr & Elec & 2.24 & 0.49 & 2.92 & -5.14 & 6.01 & 2-pyr-br-p1 & 2-pyr-cl-p2 \\ 
%   adz & Hal & pyr & Ind.Pol & 2.54 & -0.87 & 3.31 & -8.35 & 6.45 & 2-pyr-cl-p2 & 3-pyr-br-p1 \\ 
%   adz & Hal & pyr & Disp & 1.22 & -0.13 & 1.54 & -2.64 & 3.49 & 4-pyr-br-p2 & 2-pyr-br-p1 \\ 
%   adz & TILA & im & Elec & 5.74 & -0.28 & 7.35 & -11.74 & 23.72 & 3-im-dca-p6 & 4-im-tos-p2 \\ 
%   adz & TILA & im & Ind.Pol & 2.49 & 0.79 & 3.05 & -10.58 & 6.52 & 4-im-tos-p2 & 3-im-mes-p1 \\ 
%   adz & TILA & im & Disp & 2.84 & -0.24 & 3.71 & -11.54 & 6.38 & 4-im-tos-p2 & 4-im-dca-p3 \\ 
%   adz & TILA & pyr & Elec & 2.81 & 0.29 & 3.84 & -9.34 & 12.73 & 2-pyr-ntf2-p1 & 4-pyr-tos-p2 \\ 
%   adz & TILA & pyr & Ind.Pol & 1.51 & -0.42 & 1.99 & -3.68 & 7.05 & 4-pyr-tos-p2 & 2-pyr-tos-p1 \\ 
%   adz & TILA & pyr & Disp & 1.89 & -0.33 & 2.51 & -7.59 & 6.62 & 4-pyr-tos-p2 & 2-pyr-ntf2-p1 \\ 
%    \hline
% \end{tabular}
% \caption{Statistics on differences between ratio-scaled energies (Elec + Exch.Repl): EFP(aug-cc-pVDZ)}
% \end{table}
% 
% 
% \begin{table}[ht]
% \centering
% \begin{tabular}{llllrrrrrll}
%   \hline
% Basis & Halide & Cation & Energy & mean & med & sd & min & max & min.name & max.name \\ 
%   \hline
% pop & Hal & im & Elec & 8.41 & -0.03 & 9.26 & -14.66 & 11.36 & 1-im-cl-p1 & 2-im-br-p2 \\ 
%   pop & Hal & im & Ind.Pol & 8.01 & -1.26 & 9.25 & -13.76 & 17.12 & 2-im-br-p2 & 3-im-cl-p2 \\ 
%   pop & Hal & im & Disp & 5.12 & -1.98 & 5.90 & -9.93 & 8.68 & 2-im-br-p2 & 3-im-cl-p1 \\ 
%   pop & Hal & pyr & Elec & 9.00 & -0.34 & 9.60 & -15.03 & 11.24 & 4-pyr-cl-p1 & 3-pyr-br-p2 \\ 
%   pop & Hal & pyr & Ind.Pol & 7.16 & -1.11 & 7.93 & -9.97 & 15.18 & 1-pyr-br-p2 & 4-pyr-cl-p1 \\ 
%   pop & Hal & pyr & Disp & 4.20 & -0.03 & 4.39 & -5.61 & 6.18 & 3-pyr-br-p2 & 3-pyr-cl-p1 \\ 
%   pop & TILA & im & Elec & 6.17 & 2.44 & 7.42 & -11.68 & 19.71 & 3-im-dca-p6 & 1-im-ntf2-p2 \\ 
%   pop & TILA & im & Ind.Pol & 2.20 & -0.55 & 2.87 & -9.99 & 4.79 & 1-im-ntf2-p2 & 2-im-mes-p1 \\ 
%   pop & TILA & im & Disp & 4.73 & -2.21 & 5.17 & -8.09 & 8.77 & 4-im-pf6-p1 & 3-im-dca-p5 \\ 
%   pop & TILA & pyr & Elec & 3.75 & 0.20 & 4.65 & -12.10 & 10.79 & 1-pyr-dca-p3 & 3-pyr-tos-p2 \\ 
%   pop & TILA & pyr & Ind.Pol & 2.29 & -0.54 & 2.72 & -6.88 & 7.25 & 3-pyr-tos-p2 & 1-pyr-dca-p3 \\ 
%   pop & TILA & pyr & Disp & 2.89 & -0.45 & 3.33 & -5.73 & 6.69 & 2-pyr-pf6-p3 & 3-pyr-dca-p3 \\ 
%    \hline
% \end{tabular}
% \caption{Statistics on differences between ratio-scaled energies (Elec + Exch.Repl): EFP(6-311++G(d,p))}
% \end{table}
% 
% \begin{table}[ht]
% \centering
% \begin{tabular}{llllrrrrrll}
%   \hline
% Basis & Halide & Cation & Energy & mean & med & sd & min & max & min.name & max.name \\ 
%   \hline
% atz & Hal & im & Elec & 17.31 & 5.93 & 23.39 & -60.89 & 22.81 & 3-im-br-p2 & 2-im-br-p1 \\ 
%   atz & Hal & im & Exch.Repl & 13.93 & -2.21 & 16.90 & -18.77 & 38.95 & 3-im-br-p4 & 3-im-cl-p2 \\ 
%   atz & Hal & im & Ind.Pol & 8.69 & -4.16 & 10.69 & -17.66 & 25.80 & 1-im-cl-p1 & 3-im-br-p2 \\ 
%   atz & Hal & im & Disp & 4.34 & -1.30 & 5.04 & -10.21 & 7.68 & 3-im-br-p2 & 4-im-br-p4 \\ 
%   atz & Hal & pyr & Elec & 5.49 & 0.35 & 6.88 & -14.17 & 10.15 & 2-pyr-cl-p1 & 2-pyr-br-p2 \\ 
%   atz & Hal & pyr & Exch.Repl & 5.90 & -0.28 & 7.33 & -11.57 & 12.36 & 2-pyr-br-p2 & 3-pyr-cl-p3 \\ 
%   atz & Hal & pyr & Ind.Pol & 2.91 & 0.37 & 3.43 & -5.74 & 5.52 & 3-pyr-cl-p1 & 2-pyr-br-p1 \\ 
%   atz & Hal & pyr & Disp & 0.87 & -0.43 & 1.06 & -1.75 & 2.03 & 2-pyr-br-p3 & 3-pyr-cl-p1 \\ 
%   atz & TILA & im & Elec & 7.05 & 0.10 & 8.90 & -18.38 & 19.74 & 3-im-tos-p2 & 1-im-ntf2-p2 \\ 
%   atz & TILA & im & Exch.Repl & 13.04 & 9.94 & 14.07 & -29.72 & 20.81 & 2-im-dca-p1 & 4-im-tos-p1 \\ 
%   atz & TILA & im & Ind.Pol & 2.73 & 0.18 & 3.30 & -5.29 & 7.20 & 2-im-ntf2-p1 & 3-im-tos-p2 \\ 
%   atz & TILA & im & Disp & 2.64 & -0.50 & 3.55 & -8.29 & 6.97 & 3-im-ntf2-p2 & 3-im-dca-p6 \\ 
%   atz & TILA & pyr & Elec & 4.97 & 0.50 & 6.42 & -12.02 & 10.90 & 2-pyr-tos-p1 & 4-pyr-dca-p1 \\ 
%   atz & TILA & pyr & Exch.Repl & 7.39 & 4.63 & 8.75 & -23.49 & 10.11 & 4-pyr-dca-p1 & 1-pyr-bf4-p1 \\ 
%   atz & TILA & pyr & Ind.Pol & 2.42 & -1.36 & 2.65 & -3.92 & 5.61 & 1-pyr-ntf2-p1 & 4-pyr-dca-p1 \\ 
%   atz & TILA & pyr & Disp & 1.38 & -0.59 & 1.67 & -2.36 & 4.49 & 2-pyr-bf4-p3 & 1-pyr-dca-p3 \\ 
%    \hline
% \end{tabular}
% \caption{Statistics on differences between ratio-scaled energies: EFP(aug-cc-pVTZ)}
% \end{table}
% 
% \begin{table}[ht]
% \centering
% \begin{tabular}{llllrrrrrll}
%   \hline
% Basis & Halide & Cation & Energy & mean & med & sd & min & max & min.name & max.name \\ 
%   \hline
% adz & Hal & im & Elec & 12.20 & -0.60 & 16.06 & -32.78 & 31.24 & 3-im-cl-p2 & 3-im-br-p4 \\ 
%   adz & Hal & im & Exch.Repl & 12.96 & -2.93 & 15.71 & -33.26 & 34.71 & 3-im-br-p4 & 3-im-cl-p2 \\ 
%   adz & Hal & im & Ind.Pol & 4.47 & -1.22 & 5.24 & -12.43 & 7.57 & 4-im-cl-p2 & 2-im-br-p3 \\ 
%   adz & Hal & im & Disp & 4.16 & -0.69 & 4.53 & -7.96 & 6.14 & 3-im-cl-p2 & 3-im-br-p4 \\ 
%   adz & Hal & pyr & Elec & 6.98 & 0.34 & 8.44 & -13.92 & 17.86 & 4-pyr-cl-p2 & 2-pyr-br-p1 \\ 
%   adz & Hal & pyr & Exch.Repl & 8.14 & 0.02 & 10.46 & -24.35 & 19.22 & 2-pyr-br-p1 & 4-pyr-cl-p2 \\ 
%   adz & Hal & pyr & Ind.Pol & 2.54 & -0.87 & 3.31 & -8.35 & 6.45 & 2-pyr-cl-p2 & 3-pyr-br-p1 \\ 
%   adz & Hal & pyr & Disp & 1.22 & -0.13 & 1.54 & -2.64 & 3.49 & 4-pyr-br-p2 & 2-pyr-br-p1 \\ 
%   adz & TILA & im & Elec & 5.06 & -0.31 & 6.96 & -12.41 & 24.20 & 3-im-ntf2-p4 & 1-im-ntf2-p2 \\ 
%   adz & TILA & im & Exch.Repl & 6.20 & 1.72 & 7.64 & -19.48 & 19.84 & 2-im-dca-p1 & 4-im-tos-p2 \\ 
%   adz & TILA & im & Ind.Pol & 2.49 & 0.79 & 3.05 & -10.58 & 6.52 & 4-im-tos-p2 & 3-im-mes-p1 \\ 
%   adz & TILA & im & Disp & 2.84 & -0.24 & 3.71 & -11.54 & 6.38 & 4-im-tos-p2 & 4-im-dca-p3 \\ 
%   adz & TILA & pyr & Elec & 3.81 & 0.13 & 5.21 & -12.26 & 9.59 & 2-pyr-mes-p1 & 3-pyr-dca-p1 \\ 
%   adz & TILA & pyr & Exch.Repl & 5.66 & 1.12 & 6.75 & -17.03 & 16.06 & 4-pyr-dca-p1 & 1-pyr-mes-p2 \\ 
%   adz & TILA & pyr & Ind.Pol & 1.51 & -0.42 & 1.99 & -3.68 & 7.05 & 4-pyr-tos-p2 & 2-pyr-tos-p1 \\ 
%   adz & TILA & pyr & Disp & 1.89 & -0.33 & 2.51 & -7.59 & 6.62 & 4-pyr-tos-p2 & 2-pyr-ntf2-p1 \\ 
%    \hline
% \end{tabular}
% \caption{Statistics on differences between ratio-scaled energies: EFP(aug-cc-pVDZ)}
% \end{table}
% 
% 
% \begin{table}[ht]
% \centering
% \caption{Statistics on differences between ratio-scaled energies: EFP(6-311++G(d,p))}
% \begin{tabular}{llllrrrrrll}
%   \hline
% Basis & Halide & Cation & Energy & mean & med & sd & min & max & min.name & max.name \\ 
%   \hline
% pop & Hal & im & Elec & 19.07 & 2.36 & 22.47 & -33.42 & 52.59 & 3-im-br-p3 & 4-im-cl-p2 \\ 
%   pop & Hal & im & Exch.Repl & 27.44 & 2.91 & 31.14 & -56.70 & 49.66 & 4-im-cl-p2 & 3-im-br-p3 \\ 
%   pop & Hal & im & Ind.Pol & 8.01 & -1.26 & 9.25 & -13.76 & 17.12 & 2-im-br-p2 & 3-im-cl-p2 \\ 
%   pop & Hal & im & Disp & 5.12 & -1.98 & 5.90 & -9.93 & 8.68 & 2-im-br-p2 & 3-im-cl-p1 \\ 
%   pop & Hal & pyr & Elec & 13.58 & -1.94 & 15.26 & -22.24 & 22.10 & 1-pyr-br-p1 & 3-pyr-cl-p1 \\ 
%   pop & Hal & pyr & Exch.Repl & 25.42 & 3.21 & 26.72 & -31.91 & 35.44 & 3-pyr-cl-p1 & 3-pyr-br-p2 \\ 
%   pop & Hal & pyr & Ind.Pol & 7.16 & -1.11 & 7.93 & -9.97 & 15.18 & 1-pyr-br-p2 & 4-pyr-cl-p1 \\ 
%   pop & Hal & pyr & Disp & 4.20 & -0.03 & 4.39 & -5.61 & 6.18 & 3-pyr-br-p2 & 3-pyr-cl-p1 \\ 
%   pop & TILA & im & Elec & 9.43 & 0.89 & 12.23 & -28.53 & 24.93 & 3-im-ntf2-p4 & 1-im-ntf2-p2 \\ 
%   pop & TILA & im & Exch.Repl & 20.46 & 16.76 & 20.90 & -33.22 & 31.89 & 2-im-dca-p1 & 3-im-ntf2-p4 \\ 
%   pop & TILA & im & Ind.Pol & 2.20 & -0.55 & 2.87 & -9.99 & 4.79 & 1-im-ntf2-p2 & 2-im-mes-p1 \\ 
%   pop & TILA & im & Disp & 4.73 & -2.21 & 5.17 & -8.09 & 8.77 & 4-im-pf6-p1 & 3-im-dca-p5 \\ 
%   pop & TILA & pyr & Elec & 6.03 & -1.16 & 7.31 & -12.64 & 13.61 & 2-pyr-mes-p1 & 4-pyr-dca-p1 \\ 
%   pop & TILA & pyr & Exch.Repl & 12.61 & 8.64 & 13.83 & -27.03 & 16.59 & 4-pyr-dca-p1 & 3-pyr-tos-p1 \\ 
%   pop & TILA & pyr & Ind.Pol & 2.29 & -0.54 & 2.72 & -6.88 & 7.25 & 3-pyr-tos-p2 & 1-pyr-dca-p3 \\ 
%   pop & TILA & pyr & Disp & 2.89 & -0.45 & 3.33 & -5.73 & 6.69 & 2-pyr-pf6-p3 & 3-pyr-dca-p3 \\ 
%    \hline
% \end{tabular}
% \end{table}


\begin{table}[h]
\centering
\footnotesize
\caption{Comparison of full and 1st order electrostatics in SAPT with electrostatics from EFP}
\label{tab:trunc-elst-stats}
\begin{tabular}{llllrrrr}
\hline
 order    & Basis                 & Cation               & Halide  &  MAE    & SD     & Max     &                               \\ \hline
 full     &                 AVDZ  & \multirow{2}{*}{im}  & Hal     &  10.5   & 9.0    & 23.5    & \ipair{mim}{3}{cl} (p2)   \\
          &                       &                      & TILA &  13.0   & 12.5   & 69.7    & \ipair{mim}{4}{tos} (p2)  \\
          &                       & \multirow{2}{*}{pyr} & Hal     &  18.3   & 7.4    & 29.4    & \ipair{mpyr}{4}{br} (p2)  \\
          &                       &                      & TILA &  9.0    & 9.5    & 44.7    & \ipair{mpyr}{4}{tos} (p2) \\ \cline{2-8}
          &                 AVTZ  & \multirow{2}{*}{im}  & Hal     &  9.3    & 11.6   & -37.3   & \ipair{mim}{4}{br} (p4)   \\
          &                       &                      & TILA &  9.2    & 6.9    & 25.5    & \ipair{mim}{2}{tos} (p1)  \\
          &                       & \multirow{2}{*}{pyr} & Hal     &  9.2    & 4.7    & -19.7   & \ipair{mpyr}{3}{br} (p1)  \\
          &                       &                      & TILA &  4.4    & 5.1    & 15.1    & \ipair{mpyr}{2}{ntf} (p2) \\ \cline{2-8}
          &                 pop   & \multirow{2}{*}{im}  & Hal     &  13.5   & 13.1   & -33.9   & \ipair{mim}{4}{br} (p2)   \\
          &                       &                      & TILA &  4.8    & 7.6    & 32.8    & \ipair{mim}{1}{ntf} (p2)  \\
          &                       & \multirow{2}{*}{pyr} & Hal     &  9.3    & 3.7    & -15.8   & \ipair{mpyr}{3}{br} (p1)  \\
          &                       &                      & TILA &  6.8    & 9.8    & 43.7    & \ipair{mpyr}{1}{tos} (p2) \\ \hline
1st order &                 AVDZ  & \multirow{2}{*}{im}  & Hal     &  10.8   & 6.8    & 27.3    & \ipair{mim}{4}{cl} (p4)   \\
          &                       &                      & TILA &  14.8   & 12.5   & 72.7    & \ipair{mim}{4}{tos} (p2)  \\
          &                       & \multirow{2}{*}{pyr} & Hal     &  23.0   & 7.2    & 34.1    & \ipair{mpyr}{4}{br} (p2)  \\
          &                       &                      & TILA &  9.6    & 8.9    & 44.2    & \ipair{mpyr}{4}{tos} (p2) \\ \cline{2-8}
          &                 AVTZ  & \multirow{2}{*}{im}  & Hal     &  9.7    & 12.1   & -31.0   & \ipair{mim}{4}{br} (p4)   \\ 
          &                       &                      & TILA &  10.9   & 6.3    & 28.5    & \ipair{mim}{2}{tos} (p1)  \\
          &                       & \multirow{2}{*}{pyr} & Hal     &  5.0    & 4.8    & -14.8   & \ipair{mpyr}{3}{br} (p1)  \\
          &                       &                      & TILA &  4.9    & 4.9    & 15.0    & \ipair{mpyr}{2}{ntf} (p2) \\ \cline{2-8}
          &                 pop   & \multirow{2}{*}{im}  & Hal     &  12.6   & 12.8   & -36.3   & \ipair{mim}{4}{cl} (p2)   \\
          &                       &                      & TILA &  4.3    & 7.3    & 30.8    & \ipair{mim}{1}{ntf} (p2)  \\
          &                       & \multirow{2}{*}{pyr} & Hal     &  5.0    & 3.6    & -10.9   & \ipair{mpyr}{3}{br} (p1)  \\
          &                       &                      & TILA &  7.6    & 9.3    & 43.4    & \ipair{mpyr}{1}{tos} (p2) \\ \hline
\end{tabular}
\end{table}


\begin{table}[h]
\centering
\footnotesize
\caption{Comparison of full and 1st order exchange in SAPT with electrostatics from EFP}
\label{tab:trunc-exch-stats}
\begin{tabular}{llllrrrr}
\hline
  order    & Basis                 & Cation               & Halide  & MAE    & SD     & Max     &                           \\ \hline
  full     &                 AVDZ  & \multirow{2}{*}{im}  & Hal     & 6.8    & 7.9    & -18.0   & \ipair{mim}{3}{br} (p4)   \\
           &                       &                      & TILA & 13.9   & 6.2    & 23.8    & \ipair{mim}{2}{mes} (p1)  \\
           &                       & \multirow{2}{*}{pyr} & Hal     & 5.0    & 5.3    & 14.3    & \ipair{mpyr}{4}{cl} (p2)  \\
           &                       &                      & TILA & 13.7   & 6.6    & 27.8    & \ipair{mpyr}{1}{mes} (p2) \\ \cline{2-8}
           &                 AVTZ  & \multirow{2}{*}{im}  & Hal     & 8.6    & 7.7    & -16.8   & \ipair{mim}{4}{br} (p1)   \\
           &                       &                      & TILA & 15.5   & 11.5   & 28.0    & \ipair{mim}{4}{mes} (p1)  \\
           &                       & \multirow{2}{*}{pyr} & Hal     & 5.0    & 4.9    & -10.2   & \ipair{mpyr}{3}{br} (p3)  \\
           &                       &                      & TILA & 16.0   & 7.7    & 25.2    & \ipair{mpyr}{1}{mes} (p1) \\ \cline{2-8}
           &                 pop   & \multirow{2}{*}{im}  & Hal     & 21.7   & 21.8   & 48.8    & \ipair{mim}{2}{br} (p1)   \\
           &                       &                      & TILA & 31.2   & 13.7   & 50.2    & \ipair{mim}{4}{mes} (p1)  \\
           &                       & \multirow{2}{*}{pyr} & Hal     & 25.2   & 19.3   & 48.1    & \ipair{mpyr}{1}{br} (p1)  \\
           &                       &                      & TILA & 29.7   & 9.4    & 42.4    & \ipair{mpyr}{1}{tos} (p1) \\ \hline
 1st order &                 AVDZ  & \multirow{2}{*}{im}  & Hal     & 21.9   & 11.0   & -43.7   & \ipair{mim}{3}{br} (p4)   \\
           &                       &                      & TILA & 3.0    & 3.1    & -10.8   & \ipair{mim}{2}{dca} (p1)  \\
           &                       & \multirow{2}{*}{pyr} & Hal     & 14.3   & 5.7    & -26.7   & \ipair{mpyr}{2}{br} (p1)  \\
           &                       &                      & TILA & 1.8    & 2.5    & -8.7    & \ipair{mpyr}{4}{dca} (p1) \\ \cline{2-8}
           &                 AVTZ  & \multirow{2}{*}{im}  & Hal     & 27.5   & 10.1   & -41.3   & \ipair{mim}{4}{br} (p1)   \\
           &                       &                      & TILA & 4.2    & 5.3    & -19.2   & \ipair{mim}{2}{dca} (p1)  \\
           &                       & \multirow{2}{*}{pyr} & Hal     & 21.4   & 5.0    & -28.0   & \ipair{mpyr}{3}{br} (p3)  \\
           &                       &                      & TILA & 4.0    & 3.3    & -11.0   & \ipair{mpyr}{4}{dca} (p1) \\ \cline{2-8}
           &                 pop   & \multirow{2}{*}{im}  & Hal     & 21.2   & 22.8   & -39.9   & \ipair{mim}{3}{cl} (p4)   \\
           &                       &                      & TILA & 16.2   & 7.6    & 31.1    & \ipair{mim}{3}{ntf} (p4)  \\
           &                       & \multirow{2}{*}{pyr} & Hal     & 18.5   & 19.2   & 29.8    & \ipair{mpyr}{1}{br} (p1)  \\
           &                       &                      & TILA & 17.1   & 5.0    & 22.4    & \ipair{mpyr}{3}{ntf} (p4) \\ \hline
\end{tabular}
\end{table}




% raw energies

\begin{table}[ht]
\centering
\footnotesize
\caption{SAPT2+3 energetic components of interaction energy -- Imidazolium TILAs}
\begin{tabular}{rllrrrrrr}
  \hline
Chain & Anion & Conf & Electrostatics & Exchange & Induction & Dispersion & Charge-transfer & Total \\ 
  \hline
   1 & bf4 & p1 & -387.4 & 91.9 & -40.7 & -43.8 & -5.0 & -380.1 \\ 
   1 & dca & p1 & -383.8 & 113.8 & -36.5 & -60.7 & -8.3 & -367.2 \\ 
   1 & dca & p2 & -361.9 & 107.9 & -53.1 & -46.0 & -9.4 & -353.2 \\ 
   1 & mes & p1 & -434.4 & 130.4 & -55.1 & -62.7 & -9.7 & -421.8 \\ 
   1 & ntf2 & p1 & -357.5 & 119.0 & -40.5 & -76.6 & -4.0 & -355.6 \\ 
   1 & ntf2 & p2 & -346.4 & 119.1 & -67.0 & -49.3 & -11.2 & -343.7 \\ 
   1 & ntf2 & p3 & -342.7 & 105.7 & -51.7 & -58.8 & -6.2 & -347.6 \\ 
   1 & pf6 & p1 & -361.9 & 87.8 & -37.3 & -44.9 & -5.1 & -356.3 \\ 
   1 & tos & p1 & -416.7 & 127.6 & -59.7 & -63.0 & -9.6 & -411.9 \\ 
   2 & bf4 & p1 & -385.9 & 97.3 & -44.4 & -47.1 & -5.8 & -380.2 \\ 
   2 & bf4 & p2 & -379.6 & 91.8 & -42.6 & -43.6 & -5.3 & -374.1 \\ 
   2 & dca & p1 & -373.6 & 108.0 & -38.9 & -61.7 & -6.7 & -366.1 \\ 
   2 & dca & p2 & -371.3 & 105.7 & -39.2 & -59.2 & -6.8 & -364.1 \\ 
   2 & dca & p3 & -382.2 & 116.5 & -39.9 & -62.2 & -8.0 & -368.0 \\ 
   2 & dca & p4 & -376.4 & 109.3 & -38.4 & -59.3 & -7.2 & -364.8 \\ 
   2 & dca & p5 & -378.5 & 115.0 & -37.6 & -61.5 & -8.6 & -362.6 \\ 
   2 & dca & p6 & -374.3 & 110.5 & -35.4 & -60.8 & -7.6 & -360.0 \\ 
   2 & mes & p1 & -430.8 & 137.8 & -61.3 & -67.5 & -11.0 & -421.7 \\ 
   2 & mes & p2 & -426.1 & 130.8 & -57.1 & -63.8 & -9.9 & -416.2 \\ 
   2 & ntf2 & p1 & -356.7 & 126.2 & -42.5 & -84.4 & -5.0 & -357.4 \\ 
   2 & ntf2 & p2 & -342.8 & 118.1 & -63.1 & -59.3 & -9.4 & -347.0 \\ 
   2 & ntf2 & p3 & -342.4 & 112.3 & -54.8 & -63.2 & -7.0 & -348.0 \\ 
   2 & ntf2 & p4 & -348.9 & 117.6 & -41.8 & -75.8 & -4.9 & -348.9 \\ 
   2 & pf6 & p1 & -361.8 & 95.1 & -40.4 & -49.7 & -6.0 & -356.8 \\ 
   2 & pf6 & p2 & -354.0 & 86.7 & -38.0 & -45.2 & -5.1 & -350.5 \\ 
   2 & tos & p1 & -417.1 & 135.7 & -62.9 & -69.6 & -9.6 & -414.0 \\ 
   2 & tos & p2 & -411.1 & 128.2 & -59.8 & -64.7 & -9.2 & -407.3 \\ 
   3 & bf4 & p1 & -383.8 & 98.6 & -46.5 & -48.6 & -6.1 & -380.3 \\ 
   3 & bf4 & p2 & -381.2 & 94.8 & -45.2 & -46.8 & -5.6 & -378.4 \\ 
   3 & dca & p1 & -370.8 & 109.3 & -39.9 & -63.0 & -6.8 & -364.3 \\ 
   3 & dca & p2 & -370.8 & 110.8 & -38.4 & -64.0 & -6.7 & -362.4 \\ 
   3 & dca & p3 & -360.1 & 109.6 & -51.0 & -55.9 & -8.4 & -357.3 \\ 
   3 & dca & p4 & -382.3 & 117.9 & -40.4 & -66.0 & -7.6 & -370.9 \\ 
   3 & dca & p5 & -372.0 & 109.4 & -36.1 & -62.4 & -6.0 & -361.1 \\ 
   3 & dca & p6 & -370.5 & 109.7 & -35.6 & -60.9 & -7.5 & -357.3 \\ 
   3 & mes & p1 & -430.7 & 143.8 & -65.7 & -71.0 & -11.8 & -423.7 \\ 
   3 & mes & p2 & -422.4 & 131.2 & -58.2 & -64.4 & -9.9 & -413.9 \\ 
   3 & ntf2 & p1 & -356.7 & 130.7 & -43.9 & -88.7 & -5.5 & -358.6 \\ 
   3 & ntf2 & p2 & -342.8 & 122.3 & -63.8 & -64.2 & -9.6 & -348.5 \\ 
   3 & ntf2 & p3 & -341.5 & 114.7 & -56.3 & -65.5 & -7.3 & -348.7 \\ 
   3 & ntf2 & p4 & -340.4 & 113.9 & -40.0 & -75.7 & -4.4 & -342.3 \\ 
   3 & pf6 & p1 & -346.2 & 90.3 & -38.5 & -48.7 & -5.3 & -343.2 \\ 
   3 & pf6 & p2 & -350.8 & 86.7 & -38.5 & -45.4 & -5.2 & -348.1 \\ 
   3 & tos & p1 & -416.7 & 140.2 & -66.0 & -73.5 & -10.0 & -416.0 \\ 
   3 & tos & p2 & -412.1 & 134.9 & -64.0 & -70.6 & -9.7 & -411.8 \\ 
   4 & bf4 & p1 & -382.1 & 99.0 & -47.7 & -49.3 & -6.2 & -380.1 \\ 
   4 & bf4 & p2 & -379.0 & 94.8 & -46.1 & -47.1 & -5.6 & -377.5 \\ 
   4 & dca & p1 & -369.5 & 110.0 & -40.6 & -63.8 & -6.9 & -364.0 \\ 
   4 & dca & p2 & -380.7 & 118.6 & -41.2 & -67.8 & -7.6 & -371.1 \\ 
   4 & dca & p3 & -369.7 & 108.6 & -36.3 & -62.7 & -5.9 & -360.1 \\ 
   4 & mes & p1 & -428.7 & 145.0 & -67.4 & -72.5 & -11.9 & -423.6 \\ 
   4 & mes & p2 & -420.1 & 131.2 & -58.9 & -64.6 & -10.0 & -412.5 \\ 
   4 & pf6 & p1 & -359.7 & 98.8 & -43.5 & -53.5 & -6.5 & -357.9 \\ 
   4 & pf6 & p2 & -348.9 & 86.5 & -38.9 & -45.5 & -5.2 & -346.9 \\ 
   4 & tos & p1 & -415.6 & 143.0 & -67.9 & -76.0 & -10.2 & -416.5 \\ 
   4 & tos & p2 & -411.1 & 135.5 & -66.8 & -70.6 & -9.9 & -413.0 \\ 
   \hline
\end{tabular}
\end{table}

\begin{table}[ht]
\centering
\footnotesize
\caption{SAPT2+3 energetic components of interaction energy -- Imidazolium halides}
\begin{tabular}{rllrrrrrr}
  \hline
Chain & Anion & Conf & Electrostatics & Exchange & Induction & Dispersion & Charge-transfer & Total \\ 
  \hline
   1 & br & p1 & -468.8 & 196.9 & -76.8 & -59.5 & -67.2 & -408.3 \\ 
   1 & br & p2 & -409.2 & 151.5 & -87.9 & -38.0 & -31.2 & -383.5 \\ 
   1 & cl & p1 & -479.1 & 194.4 & -83.3 & -55.3 & -58.2 & -423.4 \\ 
   1 & cl & p2 & -436.0 & 170.1 & -101.2 & -38.9 & -32.1 & -405.9 \\ 
   2 & br & p1 & -455.3 & 194.2 & -75.7 & -60.1 & -58.5 & -396.9 \\ 
   2 & br & p2 & -410.1 & 157.8 & -92.9 & -42.5 & -32.5 & -387.6 \\ 
   2 & br & p3 & -410.6 & 155.8 & -86.2 & -42.7 & -30.4 & -383.8 \\ 
   2 & br & p4 & -461.3 & 195.6 & -77.1 & -60.6 & -65.0 & -403.4 \\ 
   2 & cl & p1 & -465.8 & 191.5 & -82.8 & -56.2 & -49.7 & -413.4 \\ 
   2 & cl & p2 & -434.3 & 173.2 & -106.4 & -42.3 & -32.8 & -409.7 \\ 
   2 & cl & p3 & -434.7 & 170.8 & -98.8 & -42.7 & -30.6 & -405.4 \\ 
   2 & cl & p4 & -470.6 & 192.0 & -83.6 & -56.1 & -56.0 & -418.2 \\ 
   3 & br & p1 & -453.3 & 195.9 & -77.9 & -61.3 & -58.9 & -396.5 \\ 
   3 & br & p2 & -406.8 & 158.1 & -94.2 & -43.2 & -32.6 & -386.2 \\ 
   3 & br & p3 & -412.5 & 161.3 & -86.9 & -47.4 & -30.2 & -385.5 \\ 
   3 & br & p4 & -463.8 & 202.5 & -80.9 & -64.0 & -68.0 & -406.2 \\ 
   3 & cl & p1 & -461.9 & 191.9 & -84.8 & -57.1 & -49.2 & -411.9 \\ 
   3 & cl & p2 & -430.5 & 173.0 & -107.7 & -42.8 & -32.8 & -408.0 \\ 
   3 & cl & p3 & -434.2 & 172.5 & -98.5 & -46.2 & -29.7 & -406.4 \\ 
   3 & cl & p4 & -471.5 & 197.5 & -87.4 & -58.9 & -57.8 & -420.4 \\ 
   4 & br & p1 & -452.0 & 196.8 & -79.0 & -62.4 & -58.3 & -396.6 \\ 
   4 & br & p2 & -405.3 & 158.9 & -95.7 & -44.6 & -32.7 & -386.6 \\ 
   4 & br & p3 & -411.7 & 163.4 & -88.1 & -49.5 & -30.1 & -385.9 \\ 
   4 & br & p4 & -461.3 & 202.1 & -81.4 & -64.5 & -67.1 & -405.1 \\ 
   4 & cl & p1 & -459.8 & 191.7 & -85.8 & -57.7 & -48.3 & -411.6 \\ 
   4 & cl & p2 & -428.2 & 173.2 & -109.2 & -43.7 & -32.8 & -408.1 \\ 
   4 & cl & p3 & -432.5 & 173.4 & -99.5 & -47.9 & -29.3 & -406.5 \\ 
   4 & cl & p4 & -468.9 & 197.1 & -88.1 & -59.4 & -57.2 & -419.3 \\ 
   \hline
\end{tabular}
\end{table}

\begin{table}[ht]
\centering
\footnotesize
\caption{SAPT2+3 energetic components of interaction energy -- Pyrrolidinium TILAs}
\begin{tabular}{rllrrrrrr}
  \hline
Chain & Anion & Conf & Electrostatics & Exchange & Induction & Dispersion & Charge-transfer & Total \\ 
  \hline
 1 & bf4 & p1 & -370.9 & 86.0 & -47.1 & -38.1 & -6.8 & -370.2 \\ 
   1 & bf4 & p2 & -365.8 & 84.6 & -46.5 & -34.3 & -6.9 & -361.9 \\ 
   1 & dca & p1 & -348.5 & 85.8 & -43.5 & -42.6 & -7.2 & -348.9 \\ 
   1 & dca & p2 & -351.4 & 80.6 & -39.6 & -44.5 & -5.7 & -354.8 \\ 
   1 & dca & p3 & -330.7 & 80.5 & -43.1 & -42.5 & -5.6 & -335.8 \\ 
   1 & dca & p4 & -353.9 & 81.6 & -38.5 & -43.2 & -5.2 & -353.9 \\ 
   1 & mes & p1 & -407.5 & 117.8 & -65.4 & -52.3 & -11.1 & -407.4 \\ 
   1 & mes & p2 & -403.4 & 125.6 & -68.9 & -46.3 & -13.4 & -392.9 \\ 
   1 & ntf2 & p1 & -315.7 & 71.4 & -42.9 & -49.8 & -4.2 & -336.9 \\ 
   1 & ntf2 & p2 & -312.2 & 70.7 & -45.0 & -44.9 & -4.7 & -331.4 \\ 
   1 & ntf2 & p3 & -319.9 & 79.9 & -46.1 & -48.6 & -5.4 & -334.6 \\ 
   1 & ntf2 & p5 & -314.7 & 72.7 & -45.7 & -45.9 & -4.5 & -333.6 \\ 
   1 & pf6 & p1 & -341.6 & 74.5 & -41.2 & -37.0 & -6.1 & -345.4 \\ 
   1 & pf6 & p2 & -337.7 & 73.7 & -40.8 & -33.6 & -6.2 & -338.4 \\ 
   1 & tos & p1 & -393.9 & 115.0 & -67.6 & -53.0 & -10.3 & -399.4 \\ 
   1 & tos & p2 & -383.8 & 107.9 & -64.6 & -46.4 & -9.8 & -386.9 \\ 
   2 & bf4 & p1 & -365.6 & 85.5 & -48.1 & -39.2 & -6.8 & -367.5 \\ 
   2 & bf4 & p2 & -354.0 & 84.6 & -46.7 & -36.0 & -7.2 & -352.2 \\ 
   2 & bf4 & p3 & -361.6 & 86.7 & -48.3 & -35.6 & -7.4 & -358.8 \\ 
   2 & dca & p1 & -342.0 & 84.7 & -43.9 & -43.1 & -7.1 & -344.4 \\ 
   2 & dca & p2 & -342.0 & 78.8 & -39.4 & -44.5 & -5.2 & -347.0 \\ 
   2 & dca & p3 & -345.7 & 80.6 & -40.3 & -45.2 & -5.7 & -350.6 \\ 
   2 & mes & p1 & -401.8 & 115.7 & -65.9 & -54.2 & -10.8 & -406.2 \\ 
   2 & mes & p2 & -386.2 & 108.2 & -61.8 & -49.4 & -10.1 & -389.2 \\ 
   2 & mes & p3 & -397.1 & 121.0 & -68.2 & -48.2 & -12.5 & -392.5 \\ 
   2 & ntf2 & p1 & -309.4 & 66.5 & -41.6 & -51.5 & -3.9 & -335.9 \\ 
   2 & ntf2 & p2 & -307.6 & 72.2 & -45.5 & -47.5 & -5.2 & -328.5 \\ 
   2 & ntf2 & p3 & -306.3 & 69.5 & -40.7 & -50.2 & -4.0 & -327.6 \\ 
   2 & ntf2 & p4 & -312.4 & 76.3 & -45.2 & -49.2 & -5.2 & -330.5 \\ 
   2 & ntf2 & p5 & -303.4 & 70.4 & -42.6 & -48.3 & -4.4 & -323.9 \\ 
   2 & ntf2 & p6 & -305.9 & 70.1 & -44.6 & -46.2 & -4.6 & -326.6 \\ 
   2 & pf6 & p1 & -336.1 & 76.1 & -42.1 & -37.8 & -6.4 & -339.9 \\ 
   2 & pf6 & p2 & -325.3 & 71.9 & -40.2 & -35.1 & -6.2 & -328.6 \\ 
   2 & pf6 & p3 & -332.8 & 74.1 & -41.7 & -34.7 & -6.4 & -335.2 \\ 
   2 & tos & p1 & -388.5 & 113.4 & -68.0 & -54.9 & -10.1 & -398.1 \\ 
   2 & tos & p2 & -373.2 & 105.2 & -63.2 & -50.1 & -9.3 & -381.3 \\ 
   3 & bf4 & p1 & -362.0 & 84.1 & -48.9 & -39.9 & -6.7 & -366.6 \\ 
   3 & bf4 & p2 & -349.3 & 83.0 & -47.7 & -36.4 & -7.0 & -350.4 \\ 
   3 & bf4 & p3 & -358.2 & 86.2 & -49.3 & -36.2 & -7.4 & -357.5 \\ 
   3 & dca & p1 & -338.8 & 80.8 & -43.4 & -44.1 & -6.7 & -345.5 \\ 
   3 & dca & p2 & -338.3 & 78.3 & -40.1 & -44.7 & -5.2 & -344.9 \\ 
   3 & dca & p3 & -313.6 & 72.3 & -40.1 & -41.6 & -4.8 & -322.9 \\ 
   3 & dca & p4 & -336.1 & 77.4 & -40.9 & -43.6 & -5.9 & -343.3 \\ 
   3 & dca & p5 & -342.9 & 79.8 & -41.0 & -45.5 & -5.6 & -349.6 \\ 
   3 & mes & p1 & -396.9 & 112.3 & -66.2 & -55.2 & -10.4 & -405.9 \\ 
   3 & mes & p2 & -381.7 & 107.5 & -63.0 & -49.9 & -10.1 & -387.0 \\ 
   3 & mes & p3 & -393.7 & 120.6 & -69.3 & -49.4 & -12.4 & -391.9 \\ 
   3 & ntf2 & p1 & -307.7 & 69.2 & -42.7 & -54.2 & -4.1 & -335.4 \\ 
   3 & ntf2 & p2 & -304.1 & 73.2 & -45.3 & -47.3 & -5.4 & -323.5 \\ 
   3 & ntf2 & p3 & -304.4 & 71.3 & -41.6 & -50.8 & -4.2 & -325.4 \\ 
   3 & ntf2 & p4 & -310.8 & 77.6 & -46.1 & -51.8 & -5.3 & -331.2 \\ 
   3 & ntf2 & p5 & -300.3 & 68.9 & -42.7 & -48.4 & -4.4 & -322.5 \\ 
   3 & ntf2 & p6 & -302.7 & 69.5 & -44.7 & -48.0 & -4.7 & -325.8 \\ 
   3 & pf6 & p1 & -334.2 & 73.6 & -42.5 & -39.3 & -6.1 & -342.4 \\ 
   3 & pf6 & p2 & -321.3 & 70.8 & -40.8 & -35.3 & -6.1 & -326.6 \\ 
   3 & pf6 & p3 & -329.9 & 73.5 & -42.3 & -35.4 & -6.4 & -334.2 \\ 
   3 & tos & p1 & -384.2 & 110.4 & -68.1 & -56.0 & -9.7 & -397.9 \\ 
   3 & tos & p2 & -368.4 & 104.1 & -64.1 & -50.6 & -9.2 & -379.1 \\ 
   4 & bf4 & p1 & -359.8 & 84.1 & -49.7 & -40.2 & -6.8 & -365.5 \\ 
   4 & bf4 & p2 & -346.7 & 82.4 & -48.4 & -36.7 & -7.0 & -349.3 \\ 
   4 & bf4 & p3 & -356.6 & 87.0 & -50.2 & -36.6 & -7.4 & -356.5 \\ 
   4 & dca & p1 & -330.5 & 69.8 & -37.4 & -42.9 & -4.9 & -341.0 \\ 
   4 & dca & p2 & -335.9 & 78.0 & -40.6 & -44.9 & -5.2 & -343.4 \\ 
   4 & dca & p3 & -336.3 & 78.3 & -43.0 & -44.8 & -6.2 & -345.8 \\ 
   4 & dca & p4 & -335.9 & 78.1 & -40.7 & -44.9 & -5.2 & -343.4 \\ 
   4 & dca & p5 & -340.6 & 79.3 & -39.6 & -44.6 & -5.2 & -345.4 \\ 
   4 & dca & p6 & -340.5 & 79.4 & -41.2 & -45.6 & -5.6 & -347.9 \\ 
   4 & mes & p1 & -394.0 & 111.9 & -67.0 & -55.3 & -10.4 & -404.4 \\ 
   4 & mes & p2 & -378.7 & 106.9 & -63.7 & -50.6 & -10.0 & -386.1 \\ 
   4 & mes & p3 & -391.3 & 120.6 & -70.2 & -49.8 & -12.4 & -390.7 \\ 
   %4 & ntf2 & p1 &  &  &  &  &  &  \\ 
   %4 & ntf2 & p2 &  &  &  &  &  &  \\ 
   %4 & ntf2 & p3 &  &  &  &  &  &  \\ 
   %4 & ntf2 & p4 &  &  &  &  &  &  \\ 
   %4 & ntf2 & p5 &  &  &  &  &  &  \\ 
   %4 & ntf2 & p6 &  &  &  &  &  &  \\ 
   4 & pf6 & p1 & -332.3 & 73.8 & -43.3 & -39.5 & -6.2 & -341.2 \\ 
   4 & pf6 & p2 & -319.2 & 70.6 & -41.3 & -35.9 & -6.1 & -325.8 \\ 
   4 & pf6 & p3 & -328.1 & 73.2 & -42.8 & -35.8 & -6.3 & -333.5 \\ 
   %4 & tos & p1 &  &  &  &  &  &  \\ 
   4 & tos & p2 & -365.7 & 103.6 & -64.8 & -51.4 & -9.2 & -378.3 \\ 
   \hline
\end{tabular}
\end{table}

\begin{table}[ht]
\centering
\footnotesize
\caption{SAPT2+3 energetic components of interaction energy -- Pyrrolidinium halides}
\begin{tabular}{rllrrrrrr}
  \hline
Chain & Anion & Conf & Electrostatics & Exchange & Induction & Dispersion & Charge-transfer & Total \\ 
  \hline
 1 & br & p1 & -415.4 & 150.5 & -71.5 & -47.5 & -30.6 & -383.9 \\ 
   1 & br & p2 & -408.9 & 146.4 & -70.7 & -44.2 & -28.6 & -377.4 \\ 
   1 & cl & p1 & -423.9 & 140.6 & -74.1 & -43.3 & -23.8 & -400.8 \\ 
   1 & cl & p2 & -420.1 & 139.5 & -73.8 & -41.0 & -23.0 & -395.4 \\ 
   2 & br & p1 & -410.3 & 150.4 & -72.0 & -49.0 & -30.4 & -380.9 \\ 
   2 & br & p2 & -393.6 & 146.4 & -71.9 & -45.8 & -27.3 & -364.9 \\ 
   2 & br & p3 & -403.6 & 147.3 & -71.6 & -45.6 & -28.8 & -373.5 \\ 
   2 & cl & p1 & -417.6 & 139.9 & -74.9 & -44.4 & -23.5 & -397.0 \\ 
   2 & cl & p2 & -401.7 & 136.8 & -74.5 & -41.7 & -21.4 & -381.1 \\ 
   2 & cl & p3 & -412.8 & 138.6 & -74.7 & -41.8 & -22.6 & -390.7 \\ 
   3 & br & p1 & -407.9 & 151.2 & -73.3 & -50.2 & -30.5 & -380.3 \\ 
   3 & br & p2 & -388.3 & 143.5 & -71.3 & -46.0 & -26.3 & -362.0 \\ 
   3 & br & p3 & -401.2 & 147.7 & -72.5 & -46.7 & -28.7 & -372.6 \\ 
   3 & cl & p1 & -414.2 & 139.4 & -76.1 & -45.2 & -23.3 & -396.0 \\ 
   3 & cl & p2 & -397.3 & 136.3 & -75.7 & -42.3 & -21.1 & -379.0 \\ 
   3 & cl & p3 & -409.4 & 138.0 & -75.6 & -42.4 & -22.4 & -389.3 \\ 
   4 & br & p1 & -405.9 & 151.6 & -74.2 & -50.6 & -30.6 & -379.1 \\ 
   4 & br & p2 & -385.4 & 143.0 & -72.1 & -46.5 & -26.2 & -361.0 \\ 
   4 & br & p3 & -399.5 & 148.6 & -73.4 & -47.1 & -28.9 & -371.4 \\ 
   4 & cl & p1 & -411.5 & 139.0 & -76.7 & -45.3 & -23.2 & -394.6 \\ 
   4 & cl & p2 & -395.1 & 136.7 & -77.1 & -43.1 & -21.2 & -378.6 \\ 
   4 & cl & p3 & -406.6 & 137.5 & -76.2 & -42.5 & -22.2 & -387.9 \\ 
   \hline
\end{tabular}
\end{table}



\begin{table}[ht]
\centering
\footnotesize
\caption{EFP energetic components of interaction energy -- Imidazolium TILAs (aug-cc-pVDZ)}
\begin{tabular}{rllrrrrrr}
  \hline
Chain & Anion & Conf & Electrostatic & Repulsion & Polarization & Dispersion & Charge-tranfer & Total \\ 
  \hline
 1 & bf4 & p1 & -391.7 & 76.1 & -34.0 & -54.3 & -1.4 & -405.3 \\ 
   1 & dca & p1 & -395.9 & 105.7 & -29.0 & -83.4 & -2.0 & -404.7 \\ 
   1 & dca & p2 & -379.8 & 94.9 & -39.8 & -52.6 & -4.7 & -382.0 \\ 
   1 & mes & p1 & -442.5 & 107.7 & -48.9 & -82.5 & -2.5 & -468.8 \\ 
   1 & ntf2 & p1 & -377.4 & 110.1 & -31.5 & -101.7 & -2.2 & -402.7 \\ 
   1 & ntf2 & p2 & -342.2 & 105.2 & -41.4 & -52.1 & -6.7 & -337.1 \\ 
   1 & ntf2 & p3 & -357.9 & 90.7 & -40.5 & -70.8 & -2.0 & -380.5 \\ 
   1 & pf6 & p1 & -372.3 & 75.1 & -31.0 & -56.1 & -1.7 & -386.0 \\ 
   1 & tos & p1 & -425.6 & 108.6 & -51.2 & -76.0 & -2.2 & -446.3 \\ 
   2 & bf4 & p1 & -392.6 & 80.5 & -37.5 & -57.5 & -1.6 & -408.7 \\ 
   2 & bf4 & p2 & -386.4 & 74.8 & -35.6 & -54.1 & -1.3 & -402.5 \\ 
   2 & dca & p1 & -394.7 & 112.4 & -33.6 & -82.0 & -2.2 & -400.1 \\ 
   2 & dca & p2 & -385.6 & 99.0 & -34.8 & -78.5 & -3.0 & -402.8 \\ 
   2 & dca & p3 & -394.7 & 112.4 & -33.6 & -82.0 & -2.2 & -400.1 \\ 
   2 & dca & p4 & -391.3 & 102.9 & -32.6 & -81.0 & -2.1 & -404.1 \\ 
   2 & dca & p5 & -393.5 & 107.5 & -30.0 & -83.8 & -1.7 & -401.5 \\ 
   2 & dca & p6 & -378.5 & 103.0 & -29.6 & -84.8 & -2.0 & -392.0 \\ 
   2 & mes & p1 & -428.1 & 114.0 & -54.8 & -86.5 & -2.5 & -457.8 \\ 
   2 & mes & p2 & -443.0 & 108.0 & -50.1 & -83.9 & -2.0 & -471.0 \\ 
   2 & ntf2 & p1 & -374.0 & 116.5 & -33.7 & -110.6 & -2.3 & -404.1 \\ 
   2 & ntf2 & p2 & -357.1 & 104.6 & -45.9 & -67.0 & -5.5 & -370.9 \\ 
   2 & ntf2 & p3 & -354.6 & 94.9 & -43.5 & -75.4 & -2.4 & -380.9 \\ 
   2 & ntf2 & p4 & -352.6 & 103.1 & -31.9 & -101.7 & -2.1 & -385.1 \\ 
   2 & pf6 & p1 & -374.7 & 81.1 & -34.1 & -61.6 & -2.0 & -391.3 \\ 
   2 & pf6 & p2 & -363.1 & 73.8 & -31.9 & -56.7 & -1.5 & -379.4 \\ 
   2 & tos & p1 & -440.5 & 114.1 & -51.2 & -84.8 & -2.3 & -464.7 \\ 
   2 & tos & p2 & -459.7 & 107.4 & -50.2 & -79.7 & -2.1 & -484.3 \\ 
   3 & bf4 & p1 & -395.4 & 82.3 & -40.0 & -59.0 & -1.6 & -413.8 \\ 
   3 & bf4 & p2 & -386.2 & 78.3 & -37.2 & -57.2 & -1.6 & -404.0 \\ 
   3 & dca & p1 & -384.7 & 102.3 & -34.3 & -83.3 & -3.3 & -403.3 \\ 
   3 & dca & p2 & -376.6 & 103.3 & -33.2 & -87.4 & -1.5 & -395.4 \\ 
   3 & dca & p3 & -376.1 & 99.5 & -38.4 & -68.2 & -3.2 & -386.4 \\ 
   3 & dca & p4 & -398.8 & 112.4 & -35.7 & -88.3 & -2.1 & -412.5 \\ 
   3 & dca & p5 & -387.0 & 102.6 & -27.7 & -88.6 & -1.9 & -402.6 \\ 
   3 & dca & p6 & -375.5 & 102.2 & -29.8 & -85.7 & -3.2 & -392.0 \\ 
   3 & mes & p1 & -428.1 & 120.3 & -60.0 & -90.2 & -2.5 & -460.6 \\ 
   3 & mes & p2 & -432.0 & 108.3 & -50.5 & -85.0 & -2.6 & -461.8 \\ 
   3 & ntf2 & p1 & -371.1 & 120.9 & -35.3 & -115.7 & -2.3 & -403.5 \\ 
   3 & ntf2 & p2 & -350.6 & 108.5 & -46.6 & -73.4 & -4.8 & -366.9 \\ 
   3 & ntf2 & p3 & -352.2 & 97.4 & -44.8 & -77.9 & -2.3 & -379.7 \\ 
   3 & ntf2 & p4 & -339.6 & 98.6 & -30.0 & -103.6 & -1.9 & -376.6 \\ 
   3 & pf6 & p1 & -346.3 & 76.5 & -32.8 & -62.3 & -1.4 & -366.2 \\ 
   3 & pf6 & p2 & -357.2 & 73.5 & -32.3 & -57.2 & -1.6 & -374.7 \\ 
   3 & tos & p1 & -448.1 & 119.1 & -55.1 & -89.5 & -2.1 & -475.7 \\ 
   3 & tos & p2 & -451.2 & 114.8 & -54.5 & -86.0 & -2.5 & -479.4 \\ 
   4 & bf4 & p1 & -382.8 & 82.9 & -40.9 & -58.5 & -1.5 & -400.8 \\ 
   4 & bf4 & p2 & -386.5 & 78.5 & -37.8 & -56.6 & -1.7 & -404.0 \\ 
   4 & dca & p1 & -382.6 & 102.6 & -35.3 & -83.0 & -3.2 & -401.4 \\ 
   4 & dca & p2 & -392.7 & 113.0 & -36.6 & -88.4 & -2.1 & -406.8 \\ 
   4 & dca & p3 & -377.8 & 101.8 & -28.6 & -88.2 & -2.0 & -394.8 \\ 
   4 & mes & p1 & -439.5 & 121.6 & -61.4 & -90.1 & -2.5 & -471.9 \\ 
   4 & mes & p2 & -426.7 & 108.3 & -52.2 & -81.7 & -2.5 & -454.7 \\ 
   4 & pf6 & p1 & -366.3 & 85.2 & -37.0 & -64.0 & -2.1 & -384.1 \\ 
   4 & pf6 & p2 & -337.7 & 73.4 & -33.1 & -54.8 & -1.6 & -353.7 \\ 
   4 & tos & p1 & -609.1 & 121.7 & -55.1 & -90.5 & -2.2 & -635.2 \\ 
   4 & tos & p2 & -480.7 & 114.2 & -52.6 & -84.3 & -2.2 & -505.6 \\ 
   \hline
\end{tabular}
\end{table}

\begin{table}[ht]
\centering
\footnotesize
\caption{EFP energetic components of interaction energy -- Imidazolium halides (aug-cc-pVDZ)}
\begin{tabular}{rllrrrrrr}
  \hline
Chain & Anion & Conf & Electrostatic & Repulsion & Polarization & Dispersion & Charge-tranfer & Total \\ 
  \hline
 1 & br & p1 & -466.9 & 202.3 & -65.9 & -51.4 & -2.4 & -384.4 \\ 
   1 & br & p2 & -430.8 & 143.3 & -85.9 & -28.5 & -8.0 & -410.0 \\ 
   1 & cl & p1 & -482.3 & 192.6 & -72.9 & -49.5 & -2.0 & -414.1 \\ 
   1 & cl & p2 & -458.1 & 157.4 & -98.2 & -30.6 & -8.3 & -437.7 \\ 
   2 & br & p1 & -449.3 & 194.3 & -67.9 & -51.8 & -1.5 & -376.2 \\ 
   2 & br & p2 & -420.1 & 151.5 & -85.4 & -31.1 & -9.1 & -394.2 \\ 
   2 & br & p3 & -424.5 & 148.1 & -86.6 & -33.6 & -5.3 & -401.9 \\ 
   2 & br & p4 & -458.9 & 204.0 & -71.3 & -52.7 & -1.9 & -380.8 \\ 
   2 & cl & p1 & -468.8 & 197.7 & -67.6 & -49.0 & -1.2 & -388.9 \\ 
   2 & cl & p2 & -452.0 & 170.0 & -85.4 & -31.9 & -9.4 & -408.6 \\ 
   2 & cl & p3 & -451.8 & 166.4 & -80.2 & -34.6 & -5.4 & -405.6 \\ 
   2 & cl & p4 & -474.8 & 192.3 & -71.8 & -50.4 & -1.6 & -406.4 \\ 
   3 & br & p1 & -448.5 & 200.3 & -73.5 & -53.5 & -1.5 & -376.6 \\ 
   3 & br & p2 & -420.1 & 148.9 & -85.5 & -31.8 & -7.4 & -395.9 \\ 
   3 & br & p3 & -423.6 & 157.7 & -76.0 & -38.0 & -3.9 & -383.8 \\ 
   3 & br & p4 & -464.5 & 220.4 & -63.1 & -54.7 & -3.1 & -365.0 \\ 
   3 & cl & p1 & -468.0 & 195.8 & -72.2 & -49.9 & -0.9 & -395.1 \\ 
   3 & cl & p2 & -454.0 & 157.3 & -108.7 & -33.1 & -7.6 & -446.0 \\ 
   3 & cl & p3 & -451.0 & 165.1 & -86.7 & -38.1 & -4.3 & -415.0 \\ 
   3 & cl & p4 & -483.8 & 204.1 & -71.9 & -52.6 & -2.9 & -407.2 \\ 
   4 & br & p1 & -446.2 & 202.8 & -73.0 & -53.4 & -1.2 & -370.9 \\ 
   4 & br & p2 & -414.9 & 167.6 & -69.8 & -31.2 & -5.8 & -354.1 \\ 
   4 & br & p3 & -417.6 & 170.2 & -71.4 & -38.4 & -3.4 & -360.5 \\ 
   4 & br & p4 & -474.4 & 210.1 & -73.4 & -55.8 & -5.1 & -398.7 \\ 
   4 & cl & p1 & -457.6 & 195.5 & -71.7 & -49.1 & -0.5 & -383.4 \\ 
   4 & cl & p2 & -441.9 & 177.9 & -78.6 & -32.0 & -5.8 & -380.4 \\ 
   4 & cl & p3 & -442.3 & 165.5 & -90.7 & -39.4 & -3.7 & -410.6 \\ 
   4 & cl & p4 & -490.7 & 206.6 & -71.2 & -51.4 & -4.9 & -411.6 \\ 
   \hline
\end{tabular}
\end{table}

\begin{table}[ht]
\centering
\footnotesize
\scriptsize
\caption{EFP energetic components of interaction energy -- Pyrrolidinium TILAs (aug-cc-pVDZ)}
\begin{tabular}{rllrrrrrr}
  \hline
Chain & Anion & Conf & Electrostatic & Repulsion & Polarization & Dispersion & Charge-tranfer & Total \\ 
  \hline
 1 & bf4 & p1 & -375.2 & 67.5 & -37.5 & -42.4 & -1.8 & -389.4 \\ 
   1 & bf4 & p2 & -369.2 & 65.2 & -36.8 & -37.1 & -2.0 & -379.9 \\ 
   1 & dca & p1 & -358.1 & 75.6 & -35.8 & -49.0 & -1.0 & -368.3 \\ 
   1 & dca & p2 & -365.8 & 75.1 & -32.1 & -52.1 & -1.1 & -375.9 \\ 
   1 & dca & p3 & -334.1 & 71.3 & -39.0 & -54.4 & -1.3 & -357.6 \\ 
   1 & dca & p4 & -367.7 & 76.5 & -31.0 & -51.8 & -1.1 & -375.0 \\ 
   1 & mes & p1 & -417.8 & 94.9 & -55.7 & -61.8 & -2.3 & -442.7 \\ 
   1 & mes & p2 & -422.6 & 97.9 & -60.7 & -51.7 & -2.8 & -439.9 \\ 
   1 & ntf2 & p1 & -304.8 & 63.3 & -33.4 & -60.5 & -1.7 & -337.0 \\ 
   1 & ntf2 & p2 & -324.0 & 60.4 & -37.6 & -51.7 & -0.9 & -353.7 \\ 
   1 & ntf2 & p3 & -326.4 & 68.9 & -35.1 & -57.0 & -1.7 & -351.2 \\ 
   1 & ntf2 & p5 & -322.9 & 61.3 & -37.3 & -52.6 & -0.7 & -352.2 \\ 
   1 & pf6 & p1 & -351.9 & 60.4 & -33.7 & -41.4 & -1.7 & -368.3 \\ 
   1 & pf6 & p2 & -349.1 & 58.6 & -33.3 & -36.8 & -1.9 & -362.4 \\ 
   1 & tos & p1 & -400.3 & 93.8 & -55.4 & -59.0 & -2.2 & -423.1 \\ 
   1 & tos & p2 & -403.8 & 87.7 & -55.2 & -50.1 & -2.0 & -423.4 \\ 
   2 & bf4 & p1 & -360.1 & 67.7 & -39.9 & -43.6 & -2.0 & -378.0 \\ 
   2 & bf4 & p2 & -356.8 & 66.2 & -36.9 & -38.8 & -1.7 & -368.0 \\ 
   2 & bf4 & p3 & -362.1 & 67.1 & -39.0 & -38.4 & -1.9 & -374.2 \\ 
   2 & dca & p1 & -353.5 & 75.7 & -37.4 & -49.6 & -1.1 & -365.9 \\ 
   2 & dca & p2 & -350.1 & 74.1 & -31.7 & -52.5 & -1.1 & -361.3 \\ 
   2 & dca & p3 & -358.2 & 75.8 & -33.4 & -52.8 & -1.0 & -369.6 \\ 
   2 & mes & p1 & -402.9 & 93.9 & -59.3 & -64.2 & -2.9 & -435.3 \\ 
   2 & mes & p2 & -392.8 & 87.5 & -53.8 & -57.1 & -2.1 & -418.3 \\ 
   2 & mes & p3 & -408.9 & 95.5 & -60.6 & -54.6 & -2.5 & -431.0 \\ 
   2 & ntf2 & p1 & -297.1 & 58.9 & -33.2 & -63.3 & -1.5 & -336.2 \\ 
   2 & ntf2 & p2 & -308.2 & 61.6 & -38.9 & -54.3 & -1.0 & -340.9 \\ 
   2 & ntf2 & p3 & -300.8 & 62.3 & -30.6 & -60.2 & -1.1 & -330.3 \\ 
   2 & ntf2 & p4 & -320.3 & 65.4 & -34.8 & -58.5 & -1.7 & -349.9 \\ 
   2 & ntf2 & p5 & -306.5 & 60.5 & -35.5 & -55.4 & -0.6 & -337.5 \\ 
   2 & ntf2 & p6 & -312.6 & 59.5 & -37.1 & -53.1 & -0.7 & -344.0 \\ 
   2 & pf6 & p1 & -334.9 & 61.8 & -35.2 & -42.0 & -1.9 & -352.2 \\ 
   2 & pf6 & p2 & -330.4 & 57.9 & -32.5 & -38.1 & -1.5 & -344.7 \\ 
   2 & pf6 & p3 & -342.4 & 59.5 & -34.8 & -37.8 & -1.8 & -357.4 \\ 
   2 & tos & p1 & -370.4 & 92.9 & -59.6 & -61.3 & -2.7 & -401.1 \\ 
   2 & tos & p2 & -402.3 & 86.6 & -53.8 & -54.7 & -1.9 & -426.0 \\ 
   3 & bf4 & p1 & -360.9 & 66.9 & -40.5 & -42.9 & -1.8 & -379.3 \\ 
   3 & bf4 & p2 & -350.1 & 64.5 & -39.2 & -37.8 & -1.6 & -364.2 \\ 
   3 & bf4 & p3 & -358.8 & 66.7 & -39.7 & -38.1 & -1.9 & -371.9 \\ 
   3 & dca & p1 & -355.6 & 75.2 & -34.7 & -49.5 & -0.8 & -365.5 \\ 
   3 & dca & p2 & -344.7 & 73.4 & -33.0 & -52.9 & -1.1 & -358.4 \\ 
   3 & dca & p3 & -332.0 & 64.8 & -35.0 & -52.3 & -1.2 & -355.7 \\ 
   3 & dca & p4 & -356.2 & 71.4 & -34.3 & -51.3 & -0.5 & -370.9 \\ 
   3 & dca & p5 & -357.6 & 74.9 & -33.9 & -53.3 & -0.9 & -370.7 \\ 
   3 & mes & p1 & -397.9 & 91.2 & -59.4 & -63.6 & -2.5 & -432.1 \\ 
   3 & mes & p2 & -385.4 & 86.7 & -57.1 & -56.0 & -2.0 & -413.9 \\ 
   3 & mes & p3 & -402.6 & 95.0 & -61.5 & -54.8 & -2.3 & -426.2 \\ 
   3 & ntf2 & p1 & -296.2 & 60.8 & -34.6 & -66.0 & -1.5 & -337.6 \\ 
   3 & ntf2 & p2 & -308.0 & 61.5 & -37.8 & -52.5 & -0.8 & -337.6 \\ 
   3 & ntf2 & p3 & -293.0 & 63.7 & -31.7 & -59.1 & -1.1 & -321.2 \\ 
   3 & ntf2 & p4 & -317.5 & 66.5 & -36.1 & -59.7 & -1.6 & -348.4 \\ 
   3 & ntf2 & p5 & -302.3 & 59.2 & -36.1 & -55.8 & -0.6 & -335.6 \\ 
   3 & ntf2 & p6 & -306.6 & 59.4 & -37.5 & -54.9 & -0.7 & -340.3 \\ 
   3 & pf6 & p1 & -337.2 & 60.7 & -35.8 & -42.5 & -1.7 & -356.5 \\ 
   3 & pf6 & p2 & -322.3 & 56.7 & -33.9 & -36.9 & -1.5 & -337.9 \\ 
   3 & pf6 & p3 & -336.7 & 59.1 & -35.5 & -37.7 & -1.8 & -352.6 \\ 
   3 & tos & p1 & -384.4 & 90.7 & -59.7 & -60.9 & -2.4 & -416.7 \\ 
   3 & tos & p2 & -380.9 & 85.3 & -57.4 & -53.6 & -1.8 & -408.3 \\ 
   4 & bf4 & p1 & -360.8 & 66.8 & -41.5 & -44.5 & -1.6 & -381.6 \\ 
   4 & bf4 & p2 & -351.9 & 64.1 & -39.8 & -38.8 & -1.5 & -367.9 \\ 
   4 & bf4 & p3 & -359.8 & 67.3 & -40.6 & -39.1 & -1.7 & -373.9 \\ 
   4 & dca & p1 & -345.0 & 72.9 & -33.0 & -52.8 & -1.1 & -359.0 \\ 
   4 & dca & p2 & -352.2 & 72.9 & -35.0 & -52.5 & -0.5 & -367.2 \\ 
   4 & dca & p3 & -354.3 & 74.6 & -34.7 & -52.9 & -0.6 & -367.9 \\ 
   4 & dca & p4 & -346.3 & 73.0 & -33.0 & -52.2 & -1.1 & -359.6 \\ 
   4 & dca & p5 & -362.6 & 74.7 & -33.9 & -52.8 & -0.6 & -375.2 \\ 
   4 & dca & p6 & -354.3 & 74.6 & -34.7 & -52.7 & -0.6 & -367.7 \\ 
   4 & mes & p1 & -397.4 & 90.7 & -60.0 & -65.5 & -2.0 & -434.1 \\ 
   4 & mes & p2 & -387.9 & 86.1 & -56.7 & -58.1 & -1.6 & -418.2 \\ 
   4 & mes & p3 & -399.4 & 95.1 & -62.1 & -56.2 & -2.0 & -424.6 \\ 
   4 & ntf2 & p1 & -298.9 & 61.5 & -35.6 & -66.1 & -1.3 & -340.4 \\ 
   4 & ntf2 & p2 & -308.7 & 61.1 & -39.8 & -61.4 & -0.5 & -349.2 \\ 
   4 & ntf2 & p3 & -298.1 & 65.8 & -33.5 & -61.6 & -1.1 & -328.5 \\ 
   4 & ntf2 & p4 & -318.1 & 66.6 & -36.7 & -62.7 & -1.4 & -352.2 \\ 
   4 & ntf2 & p5 & -301.8 & 58.7 & -36.2 & -55.9 & -0.5 & -335.8 \\ 
   4 & ntf2 & p6 & -313.1 & 60.0 & -37.4 & -56.2 & -0.5 & -347.2 \\ 
   4 & pf6 & p1 & -335.8 & 60.8 & -36.7 & -44.0 & -1.6 & -357.3 \\ 
   4 & pf6 & p2 & -322.9 & 56.6 & -33.9 & -38.4 & -1.4 & -340.0 \\ 
   4 & pf6 & p3 & -336.6 & 59.2 & -35.8 & -38.9 & -1.7 & -353.9 \\ 
   4 & tos & p1 & -387.8 & 90.6 & -60.3 & -59.3 & -1.9 & -418.7 \\ 
   4 & tos & p2 & -410.4 & 84.6 & -56.9 & -55.4 & -1.6 & -439.7 \\ 
   \hline
\end{tabular}
\end{table}

\begin{table}[ht]
\centering
\footnotesize
\caption{EFP energetic components of interaction energy -- Pyrrolidinium halides (aug-cc-pVDZ)}
\begin{tabular}{rllrrrrrr}
  \hline
Chain & Anion & Conf & Electrostatic & Repulsion & Polarization & Dispersion & Charge-tranfer & Total \\ 
  \hline
 1 & br & p1 & -421.6 & 151.1 & -65.3 & -37.1 & -0.7 & -373.8 \\ 
   1 & br & p2 & -425.6 & 145.6 & -62.7 & -33.8 & -0.0 & -376.5 \\ 
   1 & cl & p1 & -436.5 & 138.0 & -64.7 & -34.8 & -0.5 & -398.6 \\ 
   1 & cl & p2 & -442.4 & 128.8 & -68.4 & -33.1 & 0.2 & -414.8 \\ 
   2 & br & p1 & -412.4 & 158.5 & -63.6 & -37.9 & -1.4 & -356.9 \\ 
   2 & br & p2 & -412.8 & 141.5 & -71.4 & -35.1 & 1.4 & -376.3 \\ 
   2 & br & p3 & -415.4 & 150.1 & -63.5 & -34.7 & 0.4 & -363.1 \\ 
   2 & cl & p1 & -428.0 & 137.4 & -69.3 & -35.9 & -1.1 & -397.0 \\ 
   2 & cl & p2 & -424.1 & 133.5 & -62.4 & -32.6 & 1.4 & -384.1 \\ 
   2 & cl & p3 & -435.3 & 128.6 & -72.8 & -33.8 & 0.6 & -412.7 \\ 
   3 & br & p1 & -415.2 & 151.1 & -71.3 & -38.8 & -0.6 & -374.8 \\ 
   3 & br & p2 & -409.3 & 141.5 & -67.7 & -34.5 & 2.6 & -367.4 \\ 
   3 & br & p3 & -415.2 & 144.0 & -73.6 & -35.8 & 0.2 & -380.4 \\ 
   3 & cl & p1 & -434.9 & 137.0 & -68.1 & -35.7 & -0.7 & -402.5 \\ 
   3 & cl & p2 & -422.3 & 125.0 & -73.1 & -33.7 & 1.5 & -402.7 \\ 
   3 & cl & p3 & -434.0 & 129.4 & -71.4 & -33.6 & 0.3 & -409.4 \\ 
   4 & br & p1 & -423.1 & 154.9 & -67.6 & -39.2 & 0.5 & -374.5 \\ 
   4 & br & p2 & -414.8 & 136.2 & -74.1 & -36.1 & 3.4 & -385.5 \\ 
   4 & br & p3 & -419.2 & 149.2 & -70.5 & -36.2 & 0.5 & -376.3 \\ 
   4 & cl & p1 & -435.7 & 134.8 & -70.5 & -36.7 & 0.2 & -407.9 \\ 
   4 & cl & p2 & -421.9 & 122.4 & -75.4 & -34.1 & 1.4 & -407.6 \\ 
   4 & cl & p3 & -433.7 & 130.7 & -71.2 & -33.8 & 0.6 & -407.3 \\ 
   \hline
\end{tabular}
\end{table}


\begin{table}[ht]
\centering
\footnotesize
\caption{EFP energetic components of interaction energy -- Imidazolium TILAs (aug-cc-pVTZ)}
\begin{tabular}{rllrrrrrr}
  \hline
Chain & Anion & Conf & Electrostatic & Repulsion & Polarization & Dispersion & Charge-tranfer & Total \\ 
  \hline
 1 & bf4 & p1 & -392.9 & 71.5 & -32.7 & -55.6 & -2.5 & -412.1 \\ 
   1 & dca & p1 & -394.7 & 116.2 & -29.9 & -80.4 & -3.7 & -392.7 \\ 
   1 & dca & p2 & -366.0 & 102.2 & -42.0 & -52.1 & -9.9 & -367.8 \\ 
   1 & mes & p1 & -440.3 & 106.5 & -50.0 & -79.6 & -3.9 & -467.3 \\ 
   1 & ntf2 & p1 & -361.9 & 107.2 & -31.1 & -96.8 & -2.5 & -385.1 \\ 
   1 & ntf2 & p2 & -355.6 & 106.9 & -46.3 & -51.6 & -10.9 & -357.5 \\ 
   1 & ntf2 & p3 & -357.5 & 87.8 & -39.9 & -67.2 & -4.3 & -381.1 \\ 
   1 & pf6 & p1 & -371.7 & 70.0 & -29.1 & -57.7 & -2.2 & -390.7 \\ 
   1 & tos & p1 & -430.4 & 107.2 & -55.6 & -81.7 & -3.7 & -464.1 \\ 
   2 & bf4 & p1 & -391.6 & 75.2 & -35.9 & -58.6 & -2.9 & -413.8 \\ 
   2 & bf4 & p2 & -387.2 & 71.0 & -35.5 & -54.3 & -2.6 & -408.5 \\ 
   2 & dca & p1 & -388.5 & 120.7 & -35.2 & -82.9 & -5.7 & -391.5 \\ 
   2 & dca & p2 & -385.1 & 106.9 & -34.3 & -78.1 & -6.0 & -396.5 \\ 
   2 & dca & p3 & -388.5 & 120.7 & -35.2 & -81.6 & -5.7 & -390.1 \\ 
   2 & dca & p4 & -390.9 & 112.6 & -34.0 & -81.0 & -5.1 & -398.3 \\ 
   2 & dca & p5 & -391.0 & 117.4 & -31.5 & -84.6 & -3.8 & -393.5 \\ 
   2 & dca & p6 & -382.3 & 113.2 & -29.7 & -84.9 & -3.6 & -387.3 \\ 
   2 & mes & p1 & -431.5 & 110.9 & -53.1 & -86.3 & -4.6 & -464.6 \\ 
   2 & mes & p2 & -436.0 & 106.7 & -52.8 & -84.5 & -3.8 & -470.4 \\ 
   2 & ntf2 & p1 & -359.2 & 112.7 & -32.3 & -109.5 & -2.6 & -391.0 \\ 
   2 & ntf2 & p2 & -349.3 & 101.9 & -48.2 & -63.7 & -9.9 & -369.2 \\ 
   2 & ntf2 & p3 & -358.6 & 92.7 & -43.4 & -73.7 & -4.9 & -387.9 \\ 
   2 & ntf2 & p4 & -349.9 & 99.7 & -32.7 & -99.0 & -2.4 & -384.4 \\ 
   2 & pf6 & p1 & -371.5 & 75.2 & -31.8 & -62.9 & -2.6 & -393.6 \\ 
   2 & pf6 & p2 & -365.4 & 69.1 & -30.6 & -57.9 & -2.1 & -386.9 \\ 
   2 & tos & p1 & -442.7 & 110.3 & -56.5 & -90.4 & -4.4 & -483.6 \\ 
   2 & tos & p2 & -431.5 & 105.5 & -57.7 & -85.0 & -3.9 & -472.5 \\ 
   3 & bf4 & p1 & -388.0 & 76.4 & -38.0 & -59.8 & -3.2 & -412.6 \\ 
   3 & bf4 & p2 & -380.9 & 73.2 & -37.6 & -56.7 & -2.9 & -404.9 \\ 
   3 & dca & p1 & -386.8 & 111.5 & -35.1 & -83.6 & -6.0 & -399.9 \\ 
   3 & dca & p2 & -386.7 & 111.9 & -33.2 & -88.7 & -5.5 & -402.2 \\ 
   3 & dca & p3 & -356.3 & 105.5 & -41.8 & -69.1 & -9.2 & -371.0 \\ 
   3 & dca & p4 & -393.1 & 121.5 & -34.9 & -89.0 & -5.9 & -401.4 \\ 
   3 & dca & p5 & -386.2 & 113.0 & -29.6 & -87.4 & -4.5 & -394.8 \\ 
   3 & dca & p6 & -379.1 & 112.4 & -29.9 & -86.0 & -3.5 & -386.1 \\ 
   3 & mes & p1 & -431.0 & 116.1 & -58.0 & -90.4 & -5.1 & -468.5 \\ 
   3 & mes & p2 & -431.2 & 106.8 & -54.1 & -83.4 & -4.1 & -466.0 \\ 
   3 & ntf2 & p1 & -355.7 & 116.3 & -33.6 & -112.7 & -3.1 & -388.8 \\ 
   3 & ntf2 & p2 & -351.8 & 104.7 & -49.4 & -70.9 & -9.8 & -377.2 \\ 
   3 & ntf2 & p3 & -352.8 & 94.8 & -45.1 & -76.9 & -5.4 & -385.3 \\ 
   3 & ntf2 & p4 & -341.1 & 95.5 & -31.5 & -96.8 & -2.3 & -376.2 \\ 
   3 & pf6 & p1 & -356.3 & 71.9 & -32.2 & -62.0 & -1.8 & -380.4 \\ 
   3 & pf6 & p2 & -360.2 & 68.9 & -31.2 & -58.3 & -2.2 & -382.9 \\ 
   3 & tos & p1 & -433.5 & 114.1 & -60.4 & -93.9 & -4.7 & -478.3 \\ 
   3 & tos & p2 & -418.3 & 111.6 & -65.3 & -92.2 & -4.4 & -468.6 \\ 
   4 & bf4 & p1 & -383.5 & 76.8 & -38.8 & -60.4 & -3.3 & -409.3 \\ 
   4 & bf4 & p2 & -368.5 & 73.2 & -38.8 & -58.2 & -2.9 & -395.3 \\ 
   4 & dca & p1 & -386.9 & 112.0 & -35.7 & -85.3 & -6.0 & -401.9 \\ 
   4 & dca & p2 & -392.3 & 122.0 & -35.5 & -90.7 & -6.1 & -402.6 \\ 
   4 & dca & p3 & -383.5 & 112.2 & -30.2 & -87.4 & -4.3 & -393.2 \\ 
   4 & mes & p1 & -429.3 & 117.0 & -59.4 & -92.1 & -5.4 & -469.1 \\ 
   4 & mes & p2 & -428.4 & 106.7 & -55.2 & -83.9 & -3.9 & -464.6 \\ 
   4 & pf6 & p1 & -363.2 & 78.2 & -34.3 & -66.0 & -3.0 & -388.3 \\ 
   4 & pf6 & p2 & -357.0 & 68.7 & -31.7 & -58.4 & -2.1 & -380.5 \\ 
   4 & tos & p1 & -435.3 & 116.4 & -61.1 & -99.1 & -5.0 & -484.1 \\ 
   4 & tos & p2 & -404.6 & 110.5 & -64.1 & -91.2 & -4.9 & -454.2 \\ 
   \hline
\end{tabular}
\end{table}

\begin{table}[ht]
\centering
\footnotesize
\caption{EFP energetic components of interaction energy -- Imidazolium halides (aug-cc-pVTZ)}
\begin{tabular}{rllrrrrrr}
  \hline
Chain & Anion & Conf & Electrostatic & Repulsion & Polarization & Dispersion & Charge-tranfer & Total \\ 
  \hline
 1 & br & p1 & -472.1 & 209.3 & -69.5 & -55.1 & -4.6 & -391.9 \\ 
   1 & br & p2 & -413.7 & 159.0 & -81.2 & -30.2 & -18.1 & -384.2 \\ 
   1 & cl & p1 & -483.9 & 205.9 & -66.7 & -53.9 & -4.0 & -402.6 \\ 
   1 & cl & p2 & -438.4 & 180.7 & -84.1 & -32.8 & -18.8 & -393.4 \\ 
   2 & br & p1 & -450.7 & 210.4 & -66.1 & -55.5 & -6.2 & -368.1 \\ 
   2 & br & p2 & -396.9 & 157.6 & -102.9 & -34.7 & -21.3 & -398.3 \\ 
   2 & br & p3 & -413.9 & 149.5 & -110.0 & -37.9 & -17.7 & -430.1 \\ 
   2 & br & p4 & -447.5 & 295.4 & -78.5 & -86.8 & -1.0 & -318.4 \\ 
   2 & cl & p1 & -469.9 & 193.6 & -71.5 & -55.4 & -5.6 & -408.8 \\ 
   2 & cl & p2 & -418.2 & 181.3 & -93.0 & -34.4 & -20.2 & -384.5 \\ 
   2 & cl & p3 & -435.5 & 169.5 & -91.0 & -37.1 & -18.1 & -412.2 \\ 
   2 & cl & p4 & -480.0 & 198.7 & -71.4 & -55.2 & -3.7 & -411.6 \\ 
   3 & br & p1 & -447.0 & 210.4 & -68.3 & -58.1 & -7.2 & -370.1 \\ 
   3 & br & p2 & -398.0 & 150.7 & -131.9 & -36.0 & -21.9 & -437.1 \\ 
   3 & br & p3 & -411.6 & 161.7 & -87.8 & -42.0 & -18.3 & -398.1 \\ 
   3 & br & p4 & -435.9 & 215.1 & -75.5 & -61.6 & -7.1 & -364.9 \\ 
   3 & cl & p1 & -466.1 & 198.1 & -70.6 & -54.4 & -6.4 & -399.4 \\ 
   3 & cl & p2 & -424.6 & 161.8 & -135.5 & -36.3 & -20.5 & -455.0 \\ 
   3 & cl & p3 & -430.5 & 178.4 & -82.6 & -42.2 & -18.8 & -395.8 \\ 
   3 & cl & p4 & -455.2 & 208.4 & -73.3 & -56.1 & -6.4 & -382.7 \\ 
   4 & br & p1 & -447.5 & 213.6 & -70.6 & -59.3 & -7.4 & -371.2 \\ 
   4 & br & p2 & -379.0 & 172.7 & -85.4 & -34.7 & -22.9 & -349.2 \\ 
   4 & br & p3 & -409.2 & 178.8 & -74.3 & -43.9 & -18.2 & -366.8 \\ 
   4 & br & p4 & -424.0 & 210.4 & -80.3 & -62.4 & -6.8 & -363.1 \\ 
   4 & cl & p1 & -460.3 & 189.6 & -76.6 & -56.8 & -6.6 & -410.6 \\ 
   4 & cl & p2 & -411.6 & 182.6 & -95.1 & -36.1 & -20.4 & -380.6 \\ 
   4 & cl & p3 & -429.6 & 170.5 & -91.0 & -44.1 & -18.8 & -413.0 \\ 
   4 & cl & p4 & -447.9 & 207.2 & -74.8 & -56.7 & -6.2 & -378.4 \\ 
   \hline
\end{tabular}
\end{table}

\begin{table}[ht]
\centering
\footnotesize
\scriptsize
\caption{EFP energetic components of interaction energy -- Pyrrolidinium TILAs (aug-cc-pVTZ)}
\begin{tabular}{rllrrrrrr}
  \hline
Chain & Anion & Conf & Electrostatic & Repulsion & Polarization & Dispersion & Charge-tranfer & Total \\ 
  \hline
 1 & bf4 & p1 & -376.3 & 63.2 & -37.6 & -43.0 & -4.3 & -398.0 \\ 
   1 & bf4 & p2 & -367.7 & 62.3 & -37.3 & -37.7 & -4.6 & -385.0 \\ 
   1 & dca & p1 & -351.4 & 78.1 & -35.9 & -49.5 & -5.2 & -363.8 \\ 
   1 & dca & p2 & -355.9 & 76.8 & -32.6 & -51.5 & -5.3 & -368.6 \\ 
   1 & dca & p3 & -334.7 & 73.2 & -38.3 & -54.6 & -5.2 & -359.6 \\ 
   1 & dca & p4 & -358.2 & 78.8 & -31.4 & -52.2 & -5.8 & -368.7 \\ 
   1 & mes & p1 & -407.7 & 92.6 & -56.8 & -61.1 & -6.9 & -440.0 \\ 
   1 & mes & p2 & -404.6 & 101.1 & -61.6 & -52.1 & -8.2 & -425.4 \\ 
   1 & ntf2 & p1 & -324.0 & 58.1 & -33.6 & -58.9 & -4.0 & -362.3 \\ 
   1 & ntf2 & p2 & -321.1 & 56.8 & -37.3 & -50.0 & -2.8 & -354.3 \\ 
   1 & ntf2 & p3 & -326.3 & 65.8 & -36.1 & -53.8 & -4.9 & -355.3 \\ 
   1 & ntf2 & p5 & -324.9 & 57.4 & -38.1 & -52.3 & -3.0 & -360.9 \\ 
   1 & pf6 & p1 & -345.8 & 56.0 & -32.8 & -41.8 & -3.4 & -367.7 \\ 
   1 & pf6 & p2 & -342.6 & 55.3 & -32.8 & -36.7 & -3.6 & -360.4 \\ 
   1 & tos & p1 & -402.5 & 91.1 & -62.6 & -62.1 & -6.8 & -442.9 \\ 
   1 & tos & p2 & -393.1 & 85.6 & -60.1 & -53.1 & -6.7 & -427.5 \\ 
   2 & bf4 & p1 & -366.3 & 63.0 & -38.9 & -44.4 & -4.5 & -391.1 \\ 
   2 & bf4 & p2 & -351.3 & 62.9 & -36.9 & -39.3 & -4.6 & -369.2 \\ 
   2 & bf4 & p3 & -362.4 & 64.4 & -39.3 & -38.3 & -4.9 & -380.5 \\ 
   2 & dca & p1 & -342.8 & 77.5 & -36.5 & -49.5 & -5.5 & -356.8 \\ 
   2 & dca & p2 & -345.4 & 76.1 & -32.5 & -52.4 & -6.4 & -360.7 \\ 
   2 & dca & p3 & -351.4 & 76.8 & -33.9 & -52.5 & -5.8 & -366.8 \\ 
   2 & mes & p1 & -400.4 & 90.7 & -57.8 & -63.7 & -6.8 & -437.9 \\ 
   2 & mes & p2 & -380.0 & 84.6 & -53.7 & -56.7 & -7.0 & -412.9 \\ 
   2 & mes & p3 & -396.1 & 96.9 & -61.4 & -54.0 & -8.3 & -422.8 \\ 
   2 & ntf2 & p1 & -313.2 & 53.5 & -32.5 & -61.3 & -3.8 & -357.3 \\ 
   2 & ntf2 & p2 & -322.7 & 58.8 & -37.9 & -53.6 & -3.1 & -358.5 \\ 
   2 & ntf2 & p3 & -305.7 & 55.6 & -31.8 & -57.2 & -4.0 & -343.2 \\ 
   2 & ntf2 & p4 & -319.4 & 61.8 & -35.9 & -58.0 & -4.7 & -356.1 \\ 
   2 & ntf2 & p5 & -307.8 & 55.2 & -35.2 & -55.0 & -3.0 & -345.8 \\ 
   2 & ntf2 & p6 & -317.9 & 55.8 & -37.8 & -52.0 & -2.9 & -354.8 \\ 
   2 & pf6 & p1 & -335.6 & 58.0 & -33.4 & -42.3 & -3.5 & -356.8 \\ 
   2 & pf6 & p2 & -314.6 & 54.3 & -31.9 & -38.6 & -3.5 & -334.3 \\ 
   2 & pf6 & p3 & -336.8 & 55.9 & -34.1 & -38.2 & -3.8 & -357.0 \\ 
   2 & tos & p1 & -397.7 & 89.5 & -63.4 & -65.3 & -6.6 & -443.6 \\ 
   2 & tos & p2 & -372.8 & 82.9 & -57.5 & -58.1 & -6.8 & -412.3 \\ 
   3 & bf4 & p1 & -357.8 & 61.9 & -39.4 & -45.0 & -4.4 & -384.7 \\ 
   3 & bf4 & p2 & -345.4 & 61.4 & -38.6 & -39.2 & -4.7 & -366.5 \\ 
   3 & bf4 & p3 & -359.5 & 64.1 & -40.1 & -39.3 & -4.9 & -379.8 \\ 
   3 & dca & p1 & -336.2 & 76.7 & -35.5 & -51.5 & -5.6 & -352.1 \\ 
   3 & dca & p2 & -341.8 & 75.5 & -33.4 & -53.0 & -6.5 & -359.2 \\ 
   3 & dca & p3 & -318.8 & 65.0 & -34.7 & -53.3 & -5.3 & -347.0 \\ 
   3 & dca & p4 & -335.0 & 72.9 & -34.9 & -51.0 & -5.4 & -353.5 \\ 
   3 & dca & p5 & -346.5 & 75.9 & -34.3 & -54.1 & -5.8 & -364.7 \\ 
   3 & mes & p1 & -393.6 & 87.7 & -57.6 & -65.4 & -7.1 & -435.9 \\ 
   3 & mes & p2 & -377.0 & 84.1 & -55.4 & -57.7 & -7.3 & -413.2 \\ 
   3 & mes & p3 & -392.5 & 96.6 & -62.5 & -56.0 & -8.3 & -422.8 \\ 
   3 & ntf2 & p1 & -312.0 & 55.0 & -33.5 & -64.0 & -3.7 & -358.3 \\ 
   3 & ntf2 & p2 & -306.4 & 60.2 & -37.5 & -53.3 & -3.1 & -340.1 \\ 
   3 & ntf2 & p3 & -307.6 & 57.0 & -33.0 & -59.0 & -4.2 & -346.8 \\ 
   3 & ntf2 & p4 & -318.4 & 62.5 & -37.0 & -60.1 & -4.7 & -357.7 \\ 
   3 & ntf2 & p5 & -305.6 & 54.0 & -35.2 & -55.1 & -3.1 & -345.1 \\ 
   3 & ntf2 & p6 & -306.3 & 55.6 & -37.9 & -53.3 & -2.9 & -344.8 \\ 
   3 & pf6 & p1 & -334.6 & 55.6 & -34.1 & -44.5 & -3.5 & -361.0 \\ 
   3 & pf6 & p2 & -319.1 & 53.5 & -32.8 & -38.8 & -3.6 & -340.9 \\ 
   3 & pf6 & p3 & -334.5 & 55.5 & -34.7 & -38.9 & -3.8 & -356.4 \\ 
   3 & tos & p1 & -392.3 & 87.0 & -61.6 & -65.6 & -6.9 & -439.4 \\ 
   3 & tos & p2 & -369.9 & 82.0 & -59.6 & -58.5 & -7.0 & -413.1 \\ 
   4 & bf4 & p1 & -357.2 & 62.0 & -40.4 & -44.7 & -4.5 & -384.8 \\ 
   4 & bf4 & p2 & -338.4 & 61.1 & -39.2 & -40.0 & -4.7 & -361.3 \\ 
   4 & bf4 & p3 & -357.3 & 64.8 & -41.4 & -39.9 & -5.1 & -378.9 \\ 
   4 & dca & p1 & -337.2 & 75.2 & -33.8 & -53.1 & -6.4 & -355.3 \\ 
   %4 & dca & p2 & 0.0 & 0.0 & 0.0 & 0.0 & 0.0 & 0.0 \\ 
   4 & dca & p3 & -345.4 & 75.6 & -34.8 & -53.3 & -5.9 & -363.6 \\ 
   4 & dca & p4 & -336.6 & 75.3 & -33.8 & -53.2 & -6.4 & -354.8 \\ 
   4 & dca & p5 & -350.2 & 76.5 & -34.0 & -53.9 & -6.3 & -367.9 \\ 
   4 & dca & p6 & -345.4 & 75.6 & -34.8 & -53.6 & -5.9 & -364.0 \\ 
   4 & mes & p1 & -389.9 & 87.4 & -58.3 & -65.9 & -7.2 & -433.9 \\ 
   4 & mes & p2 & -371.9 & 83.5 & -56.0 & -58.6 & -7.3 & -410.3 \\ 
   4 & mes & p3 & -390.8 & 96.6 & -63.9 & -56.7 & -8.5 & -423.3 \\ 
   4 & ntf2 & p1 & -309.0 & 55.8 & -34.3 & -64.8 & -3.8 & -356.2 \\ 
   4 & ntf2 & p2 & -308.2 & 56.3 & -38.2 & -60.5 & -2.9 & -353.5 \\ 
   4 & ntf2 & p3 & -305.9 & 59.8 & -34.0 & -60.8 & -4.5 & -345.3 \\ 
   4 & ntf2 & p4 & -316.0 & 62.4 & -37.8 & -62.3 & -4.7 & -358.5 \\ 
   4 & ntf2 & p5 & -300.9 & 53.6 & -35.0 & -55.3 & -3.1 & -340.8 \\ 
   4 & ntf2 & p6 & -304.7 & 56.0 & -39.3 & -55.3 & -3.0 & -346.3 \\ 
   4 & pf6 & p1 & -332.0 & 55.8 & -34.9 & -44.9 & -3.5 & -359.6 \\ 
   4 & pf6 & p2 & -312.7 & 53.3 & -33.3 & -39.6 & -3.6 & -335.9 \\ 
   4 & pf6 & p3 & -331.6 & 55.3 & -35.8 & -39.6 & -3.8 & -355.6 \\ 
   4 & tos & p1 & -384.3 & 87.2 & -62.7 & -67.3 & -7.0 & -434.1 \\ 
   4 & tos & p2 & -358.9 & 81.3 & -60.0 & -59.6 & -7.1 & -404.2 \\ 
   \hline
\end{tabular}
\end{table}

\begin{table}[ht]
\centering
\footnotesize
\caption{EFP energetic components of interaction energy -- Pyrrolidinium halides (aug-cc-pVTZ)}
\begin{tabular}{rllrrrrrr}
  \hline
Chain & Anion & Conf & Electrostatic & Repulsion & Polarization & Dispersion & Charge-tranfer & Total \\ 
  \hline
 1 & br & p1 & -409.4 & 157.8 & -73.7 & -41.6 & -10.7 & -377.6 \\ 
   1 & br & p2 & -404.9 & 152.5 & -72.4 & -38.0 & -11.2 & -373.9 \\ 
   1 & cl & p1 & -414.9 & 138.7 & -73.5 & -38.9 & -9.3 & -397.9 \\ 
   1 & cl & p2 & -414.6 & 138.7 & -71.4 & -35.8 & -9.8 & -393.0 \\ 
   2 & br & p1 & -401.7 & 155.5 & -80.6 & -43.0 & -11.8 & -381.5 \\ 
   2 & br & p2 & -378.8 & 156.3 & -72.1 & -37.9 & -11.8 & -344.3 \\ 
   2 & br & p3 & -400.4 & 154.2 & -76.6 & -39.0 & -12.0 & -373.7 \\ 
   2 & cl & p1 & -407.6 & 136.0 & -81.1 & -40.4 & -10.2 & -403.3 \\ 
   2 & cl & p2 & -389.1 & 135.0 & -72.9 & -36.3 & -10.3 & -373.6 \\ 
   2 & cl & p3 & -409.0 & 137.6 & -74.1 & -36.8 & -10.5 & -392.8 \\ 
   3 & br & p1 & -388.2 & 159.5 & -73.8 & -43.5 & -12.4 & -358.4 \\ 
   3 & br & p2 & -378.6 & 150.4 & -73.7 & -39.3 & -12.4 & -353.7 \\ 
   3 & br & p3 & -397.2 & 157.9 & -72.1 & -39.6 & -12.2 & -363.2 \\ 
   3 & cl & p1 & -401.8 & 143.2 & -69.2 & -40.1 & -10.7 & -378.5 \\ 
   3 & cl & p2 & -388.1 & 132.4 & -81.1 & -37.2 & -10.8 & -384.7 \\ 
   3 & cl & p3 & -405.4 & 134.6 & -78.4 & -37.9 & -10.5 & -397.6 \\ 
   4 & br & p1 & -390.2 & 159.3 & -76.3 & -44.5 & -12.5 & -364.1 \\ 
   4 & br & p2 & -372.8 & 152.2 & -70.2 & -39.6 & -12.5 & -343.0 \\ 
   4 & br & p3 & -393.9 & 156.4 & -79.5 & -40.5 & -12.8 & -370.2 \\ 
   4 & cl & p1 & -398.8 & 141.1 & -71.9 & -40.1 & -10.6 & -380.3 \\ 
   4 & cl & p2 & -379.8 & 137.9 & -73.7 & -37.2 & -11.4 & -364.1 \\ 
   4 & cl & p3 & -402.1 & 137.3 & -76.7 & -37.5 & -10.7 & -389.7 \\ 
   \hline
\end{tabular}
\end{table}


\begin{table}[ht]
\centering
\footnotesize
\caption{EFP energetic components of interaction energy -- Imidazolium TILAs (6-311++G(d,p))}
\begin{tabular}{rllrrrrrr}
  \hline
Chain & Anion & Conf & Electrostatic & Repulsion & Polarization & Dispersion & Charge-tranfer & Total \\ 
  \hline
 1 & bf4 & p1 & -386.4 & 56.1 & -30.9 & -43.0 & -1.1 & -405.2 \\ 
   1 & dca & p1 & -382.4 & 99.7 & -25.4 & -71.7 & -3.1 & -382.9 \\ 
   1 & dca & p2 & -361.3 & 89.5 & -35.4 & -45.6 & -6.5 & -359.3 \\ 
   1 & mes & p1 & -437.6 & 84.3 & -45.3 & -69.8 & -3.6 & -472.0 \\ 
   1 & ntf2 & p1 & -368.4 & 77.8 & -30.3 & -90.7 & -2.5 & -414.0 \\ 
   1 & ntf2 & p2 & -379.3 & 97.4 & -39.5 & -47.3 & -8.5 & -377.2 \\ 
   1 & ntf2 & p3 & -353.0 & 65.9 & -39.3 & -61.7 & -3.0 & -391.1 \\ 
   1 & pf6 & p1 & -362.9 & 53.2 & -27.4 & -42.0 & -1.4 & -380.5 \\ 
   %1 & tos & p1 &  &  &  &  &  &  \\ 
   2 & bf4 & p1 & -381.6 & 58.9 & -34.6 & -45.8 & -1.2 & -404.2 \\ 
   2 & bf4 & p2 & -376.2 & 54.7 & -32.3 & -43.0 & -1.2 & -398.0 \\ 
   2 & dca & p1 & -382.4 & 102.9 & -29.9 & -70.2 & -3.5 & -383.1 \\ 
   2 & dca & p2 & -368.5 & 91.9 & -30.6 & -67.4 & -4.5 & -379.1 \\ 
   2 & dca & p3 & -382.4 & 102.9 & -29.9 & -70.2 & -3.5 & -383.1 \\ 
   2 & dca & p4 & -375.3 & 95.2 & -28.3 & -69.6 & -3.9 & -382.0 \\ 
   2 & dca & p5 & -376.3 & 100.7 & -26.3 & -72.3 & -3.1 & -377.3 \\ 
   2 & dca & p6 & -370.7 & 96.9 & -25.7 & -73.1 & -3.1 & -375.8 \\ 
   2 & mes & p1 & -427.4 & 89.5 & -52.3 & -73.4 & -3.8 & -467.4 \\ 
   2 & mes & p2 & -416.9 & 84.1 & -46.9 & -71.3 & -3.6 & -454.7 \\ 
   2 & ntf2 & p1 & -369.9 & 83.7 & -32.6 & -99.0 & -2.7 & -420.5 \\ 
   2 & ntf2 & p2 & -344.8 & 86.6 & -43.9 & -60.1 & -7.7 & -369.8 \\ 
   2 & ntf2 & p3 & -348.1 & 69.5 & -42.3 & -65.6 & -3.4 & -389.9 \\ 
   2 & ntf2 & p4 & -336.5 & 69.9 & -31.0 & -90.4 & -2.4 & -390.5 \\ 
   2 & pf6 & p1 & -360.2 & 58.0 & -30.5 & -46.3 & -1.6 & -380.6 \\ 
   2 & pf6 & p2 & -353.1 & 51.9 & -28.4 & -42.6 & -1.4 & -373.6 \\ 
   %2 & tos & p1 &  &  &  &  &  &  \\ 
   %2 & tos & p2 &  &  &  &  &  &  \\ 
   3 & bf4 & p1 & -380.1 & 60.0 & -36.7 & -47.1 & -1.0 & -404.8 \\ 
   3 & bf4 & p2 & -372.0 & 57.1 & -35.0 & -45.6 & -1.2 & -396.7 \\ 
   3 & dca & p1 & -367.4 & 96.3 & -31.2 & -71.9 & -4.3 & -378.5 \\ 
   3 & dca & p2 & -370.2 & 95.2 & -28.8 & -75.2 & -2.8 & -381.7 \\ 
   3 & dca & p3 & -358.2 & 92.3 & -35.5 & -59.5 & -5.0 & -365.9 \\ 
   3 & dca & p4 & -380.7 & 102.7 & -31.4 & -76.2 & -4.0 & -389.7 \\ 
   3 & dca & p5 & -371.4 & 95.9 & -25.0 & -76.6 & -3.0 & -380.1 \\ 
   3 & dca & p6 & -367.4 & 96.2 & -25.9 & -73.9 & -3.2 & -374.3 \\ 
   3 & mes & p1 & -435.0 & 94.2 & -56.8 & -76.7 & -3.7 & -477.9 \\ 
   3 & mes & p2 & -422.0 & 84.3 & -48.2 & -72.3 & -3.8 & -462.0 \\ 
   3 & ntf2 & p1 & -377.7 & 86.6 & -34.0 & -103.5 & -2.5 & -431.1 \\ 
   3 & ntf2 & p2 & -344.8 & 87.3 & -45.1 & -66.0 & -7.1 & -375.7 \\ 
   3 & ntf2 & p3 & -350.8 & 71.8 & -43.7 & -68.0 & -3.4 & -394.2 \\ 
   3 & ntf2 & p4 & -332.4 & 65.3 & -30.2 & -92.2 & -2.5 & -392.0 \\ 
   3 & pf6 & p1 & -342.4 & 52.0 & -29.3 & -46.9 & -0.8 & -367.4 \\ 
   3 & pf6 & p2 & -346.7 & 51.6 & -28.9 & -43.0 & -1.4 & -368.4 \\ 
   %3 & tos & p1 &  &  &  &  &  &  \\ 
   %3 & tos & p2 &  &  &  &  &  &  \\ 
   4 & bf4 & p1 & -378.7 & 60.2 & -38.2 & -47.8 & -1.0 & -405.5 \\ 
   4 & bf4 & p2 & -375.0 & 57.0 & -36.0 & -45.9 & -1.1 & -401.0 \\ 
   4 & dca & p1 & -368.3 & 96.5 & -32.2 & -73.0 & -4.2 & -381.3 \\ 
   4 & dca & p2 & -381.3 & 102.9 & -32.4 & -78.2 & -3.9 & -392.9 \\ 
   4 & dca & p3 & -371.1 & 95.3 & -25.1 & -77.1 & -2.9 & -380.9 \\ 
   4 & mes & p1 & -435.5 & 94.8 & -58.8 & -78.5 & -3.8 & -481.8 \\ 
   4 & mes & p2 & -416.7 & 84.3 & -48.9 & -72.7 & -3.6 & -457.7 \\ 
   4 & pf6 & p1 & -359.6 & 60.6 & -33.6 & -49.7 & -1.5 & -383.8 \\ 
   4 & pf6 & p2 & -346.5 & 51.3 & -29.5 & -43.2 & -1.3 & -369.3 \\ 
   %4 & tos & p1 &  &  &  &  &  &  \\ 
   %4 & tos & p2 &  &  &  &  &  &  \\ 
   \hline
\end{tabular}
\end{table}

\begin{table}[ht]
\centering
\footnotesize
\caption{EFP energetic components of interaction energy -- Imidazolium halides (6-311++G(d,p))}
\begin{tabular}{rllrrrrrr}
  \hline
Chain & Anion & Conf & Electrostatic & Repulsion & Polarization & Dispersion & Charge-tranfer & Total \\ 
  \hline
 1 & br & p1 & -440.7 & 166.8 & -44.4 & -27.3 & -4.7 & -350.2 \\ 
   1 & br & p2 & -413.1 & 115.0 & -48.8 & -15.5 & -11.1 & -373.5 \\ 
   1 & cl & p1 & -466.4 & 200.0 & -59.7 & -29.6 & -5.4 & -361.1 \\ 
   1 & cl & p2 & -436.4 & 168.9 & -67.0 & -18.1 & -11.3 & -363.8 \\ 
   2 & br & p1 & -421.5 & 145.3 & -49.4 & -28.4 & -4.5 & -358.4 \\ 
   2 & br & p2 & -410.1 & 118.9 & -51.1 & -16.9 & -11.0 & -370.2 \\ 
   2 & br & p3 & -406.6 & 121.8 & -45.6 & -18.2 & -10.8 & -359.4 \\ 
   2 & br & p4 & -438.0 & 162.3 & -46.7 & -27.8 & -5.9 & -356.1 \\ 
   2 & cl & p1 & -446.4 & 196.3 & -58.9 & -29.3 & -5.2 & -343.5 \\ 
   2 & cl & p2 & -432.0 & 177.3 & -65.1 & -18.9 & -11.3 & -349.9 \\ 
   2 & cl & p3 & -433.4 & 171.5 & -61.0 & -20.6 & -10.3 & -353.8 \\ 
   2 & cl & p4 & -458.3 & 197.6 & -58.2 & -29.6 & -7.1 & -355.5 \\ 
   3 & br & p1 & -428.3 & 151.0 & -50.9 & -28.3 & -3.8 & -360.3 \\ 
   3 & br & p2 & -391.6 & 137.2 & -46.7 & -16.8 & -9.7 & -327.7 \\ 
   3 & br & p3 & -408.1 & 114.7 & -52.7 & -20.8 & -9.2 & -376.1 \\ 
   3 & br & p4 & -454.5 & 158.5 & -49.9 & -30.4 & -5.5 & -381.6 \\ 
   3 & cl & p1 & -445.5 & 196.3 & -60.3 & -29.8 & -4.2 & -343.5 \\ 
   3 & cl & p2 & -416.3 & 162.1 & -82.4 & -19.8 & -9.7 & -366.1 \\ 
   3 & cl & p3 & -432.5 & 176.9 & -61.6 & -22.8 & -8.8 & -348.9 \\ 
   3 & cl & p4 & -473.7 & 210.7 & -57.9 & -31.1 & -5.9 & -357.9 \\ 
   4 & br & p1 & -423.6 & 152.6 & -52.1 & -28.8 & -3.9 & -355.7 \\ 
   4 & br & p2 & -371.4 & 126.3 & -49.1 & -17.4 & -8.1 & -319.8 \\ 
   4 & br & p3 & -399.1 & 122.9 & -51.6 & -22.0 & -9.1 & -358.9 \\ 
   4 & br & p4 & -460.0 & 162.7 & -50.5 & -30.4 & -6.7 & -384.9 \\ 
   4 & cl & p1 & -445.2 & 188.6 & -67.5 & -30.6 & -4.1 & -358.8 \\ 
   4 & cl & p2 & -394.9 & 180.1 & -61.5 & -19.5 & -7.9 & -303.7 \\ 
   4 & cl & p3 & -427.1 & 170.3 & -70.6 & -24.1 & -8.5 & -360.0 \\ 
   4 & cl & p4 & -487.5 & 201.3 & -62.3 & -31.6 & -7.7 & -387.9 \\ 
   \hline
\end{tabular}
\end{table}

\begin{table}[ht]
\centering
\scriptsize
\caption{EFP energetic components of interaction energy -- Pyrrolidinium TILAs (6-311++G(d,p))}
\begin{tabular}{rllrrrrrr}
  \hline
Chain & Anion & Conf & Electrostatic & Repulsion & Polarization & Dispersion & Charge-tranfer & Total \\ 
  \hline
 1 & bf4 & p1 & -367.0 & 50.0 & -33.9 & -34.2 & -2.4 & -387.5 \\ 
   1 & bf4 & p2 & -363.6 & 49.3 & -33.1 & -29.9 & -2.5 & -379.7 \\ 
   1 & dca & p1 & -347.6 & 67.0 & -32.7 & -42.8 & -3.5 & -359.5 \\ 
   1 & dca & p2 & -352.2 & 66.0 & -28.6 & -45.2 & -3.3 & -363.4 \\ 
   1 & dca & p3 & -333.4 & 63.2 & -37.7 & -48.9 & -3.6 & -360.4 \\ 
   1 & dca & p4 & -354.3 & 67.3 & -27.3 & -44.8 & -3.7 & -362.9 \\ 
   1 & mes & p1 & -412.6 & 75.5 & -52.6 & -53.2 & -5.4 & -448.3 \\ 
   1 & mes & p2 & -411.3 & 85.7 & -54.8 & -44.4 & -6.6 & -431.4 \\ 
   1 & ntf2 & p1 & -329.1 & 41.1 & -32.7 & -54.9 & -2.9 & -378.5 \\ 
   1 & ntf2 & p2 & -318.8 & 42.7 & -37.4 & -46.1 & -2.0 & -361.5 \\ 
   1 & ntf2 & p3 & -329.2 & 47.4 & -34.3 & -51.8 & -3.5 & -371.4 \\ 
   1 & ntf2 & p5 & -322.3 & 41.9 & -37.4 & -47.0 & -1.9 & -366.7 \\ 
   1 & pf6 & p1 & -343.7 & 41.0 & -29.3 & -31.5 & -2.2 & -365.7 \\ 
   1 & pf6 & p2 & -341.3 & 40.1 & -28.7 & -28.0 & -2.2 & -360.1 \\ 
   1 & tos & p1 & -426.9 & 72.6 & -49.6 & -56.3 & -5.1 & -465.4 \\ 
   1 & tos & p2 & -427.5 & 69.1 & -53.1 & -47.7 & -5.0 & -464.2 \\ 
   2 & bf4 & p1 & -360.3 & 49.7 & -35.5 & -35.3 & -2.3 & -383.6 \\ 
   2 & bf4 & p2 & -354.9 & 50.8 & -32.7 & -31.3 & -2.5 & -370.6 \\ 
   2 & bf4 & p3 & -359.4 & 51.2 & -34.5 & -31.0 & -2.5 & -376.2 \\ 
   2 & dca & p1 & -340.6 & 66.8 & -33.1 & -43.3 & -3.3 & -353.5 \\ 
   2 & dca & p2 & -343.7 & 63.7 & -27.9 & -45.5 & -3.6 & -357.1 \\ 
   2 & dca & p3 & -347.0 & 65.9 & -29.2 & -46.0 & -3.3 & -359.6 \\ 
   2 & mes & p1 & -404.9 & 73.3 & -53.7 & -55.3 & -5.3 & -446.0 \\ 
   2 & mes & p2 & -392.5 & 69.8 & -48.8 & -49.1 & -5.0 & -425.6 \\ 
   2 & mes & p3 & -406.6 & 81.1 & -54.1 & -47.0 & -6.2 & -432.8 \\ 
   2 & ntf2 & p1 & -320.5 & 37.3 & -31.9 & -57.4 & -2.4 & -374.8 \\ 
   2 & ntf2 & p2 & -315.6 & 44.9 & -38.1 & -48.3 & -2.0 & -359.2 \\ 
   2 & ntf2 & p3 & -315.6 & 38.0 & -30.1 & -54.7 & -2.5 & -364.9 \\ 
   2 & ntf2 & p4 & -323.6 & 43.7 & -34.0 & -53.1 & -3.3 & -370.2 \\ 
   2 & ntf2 & p5 & -311.5 & 40.3 & -34.3 & -49.3 & -1.7 & -356.5 \\ 
   2 & ntf2 & p6 & -311.4 & 41.3 & -37.7 & -47.2 & -1.8 & -356.7 \\ 
   2 & pf6 & p1 & -337.8 & 43.6 & -30.1 & -32.1 & -2.1 & -358.6 \\ 
   2 & pf6 & p2 & -329.1 & 40.9 & -28.0 & -29.0 & -2.1 & -347.4 \\ 
   2 & pf6 & p3 & -337.5 & 40.9 & -29.6 & -28.8 & -2.2 & -357.2 \\ 
   %2 & tos & p1 &  &  &  &  &  &  \\ 
   2 & tos & p2 & -407.8 & 67.1 & -46.3 & -52.1 & -4.7 & -443.8 \\ 
   3 & bf4 & p1 & -355.0 & 48.7 & -36.4 & -36.0 & -2.0 & -380.6 \\ 
   3 & bf4 & p2 & -352.4 & 48.8 & -33.9 & -31.6 & -2.3 & -371.5 \\ 
   3 & bf4 & p3 & -357.8 & 50.9 & -34.8 & -31.6 & -2.3 & -375.6 \\ 
   3 & dca & p1 & -338.3 & 66.7 & -31.0 & -44.8 & -2.6 & -350.0 \\ 
   3 & dca & p2 & -341.9 & 63.1 & -28.6 & -46.0 & -3.3 & -356.6 \\ 
   3 & dca & p3 & -317.5 & 54.3 & -32.0 & -48.4 & -2.8 & -346.4 \\ 
   3 & dca & p4 & -335.1 & 62.4 & -29.9 & -44.6 & -2.7 & -349.9 \\ 
   3 & dca & p5 & -343.7 & 65.1 & -29.5 & -46.4 & -3.0 & -357.5 \\ 
   3 & mes & p1 & -398.3 & 70.6 & -54.0 & -56.7 & -4.5 & -442.9 \\ 
   3 & mes & p2 & -390.4 & 69.0 & -49.7 & -49.8 & -4.7 & -425.6 \\ 
   3 & mes & p3 & -398.7 & 80.7 & -54.2 & -48.4 & -5.7 & -426.3 \\ 
   3 & ntf2 & p1 & -319.5 & 37.5 & -33.3 & -59.8 & -2.2 & -377.4 \\ 
   3 & ntf2 & p2 & -309.0 & 46.6 & -38.0 & -48.3 & -2.0 & -350.7 \\ 
   3 & ntf2 & p3 & -313.4 & 38.9 & -30.6 & -55.4 & -2.5 & -363.1 \\ 
   3 & ntf2 & p4 & -314.5 & 44.1 & -34.3 & -56.0 & -3.0 & -363.7 \\ 
   3 & ntf2 & p5 & -312.8 & 39.1 & -33.9 & -49.6 & -1.6 & -358.8 \\ 
   3 & ntf2 & p6 & -308.4 & 42.0 & -36.6 & -48.8 & -1.7 & -353.4 \\ 
   3 & pf6 & p1 & -334.4 & 40.6 & -31.2 & -33.7 & -1.8 & -360.4 \\ 
   3 & pf6 & p2 & -325.0 & 39.9 & -28.5 & -29.2 & -2.0 & -344.7 \\ 
   3 & pf6 & p3 & -331.4 & 40.6 & -29.6 & -29.6 & -2.0 & -351.9 \\ 
   3 & tos & p1 & -418.2 & 68.8 & -53.7 & -60.0 & -4.3 & -467.5 \\ 
   3 & tos & p2 & -411.4 & 66.1 & -47.5 & -52.7 & -4.4 & -449.9 \\ 
   4 & bf4 & p1 & -356.4 & 48.7 & -37.4 & -36.3 & -1.9 & -383.2 \\ 
   4 & bf4 & p2 & -345.5 & 48.6 & -34.6 & -32.0 & -2.2 & -365.7 \\ 
   4 & bf4 & p3 & -356.6 & 51.6 & -35.5 & -32.0 & -2.4 & -375.0 \\ 
   4 & dca & p1 & -338.8 & 62.8 & -28.9 & -46.1 & -3.2 & -354.2 \\ 
   4 & dca & p2 & -340.0 & 63.5 & -31.1 & -46.2 & -2.2 & -356.0 \\ 
   4 & dca & p3 & -341.4 & 64.9 & -29.9 & -46.6 & -3.0 & -356.1 \\ 
   4 & dca & p4 & -338.9 & 62.8 & -28.9 & -46.1 & -3.2 & -354.3 \\ 
   4 & dca & p5 & -344.4 & 65.3 & -28.7 & -46.5 & -3.2 & -357.6 \\ 
   4 & dca & p6 & -341.4 & 64.9 & -29.9 & -46.6 & -3.0 & -356.1 \\ 
   4 & mes & p1 & -398.4 & 70.3 & -54.9 & -56.9 & -4.4 & -444.3 \\ 
   4 & mes & p2 & -381.4 & 68.5 & -50.4 & -50.5 & -4.7 & -418.5 \\ 
   4 & mes & p3 & -398.6 & 80.8 & -54.9 & -48.9 & -5.8 & -427.4 \\ 
   4 & ntf2 & p1 & -320.7 & 38.3 & -34.5 & -60.4 & -2.3 & -379.6 \\ 
   4 & ntf2 & p2 & -321.5 & 42.7 & -38.8 & -55.7 & -1.5 & -374.7 \\ 
   4 & ntf2 & p3 & -309.2 & 41.4 & -32.1 & -56.9 & -2.6 & -359.3 \\ 
   4 & ntf2 & p4 & -319.0 & 44.0 & -34.5 & -57.5 & -3.0 & -370.0 \\ 
   4 & ntf2 & p5 & -305.9 & 38.7 & -34.0 & -50.6 & -1.5 & -353.3 \\ 
   4 & ntf2 & p6 & -303.9 & 42.1 & -37.1 & -50.9 & -1.8 & -351.6 \\ 
   4 & pf6 & p1 & -334.4 & 40.8 & -32.0 & -33.9 & -1.8 & -361.3 \\ 
   4 & pf6 & p2 & -318.4 & 39.8 & -29.0 & -29.8 & -1.9 & -339.4 \\ 
   4 & pf6 & p3 & -329.2 & 40.5 & -30.1 & -30.0 & -2.0 & -350.9 \\ 
   %4 & tos & p1 &  &  &  &  &  &  \\ 
   %4 & tos & p2 &  &  &  &  &  &  \\ 
   \hline
\end{tabular}
\end{table}

\begin{table}[ht]
\centering
\footnotesize
\caption{EFP energetic components of interaction energy -- Pyrrolidinium halides (6-311++G(d,p))}
\begin{tabular}{rllrrrrrr}
  \hline
Chain & Anion & Conf & Electrostatic & Repulsion & Polarization & Dispersion & Charge-tranfer & Total \\ 
  \hline
 1 & br & p1 & -399.8 & 102.4 & -48.9 & -20.9 & -7.1 & -374.3 \\ 
   1 & br & p2 & -400.8 & 107.2 & -43.4 & -18.5 & -6.5 & -362.0 \\ 
   1 & cl & p1 & -411.9 & 134.0 & -57.9 & -21.2 & -6.6 & -363.6 \\ 
   1 & cl & p2 & -411.9 & 131.2 & -59.0 & -20.0 & -6.1 & -365.7 \\ 
   2 & br & p1 & -395.6 & 112.7 & -45.9 & -20.9 & -7.0 & -356.8 \\ 
   2 & br & p2 & -384.4 & 101.7 & -46.9 & -19.2 & -6.4 & -355.3 \\ 
   2 & br & p3 & -396.2 & 104.8 & -46.2 & -19.4 & -6.4 & -363.5 \\ 
   2 & cl & p1 & -405.2 & 136.0 & -57.5 & -21.9 & -6.6 & -355.1 \\ 
   2 & cl & p2 & -394.3 & 130.8 & -54.1 & -19.9 & -5.9 & -343.5 \\ 
   2 & cl & p3 & -405.4 & 134.2 & -54.1 & -20.0 & -6.0 & -351.3 \\ 
   3 & br & p1 & -392.1 & 103.3 & -51.4 & -22.2 & -6.1 & -368.5 \\ 
   3 & br & p2 & -385.9 & 97.1 & -46.6 & -19.7 & -5.7 & -360.9 \\ 
   3 & br & p3 & -391.2 & 103.7 & -47.3 & -20.1 & -5.4 & -360.2 \\ 
   3 & cl & p1 & -402.7 & 136.6 & -56.7 & -22.2 & -5.4 & -350.4 \\ 
   3 & cl & p2 & -394.0 & 129.4 & -56.2 & -20.3 & -5.3 & -346.5 \\ 
   3 & cl & p3 & -400.9 & 125.0 & -62.1 & -20.9 & -5.1 & -364.0 \\ 
   4 & br & p1 & -394.2 & 112.2 & -49.0 & -21.9 & -5.7 & -358.6 \\ 
   4 & br & p2 & -372.7 & 96.4 & -46.8 & -19.9 & -5.2 & -348.2 \\ 
   4 & br & p3 & -393.5 & 103.6 & -47.9 & -20.3 & -5.8 & -363.8 \\ 
   4 & cl & p1 & -402.6 & 130.2 & -68.7 & -22.8 & -4.8 & -368.8 \\ 
   4 & cl & p2 & -387.5 & 128.1 & -58.7 & -20.6 & -5.8 & -344.5 \\ 
   4 & cl & p3 & -401.8 & 133.8 & -54.7 & -20.5 & -5.4 & -348.6 \\ 
   \hline
\end{tabular}
\end{table}

\end{document}


%\renewcommand{\bibfont}{\footnotesize}
%\printbibliography

% for natbib
\clearpage
\bibliography{sapt_efp}

\end{document}

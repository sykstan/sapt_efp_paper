
% section: method

\subsubsection{Chemical systems studied}
The chemical systems studied were single ion pairs of ionic liquids. 
Routinely used anions  such as tetrafluoroborate (\bfl), bromide (\br), chloride (\cl), dicyanamide (\dca), mesylate (\mes), tosylate (\tos), hexafluorophosphate (\pf) and bis\{(tri\-fluoro\-meth\-yl)\-sulf\-onyl\}\-amide (\ntf) were used in this study. 
These anions were combined with N-alkyl-N'-pyrrolidinium (denoted here as $ \text{C}_n\text{mpyr}$) and 1-methyl-3-alkyl-imidazolium (denoted here as $ \text{C}_n\text{mim}$) cations with varying alkyl chain from methyl, ethyl, propyl to butyl.
The names and abbreviations of the different cations and anions are tabulated in 
Table \ref{tab:cation-anion-list}. Further in the text, halides are abbreviated as "Hal", whereas the rest of the anions are referred to as typical ionic liquid anions (TILA).

\begin{table}[ht]
    \begin{centering}
    \footnotesize
    \begin{tabular}{c|c|c|c}
        \hline 
        Cations  & abbreviation & Anions & abbreviation\tabularnewline
        \hline 
        1-methyl-3-methyl-imidazolium & $\text{C}_{1}\text{mim}^{+}$ & tetrafluoroborate &  $\text{BF}_{\text{4}}^{-}$\tabularnewline
        1-methyl-3-ethyl-imidazolium & $\text{C}_{2}\text{mim}^{+}$ & bromide &  $\text{Br}^{-}$\tabularnewline
        1-methyl-3-propyl-imidazolium & $\text{C}_{3}\text{mim}^{+}$ & chloride  & $\text{Cl}^{-}$\tabularnewline
        1-methyl-3-butyl-imidazolium & $\text{C}_{4}\text{mim}^{+}$ & dicyanamide  & $\text{Dca}^{-}$\tabularnewline
        N,N’-dimethyl-pyrrolidinium & $\text{C}_{1}\text{mpyr}^{+}$ & mesylate &  $\text{Mes}^{-}$\tabularnewline
        N-ethyl-N'-methyl-pyrrolidinium & $\text{C}_{2}\text{mpyr}^{+}$ & bis\{(trifluoromethyl)sulfonyl\}amide  & $\text{NTf}_{\text{2}}^{-}$\tabularnewline
        N-propyl-N'-methyl-pyrrolidinium & $\text{C}_{3}\text{mpyr}^{+}$ & hexafluorophosphate &  $\text{PF}_{\text{6}}^{-}$\tabularnewline
        N-butyl-N'-methyl-pyrrolidinium & $\text{C}_{4}\text{mpyr}^{+}$ & tosylate  & $\text{Tos}^{-}$\tabularnewline
        \hline 
    \end{tabular}
    \caption{List of cations and anions }
    \label{tab:cation-anion-list}
    \par\end{centering}
\end{table}
%\begin{multicols}{2}


For each cation-anion combination, different configurations of these ion pairs were incorporated. 
These configurations differ by how the anion interacts with the cation. 
In the imidazolium-based cation  the anion can interact with the cation above and below the imidazolium ring and these configurations are referred to as p1 and p4, respectively.
When the anion interacts with the cation in the $\text{C}_2\text{--H}$ bond plane this configuration is referred to as in-plane interactions and denoted further in the text as p2 and p3.
The different configurations for \ipair{mim}{3}{br} are presented in Figure
\protect\ref{fig:conf-c3mim-br}
as an example. 

%\end{multicols}

% original graphics located at ~/Dropbox/QuantumChem/il_structure_images
\begin{figure}
    \centering
    \mbox{
    \subfigure[\protect\ipair{mim}{3}{br} (p1) ]{\includegraphics[scale=0.3]{./images/c3mim-br-p1.pdf}}
    \subfigure[\protect\ipair{mim}{3}{br} (p2) ]{\includegraphics[scale=0.3]{./images/c3mim-br-p2.pdf}}
    }
    \mbox{
    \subfigure[\protect\ipair{mim}{3}{br} (p3) ]{\includegraphics[scale=0.3]{./images/c3mim-br-p3.pdf}}
    \subfigure[\protect\ipair{mim}{3}{br} (p4) ]{\includegraphics[scale=0.3]{./images/c3mim-br-p4.pdf}}
    }                                 
    % need the \protect to make hyperref and the macro happy together
    \caption{Different configurations of \protect\ipair{mim}{3}{br} \label{fig:conf-c3mim-br}}
\end{figure}

For the case of pyrrolidinium-based ion pairs, the configurations studied are different as the anion tends to interact with the nitrogen centre of the cation from three energetically domineering positions denoted here as p1, p2 and p3.
\cite{Izgorodina2014a}
The \ntf anion has multiple interaction sites such as the central nitrogen and the oxygens on the sulfonyl groups, as shown in Figure \ref{fig:conf-c2mpyr-ntf2}.
Therefore, there are more ion pair configurations corresponding to the anion interacting through different centres compared to halides.
In the case of the \ntf anion there are six possible configurations (for more detail see Figure \ref{fig:conf-c2mpyr-ntf2}). 


% original graphics located at ~/Dropbox/QuantumChem/il_structure_images
\begin{figure}
    \centering
    \mbox{
    \subfigure[\protect\ipair{mpyr}{2}{ntf} (p1) ]{\includegraphics[scale=0.27]{./images/c2mpyr-ntf2-p1.pdf}}
    \subfigure[\protect\ipair{mpyr}{2}{ntf} (p2)]{\includegraphics[scale=0.27]{./images/c2mpyr-ntf2-p2.pdf}}
    }
    \mbox{
    \subfigure[\protect\ipair{mpyr}{2}{ntf} (p3) ]{\includegraphics[scale=0.27]{./images/c2mpyr-ntf2-p3.pdf}}
    \subfigure[\protect\ipair{mpyr}{2}{ntf} (p4) ]{\includegraphics[scale=0.27]{./images/c2mpyr-ntf2-p4.pdf}}
    }                                 
    \mbox{                            
    \subfigure[\protect\ipair{mpyr}{2}{ntf} (p5) ]{\includegraphics[scale=0.27]{./images/c2mpyr-ntf2-p5.pdf}}
    \subfigure[\protect\ipair{mpyr}{2}{ntf} (p6) ]{\includegraphics[scale=0.27]{./images/c2mpyr-ntf2-p6.pdf}}
    }
    % need the \protect to make hyperref and the macro happy together
    \caption{Different configurations of \protect\ipair{mpyr}{2}{ntf} \label{fig:conf-c2mpyr-ntf2}}
\end{figure}


The \catb{mim}{n}X series of ion pairs, where X represents chloride or bromide, were optimised at the MP2/aug-cc-pVDZ level, whilst for the other TILAs, MP2/6-31+G(d,p) was used.
For the \catb{mpyr}{n} series of ion pairs, geometry optimisation was performed at the B3LYP/6-31+G(d) level.
The geometries of these configurations have previously been published by our group.
\cite{Izgorodina2014a, Rigby2014a}


\subsubsection{Software}

% SAPT
The \textsc{Psi4} quantum chemistry package was used for the SAPT2+3 calculations. 
\cite{Turney2012a}
All SAPT calculations were performed using the aug-cc-pVDZ basis set, unless stated otherwise.
\cite{Izgorodina2014a}

% EFP
The GAMESS-US software package was used to perform the EFP calculations
\cite{Schmidt1993a, Gordon2005a}.
Three basis sets were used for EFP calculations: aug-cc-pVDZ, aug-cc-pVTZ and the Pople basis set 6-311++G**.\footnote{For the bromide anion, since it is not included in the 6-311++G** basis set, 6-311G** basis functions were used instead.}
Three basis sets were employed to see how consistently the method performs for this series of basis sets, and the effect of the basis set on the EFP performance for ionic liquids.

\subsubsection{Basis sets}
While three basis sets were used to give an indication of basis set dependency for the EFP method, only the aug-cc-pDVZ basis set was used for SAPT2+3. 
A number of test calculations were performed using aug-cc-pVTZ, indicated that aug-cc-pVDZ gave satisfactory accuracy.
The calculations using aug-cc-pVTZ were run on select representative systems, largely with halide anions, namely \catb{mim}{n}X, where X = Cl and Br.
The rest of the ion pairs for which SAPT2+3/(aug-cc-pVTZ) calculations were run are \catb{mpyr}{n=1,3}X, where X = \bfl and \cl.
Halides---in particular, chlorides---were selected due to smaller system sizes (monoatomic anions). 
These are strongly bound to the cation, representing challenging systems from the theoretical point of view.
\cite{Lehmann2010a}


The differences between these two basis sets are reported in the ESI.
The consistency between the two basis sets is immediately apparent: with the triple-$\zeta$ basis set, electrostatics is always underestimated by 2.7 \enUnit~ on average (3.0 \enUnit~ for imidazolium systems and 1.9 \enUnit~for pyrrolidinium systems). 
Exchange follows the same trend (an overestimation by 4.9 \enUnit~on average).
It is not surprising that the larger aug-cc-pVTZ basis set leads to greater recovery of the dispersion energy by 5.6 \enUnit~on average.
There is no such pattern observed in the induction energy, with both basis sets giving excellent agreement to each other below 1 \enUnit.
Note that the difference between the two basis sets for charge-transfer is a 8.6 \enUnit~difference on average with a standard deviation of 1.7 \enUnit.
Charge-transfer is expected to decrease with increasing basis set size in SAPT, as observed. 
Due to this, agreement between basis sets cannot be based on the comparison of charge-transfer energies.
In the total interaction energy, the largest difference of 9 \enUnit~ comes from \ipair{mim}{1}{cl}, specifically the in-plane configuration p2.
The average difference for the total interaction energy is 7.3 \enUnit~ with a standard deviation of 1.2 \enUnit.
Charge-transfer has the largest contribution to this large difference, and as the small standard deviation attests, this difference is consistent.

Excluding charge-transfer, the differences between the two basis sets for each energetic component (that is, electrostatics, exchange, induction and dispersion) range between -7.0 to 4.8 \enUnit.
Considering the low standard deviations  (0.5 to 1.5 \enUnit), this indicates that these two basis sets differ \emph{consistently} for each energetic component. 
It has to be noted that the aug-cc-pVTZ basis set requires tremendous amounts of CPU time---in most cases more than double the amount of that required for the aug-cc-pVDZ basis set.
For example, for \ipair{mim}{4}{cl}, aug-cc-pVTZ required 329 CPU hours and 22 GB of memory, compared to 26  hours and 4 GB for aug-cc-pVDZ. Taking the computational expense into account, the aug-cc-pVDZ is the largest basis set possible for a number of the bulky ionic liquid ion pairs studied, such as \ipair{pyr}{4}{ntf}.

SAPT2+3 calculations on the intermolecular complexes in the S22 and S66 data sets were also performed with charge-transfer using aug-cc-pVDZ.
\cite{Jurecka2006a, Takatani2010a, Rezac2011a}

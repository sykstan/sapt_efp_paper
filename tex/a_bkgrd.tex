
% Wed 18 Mar 2015   

% section: theoretical background

\subsection{SAPT}
Since Schr{\"o}dinger's equation is a differential equation, mathematicians and physicists naturally turn to perturbation theory in their quest for solutions.
It was first used by London et al.
\mautocite{Eisenschitz1930}
to describe the intermolecular interaction operator as a multipole expansion. 
Since then, the theory has been improved and refined; the current benchmark for calculating the intermolecular interaction energy between two dimers 
\footnote{Note that the terms 'dimer' and 'monomer' refer to the chemical entities whose interaction with each other is of interest, as opposed to the usage in polymer chemistry.}
is symmetry-adapted perturbation theory.


The SAPT method has the Hamiltonian partitioned as
\begin{equation*}
    H = F_A + F_B + W_A + W_B + V
\end{equation*}
where $ F_A, F_B $ are the Fock operators for monomers $A$ and $B$ respectively. 
Similarly, $W_A, W_B$ are the differences between the exact Coulomb operator and the Fock operator for each monomer.
The $V$ contains all the intermolecular terms.
SAPT perturbs in all of $W_A, W_B, V$ when solving for the different energy components.
The different energy components will be grouped as follows:

% note that blank lines make align unhappy, no blank lines anywhere!
\begin{flalign*}
    %\begin{split}
     E_{\text{electrostatic}} = & E^{(10)}_{\text{elec,repl}} +
                                    E^{(12)}_{\text{elec,repl}} +
                                    E^{(13)}_{\text{elec,repl}} \\ 
    E_{\text{exchange}} = & E^{(10)}_{\text{exch}} +
                                    E^{(11)}_{\text{exch}} +
                                    E^{(12)}_{\text{exch}} \\ 
    E_{\text{induction}}    = & E^{(20)}_{\text{ind,repl}} +
                                E^{(30)}_{\text{ind,repl}} f
                                    E^{(22)}_{\text{ind}} +
                                    E^{(20)}_{\text{exch-ind,repl}} + \\ 
                              & E^{(30)}_{\text{exch-ind,repl}} +
                                    E^{(22)}_{\text{exch-ind}} +
                                    \delta E^{(2)}_{\text{HF}} +
                                    \delta E^{(3)}_{\text{HF}} \\ 
    E_{\text{dispersion}}    = & E^{(20)}_{\text{disp}} +
                                    E^{(30)}_{\text{disp}} +
                                    E^{(21)}_{\text{disp}} +
                                    E^{(22)}_{\text{disp}} + 
                                    E^{(20)}_{\text{exch-disp}} + \\
                                &    E^{(30)}_{\text{exch-disp}} +
                                    E^{(30)}_{\text{ind-disp}} +
                                    E^{(30)}_{\text{exch-ind-disp}} \\ 
    E_{\text{charge-transfer}} = & E_{\text{ind}}(\text{\small dimer basis}) - 
                                    E_{\text{ind}}(\text{\small monomer basis})
\end{flalign*}

The superscripts in parenthesis denote the perturbation orders of $V$ and $W = W_A + W_B$ respectively. 
The SAPT charge-transfer energy is part of the SAPT induction energy when the dimer basis set is used. 
To extract the energy, the same calculation is performed in the monomer basis, i.e. where no charge-transfer is permitted, and the difference taken.
The individual terms in this grouping are discussed in much more detail in references
\mcite{Jeziorski1994, Hohenstein2010, Hohenstein2010a, Hohenstein2010b, Hohenstein2011, Hohenstein2012}.
Reference \mcite{Jeziorski1994} in particular provides a comprehensive review of the theory.


\subsection{EFP}
The EFP method partitions the interaction energy in the following way:
\begin{equation*}
    E_{\text{total interaction}} = E_{\text{elec}} + E_{\text{pol}} + E_{\text{disp}} + 
                                    E_{\text{exch-repl}} + E_{\text{ct}}
\end{equation*}
In order, these are the electrostatic, polarization, dispersion, exchange-repulsion and charge-transfer components.
These energies are derived from \emph{ab-initio} methods: it uses Stone's distributed multipolar analysis 
\mautocite{Stone1996} 
for the electrostatic and polarization terms.
Polarization is also treated with static polarizability tensors. 
On the other hand, dynamic polarizability tensors are used to calculate the dispersion interaction.
The exchange-repulsion is calculated using the Fock matrix.
For more details of the method, look to references
\mcite{Gordon2001, Gordon2007, Gordon2009, Mullin2009, Ghosh2010}.

% include EFP terms!!



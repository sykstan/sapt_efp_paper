% section

The interaction energies and their individual fundamental components calculated by means of the EFP method were compared with those from SAPT2+3 for an extensive series of single ion pairs of ionic liquids.
Overall, the deviations between the two methods for total interaction energy and the four fundamental components such as electrostatic, exchange-repulsion, induction and dispersion were much larger on the absolute scale than expected, falling in the range of -63 to 70 \enUnit. 
Out of all three basis sets studied for EFP, aug-cc-pVTZ gave the lowest errors.
The largest absolute errors came from the charge-transfer energy for the halide-based ion pairs and the exchange-repulsion contribution for the ILs combined with other routinely used anions. 
On the relative scale, the EFP method did not deviate from SAPT by more than 20\% \emph{on average} for all systems and energetic components (with the exception of charge transfer energy for halides). 
Electrostatics in particular showed very small relative errors, below 3\% for all basis sets on average.
As expected, out of the five fundamental components, charge-transfer from EFP produced the largest relative errors, with EFP overestimating SAPT2+3 by 40\% on average for aug-cc-pVTZ.
Apart from charge-transfer, EFP relative errors for ILs are comparable to those reported for the S22 and S66 databases.
\cite{Flick2012a}
This finding indicates that the large EFP errors are not specific to charged intermolecular complexes, as previously suggested.
Due to the sheer magnitude of each individual interaction in ionic liquids reaching up to -479 \enUnit~for electrostatics (-378 \enUnit~on average) it is not surprising that on the absolute scale the average errors were well beyond chemical accuracy, thus rendering EFP inapplicable for ILs at present.


The importance of higher-order terms in the EFP formulation has been established for the exchange-repulsion contribution in the case of typical ionic liquid anions, whereas EFP exchange-repulsion was treated rather well for halide systems.  
It has to be noted that similar relative errors from higher-order terms for electrostatics and exchange-repulsion were observed in neutral intermolecular complexes from the S22 and S66 databases. 
Electrostatics did not show appreciable contribution from higher-order terms for all ILs studied.


Although linear regression analysis of EFP against SAPT2+3 per each energetic component had very high values for the coefficient of determination, $R^2$, significant reduction in error was achieved only for the induction and dispersion terms. 
Moreover, better statistics for linear regression were found when the ion pairs were separated into cation-specific groups. 
Small errors for scaled induction and dispersion in the range of 1.2--3.2 and 0.7--11.4~\enUnit~on average were observed for both for TILAs and halides, respectively. 
This is particularly important highlighting the applicability of the EFP formulation for these two terms for ionic liquid systems. 
This finding will assist in the future development of intermolecular potentials for these complex systems.  


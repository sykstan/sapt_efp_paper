% section

The interaction energy from the EFP method was compared with that of the SAPT method.
The raw energies were discussed, and the differences between the two methods studied.
The mean and standard deviation of the differences between SAPT and EFP for each energy from each basis set is tabulated in 
tables \ref{tab:hal_sum_stats} and \ref{tab:non-hal_sum_stats}. 
For each basis set there are two columns, the first one on the right is the mean, and the second column the standard deviation.
There are two tables, on for the halides and the other for the other anions.
This is due to the differences between the two groups, especially in the exchange-repulsion and charge-transfer energies.

\end{multicols}
%\begin{table}[ht]
%    \centering
%    \small
%    \begin{tabular}{ l | r r r }
%        \hline
%        Energy & aug-cc-pVDZ & aug-cc-pVTZ & 6-311++G** \tabularnewline
%        \hline 
%        Electrostatics      & 10.12  ($\pm$ 10.81)    & 1.20  ($\pm$ 9.17)  &  -0.63 ($\pm$ 11.46) \tabularnewline
%        Exchange-Repulsion  & 10.24  ($\pm$ 8.49)     & 8.89  ($\pm$ 14.79) &  27.44 ($\pm$ 15.14) \tabularnewline
%    Induction-Polarization  & -6.29  ($\pm$ 3.78)     & -4.18 ($\pm$ 6.74)  & -14.18 ($\pm$ 8.22)  \tabularnewline
%        Dispersion          &  5.12  ($\pm$ 10.63)    & 6.94  ($\pm$ 9.35)  &  -6.01 ($\pm$ 13.67) \tabularnewline
%        Charge-transfer     & -13.2  ($\pm$ 15.03)    & -8.34 ($\pm$ 13.87) & -11.24 ($\pm$ 14.29) \tabularnewline
%Total interaction energy    & 19.18  ($\pm$ 20.45)    & 17.70 ($\pm$ 23.07) &   8.88 ($\pm$ 33.69) \tabularnewline
%        \hline
%    \end{tabular}
%    \caption{Summary statistics. All energies are in \enUnit}
%    \label{tab:sum_stats}
%\end{table}

\begin{table}[ht]
    \centering
    \small
    \begin{tabular}{l | r l | r l | r l}
        \hline
        \multicolumn{7}{c}{Halides} \tabularnewline
        \multicolumn{1}{l}{Energy} & \multicolumn{2}{c}{aug--cc-pVDZ} & 
            \multicolumn{2}{c}{aug-cc-pVTZ} & \multicolumn{2}{c}{6-311++G**}        \tabularnewline
        \hline
        Electrostatics          &  13.0 & ($\pm$  9.5)  &  -7.95    & ($\pm$  9.3)  & -10.7 & ($\pm$ 10.1)  \tabularnewline
        Exchange-repulsion      &   1.3 & ($\pm$  7.2)  &  -7.01    & ($\pm$ 15.1)  &  21.1 & ($\pm$ 20.8)  \tabularnewline
        Induction-polarization  &  -6.4 & ($\pm$  6.3)  &  -0.19    & ($\pm$ 11.6)  & -25.4 & ($\pm$  6.8)  \tabularnewline
        Dispersion              &  -9.3 & ($\pm$  1.6)  &  -4.46    & ($\pm$  4.9)  & -26.0 & ($\pm$  3.6)  \tabularnewline
        Charge-transfer         & -34.3 & ($\pm$ 14.4)  & -24.72    & ($\pm$ 18.5)  & -29.6 & ($\pm$ 15.4)  \tabularnewline
        Total Energy            &  -1.3 & ($\pm$ 18.9)  &  -9.79    & ($\pm$ 23.0)  & -36.3 & ($\pm$ 21.3)  \tabularnewline
        \hline
    \end{tabular}
    \caption{Summary statistics for halides}
    \label{tab:hal_sum_stats}
\end{table}

\begin{table}[ht]
    \centering
    \small
    \begin{tabular}{l | r l | r l | r l}
        \hline
        \multicolumn{7}{c}{Non-halides} \tabularnewline
        \multicolumn{1}{l}{Energy} & \multicolumn{2}{c}{aug--cc-pVDZ} & 
            \multicolumn{2}{c}{aug-cc-pVTZ} & \multicolumn{2}{c}{6-311++G**}    \tabularnewline
        \hline
        Electrostatics          &   9.0 & ($\pm$ 11.1)  &  4.6  &  ($\pm$ 6.5)  &  3.4  & ($\pm$  9.3)  \tabularnewline
        Exchange-repulsion      &  13.6 & ($\pm$  6.3)  & 14.8  &  ($\pm$ 9.4)  & 30.0  & ($\pm$ 11.3)  \tabularnewline
        Induction-polarization  &  -6.3 & ($\pm$  2.2)  & -5.7  &  ($\pm$ 2.2)  & -9.7  & ($\pm$  2.3)  \tabularnewline
        Dispersion              &  10.5 & ($\pm$  6.9)  & 11.1  &  ($\pm$ 6.7)  &  2.0  & ($\pm$  5.6)  \tabularnewline
        Charge-transfer         &  -5.3 & ($\pm$  1.8)  & -2.3  &  ($\pm$ 1.9)  & -3.9  & ($\pm$  1.5)  \tabularnewline
        Total Energy            &  26.9 & ($\pm$ 15.0)  & 27.8  &  ($\pm$12.4)  & 27.0  & ($\pm$ 16.1)  \tabularnewline
        \hline
    \end{tabular}
    \caption{Summary statistics for non-halides}
    \label{tab:non-hal_sum_stats}
\end{table}
\begin{multicols}{2}

It seems that for the halides, the charge-transfer energy contributes much to the error; this is followed by the repulsion and polarization components.
The aug-cc-pVTZ basis set handles the halides best overall, particularly if the charge-transfer is not considered.
For the other anions, the component with the largest error overall is repulsion.
This can be clearly seen for the tetrafluoroborates, the hexafluorophosphates, the mesylates and also \ntf. 
In the tosylates, the Coulomb interaction also contributes significantly to the difference.
The induction term is typically underestimated, while the dispersion term is overestimated consistently, indicating a systematic difference between SAPT and EFP for these two terms.
This results in some cancelling because these two energies are of a somewhat similar scale (typical magnitude around 10 \enUnit~ usually below 20 \enUnit) in the systems studied.
%The results agree with what Flick et al. observed, that the polarization terms could be improved.
Furthermore, using the Fock matrix alone to calculate the exchange-repulsion gives large errors; this is another aspect of the EFP method that can be improved.
Figure 
\ref{fig:corr-all_En} 
provides a nice summary of the differences. 
It puts into perspective the different scales of the different components.
Exchange-repulsion is consistently underestimated in the upper right, induction and dispersion terms fall on opposite sides of the diagonal line at energies between -100 and 0 \enUnit, while the charge-transfer data sits close to zero, showing how for most cases it has but a tiny contribution.
The electrostatic component along with the total interaction energy populate the lower left. 
Their proximity to one another shows that the Coulomb interaction dominates, which is expected for charged species.


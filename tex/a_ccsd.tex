
% subsection: CCSD(T) and SAPT
% and aug-cc-pVTZ as well

To ensure the results given by SAPT agree well with other benchmark methods, the CCSD(T)/CBS energies were also calculated for the ion pairs. 
In this comparison, only the total energy, the HF energy, and the correlation correction can be compared, since CCSD(T)/CBS only has these values. 
The statistics for the differences are tabulated below, with the halide systems separated from the rest.
Note that these numbers do not include \ipair{mpyr}{3}{dca} (p5) which is an outlier (0.331 \enUnit).

% numbers exclude c3mpyr-dca-p5!!!
%\end{multicols}

\begin{table}[h]
\centering
\small
    \begin{tabular}{ccc|cc}
        \multicolumn{5}{c}{$\Delta$(HF Energy) (\enUnit)}                                                \\ \hline
                & \multicolumn{2}{c}{Non-halides}        & \multicolumn{2}{c}{Halides}          \\ \hline
                & Statistics & System                    & Statistics & System                  \\ \hline
        Mean    & -1.49E-06      & -                         &  1.73E-06      & -                       \\
        Median  & -8.61E-07      & -                         &  1.03E-06      & -                       \\
        Std dev & 2.40E-05       & -                         &  1.27-05       & -                       \\
        Min     & -7.49E-05      & \ipair{mim}{4}{tos} (p1)  & -1.95E-05      & \ipair{mpyr}{2}{br} (p2) \\ 
        Max     & 6.25E-05       & \ipair{mpyr}{1}{dca} (p2) & 4.09E-05       & \ipair{mpyr}{4}{cl} (p1) \\ \hline
    \end{tabular}
    \caption{CCSD(T)/aug-cc-pVDZ and SAPT2+3 differences for HF energy}
    \label{tab:ccsd-sapt-hf}
\end{table}

\begin{table}[h]
\centering
\small
    \begin{tabular}{ccc|cc}
        \multicolumn{5}{c}{$\Delta$(Correlation correction) (\enUnit)}                                    \\ \hline
                & \multicolumn{2}{c}{Non-halides}        & \multicolumn{2}{c}{Halides}          \\ \hline
                & Statistics & System                    & Statistics & System                  \\ \hline
        Mean    & -1.89      & -                         & -3.27      & -                       \\
        Median  & -1.40      & -                         & -1.76      & -                       \\
        Std dev & 2.95       & -                         & 3.36       & -                       \\
        Min     & -9.07      & \ipair{mim}{4}{tos} (p1)  & -10.68     & \ipair{mim}{4}{cl} (p4) \\ 
        Max     & 2.37       & \ipair{mpyr}{1}{bfl} (p2) & 0.59       & \ipair{mpyr}{1}{br} (p2) \\ \hline
    \end{tabular}
    \caption{CCSD(T)/aug-cc-pVDZ and SAPT2+3 differences for correlation correction}
    \label{tab:ccsd-sapt-corr}
\end{table}

\begin{table}[h]
\centering
\small
    \begin{tabular}{ccc|cc}
        \multicolumn{5}{c}{$\Delta$(Total Energy) (\enUnit)}                                    \\ \hline
                & \multicolumn{2}{c}{Non-halides}        & \multicolumn{2}{c}{Halides}          \\ \hline
                & Statistics & System                    & Statistics & System                  \\ \hline
        Mean    & -1.77      & -                         & -1.90      & -                       \\
        Median  & -1.05      & -                         & -0.76      & -                       \\
        Std dev & 2.81       & -                         & 2.84       & -                       \\
        Min     & -8.67      & \ipair{mim}{4}{tos} (p1)  & -8.19      & \ipair{mim}{4}{cl} (p4) \\ 
        Max     & 2.13       & \ipair{mpyr}{1}{dca} (p2) & 1.50       & \ipair{mpyr}{1}{br} (p2) \\ \hline
    \end{tabular}
    \caption{CCSD(T)/CBS and SAPT2+3 differences for Total Energy}
    \label{tab:ccsd-sapt-Etot}
\end{table}

%\begin{multicols}{2}


In table \ref{tab:ccsd-sapt-hf}, the HF energies from both methods agree to the fourth decimal place in all cases.
This degree of agreement is expected, since the HF method is the same in principle for both methods.

CCSD(T)/CBS improves the HF energy by adding on the correlation correction; i.e. there are only two components, as opposed to SAPT.
Hence the difference between the SAPT2+3 Total Energy and the SAPT Hartree--Fock energy was compared against the correlation correction; this is done in the second table above.
The difference between the two correlation correction values showed excellent agreement. 
In fact, this difference can be considered as the difference between the SAPT and CCSD(T), since the difference in the HF energy is so small.
Thus that is why the statistics for the correlation correction are almost identical to the statistics for the difference between CCSD(T)/aug-cc-pVDZ and SAPT2+3 total energies.
Therefore the differences in total energy between CCSD(T)/aug-cc-pVDZ and SAPT2+3 are not shown. 


In table \ref{tab:ccsd-sapt-Etot} the difference between total energy for CCSD(T)/aug-cc-pVQZ and SAPT are shown.
The reason for the smaller values than the correlation correction are due to the fact that CCSD(T)/CBS results are used, while in the correlation comparison the aug-cc-pVDZ basis set was used for consistency. 
This indicates that the SAPT total energy results are actually converging with CCSD(T)/CBS, and points very strongly to SAPT2+3 being a robust method that approaches to the complete basis set limit.

%\end{multicols}

\begin{table}[h]
\centering
\small
    \begin{tabular}{ccccccc}
\hline
Statistic & $\Delta$(HF Energy) & System                 & $\Delta$(Corr) & System                   & $\Delta$(Total Energy) & System                  \\ \hline  
Mean      & 1.68E-06            & -                      & -10.52         & -                        & -10.39                 & -                       \\   
Median    & -8.99E-07           & -                      & -10.40         & -                        & -10.29                 & -                       \\   
Std dev   & 5.67E-06            & -                      & 2.80           & -                        & 2.76                   & -                       \\    
Min       & -5.54E-06           & \ipair{mim}{4}{cl} (p4)& -15.04         & \ipair{mim}{4}{cl} (p4)  & -14.87                 & \ipair{mim}{4}{cl} (p4) \\    
Max       & 1.17E-05            & \ipair{mim}{2}{cl} (p2)& -6.04          & \ipair{mpyr}{3}{cl} (p2) & -5.95                  & \ipair{mpyr}{3}{cl} (p2)\\ 
\hline
    \end{tabular}
    \caption{CCSD(T) and SAPT2+3/aug-cc-pVTZ differences in \enUnit}
    \label{tab:ccsd-sapt-atz}
\end{table}

%\begin{multicols}{2}

Comparing the SAPT2+3/aug-cc-pVTZ with the CCSD(T) results, the same conclusion can be derived. 
The statistics for the correlation correction difference are given in table \ref{tab:ccsd-sapt-atz}.

These errors are relatively large because of the difficulty in modelling halides. 
Furthermore, there are a lot less pyrrolidinium systems compared to imidazolium, which tend to have larger variances in interaction energies.
However, HF energy differences are on the same scale as the previous aug-cc-pVDZ:

Indeed, comparing the aug-cc-pVTZ total energies with CCSD(T)/CBS yields statistics very similar to that of the differences between the correlation correction.
This indicates most of the differences are coming from the correlation correction. 
Strictly speaking, since SAPT2+3 treats the intermolecular interaction in a very different way mathematically, it is does not correct for electronic correlation.
The correlation correction obtained as the difference of the Total Energy and the HF energy is purely for comparison purposes with CCSD(T).

The agreement between SAPT and CCSD(T)/CBS results validates the reliability of SAPT as a method  and aug-cc-pVDZ as a basis set to accurately determine the decomposition of the total interaction energy.


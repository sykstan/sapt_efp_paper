
% section: method

\subsubsection{Chemical systems studied}
The chemical systems were single ion pairs of ionic liquids. 
Routinely used anions were in this study, such as tetrafluoroborate (\bfl), bromide (\br), chloride (\cl), dicyanamide (\dca), mesylate (\mes), tosylate (\tos), hexafluorophosphate (\pf) and bis\{(trifluoromethyl)sulfonyl\}amide (\ntf). 
The cations used are N-alkyl-N'-pyrrolidinium and 1--methyl-3-alkyl-imidazolium, with the length of the alkyl chain on the cation was varied from methyl, ethyl, propyl to butyl.
The bulk of the cation is thus varied, providing insight into how the alkyl chain of the cation affects the different energy components.
The names and abbreviations of the different cations and anions, eight each, are tabulated in 
table \ref{tab:cation-anion-list}. Further in the text, halides are abbreviated as "Hal", whereas the rest of the anions are referred to as ionic liquid anions (ILA).


%\end{multicols}
\begin{table}[ht]
    \begin{centering}
    \footnotesize
    \begin{tabular}{c|c|c|c}
        \hline 
        Cations  & abbreviation & Anions & abbreviation\tabularnewline
        \hline 
        1-methyl-3-methyl-imidazolium & $\text{C}_{1}\text{mim}^{+}$ & tetrafluoroborate &  $\text{BF}_{\text{4}}^{-}$\tabularnewline
        1-methyl-3-ethyl-imidazolium & $\text{C}_{2}\text{mim}^{+}$ & bromide &  $\text{Br}^{-}$\tabularnewline
        1-methyl-3-propyl-imidazolium & $\text{C}_{3}\text{mim}^{+}$ & chloride  & $\text{Cl}^{-}$\tabularnewline
        1-methyl-3-butyl-imidazolium & $\text{C}_{4}\text{mim}^{+}$ & dicyanamide  & $\text{Dca}^{-}$\tabularnewline
        N-N'-dimethyl-pyrrolidinium & $\text{C}_{1}\text{mpyr}^{+}$ & mesylate &  $\text{Mes}^{-}$\tabularnewline
        N-ethyl-N'-methyl-pyrrolidinium & $\text{C}_{2}\text{mpyr}^{+}$ & bis\{(trifluoromethyl)sulfonyl\}amide  & $\text{NTf}_{\text{2}}^{-}$\tabularnewline
        N-propyl-N'-methyl-pyrrolidinium & $\text{C}_{3}\text{mpyr}^{+}$ & hexafluorophosphate &  $\text{PF}_{\text{6}}^{-}$\tabularnewline
        N-butyl-N'-methyl-pyrrolidinium & $\text{C}_{4}\text{mpyr}^{+}$ & tosylate  & $\text{Tos}^{-}$\tabularnewline
        \hline 
    \end{tabular}
    \caption{List of cations and anions }
    \label{tab:cation-anion-list}
    \par\end{centering}
\end{table}
%\begin{multicols}{2}


For each cation-anion combination, different configurations of these ion pairs were studied as well. 
These configurations differ by how the anion interacts with the cation.
Both cations are ring systems. 
In the imidazolium case, if the anion interacts with the cation from above, this is referred to as the above-plane configuration (shortened to p1).
If the anion interacts with the cation in the $\text{C}_2\text{--H}$ bond plane, i.e  in the plane of the ring, then this is referred to as an in-plane interaction.
In-plane configurations are p2 and p3.
The below-plane configuration is very much similar to the above-plane configuration; this is designated as p4.
The different configurations for \ipair{mim}{3}{br} are presented in figure
\protect\ref{fig:conf-c3mim-br}
as an example. 

%\end{multicols}
\begin{figure}
    \centering
    \mbox{
    \subfigure[\protect\ipair{mim}{3}{br} (p1) ]{\includegraphics[scale=0.3]{\string~/Dropbox/QuantumChem/il_structure_images/c3mim-br-p1.png}}
    \subfigure[\protect\ipair{mim}{3}{br} (p2) ]{\includegraphics[scale=0.3]{\string~/Dropbox/QuantumChem/il_structure_images/c3mim-br-p2.png}}
    }
    \mbox{
    \subfigure[\protect\ipair{mim}{3}{br} (p3) ]{\includegraphics[scale=0.3]{\string~/Dropbox/QuantumChem/il_structure_images/c3mim-br-p3.png}}
    \subfigure[\protect\ipair{mim}{3}{br} (p4) ]{\includegraphics[scale=0.3]{\string~/Dropbox/QuantumChem/il_structure_images/c3mim-br-p4.png}}
    }                                 
    % need the \protect to make hyperref and the macro happy together
    \caption{Different configurations of \protect\ipair{mim}{3}{br} \label{fig:conf-c3mim-br}}
\end{figure}

For the pyrrolidinium case, it is very similar to imidazolium, except that the ring system is not planar.
Once again the p1 and p4 conformations refer to the anion interacting above and below the pyrrolidinium ring, while conformations p2 and p3 refer to it interacting from the sides.
The \ntf anion has multiple interaction sites such as the central nitrogen and the oxygens ond the sulfonyl groups, as shown in figure
\ref{fig:conf-c2mpyr-ntf2}.
The above-plane configurations for this anion are p1 and p2; p1 is where the nitrogen faces the ring, and p2 is where the oxygens face the ring.
The next two configurations, p3 and p4, are where the nitrogen forms a non-linear hydrogen bond from below and from the side, respectively.
The p5 and p6 configurations are the same as p3 and p4, except instead of nitrogen, it is the the oxygens forming the hydrogen bond from below and side-on, respectively.


\begin{figure}
    \centering
    \mbox{
    \subfigure[\protect\ipair{mpyr}{2}{ntf} (p1) ]{\includegraphics[scale=0.3]{\string~/Dropbox/QuantumChem/il_structure_images/c2mpyr-ntf2-p1.png}}
    \subfigure[\protect\ipair{mpyr}{2}{ntf} (p2)]{\includegraphics[scale=0.3]{\string~/Dropbox/QuantumChem/il_structure_images/c2mpyr-ntf2-p2.png}}
    }
    \mbox{
    \subfigure[\protect\ipair{mpyr}{2}{ntf} (p3) ]{\includegraphics[scale=0.3]{\string~/Dropbox/QuantumChem/il_structure_images/c2mpyr-ntf2-p3.png}}
    \subfigure[\protect\ipair{mpyr}{2}{ntf} (p4) ]{\includegraphics[scale=0.3]{\string~/Dropbox/QuantumChem/il_structure_images/c2mpyr-ntf2-p4.png}}
    }                                 
    \mbox{                            
    \subfigure[\protect\ipair{mpyr}{2}{ntf} (p5) ]{\includegraphics[scale=0.3]{\string~/Dropbox/QuantumChem/il_structure_images/c2mpyr-ntf2-p5.png}}
    \subfigure[\protect\ipair{mpyr}{2}{ntf} (p6) ]{\includegraphics[scale=0.3]{\string~/Dropbox/QuantumChem/il_structure_images/c2mpyr-ntf2-p6.png}}
    }
    % need the \protect to make hyperref and the macro happy together
    \caption{Different configurations of \protect\ipair{mpyr}{2}{ntf} \label{fig:conf-c2mpyr-ntf2}}
\end{figure}



The \catb{mim}{n}X series of ion pairs, where X represents chloride or bromide, were optimised at the MP2/aug-cc-pVDZ level, whilst for the other ILAs MP2/6-31+G(d,p) was used.
For the \catb{mpyr}{n} series of ion pairs, geometry optimisation was performed at the B3LYP/6-31+G(d) level.
All of these conformations have previously been published by our group.
\cite{Izgorodina2014a, Rigby2014a}


\subsubsection{Software}

% SAPT
The \textsc{Psi4} quantum chemistry package was used for the SAPT calculations. 
\cite{Turney2012a}
All SAPT calculations were performed at the highest available order SAPT level of theory, using the aug-cc-pVDZ basis set.
\cite{Izgorodina2014a}

% means what terms are considered?

% EFP
The GAMESS-US software package was used to perform the EFP calculations
\cite{Schmidt1993a, Gordon2005a}.
Three basis sets were used for EFP: aug-cc-pVDZ, aug-cc-pVTZ and the Pople basis set 6-311++G**. 
\footnote{For the bromide anion, since it is not included in the 6-311++G** basis set, 6-311G** basis functions were used instead.}
Three basis sets were used to see how consistent the method is between basis sets, and also the quality of the basis set that gives the best expense-to-error ratio.

\subsubsection{Basis sets}
While three basis sets were used to give an indication of basis set dependency for the EFP method, only the aug-cc-pDVZ basis set was used for SAPT. 
There were some test calculations using a larger basis set, aug-cc-pVTZ, to determine whether aug-cc-pVDZ gave satisfactory accuracy.
The calculations using aug-cc-pVTZ were run on some representative systems, largely with halide anions, namely \catb{mim}{n}X, where X = Cl, Br.
The rest of the ion pairs for which SAPT(aug-cc-pVTZ) calculations were run are \catb{mpyr}{n=1,3}X, where X = \bfl, \cl.
%The imidazolium bromide and chloride ion pairs had methyl, ethyl and butyl as alkyl chains on the cation. 
%The pyrrolidinium based cations were paired with two anions, tetrafluoroborate and chloride.
%These had methyl and propyl as alkyl chains on pyrrolidinium.
Halides, in particular chlorides, were selected because of smaller system sizes (monoatomic anions) and because they are strongly bound to the cation. 
Due to the problematic nature of the charge-transfer interaction in the halides, the following discussion will ignore the differences in charge-transfer energies between the two basis sets.


The differences between these two basis sets are presented in the ESI.
One immediately observes the consistency between the two basis sets: with the triple zeta basis set, electrostatics is always underestimated (by 2.7 \enUnit on average, 3.0 for imidazolium systems and 1.9 \enUnit~for pyrrolidinium systems), but correspondingly its exchange is also underestimated (by 4.9 \enUnit on average, 5.1 for imidazolium systems and 4.0 \enUnit~for pyrrolidinium systems).
The dispersion energy is also always stronger in the aug-cc-pVTZ basis set, on average by around 5.6 \enUnit.
While there is no such pattern in the difference in the induction energies, this is the interaction both basis sets agree best on, with absolute differences below 1 \enUnit.
In the total interaction energy, the largest difference is 9 \enUnit, for \ipair{mim}{1}{cl} (p2).
Excluding charge-transfer, the differences between the two basis sets for all energetic components range between -9.0 to 4.8 \enUnit.
Considering the low standard deviations, this indicates that while these two basis sets differ, they differ \emph{consistently} for each energetic component. 
The fact that both basis sets showed such agreement for the halides, which are typically challenging, indicates that the aug-cc-pVTZ basis set would not provide much more insight and accuracy than the significantly less expensive aug-cc-pVDZ basis.


It has to be noted that the aug-cc-pVTZ basis set requires tremendous amounts of CPU time, in most cases more than double the amount of that required for the aug-cc-pVDZ basis set.
For example, for \ipair{mim}{4}{cl}, aug-cc-pVTZ required 329 CPU hours and 22 GB of memory, compared to 26  hours and 4 GB for aug-cc-pVDZ.


%\end{multicols}


%% latex table generated in R 3.0.2 by xtable 1.7-3 package
%% Wed Mar 19 14:34:48 2014
%\begin{table}[ht]
%    \centering
%    \small
%    \begin{tabular}{lcccccc}
%	
%	  \hline
%	 Name (Conf) & Electrostatics & Exchange & Induction & Dispersion & Total Energy & Charge-transfer \\ 
%	\hline
%	   \ipair{mim}{1}{br} (p1) & 4.77 & -5.92 & 0.90 & -7.05 & -7.30 & 12.17 \\ 
%	   \ipair{mim}{1}{br} (p2) & 1.42 & -3.79 & -0.20 & -5.29 & -7.87 & 7.45 \\ 
%	   \ipair{mim}{1}{cl} (p1) & 4.23 & -5.96 & 0.89 & -6.61 & -7.45 & 9.94 \\ 
%	   \ipair{mim}{1}{cl} (p2) & 1.14 & -4.27 & -0.24 & -5.62 & -9.00 & 8.30 \\ 
%	   \ipair{mim}{2}{br} (p1) & 4.43 & -5.87 & 1.02 & -6.83 & -7.25 & 10.09 \\ 
%	   \ipair{mim}{2}{br} (p2) & 1.66 & -4.07 & -0.04 & -5.21 & -7.66 & 7.64 \\ 
%	   \ipair{mim}{2}{br} (p1) & 1.90 & -4.17 & 0.13 & -5.38 & -7.51 & 7.10 \\ 
%	   \ipair{mim}{2}{br} (p2) & 4.77 & -5.93 & 1.00 & -6.87 & -7.03 & 11.73 \\ 
%	   \ipair{mim}{2}{cl} (p1) & 3.92 & -5.89 & 0.96 & -6.41 & -7.41 & 8.63 \\ 
%	   \ipair{mim}{2}{cl} (p2) & 1.31 & -4.42 & -0.13 & -5.49 & -8.73 & 8.68 \\ 
%	   \ipair{mim}{2}{cl} (p3) & 1.58 & -4.56 & 0.10 & -5.56 & -8.45 & 7.81 \\ 
%	   \ipair{mim}{2}{cl} (p4) & 4.25 & -5.94 & 0.95 & -6.40 & -7.13 & 9.69 \\ 
%	   \ipair{mpyr}{3}{cl} (p1) & 2.00 & -4.61 & 0.90 & -4.41 & -6.11 & 6.49 \\ 
%	   \ipair{mpyr}{3}{cl} (p2) & 1.71 & -4.36 & 0.63 & -4.29 & -6.30 & 5.99 \\ 
%	   \ipair{mpyr}{3}{cl} (p3) & 1.77 & -4.50 & 0.75 & -4.47 & -6.44 &  \\ 
%	   \ipair{mim}{4}{cl} (p1) & 4.00 & -5.93 & 1.06 & -6.21 & -7.08 & 8.33 \\ 
%	   \ipair{mim}{4}{cl} (p2) & 1.40 & -4.43 & -0.10 & -5.27 & -8.41 &  \\ 
%	   \ipair{mim}{4}{cl} (p3) & 1.95 & -4.89 & 0.36 & -5.55 & -8.13 & 7.52 \\ 
%	   \ipair{mim}{4}{cl} (p4) & 4.64 & -6.17 & 1.13 & -6.29 & -6.68 &  \\ 
%	\hline
%	   Mean and Std Dev & 2.78 $\pm$ 1.43 & -5.04 $\pm$ 0.83 & 0.53 $\pm$ 0.51 & -5.75 $\pm$ 0.84 & -7.47 $\pm$ 0.81 & 8.60 $\pm$ 1.74 \\   
%	\hline 
%    \end{tabular}
%    \caption{Differences between SAPT results from two basis sets, 
%                $ E_{\text{SAPT}}(\text{aug-cc-pVTZ}) - E_{\text{SAPT}}(\text{aug-cc-pVDZ}) $. 
%                All energies are in \enUnit. }
%    \label{tab:adiff-sapt}
%\end{table}

%\begin{multicols}{2}

    %\subsubsection{Presentation format}
    %% this section may be obsolete--need better methods of presenting data
    %In
    %figure \ref{fig:pure_en-sapt_Elec}
    %is an example of how the results for each energy are plotted. 
    %This is for the Electrostatic energy from the SAPT calculations.
    %The graph is divided into a grid, with the cations on the rows and the anions forming the columns. 
    %Methyl imidazolium is "im", and methyl pyrrilidinium is "pyr". 
    %For more information about abbreviations used, please refer to 
    %table \ref{tab:cation-anion-list}.
    %Within the grid, the horizontal axis represents the length of the alkyl chain on the cation. 
    %For example, the induction energy for 
    %\ipair{mim}{1}{bfl} 
    %is plotted in the top left graph as the leftmost point.
    %The different shapes of the points represent the different configurations between the ion pairs. 
    %Thus in the previous example the first configuration, called "p1" in the legend, was used.
    %Some ion pairs have more configurations than others. 
    %Note that there are missing data points for calculations which have not yet completed successfully. 
    %For example, none of the data for 
    %\ipair{mim}{4}{ntf}  
    %is available.



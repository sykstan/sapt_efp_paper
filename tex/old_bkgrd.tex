
% Wed 18 Mar 2015   

% section: theoretical background

\subsection{SAPT}
SAPT was first used by London et al.
\cite{Eisenschitz1930a}
to describe the intermolecular interaction operator as a multipole expansion. 
The theory has been further improved and refined and is the current benchmark for calculating the intermolecular interaction energy between two dimers.
Jeziorski, Moszynski and Szalewicz have a comprehensive description of the theory in \citenum{Jeziorski1994a}.
Dimer here refers to an ion pair, and monomer refers to a single ion.
This terminology is used in the literature when discussing intermolecular interactions, with the intermolecular interaction energy defined as the difference between to total energies of the dimer and monomers:
\begin{equation}
    \energ{int} = \energ{AB} - \energ{A} - \energ{B}
\end{equation}
where A and B are the constituent monomers of the system.


In order to obtain physically appealing concepts such as the electrostatic, dispersion and induction interactions, a nonsymmetric decomposition of the Hamiltonian is used.
This means that electrons are no longer indistinguishable, and the corresponding zeroth-order wavefunction no longer obeys the Pauli exclusion principle. 
Antisymmetrisation corrects for this. 
However, the antisymmetrised wavefunction is now no longer an eigenfunction of the unperturbed sum of the monomer Hamiltonians, $H_A + H_B$.
Thus the perturbation procedure is \emph{symmetry-adapted} such that $H_A + H_B$ can be kept as the unperturbed operator whilst still utilising the antisymmetrised wavefunction.


The SAPT method has the Hamiltonian partitioned as
\begin{equation}
    H = F_A + F_B + W_A + W_B + V
\end{equation}
where $ F_A, F_B $ are the Fock operators for monomers $A$ and $B$ respectively. 
Similarly, $W_A, W_B$ are the differences between the exact Coulomb operator and the Fock operator for each monomer, whereas $V$ contains all the intermolecular terms.
SAPT perturbs all of $W_A, W_B, V$ through various orders in order to calculate the energy terms.
The different energy components will be grouped as follows:

% note that blank lines make align unhappy, no blank lines anywhere!
\begin{flalign}
    %\begin{split}
     E_{\text{electrostatic}} = & E^{(10)}_{\text{Elst,Repl}} +
                \textcolor{blue}{E^{(12)}_{\text{Elst,Repl}}}  +
                \textcolor{red}{E^{(13)}_{\text{Elst,Repl}}} \\ 
    E_{\text{exchange}} = & E^{(10)}_{\text{Exch}} +
                \textcolor{blue}{E^{(11)}_{\text{Exch}}} +
                \textcolor{blue}{E^{(12)}_{\text{Exch}}} \\ 
    E_{\text{induction}}    = & E^{(20)}_{\text{Ind,Repl}} +
                \textcolor{red}{E^{(30)}_{\text{Ind,Repl}}} +
                \textcolor{blue}{E^{(22)}_{\text{Ind}}} +
                E^{(20)}_{\text{Exch-Ind,Repl}} + \\  \nonumber
                    & \textcolor{red}{E^{(30)}_{\text{Exch-Ind,Repl}}} +
                      \textcolor{blue}{E^{(22)}_{\text{Exch-Ind}}} +
                      \textcolor{blue}{\delta E^{(2)}_{\text{HF}}} +
                      \textcolor{red}{\delta E^{(3)}_{\text{HF}}} \\ 
    E_{\text{dispersion}} = & E^{(20)}_{\text{Disp}} +
                                    \textcolor{red}{E^{(30)}_{\text{Disp}}} +
                                    \textcolor{blue}{E^{(21)}_{\text{Disp}}} +
                                    \textcolor{blue}{E^{(22)}_{\text{Disp}}} + 
                                E^{(20)}_{\text{Exch-Disp}} + \\    \nonumber
                                    &  \textcolor{red}{E^{(30)}_{\text{Exch-Disp}}} +
                                    \textcolor{red}{E^{(30)}_{\text{Ind-Disp}}} +
                                    \textcolor{red}{E^{(30)}_{\text{Exch-Ind-Disp}}} \\ 
    E_{\text{charge-transfer}} = & E_{\text{Ind}}(\text{\small dimer basis}) - 
                                    E_{\text{Ind}}(\text{\small monomer basis})
\end{flalign}

The superscripts in parenthesis denote the perturbation orders of $V$ and $W = W_A + W_B$ respectively. 
The 2+3 refer to the truncation order of the SAPT expansion.
In the equations above, the colour refers to this order. 
Blue terms are \textcolor{blue}{second order}, whereas red terms are \textcolor{red}{third order}.
The first order has only electrostatics and exchange terms, while induction and dispersion occur in the second order.
Also present in the second order is quenching via exchange-repulsion in the intramolecular contributions to electrostatics and exchange.
SAPT2+ further includes intramolecular electron correlation terms pertaining to dispersion.
The third order, SAPT2+3, consists of additional terms for dispersion, as well as quenching of induction and dispersion by third order exchange.
In perturbation theory, the induction energy can be separated into two categories: those involving excitations from the occupied orbitals of a molecule to virtual orbitals of the same molecule, and excitations from the occupied orbitals of a molecule to the virtual orbitals of another molecule.
\cite{Stone2009a}
The latter is known as charge-transfer.
To calculate this interaction, the SAPT induction energy from the monomer basis set, where no charge-transfer is permitted, is subtracted from the SAPT induction energy from the dimer basis set.
Furthermore, note that induction and dispersion include exchange components due to the quenching of forces as a result of proximity of the interaction species and non-negligible orbital overlap.
The individual terms in this grouping are discussed in much more detail in references
\citenum{Jeziorski1994a, Hohenstein2010a, Hohenstein2010b, Hohenstein2010c, Hohenstein2011a, Hohenstein2012a}.
Reference \citenum{Jeziorski1994a} in particular provides a comprehensive review of the theory.


\subsection{EFP}
The effective fragment potential method (EFP) is an \emph{ab-initio}-based potential method that models the intermolecular interactions of non-covalently bound systems.
\cite{Gordon2001a, Gordon2007a, Gordon2009a, Mullin2009a, Ghosh2010a}.
In the EFP method, the system is broken into fragments (typically each of the solvent molecules is a fragment). 
In this study, since only ion pairs are considered, there are only two fragments. 
These fragments are treated separately at an \emph{ab-initio} level of theory. 
Then the cation and anion are allowed to interact and the interaction energy is decomposed into individual components.


The EFP method partitions the interaction energy in the following way:
\begin{equation*}
    E_{\text{total interaction}} = E_{\text{Elst}} + E_{\text{Pol}} + E_{\text{Disp}} + 
                                    E_{\text{Repl}} + E_{\text{CT}}
\end{equation*}
In order, these are the electrostatic, induction (polarization), dispersion, exchange-repulsion and charge-transfer components.
Coulomb, induction and dispersion are considered long-range interactions.
They decay as $R^{-n}$, with $n = 1$ for Coulomb, $n = 2$ to 4 for induction, and $n = 6$ for dispersion.
The short-range interactions which decay exponentially are exchange-repulsion and charge-transfer.
The Coulomb interaction uses Stone's distributed multipolar analysis 
\cite{Stone1996a} 
for the electrostatic term, truncated at the octopole term.
This term dominates the binding energy, so it is expected for EFP to treat electrostatics quite robustly.
Induction, also known as polarisation in the EFP method, is the effect from the electric field of a molecule on the induced dipole moment of another.
This term is treated with localised polarisability tensors, truncated at the dipole term.
The polarisability tensors are located at the centroids of localised bond and lone pair orbitals of the molecules.
\cite{Li2006a}


Polarisation in EFP is analogous to induction in SAPT. 
Note that in SAPT, the induction term also carries the charge-transfer components, whereas it is calculated as a separate term in EFP, the interaction between the occupied orbitals on one EFP fragment with the virtual orbitals of another fragment.
Further in the text the full induction term from SAPT will be compared with EFP polarisation.


Dispersion is treated by the sum of two terms,
\begin{equation}
    \energ{Disp} = \frac{C_6}{R^6} + \frac{C_8}{R^8}.
\end{equation}
The first term is the induced dipole--induced dipole interaction.
In the EFP method, the coefficients for this $C_6$ term are calculated through the interactions between pairs of LMOs of each ion using the time-dependent HF method.
The $C_8$ is approximated as $1/3$ of the $C_6$ term
\cite{Adamovic2005a}.
This dispersion term was formulated with SAPT dispersion as the benchmark.


The exchange-repulsion is also calculated using a static LMO basis by expanding the intermolecular overlap integral, with truncation at the quadratic term for exchange-repulsion.
\begin{equation}
   \begin{split}
    E^{\text{exch}}_{ij} = & -4 \sqrt{\frac{-2}{\pi} \ln \lvert S_{ij} \rvert } \frac{S^2_{ij}}{R_{ij}} 
                             -2 S_{ij} \left( \sum_{k \in A} F^A_{ik} S_{kj} + \sum_{l \in B} F^B_{jl}S_{il} - 2 T_{ij} \right) \\
                             &  -2 S^2_{ij} \left( \sum_{I \in A} \frac{Z_I}{R_{Ij}}  + 2 \sum_{k \in A} \frac{1}{R_{kj}} + 
                                 \sum_{J \in B} \frac{Z_J}{R_{iJ}} + 2 \sum_{l \in B} \frac{1}{R_{il}} - \frac{1}{R_{ij}} \right)
   \end{split}
\end{equation}
Where $A,B$ are the effective fragments, $i, j, k$ and $l$ are the LMOs, and $I, J$ are the nuclei. 
$S$ refers to the intermolecular overlap integral, and $T$ to the kinetic energy integral.
The Fock matrix element is represented by $F$
\cite{Ghosh2010a}.
It is expected that higher order correlation effects are not well accounted for in second order exchange-repulsion.
For charge-transfer, the EFP method uses a second order perturbation at the HF level of theory
\cite{Li2006a}.
It accounts for the stabilisation that occurs when an occupied valence molecular orbital on one ion interacts with the unoccupied virtual orbitals on the other ion.
Therefore, for an ion pair, there is charge-transfer energy of the cation induced by the anion, as well as the charge-transfer energy of the anion induced by the cation.
As these calculations involve the virtual orbitals, it can be much slower as they are a large number of virtual orbitals relative to occupied orbitals. 
For example, the water molecule has five occupied orbitals but 60 virtual orbitals in the 6-31++G(3df,2p) basis set.
Whereas the other interactions involve occupied orbitals only, looping through the virtual orbitals means that the charge-transfer calculation is usually 20-30 times slower than that for the other terms.
\cite{Li2006a}
In order to trim the expense of CT calculations, quasiatomic minimal-basis-set orbitals (QUAMBOs) 
\cite{Lu2004a} 
are used.
Essentially, these are used to represent the valence virtual orbitals, which is where the most important CT interactions occur in the virtual space.


While the computational costs for each component varies depending on system size and complexity, in general the most expensive interactions to calculate are the exchange-repulsion and charge-transfer interactions.
These two components might be more than five times as computationally demanding than the other three components, which are of roughly the same cost relative to each other.


Originally the exchange-repulsion and charge-transfer were designed for neutral molecules and therefore it is suggested that these terms might not perform as well for charged species such as ionic liquids.
These interactions will be stronger as there will be greater orbital overlap.
While the higher energies might not result in higher relative errors, the absolute error might be larger.


In comparing EFP and SAPT, the Table \ref{tab:sapt-efp-energy-comp} describes which terms from each method will be compared against each other.

\begin{table}
    \centering
    \caption{Components from SAPT and EFP that are compared with each other, and the common name for energetic components}
    \label{tab:sapt-efp-energy-comp}
    \begin{tabular}{c|c|c}
        \hline
        SAPT name               & EFP name      & Common abbreviation   \\ \hline
        \energ{electrostatics}  & \energ{Elst}  & \energ{Elst}          \\
        \energ{exchange}        & \energ{Repl}  & \energ{Exch}          \\
        \energ{induction}       & \energ{Pol}   & \energ{Ind}           \\
        \energ{dispersion}      & \energ{Disp}  & \energ{Disp}          \\
        \energ{charge-transfer} & \energ{CT}    & \energ{CT}            \\ \hline
    \end{tabular}
\end{table}



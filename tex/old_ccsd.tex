
% subsection: CCSD(T) and SAPT
% and aug-cc-pVTZ as well

To ensure the results given by SAPT agree well with other benchmark methods, the SAPT interaction energies were compared with CCSD(T)/CBS energies obtained in previous work 
\cite{Rigby2014a}
on the same series of ionic liquid ion pairs, at the same configurations.


\begin{table}[h]
\centering
\footnotesize
\caption{Differences between SAPT and CCSD(T)/CBS in \enUnit.}
\label{tab:ccsd-sapt-stats}
\begin{tabular}{llrrrrr}
\hline
  Halide    & Cation & MAE  & Med & SD & Min & Max  \\ \hline
  ILA   & \catb{mim}{n} & 3.1  & -1.2  & 3.1  & -8.7 & 1.5  \\ 
  Hal       & \catb{mim}{n} & 3.6  & -3.2  & 2.8  & -8.2  & 0.6 \\ 
  ILA   & \catb{mpyr}{n} & 2.1  & 0.3  & 2.3  & -4.9  & 2.1 \\ 
  Hal       & \catb{mpyr}{n} & 0.6  & 0.3  & 0.7  & -1.1  & 1.5 \\ \hline
\end{tabular}
\end{table}

The performance for SAPT2+3 was compared to that of CCSD(T)/CBS in table \ref{tab:ccsd-sapt-stats}.
On average the SAPT2+3 performs within chemical accuracy, with a mean absolute error of 3.6 \enUnit. 
It performs equally well for halides and other typical ionic liquid anions, with a maximum error of -8.7 and -8.2 \enUnit, observed for tosylate and chloride systems, respectively.
Pyrrolidinium based ion pairs usually showed closer energies compared to CCSD(T) results.
Overall, due to small deviations between the two methods; SAPT2+3 in combination with aug-cc-pVDZ can be reliably used for studying energetics of ILs. 


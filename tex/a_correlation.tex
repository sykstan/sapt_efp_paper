% subsubsection 
% subsection: Scaled EFP

{\emph 
    Here put forth scaling of EFP by individual components.
    Also scaling by ratios.
    Discuss errors from these methods, and explain why it is not viable for EFP. 
}

For comparison, correlation scatterplots have been graphed by plotting SAPT against EFP.
These scatterplots are used to convey an idea of how closely the EFP and SAPT numbers agree for each component of the interaction energy, and how this is affected by the basis set used and the anion in the system.


These plots have the SAPT energy on the horizontal axis and the EFP energy on the vertical axis in figure 
\ref{fig:sapt-efp-corr}

The points are coloured by the anion in the ion pair, whereas the shape indicates the cation.
Linear regression has been performed on groups of points where a trend is clearly present.
The line $ y = x $ is plotted as well to give an idea of how well the two methods agree.
The closer the points lie to this line, the better the agreement. 

\paragraph{Electrostatics}
The separation between halides and non-halides is clearly seen; halides tend to have their electrostatic interaction underestimated, while non-halides have it overestimated. 
Out of all the different components of the interaction energy, electrostatics, the largest contribution, is treated best. 
Both lines enjoy the highest $R^2$ out of all the components.

\paragraph{Exchange-repulsion}
Again the halide/non-halide and underestimated/overestimated trend is seen. 
Halides are better treated overall, even though they have larger energies; usually systems with smaller energies have correspondingly smaller differences.
This occurs when the relative error stays the same.
Note also that the halides form clusters based on the cation; imidazolium clusters are  more dispersed.
For the non-halides, the dicyanamide anion has the least difference between SAPT and EFP, falling very close to the $y = x$ line.
The other anions appear to follow a different trend, becoming increasingly underestimated as the exchange-repulsion grows.
The disparity between dicyanamide and the other anions accounts for the slightly lower $R^2$.

\paragraph{Induction (polarization)}
Clusterring based on the anion is seen, but with induction the opposite trend is observed---the contribution is underestimated for non-halides.
Furthermore, it is neither over nor underestimated for the halides.
In fact, the line fitted for induction in halides has the slope closest to 1 out of all the lines fitted, indicating very little systematic differences between the methods.
However, whereas the pyrrolidinium systems cluster around the central line, the imidazolium halides are more scattered and result in a lower $R^2$.
For non-halides, the mesylates and tosylates have stronger contributions, hence are plotted between the halides and the other anions. 
The other anions are tightly clustered at the top right of the graph.

\paragraph{Dispersion}
With dispersion, the trend is once again observed: underestimation for halides, and overestimation for non-halides, especially as the energy gets larger.
The larger dispersion contributions occur for imidazolium systems.
Clustering based on anion can also be seen.
Out of all the different trend lines fitted for the various components, the dispersion regression for non-halides has the gradient farthest from 1. 

\paragraph{Charge-transfer}
Due to the disperse nature of the plot, it was decided that linear fitting would not be meaningful or informative.

\paragraph{Total Interaction Energy}
The total interaction energy is similar to electrostatics, since the other contributions are much smaller than electrostatics.
The other components add more variance to the previously tight agreement for the Coulomb interaction.
While the non-halides have a better fit, the halides actually fall closer to the central line, albeit with greater spread.
The non-halides have a better fit instead. 

%\end{multicols}
\begin{figure}
    \centering
    \mbox{
    \subfigure[Electrostatics]{\includegraphics[scale=0.5]{\string~/GoogleDrive/SAPT-EFP/images/sapt_efp_corr/iElec.pdf}}
    \subfigure[Exchange-Repulsion]{\includegraphics[scale=0.5]{{\string~/GoogleDrive/SAPT-EFP/images/sapt_efp_corr/iExch.Repl}.pdf}}
    }
    \mbox{
    \subfigure[Induction (polarization)]{\includegraphics[scale=0.5]{{\string~/GoogleDrive/SAPT-EFP/images/sapt_efp_corr/iInd.Pol}.pdf}}
    \subfigure[Dispersion]{\includegraphics[scale=0.5]{\string~/GoogleDrive/SAPT-EFP/images/sapt_efp_corr/iDisp.pdf}}
    }                                 
    \mbox{                            
    \subfigure[Charge-transfer]{\includegraphics[scale=0.5]{\string~/GoogleDrive/SAPT-EFP/images/sapt_efp_corr/iCT.pdf}}
    \subfigure[Total Interaction Energy]{\includegraphics[scale=0.5]{{\string~/GoogleDrive/SAPT-EFP/images/sapt_efp_corr/iTotal.E}.pdf}}
    }
    % need the \protect to make hyperref and the macro happy together
    \caption{Correlation plots of SAPT and EFP   \label{fig:sapt-efp-corr}}
\end{figure}

\begin{table}[ht]
\centering
\scriptsize
\begin{tabular}{lllrrrrrrrr}
  \hline
Basis & Class & Energy & Coef & Std Err & $R^2$ & resid.mean & resid.med & resid.sd & resid.min & resid.max \\ 
  \hline
aug-cc-pVDZ & non-hal & Elec & 0.9737 & 0.0025 & 0.9991 & 7.1818 & -1.6956 & 10.6218 & -27.8494 & 57.0315 \\ 
  aug-cc-pVDZ & non-hal & Exch.Repl & 1.1580 & 0.0067 & 0.9957 & 5.4561 & 1.6252 & 6.5243 & -22.0740 & 12.3055 \\ 
  aug-cc-pVDZ & non-hal & Ind.Pol & 1.1911 & 0.0064 & 0.9963 & 2.0612 & -0.1933 & 3.0010 & -17.6819 & 5.7811 \\ 
  aug-cc-pVDZ & non-hal & Disp & 0.8146 & 0.0055 & 0.9941 & 3.4071 & -1.9731 & 3.9554 & -6.9767 & 9.7747 \\ 
  aug-cc-pVDZ & non-hal & Total.E & 0.9276 & 0.0029 & 0.9988 & 9.6032 & -2.7835 & 12.7868 & -30.9796 & 55.9876 \\ 
  aug-cc-pVTZ & non-hal & Elec & 0.9863 & 0.0016 & 0.9997 & 5.1377 & -0.2945 & 6.5233 & -15.5519 & 19.4388 \\ 
  aug-cc-pVTZ & non-hal & Exch.Repl & 1.1623 & 0.0112 & 0.9882 & 9.4745 & 6.3946 & 10.8537 & -32.2931 & 12.6226 \\ 
  aug-cc-pVTZ & non-hal & Ind.Pol & 1.1668 & 0.0070 & 0.9954 & 2.7207 & -1.3758 & 3.3539 & -13.0093 & 12.1924 \\ 
  aug-cc-pVTZ & non-hal & Disp & 0.8086 & 0.0051 & 0.9949 & 3.3648 & -2.3141 & 3.7243 & -7.7728 & 8.6443 \\ 
  aug-cc-pVTZ & non-hal & Total.E & 0.9254 & 0.0023 & 0.9992 & 7.7169 & -1.6065 & 10.0614 & -19.6615 & 33.5950 \\ 
  6-311++G** & non-hal & Elec & 0.9892 & 0.0023 & 0.9993 & 6.0710 & -1.9473 & 9.2365 & -16.0404 & 39.0768 \\ 
  6-311++G** & non-hal & Exch.Repl & 1.4152 & 0.0209 & 0.9745 & 14.4691 & 11.6305 & 15.3461 & -37.5617 & 21.3902 \\ 
  6-311++G** & non-hal & Ind.Pol & 1.3052 & 0.0081 & 0.9954 & 2.5407 & -0.8368 & 3.2865 & -15.4649 & 9.3319 \\ 
  6-311++G** & non-hal & Disp & 0.9370 & 0.0080 & 0.9913 & 4.0795 & -2.0088 & 4.6669 & -7.9048 & 10.6908 \\ 
  6-311++G** & non-hal & Total.E & 0.9261 & 0.0033 & 0.9985 & 11.5133 & -4.8780 & 14.0558 & -21.3712 & 43.0391 \\ 
  aug-cc-pVDZ & hal & Elec & 0.9708 & 0.0031 & 0.9995 & 8.0129 & 0.8424 & 9.7458 & -19.1339 & 17.3136 \\ 
  aug-cc-pVDZ & hal & Exch.Repl & 1.0028 & 0.0062 & 0.9981 & 5.9842 & 1.8605 & 7.2120 & -18.5766 & 15.2414 \\ 
  aug-cc-pVDZ & hal & Ind.Pol & 1.1120 & 0.0134 & 0.9929 & 5.3400 & 0.9077 & 7.0690 & -21.8243 & 13.1596 \\ 
  aug-cc-pVDZ & hal & Disp & 1.2209 & 0.0107 & 0.9962 & 2.6489 & -1.3324 & 3.0115 & -6.4782 & 5.5002 \\ 
  aug-cc-pVDZ & hal & Total.E & 1.0016 & 0.0068 & 0.9977 & 15.4243 & -0.3512 & 18.9186 & -40.5806 & 38.7764 \\ 
  aug-cc-pVTZ & hal & Elec & 1.0178 & 0.0032 & 0.9995 & 7.3612 & 1.7070 & 9.4623 & -29.6939 & 17.9819 \\ 
  aug-cc-pVTZ & hal & Exch.Repl & 0.9674 & 0.0052 & 0.9986 & 5.0906 & -1.2181 & 6.1792 & -9.8547 & 16.4868 \\ 
  aug-cc-pVTZ & hal & Ind.Pol & 1.0136 & 0.0198 & 0.9821 & 8.1301 & -1.4680 & 11.1708 & -15.9871 & 39.5327 \\ 
  aug-cc-pVTZ & hal & Disp & 1.1057 & 0.0093 & 0.9966 & 2.4567 & -1.1375 & 2.8500 & -6.1600 & 5.1061 \\ 
  aug-cc-pVTZ & hal & Total.E & 1.0185 & 0.0076 & 0.9973 & 15.1033 & -1.8146 & 20.6345 & -35.2910 & 58.9812 \\ 
  6-311++G** & hal & Elec & 1.0252 & 0.0035 & 0.9994 & 7.3637 & 0.9038 & 10.2525 & -24.5281 & 30.8930 \\ 
  6-311++G** & hal & Exch.Repl & 1.1181 & 0.0233 & 0.9792 & 21.8097 & 1.4749 & 23.7515 & -38.1388 & 36.0272 \\ 
  6-311++G** & hal & Ind.Pol & 1.4874 & 0.0267 & 0.9845 & 8.1916 & -1.7509 & 10.4225 & -24.6313 & 25.4874 \\ 
  6-311++G** & hal & Disp & 2.1177 & 0.0248 & 0.9933 & 3.3020 & -0.7121 & 4.0348 & -7.7202 & 7.5233 \\ 
  6-311++G** & hal & Total.E & 1.0999 & 0.0088 & 0.9969 & 17.7504 & -0.5401 & 22.2757 & -73.9961 & 34.9213 \\ 
   \hline
\end{tabular}
    \caption{Coefficients and associated fitting statistics}
    \label{tab:coef_indiv}
\end{table}


\begin{table}[ht]
\centering
\footnotesize
\begin{tabular}{lllrrrrr}
  \hline
Basis & Class & Total.E & mean & med & sd & min & max \\ 
  \hline
adz & hal & rawEFP & 15.44 & -0.99 & 18.90 & -41.17 & 38.06 \\ 
  adz & hal & indivEFP & 15.42 & -0.35 & 18.92 & -40.58 & 38.78 \\ 
  adz & hal & summedEFP.sapt & 15.31 & -0.78 & 18.83 & -39.33 & 36.20 \\ 
  adz & hal & summedEFP.ccsd & 13.60 & 0.06 & 17.36 & -35.16 & 37.99 \\ 
  adz & non-hal & rawEFP & 27.42 & 26.18 & 15.04 & -6.56 & 92.61 \\ 
  adz & non-hal & indivEFP & 9.60 & -2.78 & 12.79 & -30.98 & 55.99 \\ 
  adz & non-hal & summedEFP.sapt & 7.73 & -2.79 & 11.72 & -40.60 & 54.17 \\ 
  adz & non-hal & summedEFP.ccsd & 8.32 & -1.82 & 12.58 & -35.22 & 62.41 \\ 
  atz & hal & rawEFP & 16.45 & -8.35 & 20.32 & -42.00 & 50.91 \\ 
  atz & hal & indivEFP & 15.10 & -1.81 & 20.63 & -35.29 & 58.98 \\ 
  atz & hal & summedEFP.sapt & 13.67 & -0.53 & 18.21 & -43.03 & 46.65 \\ 
  atz & hal & summedEFP.ccsd & 13.28 & 0.29 & 17.71 & -42.03 & 47.36 \\ 
  atz & non-hal & rawEFP & 28.45 & 27.28 & 12.35 & 5.68 & 69.66 \\ 
  atz & non-hal & indivEFP & 7.72 & -1.61 & 10.06 & -19.66 & 33.59 \\ 
  atz & non-hal & summedEFP.sapt & 8.21 & -0.46 & 10.77 & -23.83 & 33.45 \\ 
  atz & non-hal & summedEFP.ccsd & 9.65 & 1.54 & 12.77 & -24.06 & 41.08 \\ 
  pop & hal & rawEFP & 36.33 & -36.71 & 21.27 & -104.35 & -1.14 \\ 
  pop & hal & indivEFP & 17.75 & -0.54 & 22.28 & -74.00 & 34.92 \\ 
  pop & hal & summedEFP.sapt & 17.12 & 5.94 & 21.48 & -71.76 & 36.14 \\ 
  pop & hal & summedEFP.ccsd & 16.81 & 6.33 & 20.81 & -69.74 & 35.42 \\ 
  pop & non-hal & rawEFP & 27.75 & 23.49 & 15.90 & 4.48 & 77.32 \\ 
  pop & non-hal & indivEFP & 11.51 & -4.88 & 14.06 & -21.37 & 43.04 \\ 
  pop & non-hal & summedEFP.sapt & 14.69 & 2.68 & 18.50 & -33.63 & 52.28 \\ 
  pop & non-hal & summedEFP.ccsd & 15.95 & 2.52 & 19.92 & -35.08 & 55.70 \\ 
   \hline
\end{tabular}
    \caption{Statistics for the differences from various methods of calculating the Total Energy}
    \label{tab:si.stats.recast}
\end{table}

%\begin{multicols}{2}

The coefficients along with their errors and other statistics from the fitting are in table 
\ref{tab:coef_indiv}.
The coefficients give a clear indication of how EFP performs for each component.
A slope greater than 1 means a tendency to underestimate the SAPT total energy, and vice versa for a slope less than 1. 
Note that for the non-halides, exchange-repulsion is the component with the largest standard errors, for all three basis sets.
This is followed by induction and then by dispersion.
For the halides, however, exchange-repulsion is better treated than dispersion, which in turn performs better than induction.
For both halides and non-halides, the component with the lowest standard error is electrostatics. 
Electrostatics dominates the interaction energy, so the total energy is always the second best in terms of standard errors, due to contributions from the other components.

In table 
\ref{tab:si.stats.recast}
the statistics of the differences between the SAPT total energies and the predicted energies are tabulated.
The predicted energies are obtained through different methods. 
The first is from an EFP calculation.
The second is from regressing the EFP total energy against the SAPT total energy; the predicted energies are obtained by multiplying the EFP energies with the corresponding coefficient.
The third method is similar to the second, except it derives the total energy as a sum of the components (electrostatics, exchange-repulsion, induction-polarisation and dispersion).
From regressing the EFP components against the corresponding SAPT components, the total energy is the sum of the predicted energies of the components.
As can be seen from the table, this third method of summing the scaled components, has the best performance overall.
The only cases where the mean difference does not improve from the first to the third method are the non-halides for the triple zeta and the Pople basis sets.
In both cases it is still a significant improvement over the raw EFP total energy.


%%%%%%%%%%%%%%%%%%%%%%%%%%%%%%%%%%%%%%%%%%%%%%%%%%%%%%%%%%%%%%%%%%%%%
%%%% not sure if we should still talk about multilinear regression

Multilinear regression was performed by treating the EFP components as predictors and regressing against the total energy from both SAPT2+3 and CCSD(T)/CBS.
Furthermore, halides and non-halides were treated separately.
Since there are three basis sets for EFP, each component had twelve had twelve coefficients, for each basis set, for halide and non-haldie systems, and for different benchmarks.


\begin{equation}
    E^{\text{SAPT}}_{\text{Total Energy}} = \alpha E_{\text{Elec}}^{\text{EFP}} +
                                            \beta E_{\text{Exch-Repl}}^{\text{EFP}} +
                                            \gamma E_{\text{Ind-Pol}}^{\text{EFP}} +
                                            \delta E_{\text{Disp}}^{\text{EFP}}
\end{equation}

From the linear correlations discussed above, predicted energies for each of the components were obtained and compared with the SAPT values. 
In table 
\ref{tab:indi_scaled}
the statistics for the differences of these predicted energies and the SAPT results are tabulated. 
%\end{multicols}

\begin{table}[ht]
\centering
\scriptsize
\begin{tabular}{lllrrrrrrr}
  \hline
Basis & Halide & Stat & s.sapt.diff\_Elec & s.sapt.diff\_Exch.Repl & s.sapt.diff\_Ind.Pol & s.sapt.diff\_Disp & s.sapt.diff\_Total.E & ss.sapt.diff & ss.ccsd.diff \\ 
  \hline
adz & non-hal & mean & -0.00 & -0.00 & -0.00 & -0.00 & -0.00 & 0.00 & 65.14 \\ 
  adz & non-hal & median & -0.57 & 0.57 & 0.61 & 0.23 & 1.18 & -1.27 & 0.93 \\ 
  adz & non-hal & sd & 14.43 & 6.29 & 2.67 & 2.44 & 11.74 & 12.87 & 206.81 \\ 
  adz & non-hal & min & -30.84 & -20.05 & -17.46 & -6.29 & -29.87 & -29.03 & -23.66 \\ 
  adz & non-hal & max & 109.67 & 13.08 & 3.91 & 6.11 & 83.65 & 104.34 & 805.41 \\ 
  adz & hal & mean & -18.63 & -18.02 & 1.22 & -11.70 & -30.34 & -47.13 & -45.23 \\ 
  adz & hal & median & -16.31 & -16.23 & 1.76 & -11.61 & -28.70 & -42.47 & -41.51 \\ 
  adz & hal & sd & 13.25 & 8.73 & 6.92 & 2.45 & 14.54 & 26.64 & 24.33 \\ 
  adz & hal & min & -41.97 & -41.49 & -20.64 & -17.18 & -58.34 & -105.86 & -98.51 \\ 
  adz & hal & max & 5.35 & -2.14 & 12.47 & -7.51 & -1.57 & -3.00 & -3.17 \\ 
  atz & non-hal & mean & -0.00 & -0.00 & -0.00 & 0.00 & 0.00 & 0.00 & 65.23 \\ 
  atz & non-hal & median & 0.25 & 1.40 & -0.02 & 0.20 & 0.00 & -0.39 & 1.94 \\ 
  atz & non-hal & sd & 6.36 & 9.31 & 2.43 & 2.34 & 6.33 & 8.74 & 205.38 \\ 
  atz & non-hal & min & -15.55 & -23.97 & -13.51 & -7.30 & -11.74 & -18.67 & -18.89 \\ 
  atz & non-hal & max & 16.30 & 16.35 & 7.59 & 5.69 & 15.03 & 29.56 & 786.06 \\ 
  atz & hal & mean & -15.74 & -13.27 & 3.20 & -9.08 & -35.91 & -34.89 & -32.99 \\ 
  atz & hal & median & -14.15 & -12.94 & 3.86 & -9.34 & -35.03 & -32.34 & -30.90 \\ 
  atz & hal & sd & 8.61 & 12.85 & 10.48 & 2.94 & 17.92 & 20.72 & 19.50 \\ 
  atz & hal & min & -45.42 & -94.16 & -11.78 & -13.20 & -92.73 & -103.86 & -98.01 \\ 
  atz & hal & max & -1.88 & 3.92 & 40.73 & 6.53 & 9.48 & 15.28 & 15.99 \\ 
  pop & non-hal & mean & -0.00 & -0.00 & 0.00 & 0.00 & 0.00 & 0.00 & 57.58 \\ 
  pop & non-hal & median & -1.72 & -0.05 & 0.04 & 0.03 & -4.49 & -1.83 & -0.90 \\ 
  pop & non-hal & sd & 9.11 & 11.02 & 2.73 & 1.99 & 10.46 & 13.51 & 195.81 \\ 
  pop & non-hal & min & -16.25 & -23.55 & -15.77 & -4.81 & -12.77 & -19.75 & -21.20 \\ 
  pop & non-hal & max & 36.18 & 23.83 & 6.13 & 6.13 & 29.76 & 47.48 & 775.52 \\ 
  pop & hal & mean & -17.68 & 1.05 & -14.39 & -18.36 & -53.82 & -49.39 & -47.49 \\ 
  pop & hal & median & -15.34 & -3.27 & -12.91 & -17.76 & -54.29 & -48.70 & -46.47 \\ 
  pop & hal & sd & 9.84 & 18.00 & 9.05 & 4.67 & 18.28 & 18.80 & 17.70 \\ 
  pop & hal & min & -41.20 & -24.35 & -34.99 & -28.95 & -103.52 & -110.68 & -108.66 \\ 
  pop & hal & max & 7.36 & 29.11 & 6.45 & -11.64 & -19.88 & -17.12 & -17.84 \\ 
   \hline
\end{tabular}
\caption{Statistics for individually scaled regression \label{tab:indi_scaled} }
\end{table}

\begin{table}[ht]
    \label{tab:multiLinEFP}
\centering
\begin{tabular}{lllrrrrrr}
  \hline
Basis & Halide & Stat & diff\_Elec & diff\_Exch.Repl & diff\_Ind.Pol & diff\_Disp & diff\_Total.E & diff\_CCSD \\ 
  \hline
adz & non-hal & mean & 33.75 & -19.45 & -15.50 & -1.23 & -0.12 & -0.72 \\ 
  adz & non-hal & median & 33.42 & -16.77 & -14.63 & -2.25 & 0.21 & -1.76 \\ 
  adz & non-hal & sd & 12.46 & 8.83 & 3.37 & 3.79 & 12.46 & 11.70 \\ 
  adz & non-hal & min & 6.80 & -48.95 & -33.37 & -7.23 & -52.41 & -44.33 \\ 
  adz & non-hal & max & 102.00 & -6.44 & -10.66 & 9.36 & 53.47 & 59.76 \\ 
  atz & non-hal & mean & 31.29 & 47.08 & -35.05 & -45.64 & -0.06 & -0.52 \\ 
  atz & non-hal & median & 30.00 & 44.94 & -32.22 & -43.75 & 0.08 & -1.74 \\ 
  atz & non-hal & sd & 7.76 & 11.31 & 7.06 & 9.63 & 8.98 & 9.09 \\ 
  atz & non-hal & min & 11.94 & 24.04 & -51.90 & -74.86 & -22.65 & -20.11 \\ 
  atz & non-hal & max & 57.35 & 73.80 & -25.74 & -30.40 & 19.55 & 27.39 \\ 
  pop & non-hal & mean & 35.48 & 66.11 & -60.65 & -42.72 & 0.68 & -0.41 \\ 
  pop & non-hal & median & 32.33 & 64.16 & -55.28 & -41.42 & -0.91 & -1.13 \\ 
  pop & non-hal & sd & 10.19 & 13.55 & 12.94 & 8.29 & 11.00 & 9.14 \\ 
  pop & non-hal & min & 17.23 & 42.14 & -90.56 & -70.74 & -17.05 & -13.47 \\ 
  pop & non-hal & max & 81.35 & 103.20 & -44.96 & -29.74 & 39.71 & 38.13 \\ 
  adz & hal & mean & 4.16 & 168.01 & -82.73 & -91.51 & -34.93 & -0.18 \\ 
  adz & hal & median & 4.75 & 161.59 & -77.81 & -85.26 & -29.87 & 0.50 \\ 
  adz & hal & sd & 9.67 & 23.35 & 10.67 & 16.11 & 16.50 & 7.16 \\ 
  adz & hal & min & -15.06 & 139.32 & -109.56 & -124.93 & -67.97 & -14.07 \\ 
  adz & hal & max & 21.08 & 207.85 & -71.00 & -68.87 & -14.78 & 18.79 \\ 
  atz & hal & mean & -10.80 & 120.94 & -46.33 & -66.22 & -34.75 & -0.59 \\ 
  atz & hal & median & -8.95 & 118.08 & -45.10 & -62.07 & -33.09 & -2.04 \\ 
  atz & hal & sd & 9.13 & 16.12 & 8.19 & 10.79 & 16.18 & 13.00 \\ 
  atz & hal & min & -40.34 & 101.90 & -66.24 & -89.97 & -82.56 & -28.49 \\ 
  atz & hal & max & 5.94 & 149.00 & -34.54 & -50.29 & -3.18 & 24.58 \\ 
  pop & hal & mean & -15.70 & 129.52 & -31.25 & -84.81 & -35.09 & -0.34 \\ 
  pop & hal & median & -14.24 & 127.43 & -29.26 & -79.15 & -28.35 & 0.75 \\ 
  pop & hal & sd & 10.01 & 17.38 & 8.76 & 14.32 & 17.96 & 10.76 \\ 
  pop & hal & min & -38.95 & 105.04 & -51.90 & -112.68 & -75.69 & -32.96 \\ 
  pop & hal & max & 12.70 & 164.21 & -12.62 & -62.54 & -7.85 & 30.01 \\ 
   \hline
\end{tabular}
\caption{Statistics for multilinear regression}
\end{table}

%\begin{multicols}{2}

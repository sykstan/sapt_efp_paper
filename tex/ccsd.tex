
% subsection: theoretical procedures

\subsubsection{Chemical systems studied}
The chemical systems were all ionic liquid ion pairs. 
The anion is one of the eight anions included in this study: tetrafluoroborate, bromide, chloride, dicyanamide, mesylate, tosylate, hexafluorophosphate and bis\{(trifluoromethyl)sulfonyl\}amide. 
The cations used are based on methyl pyrrolidinium and methyl imidazolium. 
The length of the alkyl chain on the cation was varied, e.g. from dimethyl imidazolium to 1-butyl-3-methyl-imidazolium.
The bulk of the cation is varied thus, and provides some insight into how the size of the cation affects the different energy components.
The names and abbreviations of the different cations and anions, eight each, are tabulated in 
table \ref{tab:cation-anion-list}.


\end{multicols}
\begin{table}[ht]
    \begin{centering}
    \small
    \begin{tabular}{c|c|c|c}
        \hline 
        Cations  & abbreviation & Anions & abbreviation\tabularnewline
        \hline 
        dimethyl-imidazolium & $\text{C}_{1}\text{mim}^{+}$ & tetrafluoroborate &  $\text{BF}_{\text{4}}^{-}$\tabularnewline
        1-methyl-3-ethyl-imidazolium & $\text{C}_{2}\text{mim}^{+}$ & bromide &  $\text{Br}^{-}$\tabularnewline
        1-methyl-3-propyl-imidazolium & $\text{C}_{3}\text{mim}^{+}$ & chloride  & $\text{Cl}^{-}$\tabularnewline
        1-methyl-3-butyl-imidazolium & $\text{C}_{4}\text{mim}^{+}$ & dicyanamide  & $\text{Dca}^{-}$\tabularnewline
        dimethyl-pyrrolidinium & $\text{C}_{1}\text{mpyr}^{+}$ & mesylate &  $\text{Mes}^{-}$\tabularnewline
        1-ethyl-1-methyl-pyrrolidinium & $\text{C}_{2}\text{mpyr}^{+}$ & bis\{(trifluoromethyl)sulfonyl\}amide  & $\text{NTf}_{\text{2}}^{-}$\tabularnewline
        1-propyl-1-methyl-pyrrolidinium & $\text{C}_{3}\text{mpyr}^{+}$ & hexafluorophosphate &  $\text{PF}_{\text{6}}^{-}$\tabularnewline
        1-butyl-1-methyl-pyrrolidinium & $\text{C}_{4}\text{mpyr}^{+}$ & tosylate  & $\text{Tos}^{-}$\tabularnewline
        \hline 
    \end{tabular}
    \caption{List of cations and anions }
    \label{tab:cation-anion-list}
    \par\end{centering}
\end{table}
\begin{multicols}{2}


Besides the different possible combinations between ion pairs, different configurations of these ion pairs were studied as well. 
These configurations are differentiated by how the anion interacts with the cation.
Both cations are ring systems; if the anion approaches the cation from above, this is referred to as the above-plane configuration (shortened to p1).
If the anion interacts with the cation in the plane of the ring, i.e. from the side, then this is referred to as an in-plane interaction.
In-plane configurations are p2 and p3.
The below-plane configuration is very much similar to the above-plane configuration; this is designated as p4.
The different configurations for \ipair{mim}{3}{br} are presented in figure
\protect\ref{fig:conf-c3mim-br}
as an example. 

\end{multicols}
\begin{figure}
    \centering
    \mbox{
    \subfigure[\protect\ipair{mim}{3}{br} (p1) ]{\includegraphics[scale=0.3]{\string~/Dropbox/QuantumChem/il_structure_images/c3mim-br-p1.png}}
    \subfigure[\protect\ipair{mim}{3}{br} (p2) ]{\includegraphics[scale=0.3]{\string~/Dropbox/QuantumChem/il_structure_images/c3mim-br-p2.png}}
    }
    \mbox{
    \subfigure[\protect\ipair{mim}{3}{br} (p3) ]{\includegraphics[scale=0.3]{\string~/Dropbox/QuantumChem/il_structure_images/c3mim-br-p3.png}}
    \subfigure[\protect\ipair{mim}{3}{br} (p4) ]{\includegraphics[scale=0.3]{\string~/Dropbox/QuantumChem/il_structure_images/c3mim-br-p4.png}}
    }                                 
    % need the \protect to make hyperref and the macro happy together
    \caption{Different configurations of \protect\ipair{mim}{3}{br} \label{fig:conf-c3mim-br}}
\end{figure}
\begin{multicols}{2}

The \ntf anion has different configurations, as shown in figure
\ref{fig:conf-c2mpyr-ntf2}.
This is because it can interact with the cation either through the nitrogen on the ring, or the oxygens on the sulfonyl groups. 
The above-plane configurations for this anion are p1 and p2; p1 is where the nitrogen faces the ring, and p2 is where the oxygens face the ring.
The next two configurations, p3 and p4, are where the nitrogen approaches the ring obliquely from below and from the side, respectively.
The p5 and p6 configurations are the same as p3 and p4, with the oxygens approaching the ring from below, and side-on.


\end{multicols}
\begin{figure}
    \centering
    \mbox{
    \subfigure[\protect\ipair{mpyr}{2}{ntf} (p1) ]{\includegraphics[scale=0.3]{\string~/Dropbox/QuantumChem/il_structure_images/c2mpyr-ntf2-p1.png}}
    \subfigure[\protect\ipair{mpyr}{2}{ntf} (p2)]{\includegraphics[scale=0.3]{\string~/Dropbox/QuantumChem/il_structure_images/c2mpyr-ntf2-p2.png}}
    }
    \mbox{
    \subfigure[\protect\ipair{mpyr}{2}{ntf} (p3) ]{\includegraphics[scale=0.3]{\string~/Dropbox/QuantumChem/il_structure_images/c2mpyr-ntf2-p3.png}}
    \subfigure[\protect\ipair{mpyr}{2}{ntf} (p4) ]{\includegraphics[scale=0.3]{\string~/Dropbox/QuantumChem/il_structure_images/c2mpyr-ntf2-p4.png}}
    }                                 
    \mbox{                            
    \subfigure[\protect\ipair{mpyr}{2}{ntf} (p5) ]{\includegraphics[scale=0.3]{\string~/Dropbox/QuantumChem/il_structure_images/c2mpyr-ntf2-p5.png}}
    \subfigure[\protect\ipair{mpyr}{2}{ntf} (p6) ]{\includegraphics[scale=0.3]{\string~/Dropbox/QuantumChem/il_structure_images/c2mpyr-ntf2-p6.png}}
    }
    % need the \protect to make hyperref and the macro happy together
    \caption{Different configurations of \protect\ipair{mpyr}{2}{ntf} \label{fig:conf-c2mpyr-ntf2}}
\end{figure}
\begin{multicols}{2}



\subsubsection{SAPT}

The \textsc{Psi4} quantum chemistry package was used for the SAPT calculations. 
\mautocite{Turney2012}
All SAPT calculations were performed at the SAPT2+3 level of theory,using the aug-cc-pVDZ basis set.
There are some differences between this study and the work done by Flick et al. 
In the first place, they used the SAPT2+(3) level of theory, which does not include the 
$E^{(30)}_{\text{exch-disp}}, E^{(30)}_{\text{ind-disp}}, \text{and}  E^{(30)}_{\text{exch-ind-disp}}$ 
terms. 
These terms belong to the dispersion interaction, and brings the intermolecular perturbation order to 3 for all components.
Another difference is that this study considers the charge-transfer energy; this effectively doubles the computational expense, since for each chemical system the same calculation is performed twice, in both the dimer and monomer basis.

% means what terms are considered?

\subsubsection{EFP}

In the EFP method, the system is broken into fragments (typically each of the solvent molecules is a fragment). 
In this study, since only ion pairs are considered, there are only two fragments. 
These fragments are treated separately at an \emph{ab-initio} level of theory. 
Then the cation and anion are allowed to interact and the interaction energy is decomposed into individual components.

Three basis sets were used for EFP: aug-cc-pVDZ, aug-cc-pVTZ and the Pople basis set 6-311++G**. 
\footnote{For the bromide anion, since it is not included in the 6-311++G** basis set, 6-311G** basis functions were used instead.}
Three basis sets were used to see how consistent the method is between basis sets, and also the quality of the basis set that gives the best expense-to-error ratio.

\subsubsection{SAPT and CCSD(T)/CBS}

The CCSD(T)/CBS energies were also calculated for the ion pairs, and compared with the SAPT energies. 
For the total interaction energy, the statistics of the differences between the two methods are as follows: minimum -26.9 \enUnit, maximum = 25.7 \enUnit, mean -1.3 \enUnit, median -0.68 \enUnit, and standard deviation of 4.2 \enUnit. 
The outliers are \ipair{mim}{2,3}{dca} (p3): 7.47 \& -13.1 \enUnit, \ipair{im}{3}{pf} (p1): -12.0 \enUnit, \ipair{pyr}{3}{dca} (p3 \& p4): 25.9 \& -26.9 \enUnit. 

Taking the difference between SAPT2+3 and the SAPT Hartree--Fock energy, and comparing that difference with the correlation correction from CCSD(T)/CBS by subtracting the latter from the former, very good agreement is also shown. 
The minimum is -11.0 \enUnit, maximum 7.6 \enUnit, mean -2.3 \enUnit, median -1.5 \enUnit, and the standard deviation is 3.3 \enUnit.
Outliers in this case are \ipair{im}{2,3,4}{cl} (p4): -9.17, -10.5 \& -10.7 \enUnit, \ipair{im}{3,4}{br} (p4): -9.21 \& -9.35 \enUnit and \ipair{im}{3}{dca} (p2 \& p3) 7.56 \& -11.0 \enUnit.
The agreement between SAPT and CCSD(T)/CBS results validates the reliability of SAPT as a method to accurately determine the decomposition of the total interaction energy.
\subsubsection{Basis sets}
While three basis sets were used to give an indication of basis set dependency for the EFP method, only the aug-cc-pDVZ basis set was used for SAPT. 
There were some test calculations using a larger basis set, aug-cc-pVTZ, to determine whether aug-cc-pVDZ gave satisfactory accuracy.
The difference between the two basis sets are tabulated in 
table \ref{tab:adiff-sapt}.
The calculations using aug-cc-pVTZ were run on some representative halide systems. 
Halides were selected because of smaller system sizes (monoatomic anions) and because they are more difficult to model. 
Due to the problematic nature of the charge-transfer interaction in the halides, the following discussion will ignore the differences in charge-transfer energies between the two basis sets.


One immediately observes the consistency between the two basis sets: with electrostatics, the triple zeta basis set always returns weaker energies (less negative), but correspondingly its exchange is less repulsive (less positive).
The dispersion energy is also always stronger in the aug-cc-pVTZ basis set. 
While there is no such pattern in the difference in the induction energies, this is the interaction both basis sets agree best on, with absolute differences below 1 \enUnit.
In the total interaction energy, the largest difference is 9 \enUnit. 
Considering the standard deviations, where these two basis sets differ, they differ \emph{consistently} for each energetic component. 
This indicates that while they may not agree on the exact numbers, the trends between different chemical systems is well reflected.
The fact that both basis sets showed such agreement for the halides, which are typically challenging, indicates that the aug-cc-pVTZ basis set would not provide much more insight and accuracy than the significantly less expensive aug-cc-pVDZ basis.


The aug-cc-pVTZ basis set requires tremendous amounts of time, in most cases more than double the amount of time required for the aug-cc-pVDZ basis set, which is already quite time consuming.


\end{multicols}


% latex table generated in R 3.0.2 by xtable 1.7-3 package
% Wed Mar 19 14:34:48 2014
\begin{table}[ht]
    \centering
    \small
    \begin{tabular}{lcccccc}
	
	  \hline
	 Name (Conf) & Electrostatics & Exchange & Induction & Dispersion & Total Energy & Charge-transfer \\ 
	\hline
	   \ipair{mim}{1}{br} (p1) & 4.77 & -5.92 & 0.90 & -7.05 & -7.30 & 12.17 \\ 
	   \ipair{mim}{1}{br} (p2) & 1.42 & -3.79 & -0.20 & -5.29 & -7.87 & 7.45 \\ 
	   \ipair{mim}{1}{cl} (p1) & 4.23 & -5.96 & 0.89 & -6.61 & -7.45 & 9.94 \\ 
	   \ipair{mim}{1}{cl} (p2) & 1.14 & -4.27 & -0.24 & -5.62 & -9.00 & 8.30 \\ 
	   \ipair{mim}{2}{br} (p1) & 4.43 & -5.87 & 1.02 & -6.83 & -7.25 & 10.09 \\ 
	   \ipair{mim}{2}{br} (p2) & 1.66 & -4.07 & -0.04 & -5.21 & -7.66 & 7.64 \\ 
	   \ipair{mim}{2}{br} (p1) & 1.90 & -4.17 & 0.13 & -5.38 & -7.51 & 7.10 \\ 
	   \ipair{mim}{2}{br} (p2) & 4.77 & -5.93 & 1.00 & -6.87 & -7.03 & 11.73 \\ 
	   \ipair{mim}{2}{cl} (p1) & 3.92 & -5.89 & 0.96 & -6.41 & -7.41 & 8.63 \\ 
	   \ipair{mim}{2}{cl} (p2) & 1.31 & -4.42 & -0.13 & -5.49 & -8.73 & 8.68 \\ 
	   \ipair{mim}{2}{cl} (p3) & 1.58 & -4.56 & 0.10 & -5.56 & -8.45 & 7.81 \\ 
	   \ipair{mim}{2}{cl} (p4) & 4.25 & -5.94 & 0.95 & -6.40 & -7.13 & 9.69 \\ 
	   \ipair{mpyr}{3}{cl} (p1) & 2.00 & -4.61 & 0.90 & -4.41 & -6.11 & 6.49 \\ 
	   \ipair{mpyr}{3}{cl} (p2) & 1.71 & -4.36 & 0.63 & -4.29 & -6.30 & 5.99 \\ 
	   \ipair{mpyr}{3}{cl} (p3) & 1.77 & -4.50 & 0.75 & -4.47 & -6.44 &  \\ 
	   \ipair{mim}{4}{cl} (p1) & 4.00 & -5.93 & 1.06 & -6.21 & -7.08 & 8.33 \\ 
	   \ipair{mim}{4}{cl} (p2) & 1.40 & -4.43 & -0.10 & -5.27 & -8.41 &  \\ 
	   \ipair{mim}{4}{cl} (p3) & 1.95 & -4.89 & 0.36 & -5.55 & -8.13 & 7.52 \\ 
	   \ipair{mim}{4}{cl} (p4) & 4.64 & -6.17 & 1.13 & -6.29 & -6.68 &  \\ 
	\hline
	   Mean and Std Dev & 2.78 $\pm$ 1.43 & -5.04 $\pm$ 0.83 & 0.53 $\pm$ 0.51 & -5.75 $\pm$ 0.84 & -7.47 $\pm$ 0.81 & 8.60 $\pm$ 1.74 \\   
	\hline 
    \end{tabular}
    \caption{Differences between SAPT results from two basis sets, 
                $ E_{\text{SAPT}}(\text{aug-cc-pVTZ}) - E_{\text{SAPT}}(\text{aug-cc-pVDZ}) $. 
                All energies are in \enUnit. }
    \label{tab:adiff-sapt}
\end{table}

\begin{multicols}{2}

\subsubsection{Presentation format}
% this section may be obsolete--need better methods of presenting data
In
figure \ref{fig:pure_en-sapt_Elec}
is an example of how the results for each energy are plotted. 
This is for the Electrostatic energy from the SAPT calculations.
The graph is divided into a grid, with the cations on the rows and the anions forming the columns. 
Methyl imidazolium is "im", and methyl pyrrilidinium is "pyr". 
For more information about abbreviations used, please refer to 
table \ref{tab:cation-anion-list}.
Within the grid, the horizontal axis represents the length of the alkyl chain on the cation. 
For example, the induction energy for 
\ipair{mim}{1}{bfl} 
is plotted in the top left graph as the leftmost point.
The different shapes of the points represent the different configurations between the ion pairs. 
Thus in the previous example the first configuration, called "p1" in the legend, was used.
Some ion pairs have more configurations than others. 
Note that there are missing data points for calculations which have not yet completed successfully. 
For example, none of the data for 
\ipair{mim}{4}{ntf}  
is available.



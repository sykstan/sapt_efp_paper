
% subsection: CCSD(T) and SAPT
% and aug-cc-pVTZ as well

% but clear up the formulas and definitions first
Differences between each component of the total interaction energy in SAPT and EFP are calculated by subtracting the EFP energy from the SAPT energy, referred to as the absolute difference further in the text.

%\begin{equation}
%    \energ{abs diff} = \energ{SAPT}(\text{INT}) - \energ{EFP}(\text{INT})
%\end{equation}

For comparison, the relative error is calculated as follows: 

\begin{equation}
    100 \cdot \frac{ \energ{SAPT}(\text{INT}) - \energ{EFP}(\text{INT}) } { \energ{SAPT}(\text{INT})} 
\end{equation}
where $E(\text{INT})$ refers to the interaction energy.

Traditional statistics such as the mean absolute error, standard deviation, and absolute maximum were also used in the analysis of the results. 
Mean absolute error (MAE) is defined as the average of the absolute differences:

%\begin{equation}
%    \text{MAE} = \frac{1}{N} \sum \lvert E(\text{INT}) \rvert,
%\end{equation}
%where $N$ is the number of ion pairs.

Standard deviation (SD) refers to sample standard deviation, that is, using $N-1$ in the denominator,
\begin{equation}
    \text{SD} = \sqrt{\frac{\sum (X - \bar{X})^2}{N-1}} ,
\end{equation}
where $\bar{X} $ refers to the sample mean.



\subsection{CCSD(T) and SAPT}
\label{subsec:ccsd}
To ensure the results given by SAPT agree well with other benchmark methods, the SAPT interaction energies were compared with CCSD(T)/CBS energies obtained in previous work 
\cite{Rigby2014a}
on the same series of ionic liquid ion pairs, at the same configurations. The comparison is given in Table \ref{tab:ccsd-sapt-stats}.


\begin{table}[h]
\centering
\footnotesize
\caption{Differences between SAPT and CCSD(T)/CBS in \enUnit.}
\label{tab:ccsd-sapt-stats}
\begin{tabular}{llrrrrr}
\hline
  Halide  & Cation          & MAE   & SD    & Min   & Max \\ \hline
  TILA    & \catb{mim}{n}   & 3.1   & 3.1   & -8.7  & 1.5 \\ 
  Hal     & \catb{mim}{n}   & 3.6   & 2.8   & -8.2  & 0.6 \\ 
  TILA    & \catb{mpyr}{n}  & 2.1   & 2.3   & -4.9  & 2.1 \\ 
  Hal     & \catb{mpyr}{n}  & 0.6   & 0.7   & -1.1  & 1.5 \\ \hline
\end{tabular}
\end{table}

On average the SAPT2+3 method performs within chemical accuracy, with a mean absolute error of 3.6 \enUnit. 
The differences are consistent for halides as well as other typical ionic liquid anions, with a maximum error of -8.7 and -8.2 \enUnit, observed for tosylate and chloride systems, respectively.
Pyrrolidinium based ion pairs usually showed smaller differences when compared to CCSD(T) results.
Overall, due to small deviations between the two methods, SAPT2+3 in combination with aug-cc-pVDZ produces reliable energetics for ILs. 

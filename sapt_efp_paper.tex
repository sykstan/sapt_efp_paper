\documentclass[final]{article}

% generic packages and options

% generic preamble

%%%%%%%%%%%%%%%%%%%%%%%%%%%%%%%%%
% leave macro definitons and stuff in the actual file preamble
% also packages only used by that .tex file
% can \input{} this file
%%%%%%%%%%%%%%%%%%%%%%%%%%%%%%%%%


%\usepackage{gfsartemisia}
%\usepackage[math]{iwona}    %no ink traps
%\usepackage[math]{kurier}    %gaps between certain letters
%usepackage[sc]{mathpazo}
\linespread{1.1}
\usepackage[T1]{fontenc}
\usepackage[utf8]{inputenc}
\usepackage[a4paper,margin=2.0cm]{geometry}
%\usepackage{changepage}       %change page margins for figures
%\usepackage{marginnote}
%\usepackage{multicol}
%\usepackage{amsmath}
%\usepackage{breqn}
% allows onehalf & double & arbitrary spacing
\usepackage{setspace}
\usepackage[demo]{graphicx}
%\usepackage{pdfpages}
%\usepackage{subfigure}
%\usepackage{lscape}
%\usepackage{rotating}

% for pretty tables
\usepackage{booktabs}

\usepackage[toc, header]{appendix}

% for \IfEqCase conditionals
\usepackage{xstring}

%\usepackage{fancyhdr}
%%\setlength{\headhight}{14pt}
%\pagestyle{fancy}
%\fancyhf{}
%\renewcommand{\headrulewidth}{0.2pt}
%\lhead{}
%\chead{\rightmark}
%\rhead{}

\usepackage[backend=biber,sorting=none,style=numeric-comp,mcite,subentry,
                firstinits=true, autocite=superscript]{biblatex}
\bibliography{\string~/Dropbox/Papers/zot_il_lib}

% autocite capable of multicite (high-level citation markup)
% only relevant to numeric styles, since using superscript
% I mainly want it to swap between before/after punctuation superscript
\newrobustcmd*{\mautocite}{\mcitelike\autocite}

% hyperref must be at the end of all used packages, and before other settings
\usepackage[colorlinks=true,linkcolor=magenta,citecolor=blue]{hyperref}


%\newlength{\my_width}
\newcommand{\enUnit}{kJ$ \cdot \text{mol}^{-1}$}

% commands for all the anions
\newcommand{\bfl}{ \ensuremath{ \text{BF}_4^- } }
\newcommand{\br}{ \ensuremath{ \text{Br}^- } }
\newcommand{\cl}{ \ensuremath{ \text{Cl}^- } }
\newcommand{\dca}{ \ensuremath{ \text{Dca}^- } }
\newcommand{\mes}{ \ensuremath{ \text{Mes}^- } }
\newcommand{\ntf}{ \ensuremath{ \text{NTf}_{2}^{-} } }
\newcommand{\pf}{ \ensuremath{ \text{PF}_{6}^{-} } }
\newcommand{\tos}{ \ensuremath{ \text{Tos}^- } }

\newcommand{\bflb}{ $ [ \text{BF}_4 ] $ }
\newcommand{\brb}{ $ [ \text{Br} ] $ }
\newcommand{\clb}{ $ [ \text{Cl} ] $ }
\newcommand{\dcab}{ $ [ \text{Dca} ] $ }
\newcommand{\mesb}{ $ [ \text{Mes} ] $ }
\newcommand{\ntfb}{ $ [ \text{NTf}_{2} ] $ }
\newcommand{\pfb}{ $ [ \text{PF}_{6} ] $ }
\newcommand{\tosb}{ $ [ \text{Tos} ] $ }

% for cations
\newcommand{\cat}[2]{ $ \text{C}_{#2}\text{#1}^{+} $ }
\newcommand{\catb}[2]{ $ [ \text{C}_{#2}\text{#1} ] $ }

% for ion pairs
\newcommand{\ipair}[3]{
    \IfEqCase{#3} {
        {bfl} {\ensuremath{[ \text{C}_{\text{#2}}\text{#1}] [ \text{BF}_4 ]}}  
        {br} {\ensuremath{[ \text{C}_{\text{#2}}\text{#1}] [ \text{Br} ]}}
        {cl} {\ensuremath{[ \text{C}_{\text{#2}}\text{#1}] [ \text{Cl} ]}}
        {dca} {\ensuremath{[ \text{C}_{\text{#2}}\text{#1}] [ \text{Dca} ]}}
        {mes} {\ensuremath{[ \text{C}_{\text{#2}}\text{#1}] [ \text{Mes} ]}}
        {ntf} {\ensuremath{[ \text{C}_{\text{#2}}\text{#1}] [ \text{Ntf}_{2} ]}} 
        {pf} {\ensuremath{[ \text{C}_{\text{#2}}\text{#1}] [ \text{PF}_6 ]}}
        {tos} { \ensuremath{[ \text{C}_{\text{#2}}\text{#1}] [ \text{Tos} ]}}   
    } 
    [ \PackageError{ipair}{Undefined option (anion) to ipair: #3}{} ]
}

% for energies
% 2nd arg optional; \energ[Ind]{EFP}
\newcommand{\energ}[2][]{ \ensuremath{ E^{\text{#1}}_{\text{#2}} }}

% plotting macro
\newcommand{\plot}[3]{
\begin{figure}[H]
    \centering
    \centerline{\includegraphics[scale = 0.8]{#2}}
    \caption{#3}
    \label{#1}
\end{figure}
}

% for dealing with of no-break spaces in .bib
%\DeclareUnicodeCharacter{00A0}{ }
% decided to use xelatex instead since hyphens, pi and other characters
% have problems too. xelatex doesn't like declareunicode

\title{Comparison of the Effective Fragment Potential method with Symmetry-Adapted Perturbation Theory in the Calculation of Intermolecular Energies for Ionic Liquids}
\date{}
\author{Samuel Tan\\
        \texttt{samuel.tan@monash.edu} 
        \and
        Ekaterina Izgorodina\\
        \texttt{katya.pas@monash.edu}
}


% from the old_sapt_efp_paper.tex before applying the journal formatting requirements
% begin using git on this version!!
\begin{document}

%\include{report_and_timeline}
\clearpage

\maketitle
%\hline
\begin{multicols}{2}

\section{Introduction}

% Wed 18 Mar 2015
% section

Intermolecular interactions have an important effect on the physical and chemical properties of condensed chemical systems, especially where noncovalent interactions dominate. 
In ionic liquids (ILs), calculating the interacting energy not only requires accounting for the inherent covalent interactions and the ionic character that dictates much of the intermolecular dynamics, but also accurately including interactions such as hydrogen-bonding, $\pi$-$\pi$ stacking, van der Waals forces, etc.
\mautocite{Wendler2012, Bedrov2010, Izgorodina2011}
The noncovalent interactions in ionic liquids are often dominated by electrostatics (Coulomb), dispersion and induction (also known as polarization), as well as exchange-repulsion to a smaller extent. 
The complex interplay of all these interactions means that characterising the intermolecular dynamics of ionic liquids is a challenging task. 
\mautocite{Izgorodina2011}


Symmetry-adapted perturbation theory (SAPT) is the state-of-the-art method for calculating intermolecular interactions, and the separation of its components provides important insight into how the interactions affect the sturcture and properties of the chemical system in consideration.
\mautocite{Stone1996, Turney2012}
However, while accurate, it is very expensive computationally. This theory often partitions the intermolecular interaction energy into electrostatic, exchange, induction and dispersion components. The charge-transfer is considered a part of the induction energy. 


The general effective fragment potential (EFP) method was developed by Gordon et al. 
\mautocite{Jensen1998, Gordon2001, Gordon2009, Mullin2009, Gordon2012} 
as a computationally inexpensive method to model intermolecular interactions. 
This method was originally created to model solvents,
\mautocite{Day1996, Chen1996, Adamovic2006} 
but has then been generalised.
\mautocite{Gordon2007, Ghosh2010}
It belongs to class of fragmentation methods, and is an \emph{ab-initio} based method, without any empirical parameters, with each term developed independently of the rest. Each term in the EFP method represents an individual fundamental component of interaction energy such as electrostatics, exchange, polarization, dispersion and charge transfer. Thus it calculates the interaction energy as a sum of terms directly comparable with SAPT.


This work extends on the work done by Flick et al.,
\mautocite{Flick2012}
who undertook a systematic study on the performance of EFP compared against a raft of semi-empirical and correlated methods.
They used the S22 and S66 test sets of Hobza et al.
\mautocite{Jurecka2006, Rezac2011}    % get these on Mon
However, to date no systematic study has been done on the suitability of the EFP method for charged species like ionic liquids. 
While this method was not originally designed for charged species, its computational efficiency shows promise. 
This study attempts to identify how well EFP performs for ionic liquids in representing the intermolecular interactions.
The test set is a suite of ionic liquids at various configurations, and the EFP data will be compared against the SAPT results.
Three basis sets will be used for EFP to determine the accuracy gained when larger basis sets are used.



\section{Theoretical Background}

% Wed 18 Mar 2015   

% section: theoretical background

\subsection{SAPT}
SAPT was first used by London \emph{et al.}
\cite{Eisenschitz1930a}
to describe the intermolecular interaction operator as a multipole expansion. 
The theory has been further improved and refined and is the current benchmark for calculating the intermolecular interaction energy between two molecules.
Jeziorski, Moszynski and Szalewicz have a comprehensive description of the theory elsewhere. 
\cite{Jeziorski1994a}
In the context of this work, ``dimer'' here refers to an ion pair, whereas ``monomer'' refers to an individual ion.
It has to be pointed out that within the SAPT formulation the intermolecular interaction energy defined as the difference between the total energies of the dimer and constituent monomers is calculated free of basis set superposition error. 

In order to obtain physically sound concepts of intermolecular forces such as the electrostatic, dispersion and induction, a non-symmetric decomposition of the Hamiltonian is used.
This means that electrons are no longer indistinguishable, and the corresponding zeroth-order wavefunction no longer obeys the Pauli exclusion principle. 
As a result, anti-symmetrisation is required making the anti-symmetrised wavefunction no longer an eigenfunction of the unperturbed sum of the Hamiltonian's of constituent monomers, $H_A + H_B$.
To circumvent the issue, the \emph{symmetry-adapted} perturbation procedure is applied to keep the $H_A + H_B$ sum as the unperturbed operator whilst still utilising the anti-symmetrised wavefunction.


The SAPT method has the Hamiltonian partitioned as
\begin{equation}
    H = F_A + F_B + W_A + W_B + V
\end{equation}
where $ F_A, F_B $ are the Fock operators for monomers $A$ and $B$ respectively. 
Similarly, $W_A, W_B$ are the differences between the exact Coulomb operator and the Fock operator for each monomer, whereas $V$ contains all the intermolecular terms.
SAPT perturbs all of $W_A, W_B, V$ through various orders in order to calculate the individual energy terms.
The different energy components are grouped to produce five fundamental forces as follows:

% note that blank lines make align unhappy, no blank lines anywhere!
\begin{flalign}
    %\begin{split}
     E_{\text{electrostatic}} = & E^{(10)}_{\text{Elst,Repl}} +
                \textcolor{blue}{E^{(12)}_{\text{Elst,Repl}}}  +
                \textcolor{red}{E^{(13)}_{\text{Elst,Repl}}} \\ 
    E_{\text{exchange}} = & E^{(10)}_{\text{Exch}} +
                \textcolor{blue}{E^{(11)}_{\text{Exch}}} +
                \textcolor{blue}{E^{(12)}_{\text{Exch}}} \\ 
    E_{\text{induction}}    = & E^{(20)}_{\text{Ind,Repl}} +
                \textcolor{red}{E^{(30)}_{\text{Ind,Repl}}} +
                \textcolor{blue}{E^{(22)}_{\text{Ind}}} +
                E^{(20)}_{\text{Exch-Ind,Repl}} + \\  \nonumber
                    & \textcolor{red}{E^{(30)}_{\text{Exch-Ind,Repl}}} +
                      \textcolor{blue}{E^{(22)}_{\text{Exch-Ind}}} +
                      \textcolor{blue}{\delta E^{(2)}_{\text{HF}}} +
                      \textcolor{red}{\delta E^{(3)}_{\text{HF}}} \\ 
    E_{\text{dispersion}} = & E^{(20)}_{\text{Disp}} +
                                    \textcolor{red}{E^{(30)}_{\text{Disp}}} +
                                    \textcolor{blue}{E^{(21)}_{\text{Disp}}} +
                                    \textcolor{blue}{E^{(22)}_{\text{Disp}}} + 
                                E^{(20)}_{\text{Exch-Disp}} + \\    \nonumber
                                    &  \textcolor{red}{E^{(30)}_{\text{Exch-Disp}}} +
                                    \textcolor{red}{E^{(30)}_{\text{Ind-Disp}}} +
                                    \textcolor{red}{E^{(30)}_{\text{Exch-Ind-Disp}}} \\ 
    E_{\text{charge-transfer}} = & E_{\text{Ind}}(\text{\small dimer basis}) - 
                                    E_{\text{Ind}}(\text{\small monomer basis})
\end{flalign}

The superscripts in parenthesis denote the perturbation order of $V$ and $W = W_A + W_B$ respectively. 
In this work the 2+3 truncation was used in the SAPT expansion
\cite{Turney2012a}.
%SAM WE NEED A REFERENCE FOR THE 2+3 COMBINATION.
In the equations above, blue terms refer to \textcolor{blue}{second order}, whereas red terms represent \textcolor{red}{third order}.
The first order has only electrostatics and exchange terms, while induction and dispersion occur in the second order.
Also present in the second order is quenching via exchange-repulsion in the intramolecular contributions to electrostatics and exchange.
SAPT2+ further includes intramolecular electron correlation terms pertaining to dispersion.
The third order, SAPT2+3, consists of additional terms for dispersion, as well as quenching of induction and dispersion by third order exchange.
In perturbation theory, the induction energy can be separated into two categories: those involving excitations from the occupied orbitals of a molecule to virtual orbitals of the same molecule, and excitations from the occupied orbitals of a molecule to the virtual orbitals of another molecule.
\cite{Stone2009a}
The latter is known as the charge-transfer energy.
To calculate this interaction, the SAPT induction energy from the monomer basis set, where no charge-transfer is permitted, is subtracted from the SAPT induction energy from the dimer basis set.
Furthermore, note that induction and dispersion include exchange components due to the quenching of forces as a result of proximity of the interaction species and non-negligible orbital overlap.
%For more detailed analysis of individual terms see references
\cite{Jeziorski1994a, Hohenstein2010a, Hohenstein2010b, Hohenstein2010c, Hohenstein2011a, Hohenstein2012a}



\subsection{EFP}
The effective fragment potential method (EFP) is an \emph{ab-initio}-based potential method that models the intermolecular interactions of non-covalently bound systems using a cost-effective formulation.
\cite{Gordon2001a, Gordon2007a, Gordon2009a, Mullin2009a, Ghosh2010a}.
In the EFP method, the system is broken into fragments (typically each of the interacting molecules is a fragment). 
In this study, since only ion pairs are considered, individual ions are considered as two fragments. 
Each fragment is treated separately at the Hartree--Fock level of theory in order to generate potentials.
% total energy of system is interaction energy of the effective fragments and the energy of the ab-initio region in the field of the fragments (which we don't consider, so omit discussing)
%These fragments are treated separately at an \emph{ab-initio} level of theory. 
% TODO
%SAM, WHAT DOES THIS MEAN - TREATED AT AN AB INITIO LEVEL OF THEORY? WHICH ONE AND WHY? PLEASE GIVE MORE DETAIL HERE.
Then the cation and anion potentials are allowed to interact and the interaction energy is decomposed into individual components.


The EFP method partitions the interaction energy in the following way:
\begin{equation*}
    E_{\text{total interaction}} = E_{\text{Elst}} + E_{\text{Pol}} + E_{\text{Disp}} + 
                                    E_{\text{Repl}} + E_{\text{CT}}
\end{equation*}
In order of appearance, these are the electrostatic, induction (polarization), dispersion, exchange-repulsion and charge-transfer components.
Coulomb, induction and dispersion are considered long-range interactions.
They decay as $R^{-n}$, with $n = 1$ for Coulomb, $n = 2$ to 4 for induction, and $n = 6$ for dispersion.
The short-range interactions which decay exponentially are exchange-repulsion and charge-transfer.
In the EFP formulation, the Coulomb interaction uses Stone's distributed multipolar analysis 
\cite{Stone1996a} truncated at the octopole term.
Induction, also known as polarisation in the EFP method, is the effect of inducing a dipole moment in a molecule by the electric field of another.
This term is treated with dipole polarisability tensors located at the centroids of localised bond and lone pair orbitals of the molecules.
\cite{Li2006a}


Polarisation in EFP is analogous to induction in SAPT with one exception. 
The SAPT induction term also contains the charge-transfer energy, whereas it is calculated separately in the EFP method and is defined as the interaction between the occupied orbitals on one EFP fragment with the virtual orbitals of another fragment. 
For charge-transfer, the EFP method uses a second order perturbation at the HF level of theory
\cite{Li2006a}.
As these calculations involve the virtual orbitals, it becomes slower with increasing number of basis functions. 
For example, the water molecule has five occupied orbitals and 60 virtual orbitals in the 6-31++G(3df,2p) basis set. 
Thus the charge-transfer term is usually 20-30 times slower than that for the other terms.
\cite{Li2006a}
In order to trim the expense of CT calculations, quasiatomic minimal-basis-set orbitals 
\cite{Lu2004a} 
are used that include the valence virtual orbitals. The latter ensures recovery of the most important CT interactions in the virtual space.


Dispersion is treated by the sum of two terms,
\begin{equation}
    \energ{Disp} = \frac{C_6}{R^6} + \frac{C_8}{R^8}.
\end{equation}
The first term is the induced dipole--induced dipole interaction.
In the EFP method, the coefficients for this $C_6$ term are calculated through the interactions between pairs of localised molecular orbitals of each ion using the time-dependent Hartree-Fock method, with the $C_8$ coefficients being approximated as $1/3$ of the $C_6$ ones.
\cite{Adamovic2005a}.
It has to be noted that this dispersion term was formulated by comparing with SAPT dispersion as the benchmark.


The exchange-repulsion is also calculated using a static localised molecular orbital basis by expanding the intermolecular overlap integral, with truncation at the quadratic term for exchange-repulsion.
\begin{equation}
   \begin{split}
    E^{\text{exch}}_{ij} = & -4 \sqrt{\frac{-2}{\pi} \ln \lvert S_{ij} \rvert } \frac{S^2_{ij}}{R_{ij}} 
                             -2 S_{ij} \left( \sum_{k \in A} F^A_{ik} S_{kj} + \sum_{l \in B} F^B_{jl}S_{il} - 2 T_{ij} \right) \\
                             &  -2 S^2_{ij} \left( \sum_{I \in A} \frac{Z_I}{R_{Ij}}  + 2 \sum_{k \in A} \frac{1}{R_{kj}} + 
                                 \sum_{J \in B} \frac{Z_J}{R_{iJ}} + 2 \sum_{l \in B} \frac{1}{R_{il}} - \frac{1}{R_{ij}} \right)
   \end{split}
\end{equation}
where $A,B$ are the effective fragments, $i, j, k$ and $l$ are the LMOs, and $I, J$ are the nuclei. 
$S$ refers to the intermolecular overlap integral, and $T$ to the kinetic energy integral.
The Fock matrix element is represented by $F$
\cite{Ghosh2010a}.
It is expected that higher order correlation effects are not well accounted for in second order exchange-repulsion.

While the computational costs for each component varies depending on system size and complexity, in general the most expensive interactions to calculate by means of EFP are the exchange-repulsion and charge-transfer interactions.
These two components might be more than five times as computationally demanding than the other three components, which are of roughly the same cost relative to each other.


Originally the exchange-repulsion and charge-transfer were designed with optimisations for neutral molecules and therefore it is suggested that these terms might not perform as well for charged species such as ionic liquids.
These interactions will be stronger due to greater orbital overlap among ions.
While stronger interaction energies in ionic liquid ion pairs might not result in higher relative errors, absolute errors would be expected to be larger.


In comparing EFP and SAPT, the Table \ref{tab:sapt-efp-energy-comp} describes which terms from each method will be compared against each other.

\begin{table}
    \centering
    \caption{Energetic components from SAPT and EFP that are compared with each other, and abbreviation of these components used in the text.}
    \label{tab:sapt-efp-energy-comp}
    \begin{tabular}{c|c|c}
        \hline
        SAPT name               & EFP name      & Abbreviation   \\ \hline
        \energ{electrostatics}  & \energ{Elst}  & \energ{Elst}          \\
        \energ{exchange}        & \energ{Repl}  & \energ{Exch}          \\
        \energ{induction}       & \energ{Pol}   & \energ{Ind}           \\
        \energ{dispersion}      & \energ{Disp}  & \energ{Disp}          \\
        \energ{charge-transfer} & \energ{CT}    & \energ{CT}            \\ \hline
    \end{tabular}
\end{table}




\section{Results and Discussion}


% this file actually contains a rundown of the chemical systems,
% the SAPT and EFP basis sets used
% a comparison of CCSD(T)/CBS with SAPT
% for both total energy and the correlation correction (coz dispersion is different)
% as well as a comparison between the different SAPT basis sets
% a discussion of the SAPT and EFP raw energies???
subsection{Theoretical procedures}

% subsection: CCSD(T) and SAPT
% and aug-cc-pVTZ as well

% but clear up the formulas and definitions first
Differences between each component of the total interaction energy in SAPT and EFP are calculated by subtracting the EFP energy from the SAPT energy, referred to as the absolute difference further in the text.

%\begin{equation}
%    \energ{abs diff} = \energ{SAPT}(\text{INT}) - \energ{EFP}(\text{INT})
%\end{equation}

For comparison, the relative error is calculated as follows: 

\begin{equation}
    100 \cdot \frac{ \energ{SAPT}(\text{INT}) - \energ{EFP}(\text{INT}) } { \energ{SAPT}(\text{INT})} 
\end{equation}
where $E(\text{INT})$ refers to the interaction energy.

Traditional statistics such as the mean absolute error, standard deviation, and absolute maximum were also used in the analysis of the results. 
Mean absolute error (MAE) is defined as the average of the absolute differences.

%\begin{equation}
%    \text{MAE} = \frac{1}{N} \sum \lvert E(\text{INT}) \rvert,
%\end{equation}
%where $N$ is the number of ion pairs.

Standard deviation (SD) refers to sample standard deviation, that is, using $N-1$ in the denominator,
\begin{equation}
    \text{SD} = \sqrt{\frac{\sum (X - \bar{X})^2}{N-1}} ,
\end{equation}
where $\bar{X} $ refers to the sample mean.



\subsection{CCSD(T) and SAPT}
\label{subsec:ccsd}
To ensure the results given by SAPT agree well with other benchmark methods, the SAPT interaction energies were compared with CCSD(T)/CBS energies obtained in previous work 
\cite{Rigby2014a}
on the same series of ionic liquid ion pairs, at the same configurations. The comparison is given in Table \ref{tab:ccsd-sapt-stats}.


\begin{table}[h]
\centering
\footnotesize
\caption{Differences between SAPT and CCSD(T)/CBS in \enUnit.}
\label{tab:ccsd-sapt-stats}
\begin{tabular}{llrrrrr}
\hline
  Halide  & Cation          & MAE   & SD    & Min   & Max \\ \hline
  TILA    & \catb{mim}{n}   & 3.1   & 3.1   & -8.7  & 1.5 \\ 
  Hal     & \catb{mim}{n}   & 3.6   & 2.8   & -8.2  & 0.6 \\ 
  TILA    & \catb{mpyr}{n}  & 2.1   & 2.3   & -4.9  & 2.1 \\ 
  Hal     & \catb{mpyr}{n}  & 0.6   & 0.7   & -1.1  & 1.5 \\ \hline
\end{tabular}
\end{table}

On average the SAPT2+3 method performed within chemical accuracy, with a mean absolute error of 3.6 \enUnit. 
The differences are consistent for halides as well as other typical ionic liquid anions, with a maximum error of -8.7 and -8.2 \enUnit, observed for tosylate and chloride systems, respectively.
Pyrrolidinium based ion pairs usually showed smaller differences when compared to CCSD(T) results.
Overall, due to small deviations between the two methods, SAPT2+3 in combination with aug-cc-pVDZ produces reliable energetics for ILs. 


\subsection{Comparison between methods}
% work on this section--probably want a difference and a correlation part separately
\subsubsection{Differences between SAPT and EFP}
% subsection: differences between SAPT and EFP


%\begin{footnotesize}

%(probably in supplementary info)

\begin{table}[h]
\centering
\scriptsize
\caption{Statistics on differences between SAPT2+3 and EFP by basis set, cation and energy component, classified into halides and TILAs.}
\label{tab:sapt-efp-diff-stats}
\begin{tabular}{lll|rrrr|rrrr}
\hline
 Cation                               & Basis                          & Component & \multicolumn{4}{c}{Halides}                   & \multicolumn{4}{c}{TILAs}                        \\
                                      &                                &           & MAE    & SD    & Max    &                     & MAE     & SD    & Max    &                      \\ \hline
 \multirow{18}{*}{\catb{mim}{n}}      & \multirow{6}{*}{AVDZ}          & Elst      & 10.5   & 9.0   & 23.5   & \ipair{mim}{3}{cl}  & 13.0    & 12.5  & 69.7   & \ipair{mim}{4}{tos}  \\
                                      &                                & Exch      & 6.8    & 7.9   & -18.0  & \ipair{mim}{3}{br}  & 13.9    & 6.2   & 23.8   & \ipair{mim}{2}{mes}  \\
                                      &                                & Ind       & 11.6   & 7.4   & -30.6  & \ipair{mim}{4}{cl}  & 8.4     & 3.8   & -25.6  & \ipair{mim}{1}{ntf}  \\
                                      &                                & Disp      & 8.9    & 1.8   & -13.4  & \ipair{mim}{4}{br}  & 16.6    & 6.2   & 27.9   & \ipair{mim}{3}{ntf}  \\
                                      &                                & CT        & 40.7   & 16.4  & -64.8  & \ipair{mim}{3}{br}  & 5.1     & 1.8   & -9.4   & \ipair{mim}{4}{mes}  \\
                                      &                                & Total     & 18.0   & 20.1  & -41.2  & \ipair{mim}{3}{br}  & 36.8    & 15.1  & 92.6   & \ipair{mim}{4}{tos}  \\ \cline{2-11}
                                      & \multirow{6}{*}{AVTZ}          & Elst      & 9.3    & 11.6  & -37.3  & \ipair{mim}{4}{br}  & 9.2     & 6.9   & 25.5   & \ipair{mim}{2}{tos}  \\
                                      &                                & Exch      & 8.6    & 7.7   & -16.8  & \ipair{mim}{4}{br}  & 15.5    & 11.5  & 28.0   & \ipair{mim}{4}{mes}  \\
                                      &                                & Ind       & 12.6   & 14.0  & 37.7   & \ipair{mim}{3}{br}  & 7.4     & 3.4   & -20.7  & \ipair{mim}{1}{ntf}  \\
                                      &                                & Disp      & 4.5    & 2.5   & -9.8   & \ipair{mim}{4}{br}  & 17.1    & 6.0   & 25.1   & \ipair{mim}{3}{dca}  \\
                                      &                                & CT        & 31.2   & 20.9  & -62.5  & \ipair{mim}{1}{br}  & 3.1     & 2.0   & -6.6   & \ipair{mim}{3}{mes}  \\
                                      &                                & Total     & 22.9   & 26.3  & 50.9   & \ipair{mim}{3}{br}  & 36.3    & 12.5  & 69.7   & \ipair{mim}{2}{tos}  \\ \cline{2-11}
                                      & \multirow{6}{*}{6-311++G(d,p)} & Elst      & 13.5   & 13.1  & -33.9  & \ipair{mim}{4}{br}  & 4.8     & 7.6   & 32.8   & \ipair{mim}{1}{ntf}  \\
                                      &                                & Exch      & 21.7   & 21.8  & 48.8   & \ipair{mim}{2}{br}  & 31.2    & 13.7  & 50.2   & \ipair{mim}{4}{mes}  \\
                                      &                                & Ind       & 32.7   & 7.9   & -47.8  & \ipair{mim}{4}{cl}  & 11.0    & 3.4   & -27.5  & \ipair{mim}{1}{ntf}  \\
                                      &                                & Disp      & 27.2   & 3.9   & -34.1  & \ipair{mim}{4}{br}  & 6.7     & 6.3   & 16.5   & \ipair{mim}{3}{ntf}  \\
                                      &                                & CT        & 37.4   & 16.7  & -62.5  & \ipair{mim}{3}{br}  & 4.1     & 1.5   & -8.1   & \ipair{mim}{4}{mes}  \\
                                      &                                & Total     & 46.2   & 21.3  & -104.3 & \ipair{mim}{4}{cl}  & 29.4    & 15.6  & 72.6   & \ipair{mim}{3}{ntf}  \\ \hline \hline
 \multirow{18}{*}{\catb{mpyr}{n}}     & \multirow{6}{*}{AVDZ}          & Elst      & 18.3   & 7.4   & 29.4   & \ipair{mpyr}{4}{br} & 9.0     & 9.5   & 44.7   & \ipair{mpyr}{4}{tos} \\
                                      &                                & Exch      & 5.0    & 5.3   & 14.3   & \ipair{mpyr}{4}{cl} & 13.7    & 6.6   & 27.8   & \ipair{mpyr}{1}{mes} \\
                                      &                                & Ind       & 5.1    & 3.5   & -12.1  & \ipair{mpyr}{2}{cl} & 7.8     & 1.5   & -12.2  & \ipair{mpyr}{1}{tos} \\
                                      &                                & Disp      & 9.8    & 1.2   & -11.5  & \ipair{mpyr}{3}{br} & 6.2     & 2.7   & 11.9   & \ipair{mpyr}{1}{dca} \\
                                      &                                & CT        & 26.1   & 3.4   & -31.1  & \ipair{mpyr}{4}{br} & 5.4     & 1.9   & -10.6  & \ipair{mpyr}{1}{mes} \\
                                      &                                & Total     & 12.2   & 13.6  & 29.0   & \ipair{mpyr}{4}{cl} & 20.5    & 10.7  & 61.4   & \ipair{mpyr}{4}{tos} \\ \cline{2-11}
                                      & \multirow{6}{*}{AVTZ}          & Elst      & 9.2    & 4.7   & -19.7  & \ipair{mpyr}{3}{br} & 4.4     & 5.1   & 15.1   & \ipair{mpyr}{2}{ntf} \\
                                      &                                & Exch      & 5.0    & 4.9   & -10.2  & \ipair{mpyr}{3}{br} & 16.0    & 7.7   & 25.2   & \ipair{mpyr}{1}{mes} \\
                                      &                                & Ind       & 3.0    & 3.8   & 8.5    & \ipair{mpyr}{2}{br} & 7.6     & 1.5   & -9.9   & \ipair{mpyr}{1}{ntf} \\
                                      &                                & Disp      & 5.8    & 1.0   & -7.9   & \ipair{mpyr}{2}{br} & 6.7     & 2.4   & 12.1   & \ipair{mpyr}{1}{dca} \\
                                      &                                & CT        & 14.4   & 2.9   & -19.8  & \ipair{mpyr}{1}{br} & 2.0     & 1.5   & -5.2   & \ipair{mpyr}{1}{mes} \\
                                      &                                & Total     & 8.6    & 9.1   & -21.8  & \ipair{mpyr}{3}{br} & 22.4    & 8.1   & 45.5   & \ipair{mpyr}{2}{tos} \\ \cline{2-11}
                                      & \multirow{6}{*}{6-311++G(d,p)} & Elst      & 9.3    & 3.7   & -15.8  & \ipair{mpyr}{3}{br} & 6.8     & 9.8   & 43.7   & \ipair{mpyr}{1}{tos} \\
                                      &                                & Exch      & 25.2   & 19.3  & 48.1   & \ipair{mpyr}{1}{br} & 29.7    & 9.4   & 42.4   & \ipair{mpyr}{1}{tos} \\
                                      &                                & Ind       & 21.1   & 4.9   & -27.3  & \ipair{mpyr}{1}{br} & 11.8    & 2.4   & -18.0  & \ipair{mpyr}{1}{tos} \\
                                      &                                & Disp      & 24.5   & 2.6   & -28.7  & \ipair{mpyr}{4}{br} & 2.9     & 3.6   & 6.8    & \ipair{mpyr}{3}{dca} \\
                                      &                                & CT        & 19.8   & 3.3   & -25.0  & \ipair{mpyr}{4}{br} & 3.8     & 1.5   & -6.8   & \ipair{mpyr}{1}{mes} \\ 
                                      &                                & Total     & 23.8   & 13.3  & -45.6  & \ipair{mpyr}{3}{cl} & 26.6    & 16.1  & 77.3   & \ipair{mpyr}{1}{tos} \\ \hline
\end{tabular}
\end{table}


\begin{table}[h]
\centering
\scriptsize
\caption{Percentage of differences between SAPT2+3 and EFP for each energetic component}
\label{tab:sapt-efp-perc-stats}
\begin{tabular}{lll|rrrr|rrrr}
\hline
Cation                            & Basis                           & Component & \multicolumn{4}{c}{Halides}                & \multicolumn{4}{c}{TILAs}                   \\
                                  &                                 &        & MAE   & SD   & Max   &                     & MAE & SD  & Max  &                         \\ \hline       
\multirow{18}{*}{\catb{mim}{n}}   & \multirow{6}{*}{AVDZ}           & Elst   & 2.4   & 1.6  & 5.5   & \ipair{mim}{3}{cl}  & 3.4  & 2.9  & 16.9 & \ipair{mim}{4}{tos}   \\              
                                  &                                 & Exch   & 3.8   & 2.2  & 9.1   & \ipair{mim}{3}{cl}  & 12.2 & 4.6  & 18.5 & \ipair{mim}{2}{bfl}   \\              
                                  &                                 & Ind    & 12.9  & 7.3  & 28.0  & \ipair{mim}{4}{cl}  & 17.3 & 5.4  & 38.2 & \ipair{mim}{1}{ntf}   \\              
                                  &                                 & Disp   & 18.0  & 5.8  & 30.0  & \ipair{mim}{4}{br}  & 26.8 & 7.8  & 42.0 & \ipair{mim}{3}{dca}   \\              
                                  &                                 & CT     & 88.3  & 9.3  & 99.0  & \ipair{mim}{4}{cl}  & 67.8 & 10.2 & 79.9 & \ipair{mim}{2}{dca}   \\              
                                  &                                 & Total  & 4.5   & 2.9  & 10.1  & \ipair{mim}{3}{br}  & 9.7  & 3.4  & 22.4 & \ipair{mim}{4}{tos}   \\ \cline{2-11}
                                  & \multirow{6}{*}{AVTZ}           & Elst   & 2.1   & 2.1  & 8.1   & \ipair{mim}{4}{br}  & 2.4  & 1.5  & 6.1  & \ipair{mim}{2}{tos}   \\              
                                  &                                 & Exch   & 4.7   & 2.7  & 9.4   & \ipair{mim}{4}{br}  & 13.7 & 7.7  & 22.8 & \ipair{mim}{4}{bfl}   \\              
                                  &                                 & Ind    & 13.9  & 7.9  & 40.1  & \ipair{mim}{3}{br}  & 15.7 & 6.0  & 30.9 & \ipair{mim}{1}{ntf}   \\              
                                  &                                 & Disp   & 9.7   & 6.4  & 22.1  & \ipair{mim}{4}{br}  & 27.7 & 7.7  & 41.2 & \ipair{mim}{3}{dca}   \\              
                                  &                                 & CT     & 62.7  & 26.5 & 93.4  & \ipair{mim}{2}{cl}  & 41.6 & 18.8 & 66.4 & \ipair{mim}{3}{pf}    \\              
                                  &                                 & Total  & 5.7   & 3.9  & 13.2  & \ipair{mim}{3}{br}  & 9.6  & 2.8  & 16.8 & \ipair{mim}{2}{tos}   \\ \cline{2-11}
                                  & \multirow{6}{*}{6-311++G(d,p)}  & Elst   & 3.0   & 2.5  & 8.4   & \ipair{mim}{4}{br}  & 1.3  & 1.7  & 9.5  & \ipair{mim}{1}{ntf}   \\              
                                  &                                 & Exch   & 12.2  & 10.1 & 28.9  & \ipair{mim}{3}{br}  & 28.4 & 12.4 & 42.6 & \ipair{mim}{3}{ntf}   \\              
                                  &                                 & Ind    & 36.5  & 7.4  & 50.4  & \ipair{mim}{3}{br}  & 24.0 & 5.1  & 41.1 & \ipair{mim}{1}{ntf}   \\              
                                  &                                 & Disp   & 53.0  & 4.4  & 61.2  & \ipair{mim}{3}{br}  & 10.4 & 6.8  & 23.0 & \ipair{mim}{4}{dca}   \\              
                                  &                                 & CT     & 79.9  & 11.6 & 93.6  & \ipair{mim}{3}{br}  & 57.7 & 17.0 & 85.5 & \ipair{mim}{3}{pf}    \\              
                                  &                                 & Total  & 11.4  & 5.2  & 25.6  & \ipair{mim}{4}{cl}  & 7.9  & 4.1  & 20.2 & \ipair{mim}{3}{ntf}   \\ \hline \hline
\multirow{18}{*}{\catb{mpyr}{n}}  & \multirow{6}{*}{AVDZ}           & Elst   & 4.5   & 1.9  & 7.6   & \ipair{mpyr}{4}{br}  & 2.6  & 2.1  & 12.2 & \ipair{mpyr}{4}{tos}  \\              
                                  &                                 & Exch   & 3.6   & 2.9  & 10.5  & \ipair{mpyr}{4}{cl}  & 15.5 & 5.6  & 23.0 & \ipair{mpyr}{1}{bfl}  \\              
                                  &                                 & Ind    & 6.9   & 4.2  & 16.3  & \ipair{mpyr}{2}{cl}  & 16.5 & 3.7  & 24.9 & \ipair{mpyr}{2}{ntf}  \\              
                                  &                                 & Disp   & 21.6  & 1.8  & 24.9  & \ipair{mpyr}{3}{br}  & 13.6 & 5.2  & 28.0 & \ipair{mpyr}{1}{dca}  \\              
                                  &                                 & CT     & 102.0 & 4.6  & 113.1 & \ipair{mpyr}{4}{br}  & 77.9 & 6.7  & 92.2 & \ipair{mpyr}{3}{dca}  \\              
                                  &                                 & Total  & 3.2   & 2.4  & 7.7   & \ipair{mpyr}{4}{cl}  & 5.7  & 2.7  & 16.2 & \ipair{mpyr}{4}{tos}  \\ \cline{2-11}
                                  & \multirow{6}{*}{AVTZ}           & Elst   & 2.3   & 1.2  & 4.8   & \ipair{mpyr}{3}{br}  & 1.3  & 1.0  & 4.9  & \ipair{mpyr}{2}{ntf}  \\              
                                  &                                 & Exch   & 3.4   & 2.1  & 6.9   & \ipair{mpyr}{3}{br}  & 18.4 & 7.6  & 26.6 & \ipair{mpyr}{1}{bfl}  \\              
                                  &                                 & Ind    & 4.1   & 3.3  & 11.8  & \ipair{mpyr}{2}{br}  & 16.1 & 4.3  & 21.8 & \ipair{mpyr}{2}{ntf}  \\              
                                  &                                 & Disp   & 12.8  & 1.9  & 17.3  & \ipair{mpyr}{2}{br}  & 14.8 & 4.2  & 28.5 & \ipair{mpyr}{1}{dca}  \\              
                                  &                                 & CT     & 55.8  & 4.4  & 64.8  & \ipair{mpyr}{1}{br}  & 27.3 & 13.1 & 44.9 & \ipair{mpyr}{2}{pf}   \\              
                                  &                                 & Total  & 2.3   & 1.8  & 5.7   & \ipair{mpyr}{3}{br}  & 6.3  & 2.0  & 11.4 & \ipair{mpyr}{2}{tos}  \\ \cline{2-11}
                                  & \multirow{6}{*}{6-311++G(d,p)}  & Elst   & 2.3   & 0.9  & 3.9   & \ipair{mpyr}{3}{br}  & 1.9  & 2.5  & 11.7 & \ipair{mpyr}{3}{tos}  \\              
                                  &                                 & Exch   & 17.2  & 12.9 & 32.6  & \ipair{mpyr}{4}{br}  & 35.0 & 10.1 & 45.8 & \ipair{mpyr}{3}{ntf}  \\              
                                  &                                 & Ind    & 28.7  & 7.1  & 38.7  & \ipair{mpyr}{1}{br}  & 24.7 & 4.5  & 30.5 & \ipair{mpyr}{4}{dca}  \\              
                                  &                                 & Disp   & 54.2  & 3.0  & 58.1  & \ipair{mpyr}{2}{br}  & 7.1  & 5.6  & 17.3 & \ipair{mpyr}{2}{pf}   \\              
                                  &                                 & CT     & 76.7  & 3.0  & 81.5  & \ipair{mpyr}{4}{br}  & 54.3 & 11.9 & 72.0 & \ipair{mpyr}{4}{bfl}  \\              
                                  &                                 & Total  & 6.2   & 3.4  & 11.5  & \ipair{mpyr}{3}{cl}  & 7.4  & 4.2  & 20.0 & \ipair{mpyr}{1}{tos}  \\ \hline
\end{tabular}
\end{table}


% theory
%The names for the components differ between the methods. 
%For SAPT, it is electrostatics, exchange-repulsion, induction, and dispersion that make up the total interaction energy. 
%The SAPT charge-transfer energy is calculated in Psi4 as the difference in total induction between the dimer and mononmer basis sets. 
%This is because in SAPT, the charge-transfer energy is included in the total induction energy, i.e.
%
%\begin{equation*}
%    \energ[tot Ind]{SAPT} = \energ[Ind]{SAPT} + \energ[CT]{SAPT}
%\end{equation*}
%
%
%The components that make up the total EFP interaction energy, in order corresponding to their SAPT equivalents, are electrostatics, repulsion, polarizaton, dispersion, and charge-transfer. 
%In the EFP method, the charge-transfer energy is considered separate from polarization as a part of the total energy. 
%
%In comparing the two methods, the full induction energy from SAPT will be compared with the polarization energy from EFP; i.e. not the sum of polarization energy with charge-transfer.

%Therefore, to compare the induction/polarization component between the two methods, the polarization energy will be added to the charge-transfer energy in the EFP method. 
%That is, compare 
%\energ[Ind]{SAPT} with
%$ \energ[Pol]{EFP} + \energ[CT]{EFP} $.

% data analysis
The statistics on the differences between SAPT and EFP energies for each energetic component, basis set and cation type are tabulated in Table \ref{tab:sapt-efp-diff-stats}.
The columns on the right indicate the ion pair with the largest errors.
For the total interaction energy, EFP produced errors that ranged up to 20.5 \enUnit~ for pyrrolidinium based systems and to 36.8 \enUnit~ for imidazolium based systems. 
Both of these maxima were found with the aug-cc-pVDZ basis set. 
All three basis sets generated large standard deviations; for typical TILAs, it ranged from 8.1 to 16.1 \enUnit, whereas for halide systems, the standard deviation was found to be wider, from 9.1 to 26.3 \enUnit.
Out of the three basis sets,  aug-cc-pVTZ produced the smallest errors. 
EFP underperforms SAPT with a maximum error of 92.6 \enUnit~ for \ipair{mim}{4}{tos} among TILA-based systems in the aug-cc-pVDZ basis set, and -104.3 \enUnit~ for \ipair{mim}{4}{cl} among halide-based systems in the 6-311++G(d,p) basis set.
Generally, pyrrolidinium-based ion pairs had smaller errors (largest of 77.3 \enUnit in 6-311++G(d,p)) compared to imidazolium; however, these are still unacceptable.
On the absolute scale, all four components produced large errors for imidazolium ion pairs, whereas for pyrrolidinium the largest errors came from electrostatics and exchange components.


% CT comment bit?


Relative errors calculated using the SAPT energy for each energetic component, are tabulated in Table \ref{tab:sapt-efp-perc-stats}.
On average, the relative errors fall within 20\% for the TILAs. 
Exceptions to this trend include the exchange and induction terms in the 6-311++G(d,p) basis set, and the dispersion term for Dunning's basis sets.
%The analysis of these errors indicates that across all three basis sets studied, the average deviation of EFP from SAPT falls outside the 20\% range for all energetic components, except for electrostatics for TILAs, and exchange-repulsion for halides with Dunning's basis sets.
For systems with halides, it is notable that Dunning's basis sets had smaller errors for exchange-repulsion compared to the TILAs, whereas the opposite trend is observed in the 6-311++G(d,p) basis set.
With Dunning's basis sets, the relative errors are usually smaller for the halides for all components except charge-transfer, compared to the other ionic liquid anions.
Across all basis sets, it is notable that electrostatics produced the least relative errors, below 4.5\% on average.
%However, considering the same subset of EFP/aug-cc-pVTZ calculations for pyrrolidinium-based ion pairs, the halides have a lower MAE compared to the TILAs for all other components; exchange (3.4\% \emph{vs.} 18.4\%), induction (4.1\% \emph{vs.} 16.1\%), dispersion (12.8\% \emph{vs.} 14.8\%) and total interaction energy (2.3\% \emph{vs.} 6.3\%).
The aug-cc-pVTZ basis set had the least errors for the charge-transfer energy across all systems studied.


Surprisingly, the 6-311++G(d,p) basis set gave the largest relative errors of 60\% for dispersion for halides, even though it usually performed better than Dunning's basis sets for the TILAs by reducing the error to about 8\%.
%In the case of induction, the 6-311++G(d,p) basis set also gives slightly higher errors for halide systems. 
%This is rather surprising as the EFP method was designed for use with Pople basis sets.
% need to write about halide calculations (ghost atoms and manual addition)
EFP charge-transfer energies disagree with SAPT numbers by at least twofold regardless of basis set or system. 
The largest differences in charge-transfer are usually observed in systems that have anions with sulfonyl bonds such as mesylates and \ntf or $\pi$-conjugation, such as dicyanamide and tosylate.
Similar relative errors in EFP for electrostatics, exchange, dispersion and induction were observed for other intermolecular complexes from the S22 and S66 databases.
\cite{Flick2012a}
These findings indicate that errors observed in the application of EFP to ionic liquids are not specific to charged intermolecular complexes as previously suggested.
Due to the increased strength of these four components in ionic liquids, absolute errors with the EFP method fall in a much wider range.
Overall, based on these statistics presented, EFP could not be recommended for accurate calculations of interaction energies of ionic liquids where chemical accuracy is required.




%table 1 (in supplementary information) shows the mean and standard deviation of differences between sapt and efp, as well as the system with the largest difference. 
%the rows are grouped by basis set, cation base, and energy component. 
%halides and non-halides are separated side-by-side. 
%
%The averages and standard deviations are all quite large; the agreement between SAPT and EFP 
%
%The energies from each method are also plotted against each other in correlation plots, and a linear regression applied.
%
%
%%The format for the graphs plotting the difference in energy is very similar to the graphs for the energy plots. 
%%Instead of plotting the different configurations, the difference is averaged across the configurations, and the different basis sets used by EFP are compared. 
%%There are six plots, one each for electrostatics, exchange-repulsion, induction (polarization), dispersion, charge-transfer and the total interaction energy.
%%These graphs are meant to illustrate how the energy differences across the different ion pairs and basis sets.
%
%
%%The Boltzmann distribution is used to determine the average energy ($\overline{E}_{\text{comp}}$, the 'comp' refers to a generic component of the interaction energy) of an ion pair system,
%%\begin{equation*}
%%\overline{E}_{\text{comp}} = \frac{\sum_{i=1}^{N} [E_{\text{comp}}^i \times e^{ \frac{- E_{\text{comp}}^i}{RT} }]}{\sum_{i=1}^N e^{\frac{- E_{\text{comp}}^i}{RT}}}
%%\end{equation*}
%%where $N$ is the number of configurations for a particular ion pair, the $E_{\text{comp}}^i$ are the energies for the $i$-th configuration, and $RT$ is room temperature.
%
%% is RT really room temperature???
%
%
%
%
%\paragraph{Electrostatics}
%Looking at the correlation plots for electrostatics, the agreement between the two methods is clear.
%
%Considering the plots of the differences, most of the EFP values fall within 25 \enUnit of the SAPT results; only for some instances of the tosylates does EFP overestimate the energy beyond 25 \enUnit. 
%For the imidazoliums, the 6-311++G(d,p)le basis set tends to underestimate the electrostatic energy, whilst the Dunning basis sets overestimate if we exclude the halides.
%However, for pyrrolidinium systems, in general the triple zeta basis set has the weakest electrostatic interactions (except for \ntf), and is often closer to the SAPT values.
%cor both cationsf often the aug-cc-pVDZ has the largest overestimations; exceptions include the \ntf anion. 
%The EFP results indicate that a basis set of at least up to aug-cc-pVTZ quality is required to treat the tosylate systems well, especially when the system gets larger. 
%
%
%The relative difference in energy across the three basis sets showed the error to be within 5\%, except for the tosylates and some halides.
%In terms of relative error, the electrostatic energy is the best treated out of all the components of the interaction energy.
%It is crucial that electrostatics is treated well, since this is typically the largest component in the interaction energy. 
%
%
%Looking at the correlation scatterplot, the trend linse follow the centre diagonal fairly closely.
%The anions with lower electrostatic energies, i.e. towards the top right, show better agreement; it is the halides and the tosylates that deviate more at the higher energies.
%
%\paragraph{Exchange-Repulsion}
%The correlation plots show clear separation between different anions and cations.
%It is further broken down into halides and non-halides.
%
%This component has the lowest $R^2$ values, aside from charge-transfer from which no meaningful correlation could be observed. 
%This is due to the different trends that the mesylates and tosylates follow. 
%However sub-dividing with this further distinction would lead to too much fragmentation. 
%The important thing to note is that linear correlation between the two methods is still prominent.
%
%Here the separation of the different basis sets is clearly seen from the difference plots.
%As expected, the Dunning basis sets perform much better than the 6-311++G(d,p)le basis set. 
%However, excluding the halides and dicyanamide, surprisingly the triple zeta is worse than the double zeta basis set.
%In general, the exchange-repulsion interaction is underestimated by the EFP method. 
%This can clearly be seen in the correlation scatterplot, figure
%\ref{fig:sapt-efp-corr}.
%The scatterplot also shows the very clear separation between anions.
%Chloride is handled relatively well, considering that both halides have higher exchange energies compared to the rest of the anions; this is likely due to its smaller size.
%EFP tends to slightly overestimate the repulsion for imidazolium chloride systems in the 6-311++G(d,p)le and triple zeta basis sets, and overestimate it in the double zeta basis set.
%In pyrrolidinium chloride systems, EFP slightly underestimates for all three basis sets. 
%Bromide systems were well treated in the aug-cc-pVDZ basis set, but when the 6-311++G(d,p)le basis set is used the error is comparable with that of the other anions, in fact it is the highest amongst the pyrrolidinium systems.
%When using aug-cc-pVTZ basis set though, the repulsion energy for the bromides is overestimated in both imidazolium and pyrrolidinium.
%After chloride, dicyanamide is the anion with the lowest errors.
%Here, the triple zeta basis set performs the best, slightly overestimating the energy for \ipair{mim}{n}{dca} systems, and underestimating it in all other cases.
%For the rest of the anions (\bfl, \mes, \ntf, \pf, and \tos), the repulsion is underestimated. 
%The double zeta basis set gives the closest results, followed by triple zeta and then lastly the 6-311++G(d,p)le basis set.
%Excluding the 6-311++G(d,p)le basis set, most errors were under 25 \enUnit, or around 20\% relative error.
%The 6-311++G(d,p)le basis set gave errors up to nearly 50 \enUnit, for example in the case of the imidazolium mesylates.
%% what kind of systems are the mesylates/tosylates and how are they related to the halides?
%There is a very slight suggestion that the repulsion energy error increases for imidazolium systems as the length of the alkyl chain increases. 
%There is an equally slight but opposite indication for the pyrrolidinium systems.
%Referring back to the SAPT results, the trend for exchange to increase for longer alkyl chains in imidazolium is observed, whilst this decreases in pyrrolidinium. 
%Comparing with the EFP results, this pattern is also seen, but to a lesser degree.
%Hence this is also seen in the difference between the two methods.
%
%%When comparing the absolute and relative differences between methods, the electrostatic energy is in good agreement, and while the EFP method overestimates the repulsion energy when compared to the exchange energy of SAPT, the trends across the test set are largely in agreement. 
%
%
%\paragraph{Induction (Polarization)}
%Note that to compare the SAPT induction and EFP polarization energies, the EFP charge-transfer energy is summed with the EFP polarization energy. 
%This is because the SAPT method calculates the total induction, which includes the charge-transfer energy.
%% should we instead do SAPT_Ind - SAPT_CT - EFP_Pol ???
%Excepting a few cases of the halides (oddly enough, from the triple zeta basis set), the induction energy is consistently underestimated by the EFP method; most of the points in the correlation scatterplot fall above the line $ y= x $.
%The scatterplot once again highlights the separation between anions.
%Due to the variation in the halide values, no correlation was done for them; the one equation refers to the linear correlation for all the non-halides only.
%
%Looking at the difference plots, errors are usually within 10 \enUnit; the worst errors come from the halides in the 6-311++G(d,p)le basis set, which go over 30 \enUnit for the imidazolium bromides.  
%In terms of relative error this translates to within 20\%, excluding results from the 6-311++G(d,p)le basis.
%Here again the 6-311++G(d,p)le basis set has the largest deviations from the SAPT numbers. 
%The Dunning basis sets are comparable for the non-halides, except in tosylates where the triple zeta does better. 
%No data is available for the \ipair{mim}{n}{tos} systems in the 6-311++G** basis set, though it can be surmised from the corresponding pyrrolidinium systems that they would have larger errors than the Dunning basis sets.
%Furthermore none of the \ipair{mim}{n}{ntf} nor \ipair{mpyr}{n}{ntf} results are available since the SAPT results required excessive computational time.
%
%From the SAPT data, the induction energy increases with the length of the alkyl chain on the cation. 
%This trend is reflected in the EFP data as well; for the tetrafluoroborates, the mesylates, hexafluorophosphates and to a lesser extent the dicyanamides and \ntf, though the latter two show less constancy in the energy difference between the two methods.
%This is more clearly seen in the plots of the relative error, with the error being larger for smaller for the bulkier cations.
%
%The scatterplot also shows how the basis sets agree much better for the anions that have lower energies.
%The three basis sets diverge when it comes to the more strongly binding anions such as the halides.
%Surprisingly however, it is the trend line from the 6-311++G(d,p)le basis set that follows the centre diagonal the closest with these more problematic anions; the other two basis sets underestimate considerably.
%Nevertheless, while the mean of the results may agree better, the deviations within are just as large for the 6-311++G(d,p)le basis as for the Dunning basis.
%
%
%\paragraph{Dispersion}
%If the halides are excluded, the dispersion energy is usually overestimated, except for \bfl and \pf when using the 6-311++G(d,p)le basis set. 
%The halides on the other hand, usually have their dispersion underestimated.
%Looking at the correlation scatterplot, once again the clustering of the anions is observed.
%The two trend lines are differentiated once again by halides and non-halides. 
%The 6-311++G(d,p)le basis set in many cases gives results closer to the SAPT values than the other two basis sets.
%This is surprising, as one would expect the Dunning basis sets to have allow a better treatement of dispersion.
%Again excluding the halides, the absolute difference between SAPT and EFP for dispersion is usually within 10 \enUnit for pyrrolidinium, and within 25 \enUnit for imidazolium.
%In terms of relative energy, this means within 20\% for pyrrolidinium and 40\% for imidazolium.
%The halides have much larger errors, due to the 6-311++G(d,p)le basis set. 
%If the 6-311++G(d,p)le basis set is not considered, than the halides have error ranges comparable with the other anions.
%The SAPT results indicate that the dispersion interaction in general increases for bulkier cations, and this trend is well reflected in the EFP results, as there is only slight variation across the different alkyl chain lengths in the errors. 
%
%
%\paragraph{Charge-transfer}
%The absolute error for charge-transfer is small compared to the other interactions (within 10 \enUnit excluding halides), but the relative error is the highest out of all the components, with most of the pyrrolidinium results above 25\% and imidazolium results above 50\%.
%On an absolute scale this difference is not significant, but the relative error makes it obvious.
%Considering the correlation plot, it is clear that this is the component with the worst agreement between the two methods.
%Thus no attempt to linearly correlate the two methods for this component has been made.
%While the halides have much larger absolute errors, they also experience stronger charge-transfer interactions, so their relative errors are comparable with the rest of the anions.
%However, this means the halides dominate the scatterplot; if they are excluded, then the other anions show better agreement in pattern.
%The EFP method generally underestimates charge-transfer when compared with SAPT.
%Charge-transfer was best treated with the aug-cc-pVTZ basis set in every case.
%In fact, with the triple zeta basis set, the error decreases with increasing alkyl chain length, while it increases for the other two basis sets.
%
%
%\paragraph{Total interaction energy}
%Interestingly, the halides have the lowest error in the total interaction energy if the 6-311++G(d,p)le basis set is disregarded.
%This is likely due to the errors from exchange being lower for the halides.
%The electrostatic energy is usually the most dominant interaction, and the exchange-repulsion cancels this energy out.
%For imidazolium halides, the tendency is to underestimate the total interaction energy, especially when using the 6-311++G(d,p)le basis set.
%In the double zeta basis set, the pyrrolidinium bromides have a couple of systems overestimated, while all the pyrrolidinium chloride systems in this basis are overestimated.
%The rest of the pyrrolidinium halides are underestimated, with the triple zeta basis giving the closest results overall.
%
%
%The next two anions with the lowest errors are \dca and \bfl. 
%If the 6-311++G(d,p)le set is excluded, then \pf would belong to this group as well.
%Here, surprisingly, the 6-311++G(d,p)le basis set seems to have the lowest errors across the different basis sets.
%For these three systems, in general the 6-311++G(d,p)le basis set slightly underestimates the electrostatic energy, and underestimates the induction energy.
%However, it treats dispersion better than the Dunning basis sets.
%This seems to indicate that the 6-311++G(d,p)le basis set is sufficient for smaller ion pairs, but do not treat ion pairs with halides, or larger anions like mesylate and \ntf as well.
%
%
%For \ipair{mim}{n}{ntf}, there is very little difference between the two Dunning basis sets, but in \ipair{mpyr}{n}{ntf} the double zeta basis gives better results. 
%
%
%Lastly the tosylates and the mesylates have the highest errors, especially the tosylates. 
%This is probably because of the larger errors from the electrostatic and dispersion components, which were overestimated, coupled with the fact that the repulsion was underestimated. 
%This is more severe for the imidazolium tosylates.
%The mesylates show little difference between basis sets, across cations with different alkyl chain lengths.
%
%
%An interesting trend is seen when looking at the scatterplot---ion pairs with less intermolecular attraction are usually better treated than those with high binding energies.
%While the agreement between SAPT and EFP is not perfect, the EFP energy is usually higher; this consistency is not seen in the other anions.
%The energies tend to be more scattered towards the left, as the interaction energy increases; another observation is that certain anions tend to be underestimated, whilst others tend to be overestimated.
%
%%\end{par}
%
%To give a sense of how the different components contribute to the total interaction energy, figures 
%\ref{fig:corr-all_En} and
%\ref{fig:adiff_en-barplot_all}
%show all the energies on the same plot.
%In the scatterplot (\ref{fig:corr-all_En}), the scales of the different components can be seen.
%In this plot the colours now refer to different basis sets, and the shapes of the points correspond to the different energies.
%The grid of bar plots in figure \ref{fig:adiff_en-barplot_all} is of the absolute errors.
%It is meant to convey the magnitudes of the error from each component, and how they sum to give the final difference.
%The colours correspond to the different energies, and the depth of the colour indicates the basis set.
%Once again, the Boltzmann distribution was used to average the energies across different configurations.
%% make sure the average across chain lengths is done right!!!!!
%
%%\subsection{SAPT results}
%%
%%The raw SAPT energies are presented here to give an indication of how the different energies behave across the systems studied.
%%
%%\paragraph{Electrostatics} 
%%The results from this graph tend to fall into neat bands. 
%%For example, in the imidazolium cation systems, only the first and fourth configurations of bromide                                 and chloride fall below the $-450$ \enUnit ~ mark. 
%%The second and third energetically favourable configurations are between -400 and -450 \enUnit, along with systems that have the mesylates and tosylates as anions. 
%%Next up are the tetrafluoraborates and dicyanamides, followed by \pf and \ntf systems. 
%%The results are less spread out in pyrrolidinium, but the trends are the same. 
%%The mesylate, tosylate and halide systems are again similar in energy, with much less visible separation this time between the different halide configurations.  
%%Next up are \bfl, \dca, \pf and \ntf, again in that order. 
%%The consistent trend across all systems is that the electrostatic interaction weakens as the length of the alkyl chain on the cation increases.
%%This is consistent with our understanding of the chemistry, as the bulkier cations mean a greater inter-ion distance. 
%%Moreover, this matches up very well with previously proposed proton affinity scale of Izgorodina et al.
%%\cite{Izgorodina2007}
%%
%%\paragraph{Exchange}
%%Once again a strong separation into bands is observed in all the results. 
%%For the imidazolium halide systems, where previously the above plane and below plane configurations showed stronger electrostatic interactions, here they exhibit stronger exchange forces. 
%%This is because both electrostatic and exchange are strongly distance dependent. 
%%Just as the closer separations mean the electron-nuclei attraction is stronger, in the same way the electron-electron and nuclear-nuclear repulsions are stronger.
%%The exchange results tend to mirror the electrostatic numbers, but in reverse. 
%%After the halides, the mesylates and tosylates have the strongest exchange interactions, followed by \ntf and \dca. \bfl and \pf6 have the weakest exchanges. 
%%On the other hand, in the pyrrolidinium systems, \pf and \ntf have the weakest interactions, then \dca and \bfl.
%%The next few anions in order of increasing exchange force are \tos, \mes, \cl and \br. 
%%It is even more evident here that the pyrrolidinium results appear uniform than the imidazolium results.
%%This difference between the two cations is observed for all the energy components in SAPT.
%%As the length of the alkyl chain increases, in pyrrolidinium a slight reduction in exchange is noted in chloride, mesylate and tosylate systems; perhaps tetrafluoroborate, hexafluorophosphate and even dicyanamide systems too.
%%For example, in \ipair{mpyr}{n}{tos} (p2), as the alkyl chain goes from methyl to butyl, the exchange decreases as 107.9, 105.2, 104.1 and 103.6 \enUnit.
%%However, the overall trend in the imidazolium systems seems to be increasing exchange as the chain length increases. 
%%This is most clearly seen in \tos, \ntf, the first configurations of \mes and \pf, and \bfl as well as the second and third configurations of \br and \cl. 
%%Using \ipair{mim}{n}{tos} (p1) as an example, the exchange increases as 127.6, 135.7, 140.2 and 143.0 going from dimethyl imidazolium to butyl-methyl-imidazolium.
%%The second configurations of \mes and \pf, the first and fourth configurations of the halides, and all the \dca configurations show little or no variation. 
%%
%%\paragraph{Induction}
%%Separation between different chemical systems is again evident in the plot for the induction component.
%%The halides have the strongest interactions; all of them fall below -70 \enUnit~ in both cations, with no other anions having results lower.
%%The mesylates and tosylates follow the halides, as previously seen in electrostatics and exchange, though this is less obvious in the imidazolium row since the second and third configurations of \ntf fall into the same range of energies, between -70 and -50 \enUnit.
%%For pyrrolidinium, the mesylates and tosylates occupy the band between -70 and -60 \enUnit.
%%Except for in systems with dicyanamide, longer alkyl chains mean stronger induction in imidazolium systems. 
%%For instance, for the first configuration of \ipair{mim}{1}{tos} the induction energy is -59.7 \enUnit~ while for \ipair{mim}{4}{tos} it is -67.9 \enUnit.
%%This is more noticeably manifest in imidazolium, but only weakly observed in pyrrolidinium, e.g. for the mesylates and tosylates there is no visually discernable pattern.
%%
%%\paragraph{Dispersion}
%%In similar fashion as induction, dispersion increases as the chain increases in length.
%%In pyrrolidinium systems, the are roughly two bands, one containing \bfl and \pf2 (~ -40 to -30 \enUnit), and the other containing the rest (-60 to -40 \enUnit).
%%With imidazolium such a distinction is even more blurred, with the \ntf, \mes, \tos and perhaps \dca and the first and fourth configurations of the halides having stronger dispersion than the rest, at energies below -40 \enUnit.
%%This is due to the fact that imidazolium, with a delocalised ring, allows for more dispersion compared to pyrrolidnimium.
%%The electrons an anion has, the greater the dispersion.
%%The anions listed previously have greater electron density compared with the other anions.
%%The upper band is occupied by the second and third configurations of the halides, \bfl, and \pf. 
%%\ipair{1}{mim}{dca} and
%%\ipair{1}{mim}{ntf}
%%lie in the upper band as well. 
%%In all the systems, the above-plane configuration (and below-plane, if it exists) always has a stronger dispersion interaction compared to the in-plane configurations; this is due to direct interaction with the delocalised ring system.
%%
%%\paragraph{Charge-transfer}
%%There is very little variation across varying chain lengths for this energy. 
%%Once again, the halides have the strongest interactions, all of them occupying the band below -20 \enUnit. 
%%The rest of the anions all have absolute energies below 15 \enUnit.
%%There does not appear to be much separation between different configurations, except for the stark instance in the halides, between the in-plane and above/below-plane configurations.
%%The mesylates and tosylates have slightly higher energies in pyrrolidinium; this is less obvious in imidazolium.
%%
%%\paragraph{Total interaction energy}
%%This energy is the sum of all the components discussed previously, except for charge-transfer, which is included in the induction energy.
%%The interesting thing to note here is that the mesylates and tosylates have the highest energies here, followed by the halides, then the tetrafluoroborates, and then the rest.
%%While the halides had higher energies for all the interactions, the large repulsion they possessed meant that they had a weaker interaction overall. 
%%In general, and more evidently for the pyrrolidiniums, the longer the alkyl chain the lesser the energy. 
%%This energy difference may be up to 12.6 \enUnit~ (\ipair{mpyr}{n}{pf}), but is usually below 10 \enUnit.
%%The same is hard to say for the imidazoliums, with some decreasing (first configurations of \mes and \pf), while others increasing (\tos, \ntf and to some extent the first configurations of \mes and \pf).
%%For the other cations, the interaction energy does not much with increasing alkyl chain due to the different interplay between terms for each combination.
%%
%%\subsection{EFP results}
%%
%%By way of comparison with the SAPT results, a similar analysis of the raw EFP energies is given here. 
%%The results from the aug-cc-pVTZ basis set is used as a representative case, since it would be tedious to do the same for all three basis sets.
%%
%%\paragraph{Electrostatic}
%%The most obvious result from a cursory glance at the graph is that the electrostatic interaction weakens as the length of the alkyl chain increases.
%%This is not surprising as the cation gets bulkier it sterically hinders the anion. 
%%
%%The next evident fact is that the second and third (if it exists) configurations are usually lower in energy (less negative) than the first and fourth (if it exists) configurations. 
%%The first and fourth configurations correspond to above and below plane geometries; it seems that the in-plane interactions of the second and third configurations gives a lower interaction energy. 
%%
%%The pyrrolidinium ion pairs tend to be less distributed, and somewhat weaker, whereas the imidazolium systems are more spread out, and overall have stronger interactions. 
%%This is reflected as well in the SAPT results, though the EFP plots are less ordered.
%%This trend appears later on in many of the other components of the interaction energy. 
%%Another trend that is also observed in the other energies is that the halides, followed by the mesylates and tosylates, usually have stronger interactions.
%%This is more easily seen in the electrostatic interaction for the pyrrolidinium row. 
%%Once again, this was seen in the SAPT raw energies, so the EFP method does somewhat capture the differences in chemical systems, albeit in a less precise fashion.
%%
%%
%%\paragraph{Repulsion}
%%For both cations, the separation is quite clear between halides and the rest of the anions.
%%For those with an imidazolium ion, only the halides have a repulsion energy > 140 \enUnit ; those with a pyrrilidinium cation have repulsion energies above 120 \enUnit.
%%There is a possible outlier in 
%%\ipair{mim}{2}{br}.
%%Less obvious is the clustering of the non-halides. 
%%For the imidazolium cation, the lowest cluster is made up of all the ion pairs with \bfl and \pf as anions. 
%%No other anions have repulsion energies below 80 \enUnit .
%%
%%With the pyrrolidinium cation the middle cluster is composed of ion pairs with either mesylate or tosylate as the anion; the minimum repulsion for these two anions is around 81 \enUnit , and the maximum at about 101 \enUnit.
%%No other anions fall within this band. 
%%The anions that form the group with the lowest repulsion energies are hexafluorophosphate, tetrafluoroborate (as before), and \ntf, from around 65 to 53 \enUnit.
%%
%%Trends across increasing alkyl chain length not evident. 
%%In imidazolium systems the longer alkyl chains have slightly less repulsion, probably because the larger molecules are further apart. 
%%However, the pyrrolidinium systems either show no difference or a very slight increase in repulsion as che chain length increases. 
%%
%%\paragraph{Polarization}
%%Three bands can clearly be seen for the pyrrolidinium cation: the lowest from around -68 to -81 \enUnit comprising of the halides, the middle from about -50 down to -64 \enUnit representing only the mesylates and the tosylates, and the rest fall into the highest from -30 to -42 \enUnit .
%%The imidazolium cation exhibits the same behaviour, though the bands are less clear.
%%The halides are from -66 down to -135, the mesylates and tosylates from -50 to -65 \enUnit .
%%Everything else as above -50 \enUnit .
%%In both imidazolium and pyrrolidinium, the tetrafluoroborates and hexafluorophosphates are very similar to dicyanamide and \ntf .
%%Overall, the longer the alkyl chain length results in slightly stronger polarization.
%%
%%\paragraph{Dispersion}
%%For the pyrrolidinium ion two clear groups are observed, with \bfl , \br , \cl , and \pf falling in between -34 and -45 \enUnit , whereas \mes , \dca , and \ntf fall within -49 to -55 \enUnit .
%%The data for the imidazolium cation is more disperse, with \bfl , \pf and the halides having the lowest (less negative) dispersion energies. 
%%However, they do not form their own band because the \cat{mim}{1} of \dca and \ntf have lower dispersion energies too; this is likely due to the shorter alkyl chain. 
%%As the alkyl chain length increases, the dispersion force increases. 
%%This is more obvious in the imidazolium species, but can also be seen for pyrrolidinium. 
%%
%%\paragraph{Charge-transfer}
%%The second and third configurations of the halides immediately stand out for the charge-transfer energy for imidazolium.
%%These configurations have the anion interacting with the ring \textbf{in} the plane. 
%%While these configurations result in lower electrostatic interactions, they have stronger charge-transfer energies. 
%%This does not happen for pyrrolidinium systems, as no discernable difference is seen between configurations.
%%
%%Notice that the second configurations for \dca and \ntf are also lower relative to the other configurations, but not the third configurations.
%%This is because while the second configuration is still interacting side-on with the ring, because of the size of \dca and \ntf the third configuration is positioned differently.
%%In \dca , if the first configuration is thought of as above the plane, than the third configuration is below the plane, hence the similar energies.
%%In \ntf , the first configuration has the anion above the ring, the second has it perpendicular to the ring, to the side. 
%%In both configurations, it is the amide that is interacting with imidazolium. 
%%For the third configuration however, it is the carbonyl groups on either side of the amide that interacts with the ring. 
%%This occurs with the anion obliquely positioned relative to the ring, neither side-on nor fully above the plane.
%%
%%
%%Like the halides, electrostatics and charge-transfer seem to be inversely correlated for ion pairs with \dca , as can be seen from the second configurations of 
%%\ipair{mim}{1}{dca} and 
%%\ipair{mim}{3}{dca}. 
%%However, this is only for the imidazolium cation, and furthermore this is cannot be clearly seen for \ntf .
%%\ipair{mpyr}{1}{dca} and 
%%\ipair{mpyr}{3}{dca}
%%also weaker electrostatic energies, but their charge-transfer energies do not show any deviation from the other configurations.
%%
%%On the other hand, the pyrrolidinium results are recognisably less spread out. 
%%All the halides have charge-transfer energies less than -9 \enUnit , while the mesylates and tosylates occupy the narrow band from -6.6 to -8.5 \enUnit .
%%This is followed closely by the dicyanamides, then \bfl , \ntf and \pf .
%%
%%
%%
%
%
%% followed by
%% subsubsection: correlations between in SAPT and EFP for individual components
%%\end{footnotesize}


\subsubsection{Correlations for each component}
% subsubsection 


For comparison, correlation scatterplots have been graphed by plotting SAPT against EFP.
These scatterplots are used to convey an idea of how closely the EFP and SAPT numbers agree for each component of the interaction energy, and how this is affected by the basis set used and the anion in the system.


These plots have the SAPT energy on the horizontal axis and the EFP energy on the vertical axis in figure 
\ref{fig:sapt-efp-corr}

The points are coloured by the anion in the ion pair, whereas the shape indicates the cation.
Linear regression has been performed on groups of points where a trend is clearly present.
The line $ y = x $ is plotted as well to give an idea of how well the two methods agree.
The closer the points lie to this line, the better the agreement. 

\paragraph{Electrostatics}
The separation between halides and non-halides is clearly seen; halides tend to have their electrostatic interaction underestimated, while non-halides have it overestimated. 
Out of all the different components of the interaction energy, electrostatics, the largest contribution, is treated best. 
Both lines enjoy the highest $R^2$ out of all the components.

\paragraph{Exchange-repulsion}
Again the halide/non-halide and underestimated/overestimated trend is seen. 
Halides are better treated overall, even though they have larger energies; usually systems with smaller energies have correspondingly smaller differences.
This occurs when the relative error stays the same.
Note also that the halides form clusters based on the cation; imidazolium clusters are  more dispersed.
For the non-halides, the dicyanamide anion has the least difference between SAPT and EFP, falling very close to the $y = x$ line.
The other anions appear to follow a different trend, becoming increasingly underestimated as the exchange-repulsion grows.
The disparity between dicyanamide and the other anions accounts for the slightly lower $R^2$.

\paragraph{Induction (polarization)}
Clusterring based on the anion is seen, but with induction the opposite trend is observed---the contribution is underestimated for non-halides.
Furthermore, it is neither over nor underestimated for the halides.
In fact, the line fitted for induction in halides has the slope closest to 1 out of all the lines fitted, indicating very little systematic differences between the methods.
However, whereas the pyrrolidinium systems cluster around the central line, the imidazolium halides are more scattered and result in a lower $R^2$.
For non-halides, the mesylates and tosylates have stronger contributions, hence are plotted between the halides and the other anions. 
The other anions are tightly clustered at the top right of the graph.

\paragraph{Dispersion}
With dispersion, the trend is once again observed: underestimation for halides, and overestimation for non-halides, especially as the energy gets larger.
The larger dispersion contributions occur for imidazolium systems.
Clustering based on anion can also be seen.
Out of all the different trend lines fitted for the various components, the dispersion regression for non-halides has the gradient farthest from 1. 

\paragraph{Charge-transfer}
Due to the disperse nature of the plot, it was decided that linear fitting would not be meaningful or informative.

\paragraph{Total Interaction Energy}
The total interaction energy is similar to electrostatics, since the other contributions are much smaller than electrostatics.
The other components add more variance to the previously tight agreement for the Coulomb interaction.
While the non-halides have a better fit, the halides actually fall closer to the central line, albeit with greater spread.
The non-halides have a better fit instead. 

\end{multicols}
\begin{figure}
    \centering
    \mbox{
    \subfigure[Electrostatics]{\includegraphics[scale=0.5]{\string~/GoogleDrive/SAPT-EFP/images/sapt_efp_corr/iElec.pdf}}
    \subfigure[Exchange-Repulsion]{\includegraphics[scale=0.5]{{\string~/GoogleDrive/SAPT-EFP/images/sapt_efp_corr/iExch.Repl}.pdf}}
    }
    \mbox{
    \subfigure[Induction (polarization)]{\includegraphics[scale=0.5]{{\string~/GoogleDrive/SAPT-EFP/images/sapt_efp_corr/iInd.Pol}.pdf}}
    \subfigure[Dispersion]{\includegraphics[scale=0.5]{\string~/GoogleDrive/SAPT-EFP/images/sapt_efp_corr/iDisp.pdf}}
    }                                 
    \mbox{                            
    \subfigure[Charge-transfer]{\includegraphics[scale=0.5]{\string~/GoogleDrive/SAPT-EFP/images/sapt_efp_corr/iCT.pdf}}
    \subfigure[Total Interaction Energy]{\includegraphics[scale=0.5]{{\string~/GoogleDrive/SAPT-EFP/images/sapt_efp_corr/iTotal.E}.pdf}}
    }
    % need the \protect to make hyperref and the macro happy together
    \caption{Correlation plots of SAPT and EFP   \label{fig:sapt-efp-corr}}
\end{figure}

\begin{table}[ht]
\centering
\scriptsize
\begin{tabular}{lllrrrrrrrr}
  \hline
Basis & Class & Energy & Coef & Std Err & $R^2$ & resid.mean & resid.med & resid.sd & resid.min & resid.max \\ 
  \hline
aug-cc-pVDZ & non-hal & Elec & 0.9737 & 0.0025 & 0.9991 & 7.1818 & -1.6956 & 10.6218 & -27.8494 & 57.0315 \\ 
  aug-cc-pVDZ & non-hal & Exch.Repl & 1.1580 & 0.0067 & 0.9957 & 5.4561 & 1.6252 & 6.5243 & -22.0740 & 12.3055 \\ 
  aug-cc-pVDZ & non-hal & Ind.Pol & 1.1911 & 0.0064 & 0.9963 & 2.0612 & -0.1933 & 3.0010 & -17.6819 & 5.7811 \\ 
  aug-cc-pVDZ & non-hal & Disp & 0.8146 & 0.0055 & 0.9941 & 3.4071 & -1.9731 & 3.9554 & -6.9767 & 9.7747 \\ 
  aug-cc-pVDZ & non-hal & Total.E & 0.9276 & 0.0029 & 0.9988 & 9.6032 & -2.7835 & 12.7868 & -30.9796 & 55.9876 \\ 
  aug-cc-pVTZ & non-hal & Elec & 0.9863 & 0.0016 & 0.9997 & 5.1377 & -0.2945 & 6.5233 & -15.5519 & 19.4388 \\ 
  aug-cc-pVTZ & non-hal & Exch.Repl & 1.1623 & 0.0112 & 0.9882 & 9.4745 & 6.3946 & 10.8537 & -32.2931 & 12.6226 \\ 
  aug-cc-pVTZ & non-hal & Ind.Pol & 1.1668 & 0.0070 & 0.9954 & 2.7207 & -1.3758 & 3.3539 & -13.0093 & 12.1924 \\ 
  aug-cc-pVTZ & non-hal & Disp & 0.8086 & 0.0051 & 0.9949 & 3.3648 & -2.3141 & 3.7243 & -7.7728 & 8.6443 \\ 
  aug-cc-pVTZ & non-hal & Total.E & 0.9254 & 0.0023 & 0.9992 & 7.7169 & -1.6065 & 10.0614 & -19.6615 & 33.5950 \\ 
  6-311++G** & non-hal & Elec & 0.9892 & 0.0023 & 0.9993 & 6.0710 & -1.9473 & 9.2365 & -16.0404 & 39.0768 \\ 
  6-311++G** & non-hal & Exch.Repl & 1.4152 & 0.0209 & 0.9745 & 14.4691 & 11.6305 & 15.3461 & -37.5617 & 21.3902 \\ 
  6-311++G** & non-hal & Ind.Pol & 1.3052 & 0.0081 & 0.9954 & 2.5407 & -0.8368 & 3.2865 & -15.4649 & 9.3319 \\ 
  6-311++G** & non-hal & Disp & 0.9370 & 0.0080 & 0.9913 & 4.0795 & -2.0088 & 4.6669 & -7.9048 & 10.6908 \\ 
  6-311++G** & non-hal & Total.E & 0.9261 & 0.0033 & 0.9985 & 11.5133 & -4.8780 & 14.0558 & -21.3712 & 43.0391 \\ 
  aug-cc-pVDZ & hal & Elec & 0.9708 & 0.0031 & 0.9995 & 8.0129 & 0.8424 & 9.7458 & -19.1339 & 17.3136 \\ 
  aug-cc-pVDZ & hal & Exch.Repl & 1.0028 & 0.0062 & 0.9981 & 5.9842 & 1.8605 & 7.2120 & -18.5766 & 15.2414 \\ 
  aug-cc-pVDZ & hal & Ind.Pol & 1.1120 & 0.0134 & 0.9929 & 5.3400 & 0.9077 & 7.0690 & -21.8243 & 13.1596 \\ 
  aug-cc-pVDZ & hal & Disp & 1.2209 & 0.0107 & 0.9962 & 2.6489 & -1.3324 & 3.0115 & -6.4782 & 5.5002 \\ 
  aug-cc-pVDZ & hal & Total.E & 1.0016 & 0.0068 & 0.9977 & 15.4243 & -0.3512 & 18.9186 & -40.5806 & 38.7764 \\ 
  aug-cc-pVTZ & hal & Elec & 1.0178 & 0.0032 & 0.9995 & 7.3612 & 1.7070 & 9.4623 & -29.6939 & 17.9819 \\ 
  aug-cc-pVTZ & hal & Exch.Repl & 0.9674 & 0.0052 & 0.9986 & 5.0906 & -1.2181 & 6.1792 & -9.8547 & 16.4868 \\ 
  aug-cc-pVTZ & hal & Ind.Pol & 1.0136 & 0.0198 & 0.9821 & 8.1301 & -1.4680 & 11.1708 & -15.9871 & 39.5327 \\ 
  aug-cc-pVTZ & hal & Disp & 1.1057 & 0.0093 & 0.9966 & 2.4567 & -1.1375 & 2.8500 & -6.1600 & 5.1061 \\ 
  aug-cc-pVTZ & hal & Total.E & 1.0185 & 0.0076 & 0.9973 & 15.1033 & -1.8146 & 20.6345 & -35.2910 & 58.9812 \\ 
  6-311++G** & hal & Elec & 1.0252 & 0.0035 & 0.9994 & 7.3637 & 0.9038 & 10.2525 & -24.5281 & 30.8930 \\ 
  6-311++G** & hal & Exch.Repl & 1.1181 & 0.0233 & 0.9792 & 21.8097 & 1.4749 & 23.7515 & -38.1388 & 36.0272 \\ 
  6-311++G** & hal & Ind.Pol & 1.4874 & 0.0267 & 0.9845 & 8.1916 & -1.7509 & 10.4225 & -24.6313 & 25.4874 \\ 
  6-311++G** & hal & Disp & 2.1177 & 0.0248 & 0.9933 & 3.3020 & -0.7121 & 4.0348 & -7.7202 & 7.5233 \\ 
  6-311++G** & hal & Total.E & 1.0999 & 0.0088 & 0.9969 & 17.7504 & -0.5401 & 22.2757 & -73.9961 & 34.9213 \\ 
   \hline
\end{tabular}
    \caption{Coefficients and associated fitting statistics}
    \label{tab:coef_indiv}
\end{table}


\begin{table}[ht]
\centering
\footnotesize
\begin{tabular}{lllrrrrr}
  \hline
Basis & Class & Total.E & mean & med & sd & min & max \\ 
  \hline
adz & hal & rawEFP & 15.44 & -0.99 & 18.90 & -41.17 & 38.06 \\ 
  adz & hal & indivEFP & 15.42 & -0.35 & 18.92 & -40.58 & 38.78 \\ 
  adz & hal & summedEFP.sapt & 15.31 & -0.78 & 18.83 & -39.33 & 36.20 \\ 
  adz & hal & summedEFP.ccsd & 13.60 & 0.06 & 17.36 & -35.16 & 37.99 \\ 
  adz & non-hal & rawEFP & 27.42 & 26.18 & 15.04 & -6.56 & 92.61 \\ 
  adz & non-hal & indivEFP & 9.60 & -2.78 & 12.79 & -30.98 & 55.99 \\ 
  adz & non-hal & summedEFP.sapt & 7.73 & -2.79 & 11.72 & -40.60 & 54.17 \\ 
  adz & non-hal & summedEFP.ccsd & 8.32 & -1.82 & 12.58 & -35.22 & 62.41 \\ 
  atz & hal & rawEFP & 16.45 & -8.35 & 20.32 & -42.00 & 50.91 \\ 
  atz & hal & indivEFP & 15.10 & -1.81 & 20.63 & -35.29 & 58.98 \\ 
  atz & hal & summedEFP.sapt & 13.67 & -0.53 & 18.21 & -43.03 & 46.65 \\ 
  atz & hal & summedEFP.ccsd & 13.28 & 0.29 & 17.71 & -42.03 & 47.36 \\ 
  atz & non-hal & rawEFP & 28.45 & 27.28 & 12.35 & 5.68 & 69.66 \\ 
  atz & non-hal & indivEFP & 7.72 & -1.61 & 10.06 & -19.66 & 33.59 \\ 
  atz & non-hal & summedEFP.sapt & 8.21 & -0.46 & 10.77 & -23.83 & 33.45 \\ 
  atz & non-hal & summedEFP.ccsd & 9.65 & 1.54 & 12.77 & -24.06 & 41.08 \\ 
  pop & hal & rawEFP & 36.33 & -36.71 & 21.27 & -104.35 & -1.14 \\ 
  pop & hal & indivEFP & 17.75 & -0.54 & 22.28 & -74.00 & 34.92 \\ 
  pop & hal & summedEFP.sapt & 17.12 & 5.94 & 21.48 & -71.76 & 36.14 \\ 
  pop & hal & summedEFP.ccsd & 16.81 & 6.33 & 20.81 & -69.74 & 35.42 \\ 
  pop & non-hal & rawEFP & 27.75 & 23.49 & 15.90 & 4.48 & 77.32 \\ 
  pop & non-hal & indivEFP & 11.51 & -4.88 & 14.06 & -21.37 & 43.04 \\ 
  pop & non-hal & summedEFP.sapt & 14.69 & 2.68 & 18.50 & -33.63 & 52.28 \\ 
  pop & non-hal & summedEFP.ccsd & 15.95 & 2.52 & 19.92 & -35.08 & 55.70 \\ 
   \hline
\end{tabular}
    \caption{Statistics for the differences from various methods of calculating the Total Energy}
    \label{tab:si.stats.recast}
\end{table}

\begin{multicols}{2}

The coefficients along with their errors and other statistics from the fitting are in table 
\ref{tab:coef_indiv}.
The coefficients give a clear indication of how EFP performs for each component.
A slope greater than 1 means a tendency to underestimate the SAPT total energy, and vice versa for a slope less than 1. 
Note that for the non-halides, exchange-repulsion is the component with the largest standard errors, for all three basis sets.
This is followed by induction and then by dispersion.
For the halides, however, exchange-repulsion is better treated than dispersion, which in turn performs better than induction.
For both halides and non-halides, the component with the lowest standard error is electrostatics. 
Electrostatics dominates the interaction energy, so the total energy is always the second best in terms of standard errors, due to contributions from the other components.

In table 
\ref{tab:si.stats.recast}
the statistics of the differences between the SAPT total energies and the predicted energies are tabulated.
The predicted energies are obtained through different methods. 
The first is from an EFP calculation.
The second is from regressing the EFP total energy against the SAPT total energy; the predicted energies are obtained by multiplying the EFP energies with the corresponding coefficient.
The third method is similar to the second, except it derives the total energy as a sum of the components (electrostatics, exchange-repulsion, induction-polarisation and dispersion).
From regressing the EFP components against the corresponding SAPT components, the total energy is the sum of the predicted energies of the components.
As can be seen from the table, this third method of summing the scaled components, has the best performance overall.
The only cases where the mean difference does not improve from the first to the third method are the non-halides for the triple zeta and the Pople basis sets.
In both cases it is still a significant improvement over the raw EFP total energy.


%%%%%%%%%%%%%%%%%%%%%%%%%%%%%%%%%%%%%%%%%%%%%%%%%%%%%%%%%%%%%%%%%%%%%
%%%% not sure if we should still talk about multilinear regression

Multilinear regression was performed by treating the EFP components as predictors and regressing against the total energy from both SAPT2+3 and CCSD(T)/CBS.
Furthermore, halides and non-halides were treated separately.
Since there are three basis sets for EFP, each component had twelve had twelve coefficients, for each basis set, for halide and non-haldie systems, and for different benchmarks.


\begin{equation}
    E^{\text{SAPT}}_{\text{Total Energy}} = \alpha E_{\text{Elec}}^{\text{EFP}} +
                                            \beta E_{\text{Exch-Repl}}^{\text{EFP}} +
                                            \gamma E_{\text{Ind-Pol}}^{\text{EFP}} +
                                            \delta E_{\text{Disp}}^{\text{EFP}}
\end{equation}

From the linear correlations discussed above, predicted energies for each of the components were obtained and compared with the SAPT values. 
In table 
\ref{tab:indi_scaled}
the statistics for the differences of these predicted energies and the SAPT results are tabulated. 
\end{multicols}

\begin{table}[ht]
\centering
\scriptsize
\begin{tabular}{lllrrrrrrr}
  \hline
Basis & Halide & Stat & s.sapt.diff\_Elec & s.sapt.diff\_Exch.Repl & s.sapt.diff\_Ind.Pol & s.sapt.diff\_Disp & s.sapt.diff\_Total.E & ss.sapt.diff & ss.ccsd.diff \\ 
  \hline
adz & non-hal & mean & -0.00 & -0.00 & -0.00 & -0.00 & -0.00 & 0.00 & 65.14 \\ 
  adz & non-hal & median & -0.57 & 0.57 & 0.61 & 0.23 & 1.18 & -1.27 & 0.93 \\ 
  adz & non-hal & sd & 14.43 & 6.29 & 2.67 & 2.44 & 11.74 & 12.87 & 206.81 \\ 
  adz & non-hal & min & -30.84 & -20.05 & -17.46 & -6.29 & -29.87 & -29.03 & -23.66 \\ 
  adz & non-hal & max & 109.67 & 13.08 & 3.91 & 6.11 & 83.65 & 104.34 & 805.41 \\ 
  adz & hal & mean & -18.63 & -18.02 & 1.22 & -11.70 & -30.34 & -47.13 & -45.23 \\ 
  adz & hal & median & -16.31 & -16.23 & 1.76 & -11.61 & -28.70 & -42.47 & -41.51 \\ 
  adz & hal & sd & 13.25 & 8.73 & 6.92 & 2.45 & 14.54 & 26.64 & 24.33 \\ 
  adz & hal & min & -41.97 & -41.49 & -20.64 & -17.18 & -58.34 & -105.86 & -98.51 \\ 
  adz & hal & max & 5.35 & -2.14 & 12.47 & -7.51 & -1.57 & -3.00 & -3.17 \\ 
  atz & non-hal & mean & -0.00 & -0.00 & -0.00 & 0.00 & 0.00 & 0.00 & 65.23 \\ 
  atz & non-hal & median & 0.25 & 1.40 & -0.02 & 0.20 & 0.00 & -0.39 & 1.94 \\ 
  atz & non-hal & sd & 6.36 & 9.31 & 2.43 & 2.34 & 6.33 & 8.74 & 205.38 \\ 
  atz & non-hal & min & -15.55 & -23.97 & -13.51 & -7.30 & -11.74 & -18.67 & -18.89 \\ 
  atz & non-hal & max & 16.30 & 16.35 & 7.59 & 5.69 & 15.03 & 29.56 & 786.06 \\ 
  atz & hal & mean & -15.74 & -13.27 & 3.20 & -9.08 & -35.91 & -34.89 & -32.99 \\ 
  atz & hal & median & -14.15 & -12.94 & 3.86 & -9.34 & -35.03 & -32.34 & -30.90 \\ 
  atz & hal & sd & 8.61 & 12.85 & 10.48 & 2.94 & 17.92 & 20.72 & 19.50 \\ 
  atz & hal & min & -45.42 & -94.16 & -11.78 & -13.20 & -92.73 & -103.86 & -98.01 \\ 
  atz & hal & max & -1.88 & 3.92 & 40.73 & 6.53 & 9.48 & 15.28 & 15.99 \\ 
  pop & non-hal & mean & -0.00 & -0.00 & 0.00 & 0.00 & 0.00 & 0.00 & 57.58 \\ 
  pop & non-hal & median & -1.72 & -0.05 & 0.04 & 0.03 & -4.49 & -1.83 & -0.90 \\ 
  pop & non-hal & sd & 9.11 & 11.02 & 2.73 & 1.99 & 10.46 & 13.51 & 195.81 \\ 
  pop & non-hal & min & -16.25 & -23.55 & -15.77 & -4.81 & -12.77 & -19.75 & -21.20 \\ 
  pop & non-hal & max & 36.18 & 23.83 & 6.13 & 6.13 & 29.76 & 47.48 & 775.52 \\ 
  pop & hal & mean & -17.68 & 1.05 & -14.39 & -18.36 & -53.82 & -49.39 & -47.49 \\ 
  pop & hal & median & -15.34 & -3.27 & -12.91 & -17.76 & -54.29 & -48.70 & -46.47 \\ 
  pop & hal & sd & 9.84 & 18.00 & 9.05 & 4.67 & 18.28 & 18.80 & 17.70 \\ 
  pop & hal & min & -41.20 & -24.35 & -34.99 & -28.95 & -103.52 & -110.68 & -108.66 \\ 
  pop & hal & max & 7.36 & 29.11 & 6.45 & -11.64 & -19.88 & -17.12 & -17.84 \\ 
   \hline
\end{tabular}
\caption{Statistics for individually scaled regression \label{tab:indi_scaled} }
\end{table}

\begin{table}[ht]
    \label{tab:multiLinEFP}
\centering
\begin{tabular}{lllrrrrrr}
  \hline
Basis & Halide & Stat & diff\_Elec & diff\_Exch.Repl & diff\_Ind.Pol & diff\_Disp & diff\_Total.E & diff\_CCSD \\ 
  \hline
adz & non-hal & mean & 33.75 & -19.45 & -15.50 & -1.23 & -0.12 & -0.72 \\ 
  adz & non-hal & median & 33.42 & -16.77 & -14.63 & -2.25 & 0.21 & -1.76 \\ 
  adz & non-hal & sd & 12.46 & 8.83 & 3.37 & 3.79 & 12.46 & 11.70 \\ 
  adz & non-hal & min & 6.80 & -48.95 & -33.37 & -7.23 & -52.41 & -44.33 \\ 
  adz & non-hal & max & 102.00 & -6.44 & -10.66 & 9.36 & 53.47 & 59.76 \\ 
  atz & non-hal & mean & 31.29 & 47.08 & -35.05 & -45.64 & -0.06 & -0.52 \\ 
  atz & non-hal & median & 30.00 & 44.94 & -32.22 & -43.75 & 0.08 & -1.74 \\ 
  atz & non-hal & sd & 7.76 & 11.31 & 7.06 & 9.63 & 8.98 & 9.09 \\ 
  atz & non-hal & min & 11.94 & 24.04 & -51.90 & -74.86 & -22.65 & -20.11 \\ 
  atz & non-hal & max & 57.35 & 73.80 & -25.74 & -30.40 & 19.55 & 27.39 \\ 
  pop & non-hal & mean & 35.48 & 66.11 & -60.65 & -42.72 & 0.68 & -0.41 \\ 
  pop & non-hal & median & 32.33 & 64.16 & -55.28 & -41.42 & -0.91 & -1.13 \\ 
  pop & non-hal & sd & 10.19 & 13.55 & 12.94 & 8.29 & 11.00 & 9.14 \\ 
  pop & non-hal & min & 17.23 & 42.14 & -90.56 & -70.74 & -17.05 & -13.47 \\ 
  pop & non-hal & max & 81.35 & 103.20 & -44.96 & -29.74 & 39.71 & 38.13 \\ 
  adz & hal & mean & 4.16 & 168.01 & -82.73 & -91.51 & -34.93 & -0.18 \\ 
  adz & hal & median & 4.75 & 161.59 & -77.81 & -85.26 & -29.87 & 0.50 \\ 
  adz & hal & sd & 9.67 & 23.35 & 10.67 & 16.11 & 16.50 & 7.16 \\ 
  adz & hal & min & -15.06 & 139.32 & -109.56 & -124.93 & -67.97 & -14.07 \\ 
  adz & hal & max & 21.08 & 207.85 & -71.00 & -68.87 & -14.78 & 18.79 \\ 
  atz & hal & mean & -10.80 & 120.94 & -46.33 & -66.22 & -34.75 & -0.59 \\ 
  atz & hal & median & -8.95 & 118.08 & -45.10 & -62.07 & -33.09 & -2.04 \\ 
  atz & hal & sd & 9.13 & 16.12 & 8.19 & 10.79 & 16.18 & 13.00 \\ 
  atz & hal & min & -40.34 & 101.90 & -66.24 & -89.97 & -82.56 & -28.49 \\ 
  atz & hal & max & 5.94 & 149.00 & -34.54 & -50.29 & -3.18 & 24.58 \\ 
  pop & hal & mean & -15.70 & 129.52 & -31.25 & -84.81 & -35.09 & -0.34 \\ 
  pop & hal & median & -14.24 & 127.43 & -29.26 & -79.15 & -28.35 & 0.75 \\ 
  pop & hal & sd & 10.01 & 17.38 & 8.76 & 14.32 & 17.96 & 10.76 \\ 
  pop & hal & min & -38.95 & 105.04 & -51.90 & -112.68 & -75.69 & -32.96 \\ 
  pop & hal & max & 12.70 & 164.21 & -12.62 & -62.54 & -7.85 & 30.01 \\ 
   \hline
\end{tabular}
\caption{Statistics for multilinear regression}
\end{table}

\begin{multicols}{2}


%% not sure if the geodesic section is still relevant???
%\subsection{Charge-transfer and the geodesic scheme}
%
%The charge-transfer energies from SAPT and EFP differ by a significant amount. 
%It is the energy with the highest relative error; however, the contribution of the charge-transfer energy to the total interaction energy is relatively small.
%To further investigate the extent of the charge-transfer interaction, the geodesic charge allocation scheme from the GAMESS package was utilised.
%This is a computationally cheap method to fit charges to a molecular system, and is routinely used in molecular dynamics simulations.
%Using this method generates gives each atom on both the cation and anion a charge. 
%Since the ion pair system is neutral, the sum of the charges on the cation should be equal in magnitude and opposite in sign to the sum of the charges on the anion.
%If there is no charge-transfer, than the total charge on each ion should be unity. 
%However, this is not observed. 
%Instead, the sum of the charges on an ion was always less than one in magnitude. 
%Subtracting the charge on an ion from unity then gives the amount of charge-transfer that occurred. 
%The same two Dunning basis sets were used in the geodesic scheme, and both basis sets showed very close agreement, with the largest differences being less than 0.01$e$, where $e$ is the elementary charge. 
%The results from the two basis sets are plotted in 
%figures \ref{fig:geodCT-adz} and \ref{fig:geodCT-atz}.
%
%
%There is however very little correlation (in fact, negative!) between the charge-transfer calculated using SAPT and the geodesic scheme. 
%The correlation coefficient for the aug-cc-pVDZ basis set is 0.51, while the aug-cc-pVTZ basis set has a coefficient of 0.53. %(Pearson correlation)
%This means that the geodesic scheme and the SAPT charge-transfer are somewhat correlated. 
%Calculating the correlation coefficient for the halides alone gives 0.68, which is an even better correlation.
%On the other hand, the correlation coefficient of the non-halides is -0.05, very slight negative correlation. 
%This indicates that there is no relationship between the SAPT and geodesic scheme numbers for the non-halides.
%
%
%%For Kendall correlation, the coefficents are both 0.30; using Spearman's correlation, both 0.41.
%%discuss the geodesic results ... correlate with electrostatics = 0.79 (pearson), 0.69 (kendall), 0.87 (spearman)
%% halides only: kendall = 0.655, spearman = 0.837
%% non-halides: kendall = -0.064, spearman = -0.097
%
%A scatterplot summarising the data is in figure \ref{fig:geodCT-corr}. 
%As can be seen, there is an inverse relationship for the non-halides.
%While the halides do show some relationship, different systems tend to form their own clusters, indicating there are chemical differences at play here.
%
%
%% change this plot---I don't think that the y = x line means much here, we are not comparing things with equal units even



\section{Conclusion}
% section

The interaction energies and their individual fundamental components calculated by means of the EFP method were compared with those from SAPT2+3 for an extensive series of single ion pairs of ionic liquids.
Overall, the deviations between the two methods for total interaction energy and the four fundamental components such as electrostatic, exchange-repulsion, induction and dispersion were much larger on the absolute scale than expected, falling in the range of -63 to 70 \enUnit. 
Out of all three basis sets studied for EFP, aug-cc-pVTZ gave the lowest errors.
The largest absolute errors came from the charge-transfer energy for the halide-based ion pairs and the exchange-repulsion contribution for the ILs combined with other routinely used anions. 
On the relative scale, the EFP method did not deviate from SAPT by more than 20\% \emph{on average} for all systems and energetic components (with the exception of charge transfer energy for halides). 
Electrostatics in particular showed very small relative errors, below 3\% for all basis sets on average.
As expected, out of the five fundamental components, charge-transfer from EFP produced the largest relative errors, with EFP overestimating SAPT2+3 by 40\% on average for aug-cc-pVTZ.
Apart from charge-transfer, EFP relative errors for ILs are comparable to those reported for the S22 and S66 databases.
\cite{Flick2012a}
This finding indicates that the large EFP errors are not specific to charged intermolecular complexes, as previously suggested.
Due to the sheer magnitude of each individual interaction in ionic liquids reaching up to -479 \enUnit~for electrostatics (-378 \enUnit~on average) it is not surprising that on the absolute scale the average errors were well beyond chemical accuracy, thus rendering EFP inapplicable for ILs at present.


The importance of higher-order terms in the EFP formulation has been established for the exchange-repulsion contribution in the case of typical ionic liquid anions, whereas EFP exchange-repulsion was treated rather well for halide systems.  
It has to be noted that similar relative errors from higher-order terms for electrostatics and exchange-repulsion were observed in neutral intermolecular complexes from the S22 and S66 databases. 
Electrostatics did not show appreciable contribution from higher-order terms for all ILs studied.


Although linear regression analysis of EFP against SAPT2+3 per each energetic component had very high values for the coefficient of determination, $R^2$, significant reduction in error was achieved only for the induction and dispersion terms. 
Moreover, better statistics for linear regression were found when the ion pairs were separated into cation-specific groups. 
Small errors for scaled induction and dispersion in the range of 1.2--3.2 and 0.7--11.4~\enUnit~on average were observed for both for TILAs and halides, respectively. 
This is particularly important highlighting the applicability of the EFP formulation for these two terms for ionic liquid systems. 
This finding will assist in the future development of intermolecular potentials for these complex systems.  



\section*{Acknowledgements}
We gratefully acknowledge computer grants from the Monash eResearch
Centre, MASSIVE and the National Computational Infrastructure. This
work is supported by the ARC - DP Grant and a Future Fellowship for EII.


\renewcommand{\bibfont}{\footnotesize}
\printbibliography

\end{multicols}

\begin{appendices}


%\documentclass{article}
%
%\begin{document}


\section{Figures}
Due to the large number of figures, they have been included here as supplementary information. 
The first section consists of plots of all the raw energies. 
Figures 
\ref{fig:pure_en-sapt_Elec},
\ref{fig:pure_en-sapt_Exch},
\ref{fig:pure_en-sapt_Ind},
\ref{fig:pure_en-sapt_Disp},
\ref{fig:pure_en-sapt_CT} and
\ref{fig:pure_en-sapt_TotalE}
are from the SAPT method, using the aug-cc-pVDZ basis set. 
What follows are the EFP results. 
Figures
\ref{fig:pure_en-adz_efp_Elec},
\ref{fig:pure_en-adz_efp_Repl},
\ref{fig:pure_en-adz_efp_Pol},
\ref{fig:pure_en-adz_efp_Disp},
\ref{fig:pure_en-adz_efp_CT} and
\ref{fig:pure_en-adz_efp_TotalE}
were obtained from the aug-cc-pVDZ basis set as well.
Figures
\ref{fig:pure_en-atz_efp_Elec},
\ref{fig:pure_en-atz_efp_Repl},
\ref{fig:pure_en-atz_efp_Pol},
\ref{fig:pure_en-atz_efp_Disp},
\ref{fig:pure_en-atz_efp_CT} and
\ref{fig:pure_en-atz_efp_TotalE}
are from the aug-cc-pVTZ basis set.

Lastly, figures
\ref{fig:pure_en-pop_efp_Elec},
\ref{fig:pure_en-pop_efp_Repl},
\ref{fig:pure_en-pop_efp_Pol},
\ref{fig:pure_en-pop_efp_Disp},
\ref{fig:pure_en-pop_efp_CT} and
\ref{fig:pure_en-pop_efp_TotalE}
are from the 6-311++G** basis set.


The next set of plots of the differences between SAPT and EFP results.
For each energy, there is a plot of the absolute differences followed by a plot of the relative differences.
The order of energies follow that of the previous plots, i.e. electrostatics, exchange-repulsion, induction-polarization, dispersion, charge-transfer and then the total interaction energy.
These are displayed in figures
\ref{fig:adiff_en-avgBasis_Elec} through to
\ref{fig:adiff_en-avgBasis_TotalE} and figure
\ref{fig:rdiff_en-avgBasis_TotalE}.


%Scatterplots to show the correlation between SAPT and EFP data are displayed next. 
%These are in figures
%\ref{fig:corr-Elec} to
%\ref{fig:corr-TotalE}.

A final plot that has all of the different energies together is figure
\ref{fig:corr-all_En}. 
This time the shape of a point indicates which energy it is, and the colour corresponds to the basis set.


The geodesic scheme results from the two basis sets used are plotted next, in graphs 
\ref{fig:geodCT-adz} and
\ref{fig:geodCT-atz}.
The correlation with the charge-transfer energy from SAPT is shown in 
\ref{fig:geodCT-corr}.
Note the scales of the axis.


\clearpage

\renewcommand{\headrulewidth}{0pt}
%\newgeometry{top = 2.0cm, bottom = 2.0cm, left = 1.5cm, right = 1.5cm}

%\begin{adjustwidth}{-2cm}{-2cm}

%Due to the large number of figures in this report, they are placed in this appendix. 

% note that the {../path/to/file.name}.pdf is so that includegraphicx doesn't complain about 
% not knowing about the file extension; it takes everything after the first period

%\subsection{SAPT aug-cc-pVDZ}

% SAPT energies
%The SAPT energies are shown in 
%figures \ref{fig:pure_en-sapt_Elec}, \ref{fig:pure_en-sapt_Exch}, \ref{fig:pure_en-sapt_Ind}, 
%\ref{fig:pure_en-adz_efp_Disp}, \ref{fig:pure_en-sapt_CT} and \ref{fig:pure_en-sapt_TotalE}.

\plot{fig:pure_en-sapt_Elec}{\string~/Dropbox/Computational_Data/images/pure_en/il_adz_sapt_Electrostatics.pdf}{SAPT Electrostatic Energy}

\plot{fig:pure_en-sapt_Exch}{\string~/Dropbox/Computational_Data/images/pure_en/il_adz_sapt_Exchange.pdf}{SAPT Exchange Energy}

%\end{adjustwidth}

\plot{fig:pure_en-sapt_Ind}{\string~/Dropbox/Computational_Data/images/pure_en/il_adz_sapt_Induction.pdf}{SAPT Induction Energy}

\plot{fig:pure_en-sapt_Disp}{\string~/Dropbox/Computational_Data/images/pure_en/il_adz_sapt_Dispersion.pdf}{SAPT Dispersion Energy}

\plot{fig:pure_en-sapt_CT}{{\string~/Dropbox/Computational_Data/images/pure_en/il_adz_sapt_SAPT.Charge.Transfer}.pdf}{SAPT Charge-transfer Energy}

\plot{fig:pure_en-sapt_TotalE}{{\string~/Dropbox/Computational_Data/images/pure_en/il_adz_sapt_Total.SAPT2.3}.pdf}{SAPT Total Energy}

%\subsection{EFP aug-cc-pVDZ}
% EFP energies

% EFP aug-cc-pVDZ
%The EFP energies using the aug-cc-pVDZ basis set are shown in 
%figures \ref{fig:pure_en-adz_efp_Elec}, \ref{fig:pure_en-adz_efp_Repl}, \ref{fig:pure_en-adz_efp_Pol}, 
%\ref{fig:pure_en-adz_efp_Disp}, \ref{fig:pure_en-adz_efp_CT} and \ref{fig:pure_en-adz_efp_TotalE}.

\plot{fig:pure_en-adz_efp_Elec}{\string~/Dropbox/Computational_Data/images/pure_en/il_adz_efp_Electrostatic.pdf}{EFP Electrostatic Energy}

\plot{fig:pure_en-adz_efp_Repl}{\string~/Dropbox/Computational_Data/images/pure_en/il_adz_efp_Repulsion.pdf}{EFP Repulsion Energy}

\plot{fig:pure_en-adz_efp_Pol}{\string~/Dropbox/Computational_Data/images/pure_en/il_adz_efp_Polarization.pdf}{EFP Polarization Energy}

\plot{fig:pure_en-adz_efp_Disp}{\string~/Dropbox/Computational_Data/images/pure_en/il_adz_efp_Dispersion.pdf}{EFP Dispersion Energy}

\plot{fig:pure_en-adz_efp_CT}{{\string~/Dropbox/Computational_Data/images/pure_en/il_adz_efp_Charge_Transfer}.pdf}{EFP Charge-transfer Energy}

\plot{fig:pure_en-adz_efp_TotalE}{{\string~/Dropbox/Computational_Data/images/pure_en/il_adz_efp_Total_EFP_Energy}.pdf}{EFP Total Energy}

%\subsection{EFP aug-cc-pVTZ}
\clearpage

% EFP aug-cc-pVTZ
%The EFP energies using the aug-cc-pVTZ basis set are shown in 
%figures \ref{fig:pure_en-atz_efp_Elec}, \ref{fig:pure_en-atz_efp_Repl}, \ref{fig:pure_en-atz_efp_Pol}, 
%\ref{fig:pure_en-atz_efp_Disp}, \ref{fig:pure_en-atz_efp_CT} and \ref{fig:pure_en-atz_efp_TotalE}.

\plot{fig:pure_en-atz_efp_Elec}{\string~/Dropbox/Computational_Data/images/pure_en/il_atz_efp_Electrostatic.pdf}{EFP Electrostatic Energy}

\plot{fig:pure_en-atz_efp_Repl}{\string~/Dropbox/Computational_Data/images/pure_en/il_atz_efp_Repulsion.pdf}{EFP Repulsion Energy}

\plot{fig:pure_en-atz_efp_Pol}{\string~/Dropbox/Computational_Data/images/pure_en/il_atz_efp_Polarization.pdf}{EFP Polarization Energy}

\plot{fig:pure_en-atz_efp_Disp}{\string~/Dropbox/Computational_Data/images/pure_en/il_atz_efp_Dispersion.pdf}{EFP Dispersion Energy}

\plot{fig:pure_en-atz_efp_CT}{{\string~/Dropbox/Computational_Data/images/pure_en/il_atz_efp_Charge_Transfer}.pdf}{EFP Charge-transfer Energy}

\plot{fig:pure_en-atz_efp_TotalE}{{\string~/Dropbox/Computational_Data/images/pure_en/il_atz_efp_Total_EFP_Energy}.pdf}{EFP Total Energy}

%\subsection{EFP 6-311++G**}
\clearpage 

% EFP Pople basis set
%The EFP energies using the 6-311++G** basis set are shown in 
%figures \ref{fig:pure_en-pop_efp_Elec}, \ref{fig:pure_en-pop_efp_Repl}, \ref{fig:pure_en-pop_efp_Pol}, 
%\ref{fig:pure_en-pop_efp_Disp}, \ref{fig:pure_en-pop_efp_CT} and \ref{fig:pure_en-pop_efp_TotalE}.

\plot{fig:pure_en-pop_efp_Elec}{\string~/Dropbox/Computational_Data/images/pure_en/il_pop_efp_Electrostatic.pdf}{EFP Electrostatic Energy}

\plot{fig:pure_en-pop_efp_Repl}{\string~/Dropbox/Computational_Data/images/pure_en/il_pop_efp_Repulsion.pdf}{EFP Repulsion Energy}

\plot{fig:pure_en-pop_efp_Pol}{\string~/Dropbox/Computational_Data/images/pure_en/il_pop_efp_Polarization.pdf}{EFP Polarization Energy}

\plot{fig:pure_en-pop_efp_Disp}{\string~/Dropbox/Computational_Data/images/pure_en/il_pop_efp_Dispersion.pdf}{EFP Dispersion Energy}

\plot{fig:pure_en-pop_efp_CT}{{\string~/Dropbox/Computational_Data/images/pure_en/il_pop_efp_Charge_Transfer}.pdf}{EFP Charge-transfer Energy}

\plot{fig:pure_en-pop_efp_TotalE}{{\string~/Dropbox/Computational_Data/images/pure_en/il_pop_efp_Total_EFP_Energy}.pdf}{EFP Total Energy}

%\subsection{Absolute Difference between SAPT and EFP}
\clearpage

% difference between SAPT and EFP
% comparing basis sets by averaging across conf

%All the raw energies have thus been presented. What follows will be the comparisons between energies.

% first the absolute errors
% followed by the relative errors
%The absolute differences in energies, averaged across configurations, and split by basis sets, are shown in 
%figures \ref{fig:adiff_en-avgBasis_Elec}, \ref{fig:adiff_en-avgBasis_ExchRepl}, 
%\ref{fig:adiff_en-avgBasis_IndPol}, \ref{fig:adiff_en-avgBasis_Disp}, 
%\ref{fig:adiff_en-avgBasis_CT} and \ref{fig:adiff_en-avgBasis_TotalE}.

\plot{fig:adiff_en-avgBasis_Elec}{\string~/Dropbox/Computational_Data/images/diff_en/il_adiff_avgBasis_Elec.pdf}{Absolute error for Electrostatics between the different basis sets}

\plot{fig:rdiff_en-avgBasis_Elec}{\string~/Dropbox/Computational_Data/images/diff_en/il_rdiff_avgBasis_Elec.pdf}{Relative error for Electrostatics between the different basis sets}


\plot{fig:adiff_en-avgBasis_ExchRepl}{{\string~/Dropbox/Computational_Data/images/diff_en/il_adiff_avgBasis_Exch.Repl}.pdf}{Absolute error for Exchange-Repulsion between the different basis sets}

\plot{fig:rdiff_en-avgBasis_ExchRepl}{{\string~/Dropbox/Computational_Data/images/diff_en/il_rdiff_avgBasis_Exch.Repl}.pdf}{Relative error for Exchange-Repulsion between the different basis sets}
}
}
  \plot{fig:adiff_en-avgBasis_IndPol}{{\string~/Dropbox/Computational_Data/images/diff_en/il_adiff_avgBasis_Ind.Pol}.pdf}{Absolute error for Induction-Polarization between the different basis sets}
}
\plot{fig:rdiff_en-avgBasis_IndPol}{{\string~/Dropbox/Computational_Data/images/diff_en/il_rdiff_avgBasis_Ind.Pol}.pdf}{Relative error for Induction-Polarization between the different basis sets}


\plot{fig:adiff_en-avgBasis_Disp}{{\string~/Dropbox/Computational_Data/images/diff_en/il_adiff_avgBasis_Disp}.pdf}{Absolute error for Dispersion between the different basis sets}

\plot{fig:rdiff_en-avgBasis_Disp}{{\string~/Dropbox/Computational_Data/images/diff_en/il_rdiff_avgBasis_Disp}.pdf}{Relative error for Dispersion between the different basis sets}


\plot{fig:adiff_en-avgBasis_CT}{{\string~/Dropbox/Computational_Data/images/diff_en/il_adiff_avgBasis_CT}.pdf}{Absolute error for Charge-transfer between the different basis sets}

\plot{fig:rdiff_en-avgBasis_CT}{{\string~/Dropbox/Computational_Data/images/diff_en/il_rdiff_avgBasis_CT}.pdf}{Relative error for Charge-transfer between the different basis sets}


\plot{fig:adiff_en-avgBasis_TotalE}{{\string~/Dropbox/Computational_Data/images/diff_en/il_adiff_avgBasis_Total.E}.pdf}{Absolute error for Total Energy between the different basis sets}

\plot{fig:rdiff_en-avgBasis_TotalE}{{\string~/Dropbox/Computational_Data/images/diff_en/il_rdiff_avgBasis_Total.E}.pdf}{Relative error for Total Energy between the different basis sets}


\clearpage

% comparing without CT
%Here in 
%figures \ref{fig:adiff_en-avgBasis_IndPol-no_ct} 
%and \ref{fig:rdiff_en-avgBasis_IndPol-no_ct}
%are the Induction-Polarization energies \textbf{without the charge-transfer energy} added. 
%That is, the previous graphs use the total induction energy from SAPT.

%\plot{fig:adiff_en-avgBasis_IndPol-no_ct}{{../../Computational_Data/images/diff_en/il_adiff_avgBasis_Ind.Pol_old}.pdf}{Absolute error for Induction-Polarization between the different basis sets, without CT}

%\plot{fig:rdiff_en-avgBasis_IndPol-no_ct}{{../../Computational_Data/images/diff_en/il_rdiff_avgBasis_Ind.Pol_old}.pdf}{Relative error for Induction-Polarization between the different basis sets, without CT}


%% correlation plots
%\plot{fig:corr-Elec}{{\string~/Dropbox/Computational_Data/images/diff_en/corr_en_Elec}.pdf}{Correlating SAPT and EFP for Electrostatics comparing basis sets and anions}
%
%\plot{fig:corr-ExchRepl}{{\string~/Dropbox/Computational_Data/images/diff_en/corr_en_Exch.Repl}.pdf}{Correlating SAPT and EFP for Exchange-Repulsion comparing basis sets and anions}
%
%\plot{fig:corr-IndPol}{{\string~/Dropbox/Computational_Data/images/diff_en/corr_en_Ind.Pol}.pdf}{Correlating SAPT and EFP for Induction-Polarization comparing basis sets and anions}
%
%\plot{fig:corr-Disp}{{\string~/Dropbox/Computational_Data/images/diff_en/corr_en_Disp}.pdf}{Correlating SAPT and EFP for Dispersion comparing basis sets and anions}
%
%\plot{fig:corr-CT}{{\string~/Dropbox/Computational_Data/images/diff_en/corr_en_CT}.pdf}{Correlating SAPT and EFP for Charge-transfer comparing basis sets and anions}
%
%\plot{fig:corr-TotalE}{{\string~/Dropbox/Computational_Data/images/diff_en/corr_en_Total.E}.pdf}{Correlating SAPT and EFP for Total Interaction Energy comparing basis sets and anions}

\begin{sidewaysfigure}
    \centering
    \includegraphics[width = \linewidth]{{\string~/Dropbox/Computational_Data/images/diff_en/corr_en_all}.pdf}
    \caption{Correlating SAPT and EFP across all energies comparing basis sets 
            \label{fig:corr-all_En}}
\end{sidewaysfigure}


% comparing errors from different components
\begin{sidewaysfigure}[ht]
    \centering
    \includegraphics[width = \linewidth]{{\string~/Dropbox/Computational_Data/images/diff_en/il_adiff_all_CatAn_barPlot}.pdf}
    \caption{Errors from the different energy components and basis sets used, averaged across configurations and alkyl chain lengths 
            \label{fig:adiff_en-barplot_all}}
\end{sidewaysfigure}


%geodesic plots

\plot{fig:geodCT-adz}{{\string~/Dropbox/Computational_Data/images/pure_en/il_adz_geodCT}.pdf}{Charge-transfer from geodesic scheme, double zeta}

\plot{fig:geodCT-atz}{{\string~/Dropbox/Computational_Data/images/pure_en/il_atz_geodCT}.pdf}{Charge-transfer from geodesic scheme, triple zeta}

\plot{fig:geodCT-corr}{{\string~/Dropbox/Computational_Data/images/diff_en/geod_corr_ct}.pdf}{Correlation scatterplot of the charge-transfer from the geodesic scheme against the charge-transfer energy calculated from SAPT}

%\restoregeometry
%\end{adjustwidth}

%\end{document}


\end{appendices}

\end{document}
